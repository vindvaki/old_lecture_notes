% Höfundur: Hörður Freyr Yngvason

\documentclass[a4paper,icelandic,11pt]{book}

% Use utf-8 encoding for foreign characters
\usepackage[T1]{fontenc}
\usepackage[utf8]{inputenc}
\usepackage{babel}

\usepackage{enumerate}

% Setup for fullpage use
% \usepackage{fullpage}

% Document font
% \usepackage{cmbright}

% Other fonts
% \usepackage{mathrsfs}
\usepackage{ulem} % for strikethrough

% Running Headers and footers
%\usepackage{fancyhdr}

% Drawing and colors
\usepackage{framed}
\usepackage{pstricks}

\newrgbcolor{fillgreen}{.85 1 .81}
\newrgbcolor{fillorange}{.98 1 .82}
\newrgbcolor{fillblue}{.86 .86 1}

% More symbols
\usepackage{amsmath}
\usepackage{amssymb}

% Theorem
\usepackage[framed,thmmarks,amsmath,hyperref]{ntheorem}
\qedsymbol{\ensuremath{_\Box}}

\theoremseparator{.}
\theoremstyle{plain}
\theorembodyfont{\upshape}

\def\theoremframecommand{%
  \psframebox[fillstyle=solid,fillcolor=fillorange,linecolor=orange]}
\newshadedtheorem{skilgr}{Skilgreining}[chapter]

\def\theoremframecommand{%
  \psframebox[fillstyle=solid,fillcolor=fillgreen,linecolor=green]}
\newshadedtheorem{setn}{Setning}[chapter]
\newshadedtheorem{fylgisetn}{Fylgisetning}[chapter]

\def\theoremframecommand{%
  \psframebox[fillstyle=solid,fillcolor=fillblue,linecolor=blue]}
\newshadedtheorem{hjalparsetn}{Hjálparsetning}[chapter]

\theoremsymbol{\ensuremath{_\Box}}
\newtheorem{daemi}{Dæmi}[chapter]

\theoremheaderfont{\normalfont\itshape}
\theoremsymbol{}
\newtheorem*{ath}{Athugasemd}
\theoremsymbol{\ensuremath{_\blacksquare}}
\newtheorem*{sonnun}{Sönnun}


% Örvarit
\usepackage[all]{xy}

% ýmis mengi
\newcommand{\R}{\mathbb{R}}
\newcommand{\N}{\mathbb{N}}
\newcommand{\Z}{\mathbb{Z}}
\newcommand{\Q}{\mathbb{Q}}
\newcommand{\C}{\mathbb{C}}

% Föll fyrir áfangann
\DeclareMathOperator{\ssd}{ssd} % stærsti samdeilir
\DeclareMathOperator{\id}{id} % samsemdarvörpunin
\DeclareMathOperator{\adj}{adj} % aðokafylki
\DeclareMathOperator{\im}{Im} % mynd vörpunar / þverhluti tvinntölu
\DeclareMathOperator{\Ker}{Ker} % kjarni vörpunar
\DeclareMathOperator{\sign}{sign} % formerki umröðunar
\DeclareMathOperator{\ivarp}{in} % ívarp
\DeclareMathOperator{\pr}{pr} % ofanvarp
\DeclareMathOperator{\kennitala}{char} % kennitala 
\DeclareMathOperator{\stig}{stig} % stig margliðu
\DeclareMathOperator{\re}{Re} % raunhluti tvinntölu
\DeclareMathOperator{\Aut}{Aut} % sjálfmótanagrúpa
\DeclareMathOperator{\Inn}{Inn} % hlutgrúpa innri sjálfmótana


% If you want to generate a toc for each chapter (use with book)
\usepackage{minitoc}

% For indexing
\usepackage{makeidx}
\makeindex

% Hyperlinking o.fl.
\usepackage[bookmarks=true,bookmarksnumbered=true]{hyperref}

% This is now the recommended way for checking for PDFLaTeX:
\usepackage{ifpdf}

\ifpdf
\usepackage[pdftex]{graphicx}
\else
\usepackage{graphicx}
\fi

\title{\textbf{Algebra I} \\ Fyrirlestrar Reynis Axelssonar}
\author{Hörður Freyr Yngvason}
\date{Haustönn 2009}

\begin{document}

\normalem % út af \usepackage{ulem}

\maketitle
\tableofcontents

%%
%% 26. ágúst 2009
%%

\part{Einfaldir hlutir úr einfaldari talnafræði}
\chapter{Inngangsefni}
\section{Deilanleiki}
Táknum með $\N = \{ 0,1,2,3,\dots \}$ mengi allra náttúrlegu talnanna og með
$\Z = \{n\in\R:n\in\N, -n\in\N\}$. Látum $a,b,\in\Z$. Við segjum að talan $a$
\emph{gangi upp í } tölunni $b$ ef til er heiltala $c$ þ.a. $b=ac$ og við
skrifum þá 
\[
  a\mid b .
\]
Látum $m\in\Z$; við setjum \[m\Z := \{nm:n\in\Z \}. \]
Eftirfarandi skilyrði eru jafngild:
\begin{itemize}
  \item [(i)] $a\mid b$,
  \item [(ii)] $b\in a\Z$,
  \item [(iii)] $b\Z\subset a\Z $.
\end{itemize}
\begin{setn}
  [Einfaldari reglur]
  \begin{itemize}
  \item [(1)] Fyrir öll $a\in \Z$ er $1\mid a$ og $-1\mid a$. Ef $a\mid 1$, þá er $a=1$ eða $a=-1$.
  \item [(2)] Ef $a\in\Z$, þá $a\mid 0$. Ef $0\mid a$ þá er $a=0$.
  \item [(3)] Ef $a\mid b$ og $b\mid c$ þá $a\mid c$.
  \item [(4)] Ef $a\mid b$ og $b\mid a$, þá er $a=b$ eða $a=-b$.
  \item [(5)] Fyrir öll $a\in\Z$ er $a\mid a$.
  \item [(6)] Ef $a\mid b$ og $a\mid c$ þá \[a\mid nb + mc \] fyrir öll $n,m\in\Z$.
  \end{itemize}
\end{setn}
\begin{sonnun}
  Er auðveld.
\end{sonnun}
\begin{setn}
  [Um deilingu með afgangi]
  \label{sec:einfaldir-hlutir-ur-1}
  \index{deiling!með afgangi}
  Látum $m\in\Z,m\ge 1$. Fyrir sérhverja tölu $n$ eru til heilar tölur $q,r$ þ.a.
  \[ n= qm + r \qquad \text{og} \qquad 0\le r < m. \]
  Tölurnar $q$ og $r$ ákvarðast ótvírætt af þessum skilyrðum.
\end{setn}
Til að sanna þetta notum við:
\begin{setn}
  [Um minnsta stak]
  Ef $A$ er hlutmengi í $\N$ og $A\neq \emptyset$ þá hefur $A$ minnsta stak,
  þ.e. til er stak $a\in A$ þ.a. $a\le x$ fyrir öll $x\in A$.
\end{setn}
Gefum okkur þetta án sönnunar!
\begin{sonnun}
  [Sönnun á setningu \ref{sec:einfaldir-hlutir-ur-1}] Látum $n,m\in\Z,m\ge
  1$. Setjum
  \[ A:= \{x\in\N: \text {Til er } q\in\Z \text{ þ.a. } x=n-qm\} .\]
  Þá er $A\neq \emptyset$, því að $n+m = n-(-1)m \in \N$ og því $n+m\in A$.
  Þar með hefur $A$ minnsta stak $r$; og til er $q\in\Z $ þ.a. $r=n-qm$. Þá er
  $0 \le r < m$: Vegna $r\in\N$ er $r\ge 0$; ef við hefðum $r\ge m$ þá væri
  $r_1 = r-m \ge 0$ og þá er $r_1 \in \N$, ef $q$ er þ.a. $ r = n-qm,q\in\Z$,
  þá er $r_1 = n-qm - m = n-(q+1)m$, svo að $r_1 \in A$, $0\le r_1 < r$ í
  mótsögn við skilgreiningu.

  \emph{Ótvíræðni:} Ef $n=qm+r = q_1 m+r_1$ þar sem $q,q_1\in\Z, r,r_1 \in\Z,
  0\le r <m, 0\le r_1 < m$, þá er $r-r_1 \in\Z, -m < r-r_1 < m$. En $r-r_1 =
  (q_1 - q)m$, og eina heila margfeldið af $m$ sem er stærra en $-m$ og minna
  en $m$ er $0$, svo að $r-r_1 = 0$. En þá er $(q_1 - q)m=r - r_1 = 0$, svo að
  $q-q_1 = 0$. Því er $r_1 = r$ og $q_1 = q$.
\end{sonnun}
\begin{setn}
  Látum $A$ vera hlutmengi í $\Z$ þ.a. eftirfarandi tveimur skilyrðum sé fullnægt:
  \begin{itemize}
  \item [(i)] $A\neq \emptyset$
  \item [(ii)] Ef $x,y\in A$, þá er $x-y\in A$.
  \end{itemize}
  Þá er til nákvæmlega ein náttúrleg tala $m$ þ.a. $a = m\Z$.
\end{setn}
\begin{sonnun}
  Athugum: Þar sem $A\neq \emptyset$ hefur $A$ stak $x$; skv. (ii) er þá $0 =
  x-x\in A$. Höfum þá líka: Ef $x\in A$, þá er $-x = 0 -x \in A$. Ef $x,y\in
  A$ þá er $-y\in A$ og því $x+y = x-(-y)\in A$. Fáum nú: Ef $x\in A$, þá er
  $2x = x+ x \in A$ svo að $3x = 2x + x \in A$, $4x = 3x + x \in A$ og með
  þrepun fæst $nx \in A$ fyrir öll $n\in\N$. Þá er líka $(-n)x = -nx \in A$
  fyrir öll $n\in\N$. Höfum því sýnt: Ef $x\in A$, þá er $x\Z \subset A$.
  Fyrir $x\in A$ er $-x\in A$ og annaðhvort er $x\ge 0$ eða $-x \ge 0$ svo
  að $x\in\N$ eða $-x\in\N$. Ef $A=\{0\}$, þá er $A=0\Z$. Gerum ráð fyrir að
  $A\neq \{0\}$. Þá er
  \[ A \cap \N \neq \emptyset . \]
  En þá hefur $A\cap\N$ minnsta stak $m$. Höfum sýnt að $m\Z\subset A$. Látum
  nú $n\in A$. Skv. setningu um deilingu með afgangi eru til heilar tölur
  $q,r$ þ.a. $n=qm+r$ og $0\le r < m$. En $n\in A$, og $m\in A$ og því $qm\in
  A$, svo að $r = n - qm \in A$ og $r\ge 0$ svo að $r\in A\cap \N, r < m$.
  Þetta er í mótsögn við skilgreiningu á $m$ nema $r=0$. Því verður $r = 0$ að
  gilda, svo að $n = qm \in m\Z$. Höfum þá að $A\subset m\Z$ og $m\Z \subset
  A$ svo að $A = m\Z$.

  Ef $A = m\Z = m_1 \Z$ með $m,m_1 \in \Z$, þá $m \mid m_1$ og $m_1 \mid m$ svo $m =
  m_1$ eða $m = -m_1$, en $m_1 \ge 0$ svo $m = m_1$.
\end{sonnun}
Látum $a,b\in\Z$ og setjum
\[ A:= a\Z + b\Z = \{ aj + bk : j,k \in \Z \} \]
Þá fullnægir mengið $A$ skilyrðunum (i) og (ii); (i) er augljóst og ef $x = aj
+ bk$ og $y = aj_1 + bk_1$ þá er $x-y = a(j-j_1) + b(k-k_1) \in A$. Þar með er
til nákvæmlega ein náttúrleg tala $d$ þ.a.
\[ a\Z + b\Z = d\Z . \]
Höfum $a\Z, b\Z \subset a\Z + b\Z = d\Z$. Fáum:
\begin{itemize}
  \item [(i)] $d\mid a$ og $d\mid b$.
  \item [(ii)] Ef $c\mid a$ og $c\mid b$ þá $c\mid d$.
\end{itemize}
Eiginleiki (i) er augljós af $a\Z \subset d\Z$ og $b\Z \subset d\Z$. Ef $c\mid a$
og $c\mid b$, þá er $a\Z \subset c\Z$ og $b\Z \subset c\Z$. Þá er ljóst að
\[ d\Z = a\Z + b\Z \subset c\Z \]
svo að $c\mid d$. Líka: Ef $d_1$ fullnægir:
\begin{itemize}
  \item [(i')] $d_1 \mid a$ og $d_1 \mid b$.
  \item [(ii')] Ef $c\mid a$ og $c\mid b$ þá $c\mid d_1$.
\end{itemize}
þá gefa (i) og (ii') að $d\mid d_1$ og (i') og (ii) að $d_1 \mid d$; ef $d$ og $d_1$
eru bæði í $\N$ þá fæst $d=d_1$. M.ö.o. $d$ er \emph{eina} náttúrlega talan
sem fullnægir skilyrðunum (i) og (ii).

\section{Stærsti samdeilir}
\begin{skilgr}
  Látum $a,b\in\Z$. Náttúrlega talan $d$ sem fullnægir (i) og (ii) kallast
  \emph{stærsti samdeilir}\index{stærsti samdeilir} $a$ og $b$; og er táknuð
  \[ \ssd (a,b) .\]
\end{skilgr}

%%
%% 27. ágúst 2009
%%

\begin{hjalparsetn}
  Látum $a,b,q,r \in\Z$ vera þ.a.
  \[ a = bq + r .\]
  Þá er
  \[ \ssd (a,b) = \ssd (b,r) \]
\end{hjalparsetn}
\begin{sonnun}
  Ef $c\mid a$ og $c\mid b$, þá $c\mid r$ vegna $ r = a - bq $. Eins ef $c\mid
  b$ og $c\mid r$, þá
  $c\mid a$. Höfum því
  \[ (c\mid a \text{ og } c\mid b) \text{ þ.þ.a.a. } (c\mid b \text { og } c\mid r) \]
\end{sonnun}
\begin{ath}
  Ef $a,b \in \N$ og $a\mid b,b\neq 0$, þá er $a\le b$. Ef önnur talnanna $a,b$
  er núll, þá er stærsti samdeilirinn algildi hinnar tölunnar. Sér í lagi er
  $\ssd (0,0) = 0$. Það er því ljóst hver stærsti samdeilirinn er nema hvorug
  talnanna sé núll. Líka er $a\Z = (-a) \Z$, svo
  \[ \ssd (-a,b) = \ssd(a,-b) = \ssd(-a,-b) = \ssd(a,b) . \]
  Þurfum bara að finna stærsta samdeili \emph{jákvæðu} talnanna.
\end{ath}
\begin{setn}
  [Reiknirit Evklíðs]\index{Evklíðs!reiknirit}
  Látum $a,b \in \Z$, $0 < b < a$. Deilum $b$ upp í $a$ með afgangi:
\[ a = bq_1 + r_1, \qquad 0 \le r_1 < b; \]
deilum $r_1$ upp í $b$ með afgangi:
\[ b = r_1 q_2 + r_2, \qquad 0\le r_2 < r_1 ; \]
deilum $r_2$ upp í $r_2$ með afgangi:
\[ r_1 = r_2 q_3 + r_3 \]
o.s.frv. Deilum almennt $r_{k+1}$  upp í $r_k$ með afgangi, fáum
\[r_k = r_{k+1} q_{q+2} + r_{k+2}, \qquad 0\le r_{k+2} < r_{k+1} \]
meðan $r_k$ verður ekki núll. Fáum stranglega fallandi runu $(r_k)$ af
náttúrlegum tölum; hún endar með að við fáum $r_{k+1} = 0$ . Þá er
\[ \ssd (a,b) = r_k . \]
\end{setn}
\begin{sonnun}
  Skv. hjálparsetningu er
  \[ \ssd(a,b) = \ssd(b,r_2) = \ssd(r_1,r_2) = \ssd(r_2,r_3) =\cdots =
  \ssd(r_k,r_{k+1}) =\ssd(r_k,0) = r_k.\]
\end{sonnun}
\begin{ath}
  Látum $d=\ssd(a,b)$. Það þýðir að $a\Z + b\Z = d\Z$. Sér í lagi hefur jafnan
  \[ ax + by = d \]
  lausn í heilum tölum $x$ og $y$. Þessi jafna kallast
  \emph{Bézout-jafnan}\index{Bézout-jafnan}
  fyrir $a$ og $b$. Með því að „fara afturábak“ gegnum reiknirit Evklíðs fæst
  \emph{ein} lausn á Bézout-jöfnunni.
\end{ath}
\begin{daemi}
  Reiknum stærsta samdeili talnanna $75$ og $261$:
  \begin{align*}
    261 &= 3\cdot 75 + 36 \\
    75 &= 2\cdot 36 + 3 \\
    36 &= 3\cdot 12 + 0,
  \end{align*}
  Svo stærsti samdeilirinn er $3$.
  Höfum þá:
  \begin{align*}
    75 &= 2\cdot 36 + 3, \\
    261 &= 3\cdot 75 + 36,
  \end{align*}
  það gefur:
  \begin{align*}
    3 &= 75 - 36 \cdot 2 \\
    &= 75 - (261 - 3\cdot 75) \cdot 2 \\
    &= 75 \cdot 7 - 261 \cdot 2,
  \end{align*}
  svo að $x = 7, y= -2$ er ein lausn Bézout-jöfnunnar
  \[ 75x + 261y = 3. \]
\end{daemi}
\begin{skilgr}
  Segjum að heilu tölurnar $a,b$ séu
  \emph{ósamþátta}\index{osamthatta@ósamþátta!tölur}\index{osamthatta@ósamþátta}
  ef $\ssd(a,b) = 1$.
\end{skilgr}
\begin{ath}
  Tölurnar $a,b$ eru ósamþátta þ.þ.a.a. jafnan \[ ax+by = 1 \] hafi lausn í
  heilum tölum $x,y$. Almennt hefur jafna \[ax + by = c\] lausn í heilum tölum
  þ.þ.a.a. $\ssd(a,b) \mid c$.
\end{ath}

\section{Frumtölur}
\begin{skilgr}
  \emph{Frumtala}\index{frumtala} er náttúrleg tala $p>1$ þ.a. engin náttúrleg tala
  gangi upp í $p$ nema $1$ og $p$.
\end{skilgr}
\begin{ath}
  Tölurnar sem ganga upp í frumtölu $p$ eru þá $1,p,-1$ og $-p$.
\end{ath}
\begin{hjalparsetn}
  [Hjálparsetning Evklíðs]\index{Evklíðs!hjálparsetning}
  Ef $p$ er frumtala, $a$ og $b$ eru heilar tölur og $p \mid ab$, þá $p\mid a$
  eða $p\mid b$.
\end{hjalparsetn}
\begin{sonnun}
  Gerum ráð fyrir að $p\mid ab$ en að $p$ gangi ekki upp í $a$, þá er $\ssd(p,a) =
  1$, því að $\ssd(p,a)$ gengur upp í $p$, er því $1$ eða $p$ en getur ekki
  verið $p$. Þar með eru til heilar tölur $x$ og $y$ þ.a.
  \[ px + ay = 1 . \]
  En þá er $ b =  p(bx) + ab\cdot y$, og $p\mid ab$, svo að $p\mid b$.
\end{sonnun}
\begin{skilgr}
  Látum $n\in\N,n\ge 1$. \emph{Frumþáttun}\index{frumþáttun}
  tölunnar er að
  skrifa $n$ sem margfeldi
  \[ n = p_1 \cdots p_r \]
  þar sem $p_1, \dots , p_r$ eru frumtölur, sem kallast þá \emph{frumþættir}
  \index{frumþættir}frumþáttunarinnar.
\end{skilgr}
Leyfum $r = 1$, þá er $n$ frumtala; þægilegt er að leyfa $r = 0$ með því
samkomulagi að $\prod_{k=1}^0 p_k = 1$. Þá má segja:
\begin{setn}
  [Undirstöðusetning reikningslistarinnar]
  \index{Undirstöðusetning!reikningslistarinnar}
  Sérhver náttúrleg tala $n$ þ.a. $n\ge 1$ hefur frumþáttun og frumþættirnir
  ákvarðast ótvírætt burtséð frá röð.
\end{setn}
\begin{skilgr}
  Getum því kallað þá frumþætti \emph{tölunnar}\index{frumþættir!tölu} $n$.
\end{skilgr}
\begin{setn}
  [Evklíð]
  Til eru óendanlega margar frumtölur.
\end{setn}
\begin{sonnun}
  Gerum ráð fyrir að þær séu endanlega margar, segjum $p_1,\dots,p_r$. Setjum \[
  x := p_1\cdots p_r +1.
  \]
  Nú er til frumtala $p$ sem gengur upp í $x$. En $p\neq p_j$ fyrir
  $j=1,\dots,r$: Ef $p_j\mid x$ þá gengi $p_j$ upp í $x-p_1\cdots p_r=1$.
\end{sonnun}
\begin{sonnun}[á undirstöðusetningu reikningslistarinnar]
  (1) Tala $n\ge 1$ hefur frumþáttun. Annars væri til \emph{minnsta} tala $n$
  sem hefur ekki frumþáttun. Hún er þá ekki talan $1$ og ekki frumtala og því
  má skrifa $n = km$, þar sem $k,m$ eru náttúrlegar tölur ólíkar $n$ og $1$;
  en þá eru $k,m < n$ og því hafa $k$ og $m$ frumþáttanir $k = p_1 \cdots p_r$
  og $m = q_1 \cdots q_s$, en þá er
  \[ n = p_1 \cdots p_r \cdot q_1 \cdots q_s \]
  frumþáttun á $n$; mótsögn.
  
  (2) Látum $n = p_1 \cdots p_r = q_1 \cdots q_s$ vera frumþáttanir. Þá gildir
  $p_r \mid q_1 \cdots q_s$; skv. hjálparsetningu Evklíðs og þrepun, þá gengur
  $p_r$ upp í einni af tölunum $q_1,\dots,q_s$. Með því að breyta um röð á
  $q_1, \dots, q_s$ má gera ráð fyrir að $p_r \mid q_s$. En $q_s$ er frumtala,
  $p_r > 1$, svo að $p_r = q_s$. En þá er $p_1 \cdots p_{r-1} = q_1 \cdots
  q_{s-1}$. Með þrepun sést að $r - 1 = s - 1$ og $p_1, \dots, p_{r-1}$ eru
  tölurnar $q_1, \dots, q_{s-1}$. En þá er $r = s$ og $p_1, \dots, p_r$ og
  $q_1, \dots, q_s$ eru sömu tölurnar, en hugsanlega í annarri röð.
\end{sonnun}
%%
%% Fyrirlestur 28. ágúst 2009
%%
Látum $a\in\Z$ og $m\in\N,m\ge 1$. Skrifum \[ a\bmod m \] fyrir afganginn sem
fæst þegar við deilum $m$ upp í $a$; m.ö.o. ef $a = mq+r$ og $0\le r < m$; þá
er \[ a\bmod m := r .\]
\begin{ath}
  Í kennslubók stendur (bls. 9):
  \begin{quote}
    When $a = qn+r$, where $q$ is the quotitent and $r$ is the remainder upon
    dividing $a$ by $n$, we write $a\bmod n = r$ \sout{or $a=r\bmod n$}
    \footnote{Við munum ekki nota yfirstrikaða hlutann, hunsum hann}.
  \end{quote}
\end{ath}
Hins vegar:

\section{Samleifing, mátreikningur}
\begin{skilgr}
  Látum $m\in\N,m\ge 1$, og $a,b,\in\Z$. Við skrifum
  \begin{equation}
     a\equiv b \quad (\bmod\;m ) \label{eq:mod_skilgr}
  \end{equation}
  ef $m \mid a-b$ og segjum að $a$ og $b$ séu \emph{samleifa}\index{samleifa}
  m.t.t. $m$ (eða \emph{mátað við} $m$); fullyrðingin \eqref{eq:mod_skilgr}
  kallast \emph{samleifing}\index{samleifing}.
\end{skilgr}
\begin{ath}
  Höfum $a\equiv b \; (\bmod\; m)$ þ.þ.a.a. $a\bmod m = b\mod m$.
\end{ath}
\begin{ath}
  Ef $a\equiv b \;(\bmod\; m)$ þá segjum við að $a$ og $b$ séu \emph{leifar}
  hvors annars m.t.t. $m$; \emph{leifar}\index{leif}\index{afgangur} tölunnar
  $a$ m.t.t. $m$ eru allar tölur $b$ þ.a. $a\equiv b \; (\bmod\; m)$, og
  afgangurinn $a\mod m$ er sú leif $r$ sem fullnægir $0\le r < m$.
\end{ath}
\begin{setn}
  [Reiknireglur]
  Látum $m\in\N$, $m\ge 1$ og $a,b,c,d\in\Z$.
  \begin{itemize}
    \item [(1)] $a\equiv a \; (\bmod \; m)$
    \item [(2)] Ef $a\equiv b \; (\bmod \; m)$, þá er $b\equiv a \; (\bmod \; m)$.
    \item [(3)] Ef $a\equiv b \, (\bmod \; m)$ og $b\equiv c \; (\bmod \; m)$,
      þá er $a\equiv c \; (\bmod \; m)$.
    \item [(4)] Ef $a\equiv c\; (\bmod \; m)$ og $b\equiv d \; (\bmod \; m)$, þá
      er
      \begin{align*}
        a+b &\equiv c+d \; (\bmod \; m), \\
        a-b &\equiv c-d \; (\bmod \; m), \\
        ab & \equiv cd \; (\bmod \; m)
      \end{align*}
      og fyrir allar náttúrlegar tölur $n$ er \[ a^n \equiv c^n \; (\bmod \; m) . \]
    \item [(5)] Ef $a\equiv b \; (\bmod \; m)$ og $d\in\N$ þannig að $d\mid m$ þá er
      \[ a\equiv b \; (\bmod \; d). \]
  \end{itemize}
\end{setn}
\begin{sonnun}
  Er einföld; t.d. fæst þriðja reglan í (4) með
  \[ ab -cd = ab -cb + cb -cd = (a-c)b + c(b-d). \]
\end{sonnun}
Liðir (1)-(3) segja að samleifing sé dæmi um \emph{jafngildisvensl}.
\begin{skilgr}
  \emph{Vensl}\index{vensl}
  $R$ á mengi $X$ er mengi af tvenndum $(x,y)$, þar sem $x,y\in
  X$; m.ö.o. er $R\subset X\times X$. Við skrifum $xRy$ í stað $(x,y)\in R$.
\end{skilgr}
Venja er að nota sérstök tákn fyrir vensl í stað bókstafa. Samleifing $(\bmod \;
m)$ skilgreinir vensl á $\Z$. Sem mengi eru venslin $\{ (a,b) \in \Z\times\Z :
a\equiv b \;(\bmod \; m) \}$.
\begin{skilgr}
  Látum $X$ vera mengi.
  \emph{Jafngildisvensl}\index{jafngildisvensl}\index{vensl!jafngildisvensl}
  á menginu $X$ eru vensl $~$ á $X$ þ.a. eftirfarandi þremur 
  skilyrðum sé fullnægt:
  \begin{itemize}
    \item [(i)] $x \sim x$ fyrir öll $x\in X$.
    \item [(ii)] Ef $x \sim y$, þá er $y \sim x$.
    \item [(iii)] Ef $x\sim y$ og $y\sim z$, þá er $x\sim z$.
  \end{itemize}
\end{skilgr}
\begin{skilgr}
  Látum $X$ vera mengi. \emph{Deildaskipting}\index{deildaskipting} mengisins
  $X$ er mengi $\mathcal D$ af hlutmengjum í $X$ sem eru ekki tóm þ.a. sérhvert
  stak í $X$ sé innihaldið í \emph{nákvæmlega einu} af mengjunum sem eru stök í
  $\mathcal D$; köllum stökin í $\mathcal D$ \emph{deildir}\index{deild}
  deildaskiptingarinnar. Þetta þýðir:
  \begin{itemize}
  \item [(i)] Ef $A\in\mathcal D$, þa er $\emptyset\neq A\subset X$.
  \item [(ii)] Ef $A,B\in \mathcal D$, $A\neq B$, þá er $A\cap B = \emptyset$.
  \item [(iii)] $X=\bigcup_{A\in\mathcal D} A$.
  \end{itemize}
\end{skilgr}
\begin{skilgr}
  Látum nú $\sim$ vera jafngildisvensl á mengi $X$ og $x\in X$. Mengið
  \[ [x] := \{ y\in X : x\sim y \} \]
  kallast \emph{jafngildisflokkur}\index{jafngildisflokkur}
  staksins $x$ með tilliti til jafngildisvenslanna $\sim$.
\end{skilgr}
\begin{setn}
  Látum $\sim$ vera jafngildisvensl á mengi $X$. Þá mynda jafngildisflokkarnir
  deildaskiptingu á $X$. Fyrir sérhverja deildaskiptingu $\mathcal D$
  mengisins $X$ eru til nákvæmlega ein jafngildisvensl $\sim$ á $X$ þ.a.
  $\mathcal D$ sé mengi jafngildisflokkanna með tilliti til $\sim$.
\end{setn}
\begin{sonnun}
  (1) Látum $\sim$ vera jafngildisvensl á $X$. Höfum $x\sim x$ svo að $x\in
  [x]$ og því er $[x]\neq \emptyset$. G.r.f. að $x,y\in X$ og $[x]\cap [y]
  \neq \emptyset$ og veljum $z$ úr $[x]\cap [y]$. Ef $t\in [x]$, þá er $x\sim
  t$; en líka er $x\sim z$ og $y\sim z$. Af $x\sim z$ leiðir $z\sim x$. Af
  $y\sim z$ og $z\sim x$ leiðir $y\sim x$. Af $y\sim x$ og $x\sim t$ leiðir
  $y\sim t$. Því er $t\in [y]$. Höfum sýnt að $[x] \subset [y]$. Af
  samhverfuástæðum er líka $[y]\subset [x]$, og því $[x] = [y]$. Vegna $x\in
  [x]$ er $X$ sammengi allra flokkanna.

  (2) Látum $\mathcal D$ vera deildaskiptingu mengisins $X$. Skrfium $x\sim y$
  þ.þ.a.a. $x$ og $y$ séu í sömu deild skiptingarinnar. Þá er ljóst að $\sim$
  eru jafngildisvensl og að jafngildisflokkarnir eru deildir $\mathcal D$.
\end{sonnun}

\begin{daemi}
  Jafngildisflokkur heillar tölu $a$ m.t.t. jafngildisvenslanna $ ? \equiv
  \;?\; (\bmod \; m)$ er mengið
  \[ a + m\Z = \{ a + mk : k\in \Z \} , \]
  sem er mengi allra heilla talna sem eru samleifa $a\; (\bmod \; m)$.
  Jafngildisflokkarnir eru $m$ talsins, nefnilega
  \[ 0+m\Z, 1+m\Z, 2+m\Z, \dots, (m-1) + m\Z . \]
  Þeir kallast \emph{leifaflokkarnir}\index{leifaflokkur} $(\bmod \; m)$.
\end{daemi}

\part{Grúpur}
\chapter{Grúpur}

\section{Reikniaðgerðir}
\begin{skilgr}
  \emph{Reikniaðgerð}\index{reikniaðgerð}
  á mengi $X$ er vörpun $X\times X \to X$. Reikniaðgerðir
  eru (eins og vensl) táknaðar með sérstökum táknum eins og „ $\cdot$ “ eða
  „$+$“. Ef $\top$ er reikniaðgerð á $X$ þá skrifum við
  \[ x\top y \qquad \text{í stað} \qquad \top(x,y) . \]
  Reikniaðgerð $\top$ á $X$ kallast
  \emph{tengin}\index{reikniaðgerð!tengin}\index{tengni}
  ef
  \[ (x\top y)\top z = x\top (y\top z) \]
  fyrir öll $x,y,z \in X$. Hún kallast 
  \emph{víxlin}\index{reikniaðgerð!víxlin}\index{víxlni}
  ef
  \[ x\top y = y\top x \]
  fyrir öll $x,y \in X$. Stak $e$ í $X$ kallast
  \emph{hlutleysa}\index{hlutleysa} fyrir $\top$ ef
  \[ x\top e = e\top x = x \]
  fyrir öll $x\in X$.
\end{skilgr}
\begin{setn}
  Reikniaðgerð getur ekki haft nema eina hlutleysu.
\end{setn}
\begin{sonnun}
  Ef $e$ og $e'$ eru hlutleysur fyrir aðgerð $\top$, þá er \[ e = e\top e' = e'. \]
\end{sonnun}
\begin{skilgr}
  Látum $\top$ vera reikniaðgerð með hlutleysu $e$ á mengi $X$ og $x\in X$.
  Stak $x'$ í $X$ kallast \emph{umhverfa}\index{umhverfa} staksins $x$ m.t.t.
  $\top$ ef
  \[ x\top x' = x'\top x = e . \]
\end{skilgr}
\begin{setn}
  Látum $\top$ vera \emph{tengna} reikniaðgerð með hlutleysu $e$ á mengi $X$.
  Stak úr $X$ hefur í hæsta lagi eina umhverfu m.t.t. $\top$.
\end{setn}
\begin{sonnun}
  Látum $x'$ og $x''$ vera umhverfur staksins $x$ m.t.t. $\top$. Þá er
  \[ x' = x'\top e = x'\top (x\top x'') = (x' \top x) \top x'' = e\top x'' = x''. \]
\end{sonnun}

\section{Grúpur}
\begin{skilgr}
  \emph{Grúpa}\index{grúpa} er mengi $G$ með reikniaðgerð á $G$ þ.a. aðgerðin sé
  tengin, hafi hlutleysu og þ.a. sérhvert stak í $G$ hafi umhverfu m.t.t.
  aðgerðarinnar.

  Grúpa kallast \emph{víxlin}\index{víxlni}
  ef aðgerð hennar er víxlin, og grúpan kallast þá
  líka \emph{víxlgrúpa}\index{víxlgrúpa}\index{grúpa!víxlgrúpa} eða 
  \emph{Abelgrúpa}\index{Abelgrúpa}\index{grúpa!Abelgrúpa}.
\end{skilgr}
\begin{ath}
  [um rithátt]
  Þótt reikniaðgerðir séu ofboðslega margar, þá eru þær flestar táknaðar með
  annaðhvort „ $\cdot$ “ eða „$+$“. Aðgerð táknuð með „ $\cdot$ “ kallast
  \emph{margföldun}\index{margföldun}, aðgerð táknuð með „$+$“ kallast 
  \emph{samlagning}\index{samlagning}. Ef
  aðgerð er skrifuð sem margföldun, þá er venja að sleppa margföldunarmerkinu
  og skrifa $ab$ í stað $a\cdot b$; \emph{ef} það veldur ekki misskilning.
  Venja er, þegar talað er um almennar grúpur, að skrifa aðgerðina sem
  margföldun og láta lesanda eftir að sjá hvernig niðurstöður verða ef
  aðgerðin er táknuð öðruvísi.
\end{ath}
Umhverfa\index{umhverfa!margföldunarumhverfa} staks $x$
m.t.t. margföldunar er rituð $x^{-1}$. 
Umhverfa\index{umhverfa!samlagningarumhverfa} $x$ m.t.t.
samlagningar er rituð $-x$.
\begin{daemi}
  (1) $\Z, \Q, \R, \C$ eru grúpur m.t.t. samlagningar en ekki m.t.t. margföldunar.

  (2) $\Q \setminus \{0\}$, $\R\setminus \{0\}$, $\C\setminus\{0\}$ eru
  víxlnar grúpur m.t.t. margföldunar.

  (3) Látum $X$ vera mengi og $\mathfrak S (X)$ vera mengi allra gagntækra
  varpana $f: X\to X$. Þá er $\mathfrak S (X)$ grúpa m.t.t. samskeytingar
  varpana $\circ$, þar sem
  \[ (f\circ g) (x) = f(g(x)) \qquad \text{fyrir öll } x\in X . \]
  Hlutleysan er $\id_X : X\to X$ sem er gefin með \[\id_X (x) := x \] fyrir
  öll $x$ úr $X$; umhverfa gagntækrar vörpunar $f$ er
  \emph{andhverfa}\index{andhverfa}\index{umhverfa!andhverfa vörpunar} hennar.
%%
%% 2. september 2009
%%

  Vorum að athuga $\mathfrak S(X)$, sem var mengi allra gagntækra varpana
  $f:X\to X$, þar sem $X$ er gefið mengi; fyrir $f,g\in \mathfrak S(X)$ er
  samskeytingin $f\circ g$ þ.a.
  \[ (f\circ g)(x) := f(g(x)) \]
  fyrir öll $x\in X$, líka gagntæk vörpun, þannig að vörpunin
  \[ \mathfrak S(X)\times\mathfrak S (X) \to \mathfrak S(X), (f,g)\mapsto f\circ g \]
  er reikniaðgerð á menginu $\mathfrak S (X)$ sem gerir $\mathfrak S(X)$ að
  grúpu. Athugum sérstaklega
  \[ \mathfrak S_n := \mathfrak S (\{1,\dots,n\} . \]
  Köllum þetta
  \emph{uppstokkanagrúpu}\index{uppstokkanagrúpa}\index{grúpa!uppstokkanagrúpa}
  mengisins $\{1,\dots,n\}$; notum orðið
  líka fyrir almennari mengi $X$. Vörpunin
  $\sigma:\{1,\dots,n\}\to\{1,\dots,n\}$ má lýsa með „töflu“, nefnilega þannig
  að við skrifum
  \[ \sigma = \left (
    \begin{matrix}
      1 & 2 & 3 & \cdots & n \\
      \sigma (1) & \sigma (2) & \sigma(3) &\cdots &\sigma(n)
    \end{matrix}
  \right ). \]
  Til dæmis er
  \[ \sigma = \left(
  \begin{matrix}
    1 & 2 & 3 & 4\\
    4 & 2 & 1 & 3 
  \end{matrix}
  \right ) \]
  vörpunin $\sigma : \{1,2,3,4\} \to \{1,2,3,4\}$ þannig að $\sigma(1) =
  4,\sigma(2) = 2, \sigma(3) = 1, \sigma(4) = 3$. Í neðri línunni koma fyrir
  allar tölurnar $1,\dots,n$ en venjulega í annarri röð. Þess vegna er svona
  vörpun stundum kölluð \emph{umröðun}\index{umröðun}; en almennar notum við orðið
  \emph{uppstokkun}\index{uppstokkun}.
  Ef t.d.
  \[ \sigma = \left(
    \begin{matrix}
      1 & 2 & 3 & 4 & 5 \\
      5 & 2 & 1 & 3 & 4
    \end{matrix}
  \right ),  \]
  \[ \tau = \left (
    \begin{matrix}
      1 & 2 & 3 & 4 & 5 \\
      3 & 4 & 5 & 2 & 1
    \end{matrix}
  \right ) \]
  þá er auðvelt að reikna
  \[ \tau \circ \sigma = \left (
    \begin{matrix}
      1 & 2 & 3 & 4 & 5 \\
      1 & 4 & 3 & 5 & 2
    \end{matrix}
  \right ). \]
  Fyrir $\sigma,\tau$ úr $\mathfrak S_n$ er venja að skrifa $\tau \sigma$ í stað
  $\tau\circ \sigma$.
  
  (4) Látum $m\in\N,m\ge 1$ og setjum $Z_m := \{0,1,\dots,m-1\}$. Skilgreinum
  reikniaðgerð $\dotplus$ á $Z_m$ með
  \[ a\dotplus b := (a+b)\bmod m. \]
  Þetta gerir $Z_m$ að grúpu: Við sjáum að
  \[ ( a\dotplus b)\dotplus c = (a+b+c)\bmod m = a\dotplus (b\dotplus c), \]
  núll er hlutleysa, og fyrir $j$ úr $Z_m$ er $m-j$ umhverfa staksins $j$, því að
  \[ j \dotplus (m-j) = (j+m-j) \bmod m = m\bmod m = 0. \]
\end{daemi}

\section{Hlutgrúpur}
\begin{skilgr}
  Látum $G$ vera mengi með reikniaðgerð $\top$. Við segjum að hlutmengi $H$ í
  $G$ sé \emph{lokað}\index{lokað með tilliti til reikniaðgerðar}
  með tilliti til $\top$ ef $a\top b\in H$ fyrir öll $a,b \in H$.
  Þá gefur reikniaðgerðin af sér reikniaðgerð
  \[ H\times H \to H, (a,b) \mapsto a\top b \]
  á $H$ með einskroðun.
\end{skilgr}
\begin{skilgr}
  Látum nú $G$ vera grúpu.
  \emph{Hlutgrúpa}\index{hlutgrúpa}\index{grúpa!hlutgrúpa}
  í $G$ er hlutmengi $H$ í $G$ sem
  er lokað m.t.t. reikniaðgerðarinnar og þannig að aðgerðin á $H$, sem aðgerðin
  á $G$ gefur af sér geri, $H$ að grúpu.
\end{skilgr}
\begin{ath}
  Ef $H$ er hlutgrúpa í grúpu $G$, þá er hlutleysan í $H$ hlutleysan í $G$: Ef
  $e'$ er hlutleysan í $H$ og $e$ er hlutleysan í $G$, þá er
  \[e'\cdot e' = e' =e\cdot e' .\]
  Látum nú $e''$ vera umhverfu $e'$ í $G$; þá fæst
  \[ e' = e'e = e'(e'e'') = (e'e')e'' = e'e'' = e. \]
  Sér í lagi er $e\in H$. Þá er líka ljóst að umhverfa staks í $H$ er líka
  umhverfa þess í $G$: Ef $x\in H$, $x'$ er umhverfa $x$ í $H$ og $x''$ er
  umhverfan í $G$, þá er
  \[ x' = x'e  x'(xx'') = (x'x) x''= ex'' = x'' .\]
  Ef tengin aðgerð á mengi $G$ gefur af sér aðgerð á hlutmengi $H$, þá verður
  sú aðgerð líka tengin. En þá er ljóst:
\end{ath}
\begin{setn}
  Hlutmengi $H$ í grúpu $G$ er hlutgrúpa í $G$ þ.þ.a.a. eftirfarandi þremur
  skilyrðum sé fullnægt:
  \begin{itemize}
    \item [(i)] $e\in H$, þar sem $e$ er hlutleysan í $G$
    \item [(ii)] Ef $a,b\in H$, þá er $ab\in H$.
    \item [(iii)] Ef $a\in H$, þá er umhverfan $a^{-1}\in H$.
  \end{itemize}
\end{setn}
\begin{ath}
  Ef aðgerð í grúpu er skrifuð sem margföldun, þá er umhverfa staks $x$ táknuð
  $x^{-1}$; en ef hún er skrifuð sem samlagning, þá er umhverfan táknuð $-x$.
\end{ath}

Ef aðgerðin er skrifuð sem samlagning (sem er venjulega ekki gert nema hún sé
víxlin), þá er hlutleysan yfirleitt kölluð \emph{núll}\index{núll} eða
\emph{núllstak}\index{núllstak} og táknuð með $0$; við skrifum $a-b$ í stað
$a+(-b)$; höfum þá reikniaðgerð
\[ G\times G \to G, (a,b)\mapsto a-b \]
sem kallast \emph{frádráttur}\index{frádráttur}\index{reikniaðgerð!frádráttur};
hún er yfirleitt ekki tengin, né hefur
hlutleysu. Ef aðgerðin er skrifuð sem samlagning, þá verða skilyrðin í
síðustu setningu svona:
\begin{itemize}
  \item [(i)] $0\in H$.
  \item [(ii)] Ef $a,b\in H$, þá er $a+b\in H$.
  \item [(iii)] Ef $a\in H$, þá er $-a\in H$.
\end{itemize}
Gífurlegur fjöldi dæma um grúpur fæst með því að athuga hlutgrúpur í
$\mathfrak S (X)$ fyrir sérstök mengi $X$.
\begin{daemi} 

  (1) Látum $V$ vera línulegt rúm yfir $\R$; þá myndar mengi allra gagntækra
  $\R$\emph{-línulegra} varpana\index{grúpa!gagntækra línulegra varpana}
  $\phi : V \to V$ hlutgrúpu í $\mathfrak S(V)$.
  Við getum kallað hana \[ GL_\R (V) \] (GL stendur fyrir \emph{general linar
  group}). Nú má lýsa línulegu vörpuninni $\phi : V\to V$ á $n$-víðu rúmi $V$
  með $n\times n$ fylki $A$, og samskeyting varpana samsvarar fylkjamargföldun.
  Nú skilgreinir $n\times n$ fylki $A$ gagntæka vörpun $V\to V$ m.t.t. grunns í
  $V$ þ.þ.a.a. $\det A \neq 0$. Við sjáum að mengi allra $n\times n$-fylkja $A$
  yfir $\R$ þannig að $\det A \neq 0$ myndar grúpu m.t.t. fylkjamargföldunar;
  köllum hana \[ GL(n,\R) .\] Eins fæst \[ GL(n,\C) \] fyrir umhverfanleg
  $n\times n$-tvinntalnafylki.

  Látum $SL(n,\R)$ vera mengi allra $n\times n$-fylkja $A$ yfir $\R$ þ.a.  $\det
  A = 1$. Þá er $SL(n,\R)$ hlutgrúpa í $GL(n,\R)$: Hlutleysan i $GL(n,\R)$ er
  einingarfylkið $I$, og $\det I = 1$, svo að $I\in SL(n,\R)$, þá er $\det (AB)
  = \det A\cdot \det B = 1\cdot 1 = 1$, svo að $AB\in SL(n,\R)$, og $\det A^{-1}
  = \frac 1{\det A} = \frac 11 = 1$, svo að $A^{-1} \in SL(n,\R)$. Líka getum
  við athugað $SL(n,\Z)$, sem er mengi allra $n\times n$-fylkja með stök í $\Z$.
  Það er ljóst að $I\in SL(n,\Z)$; ef $A,b\in SL(n,\Z)$ þá er $AB\in SL(n,\R)$;
  og $A^{-1} = \frac 1{\det A} \cdot \adj A = \adj A$ hefur líka heiltölustuðla,
  svo að $A^{-1}\in SL(n,\Z)$.

  (2) Látum $V$ vera línulegt rúm. \emph{Vildarvörpun}\index{vildarvörpun}
  $\psi:V\to V$ er vörpun
  af gerðinni $\psi(x) = \varphi(x) + b$, þar sem $\varphi: V\to V$ er línuleg
  vörpun og $b$ er fasti í $V$. \emph{Gagntækar} vildarvarpanir $V\to V$ mynda
  grúpu með tilliti til samskeytinga. Látum nú $V = \R^n$.  Vörpun $\psi : \R^n
  \to \R^n $ kallast \emph{firðrækin}\index{firðrækin vörpun}
  eða \emph{flutningur}\index{flutningur} ef $\| \psi(x) -
  \psi (y) \| = \| x-y\|$ fyrir öll $x,y\in \R^n$. Sýna má (og á að gera í
  línulegri algebru) að þetta eru nákvæmlega allar vildarvarpanir $\psi (x) =
  \varphi(x) + b$, þar sem $\varphi$ er
  \emph{þverstöðluð}\index{zþverstöðluð@þverstöðluð línuleg vörpun} línuleg
  vörpun; það þýðir að hún varpi þverstöðluðum grunni í $\R^n$ á þverstaðlaðan
  grunn í $\R^n$; jafngilt er að $\langle \phi(x), y\rangle = \langle x,\phi
  (y)\rangle$ fyrir öll $x,y\in\R^n$; og þetta jafngildir því að fylkið $A$
  fyrir $\varphi$ í venjulega grunninum sé þverstaðlað, sem þýðir að $^t A\cdot
  A = I$, þar sem $^t A$ er bylta fylkið af $A$. Flutningarnir mynda grúpu með
  tilliti til samskeytingar!

  Látum nú $X\subset \R^n$; þá er \[ S(X) := \{ \varphi:\R^n \to \R^n: \varphi
  \text{ er flutningur þannig að } \varphi[X] = X \} \] augljóslega hlutgrúpa í
  flutningsgrúpunni: $\id_{\R^n} [X] = X$; ef $\varphi[X] = X$ og $\psi[X] = X$,
  þá er $(\varphi \circ \psi)[X] = \varphi[\psi[X]] = \varphi [X] = X$; og
  $\varphi^{-1}[X] = \varphi^{-1}\circ \varphi[X] = X$.  Köllum þetta
  \emph{samhverfugrúpu}\index{samhverfugrúpa} mengisins $X$.

  Flutningar í $\R^n$ eru hliðranir, snúningar um punkt, speglanir um línu og
  rennispeglanir; rennispeglun er speglun um línu ásamt hliðrun í stefnu
  línunnar. Samhverfugrúpa jafnhliða þríhyrnings hefur sex stök, nefnilega
  snúning um $0^\circ$ (það er samsemdavörpunin $\id_{\R^n}$, setjum $p_0 =
  \id_{\R^n}$), um $120^\circ$, köllum hana $\rho_1$, og um $240^\circ$, köllum
  hana $\rho_2$, tölusetjum hornpunktana með $1,2,3$, látum $\sigma_k$ vera
  speglunina um hæðina hjá hornpunkti númer $k$.  Sjáum að
  \[ \rho_0 \text{ varpar } 1,2,3 \text{ á } 1,2,3 \text{ í þessari röð}, \]
  \[ \rho_1 \text{ varpar } 1,2,3 \text{ á } 2,3,1 \text{ í þessari röð}, \]
  \[ \rho_2 \text{ varpar } 1,2,3 \text{ á } 3,1,2 \text{ í þessari röð}, \]
  \[ \sigma_1 \text{ varpar } 1,2,3 \text{ á } 1,3,2 \text{ í þessari röð}, \]
  \[ \sigma_2 \text{ varpar } 1,2,3 \text{ á } 3,2,1 \text{ í þessari röð}, \]
  \[ \sigma_3 \text{ varpar } 1,2,3 \text{ á } 2,1,3 \text{ í þessari röð}. \]
  \begin{table}[h]
    \centering
    \begin{tabular}{c|cccccc}
      $\circ$ & $\rho_0$ & $\rho_1$ & $\rho_2$ & $\sigma_1$ & $\sigma_2$ & $\sigma_3$ 
      \\\hline
      $\rho_0$ & $\rho_0$ & $\rho_1$ & $\rho_2$ & $\sigma_1$ & $\sigma_2$ & $\sigma_3$ \\
      $\rho_1$ & $\rho_1$ & $\rho_2$ & $\rho_0$ & $\sigma_3$ & & \\
      $\rho_2$ & $\rho_2$ & $\rho_0$ & $\rho_1$ & $\sigma_2$ && \\
      $\sigma_1$ & $\sigma_1$ &&&&& \\
      $\sigma_2$ & $\sigma_2$ &&&&& \\
      $\sigma_3$ & $\sigma_3$ &&&&&
    \end{tabular}
    \caption{Margföldunartafla fyrir samhverfugrúpu þríhyrningsins, \emph{ókláruð}.}
    \label{tab:gruputafla_snuningur_speglun}
  \end{table}
\end{daemi}
\begin{ath}
  Ef við höfum endanlega grúpu, þá má tölusegja stökin og búa til
  „margföldunartöflu“ yfir reikniaðgerðina. Við sjáum strax af slíkri töflu
  hvort við höfum hlutleysu og hvort aðgerðin er víxlin (það þýðir að taflan
  er samhverf um \emph{aðalhornalínuna}). Það er ekki ljóst hvernig sjá má
  af töflunni hvort aðgerðin er tengin.
  
  \emph{Grúputafla}\index{grúputafla}\index{grúpa!grúputafla} hefur þann
  eiginleika að í hverri línu og hverjum dálki kemur sérhvert stak úr grúpunni
  fyrir nákvæmlega einu sinni. Það er vegna þess að í grúpu gilda
  \emph{styttireglur}:
\end{ath}  
\begin{setn}
  [Styttireglur]\index{styttireglur}
  \begin{itemize}
    \item [(i)] Ef $ab=ac$, þá er $b = c$.
    \item [(ii)] Ef $ba = ca$, þá er $b = c$.
  \end{itemize}
\end{setn}
\begin{sonnun}
  Ef $ab = ac$ þá er $ b = eb = a^{-1} ab = a^{-1}ac = ec = c$; hitt er eins!
\end{sonnun}
Mengi með reikniaðgerð, sem er \emph{tengin}\index{tengni},
kallast \emph{hálfgrúpa}\index{hálfgrúpa}\index{grúpa!hálfgrúpa}. Ef
við höfum hálfgrúpu $H$ með hlutleysu $e$ og látum $G$ vera mengi allra
staka í $H$ sem hafa umhverfu í $H$ þá er $G$ lokað með tilliti til
aðgerðarinnar og myndar grúpu; því að
\[ (ab)^{-1} = b^{-1} a^{-1} \qquad \text{og} \qquad (a^{-1})^{-1} = a. \]
%%
%% 9. september 2009
%%
Nákvæmar:
\begin{setn}
  Látum $H$ vera \emph{hálfgrúpu} með hlutleysu $e$ og skrifum reikniaðgerðina í
  $H$ sem margföldun. Þá gildir:
  \begin{itemize}
    \item [(i)] Hlutleysan $e$ hefur umhverfu og $e^{-1} = e$.
    \item [(ii)] Ef $a$ og $b$ hafa umhverfur $a^{-1}$ og $b^{-1}$ þá hefur $ab$
      umhverfu og $(ab)^{-1} = b^{-1}a^{-1}$ .
    \item [(iii)] Ef $a$ hefur umhverfu $a^{-1}$ þá hefur $a^{-1}$ umhverfu, og
      $\left(a^{-1}\right)^{-1} = a$.
  \end{itemize}
\end{setn}
\begin{sonnun}
  (i) Er afleiðing af því að $ee=e$.

  (ii) Höfum
  \[ (ab)(b^{-1}a^{-1}) = a(bb^{-1})a^{-1} = aea^{-1} = aa^{-1} = e \]
  og
  \[ (b^{-1}a^{-1}) (ab) = b^{-1}(a^{-1}a)b = b^{-1}eb = b{-1}b = e \]
  svo að $b^{-1}a^{-1}$ er umhverfa staksins $ab$.

  (iii) Jöfnurnar $aa^{-1} = a^{-1}a = e$ segja ekki bara að $a^{-1}$ sé
  umhverfa staksins $a$ heldur líka að $a$ sé umhverfa staksins $a^{-1}$.
\end{sonnun}
\begin{fylgisetn}
  Látum $H$ vera hálfgrúpu með hlutleysu og $G$ vera mengi allra staka í $H$
  sem hafa umhverfu, þá gefur reikniaðgerðin í $H$ af sér reikniaðgerð á $G$
  með einskorðun; og aðgerðinni á $G$, sem fæst þannig, gerir $G$ að grúpu.
\end{fylgisetn}
\begin{fylgisetn}
  Ef $a_1, \dots, a_n$ eru stök í grúpu $G$, þá er
  \[ (a_1 a_2 \cdots a_n )^{-1} = a_n^{-1}a_{n-1}^{-1}\cdots a_1^{-1},  \]
  þ.e. umhverfa margfeldis endanlega margra staka í grúpu er margfeldið af
  umhverfunum í „öfugri röð“.
\end{fylgisetn}
\begin{daemi}
  (1) Rauntalnamengið er hálfgrúpa m.t.t. margföldunar og hefur hlutleysu $1$;
  grúpan af umhverfum er $\R\setminus\{ 0 \} $. Eins eru $\Q,\C$ hálfgrúpur
  m.t.t. margföldunar og grúpurnar með umhverfunum eru $\Q\setminus\{0\},
  \C\setminus\{0\}$. Heilu tölurnar, $\Z$, mynda hálfgrúpu með hlutleysu $1$
  með tilliti til margföldunar og grúpan af hlutleysunum er $\{-1,1\}$.

  (2) Látum $X$ vera mengi og $H$ vera mengi allra varpana $f:X\to X$. Þá
  myndar $H$ hálfgrúpu m.t.t. samskeytingar varpana; hlutleysan er
  samsemdavörpunin $\id_X: X\to X; \id_X(x):=x$ fyrir öll $x\in X$ og grúpan
  af umhverfunum er grúpan $\mathfrak S (X)$ af \emph{gagntækum} vörpunum
  $X\to X$.

  (3) Látum $\R^{n\times n}$ vera mengi allra $n\times n$-fylkja yfir $\R$.
  Það er hálfgrúpa m.t.t. margföldunar fylkja, hlutleysan er einingarfylkið
  $I$. Grúpan af umhverfunum er $GL(n,\R) := \{ A\in \R^{n\times n}: \det (A)
  \neq 0 \}$. Látum $\Z^{n\times n}$ vera mengi allra $n\times n$-fylkja með
  heiltölustökum; það er hálfgrúpa m.t.t. fylkjamargföldunar (ef við
  margföldum saman tvö fylki með heilum stuðlum fæst aftur fylki með heilum
  stuðlum), hlutleysan er einingarfylkið $I$. Grúpan af umhverfunum er
  \[ \{ A\in \Z^{n\times n} : \det (A) = 1 \text{ eða } \det(A) = -1 \}. \]
  \emph{Sönnum þetta: } Ef $A\in \Z^{n\times n}$ og $\det (A) \neq 0$, þá er
  \[ A^{-1} := \frac 1 {\det(A)} \cdot \adj (A) \]
  umhverfan í $\R^{n\times n}$, þar sem $\adj(A)$ er bylta hjáþáttafylkið.
  Höfum ljóslega $\adj(A)\in \Z^{n\times n}$, ef $\det(A) = 1$ eða $\det(A) =
  -1$ þá er $\adj(A)\in \Z^{n\times n}$ og er umhverfa $A$ í $\Z^{n\times n}$.
  Öfugt, ef $A$ hefur umhverfu $B\in \Z^{n\times n}$, þá er $AB = I$ , svo að
  $\det(A)\det(B) = \det I = 1$ og $\det(A)\det(B) \in \Z$; en þá er $\det(A)
  = 1$ eða $\det(A) = -1$.
\end{daemi}
Höfum minnst á \emph{styttiregluna}. Getum sett hana fram aðeins almennar:
\begin{setn}
  [Almennari styttiregla]\index{styttiregla!almennari}
  Látum $H$ vera hálfgrúpu með hlutleysu $e$ og $a$ vera stak í $H$ sem hefur
  umhverfu $a^{-1}$.
  \begin{itemize}
    \item [(i)] Ef $x,y\in H$ og $ax = ay$, þá er $x=y$.
    \item [(ii)] Ef $x,y\in H$ og $xa = ya$, þá er $x=y$.
  \end{itemize}
\end{setn}
\begin{ath}
  Við orðum þetta þannig: Umhverfanlegt stak er styttanlegt bæði frá hægri og vinstri.
\end{ath}
\begin{sonnun}
 (i) Ef $ax = ay$, þá er
 \[ x = ex = (a^{-1}a) x = a^{-1}(ax) = a^{-1}ay = (a^{-1}a)y = ey = y. \]

 (ii) Ef $xy = ya$, þá er
 \[ x = xe = x(aa^{-1} = (xa)a^{-1} = (ya)a^{-1} = y(aa^{-1}) = ye = y. \]
\end{sonnun}
\begin{ath}
  Ef $xa = ay$ er ekki hægt að álykta að $x = y$, þótt að $a$ sé
  umhverfanlegt, \emph{nema} að reikniaðgerðin sé víxlin.
\end{ath}
Höfum því sýnt:
\begin{setn}
  Hlutmengi $H$ í grúpu $G$ er hlutgrúpa í $G$ þ.þ.a.a. eftirfarandi skilyrðum
  sé fullnægt:
  \begin{itemize}
    \item [(i)] $e\in H$, þar sem $e$ er hlutleysan í $G$.
    \item [(ii)] Ef $a,b \in H$, þá er $ab\in H$.
    \item [(iii)] Ef $a\in H$, þá er $a^{-1}\in H$
  \end{itemize}
\end{setn}
\begin{setn}
  Skilyrðin (i), (ii) og (iii) í síðustu setningu eru jafngild eftirfarandi
  skilyrðum:
  \begin{itemize}
    \item [(i')] $H\neq \emptyset$.
    \item [(ii')] Ef $a,b\in H$, þá er $ab^{-1} \in H$.
  \end{itemize}
\end{setn}
\begin{sonnun}
  Gefum okkur að (i)-(iii) gildi. Vegna $e\in H$ skv. (i) er $H\neq
  \emptyset$, svo að (i') gildir. Látum $a,b\in H$. Skv. (iii) er $b^{-1}\in
  H$ , þá höfum við $a,b^{-1}\in H$, svo að $ab^{-1}\in H$ skv. (iii) og þar
  með gildir (ii').

  Gefum okkur nú að (i') og (ii') gildi. Þá er $H\neq \emptyset$ svo til er
  $c\in H$. Skv. (ii') er þá $e = cc^{-1}\in H$, svo að (i) gildir. Látum nú
  $a\in H$. Við vitum að $e\in H$ svo að $e,a\in H$ og þá $a^{-1} = ea^{-1}
  \in H$ skv. (ii'). Látum þá $a,b\in H$, vitum að $b^{-1}\in H$, af
  $a,b^{-1}\in H$ leiðir að $ab = a(b^{-1})^{-1} \in H$ skv. (ii') svo að
  (iii) gildir líka.
\end{sonnun}
\begin{ath}
  Ef aðgerðin í $G$ er skrifuð sem samlagning þá líta skilyrðin svona út:
  \begin{itemize}
    \item [(i)] $0\in H$.
    \item [(ii)] Ef $a,b\in H$, þá er $a+b\in H$.
    \item [(iii)] Ef $a\in H$ þá er $-a\in H$.
  \end{itemize}
  og hin skilyrðin verða
  \begin{itemize}
    \item [(i')] $H\neq \emptyset$.
    \item [(ii')] Ef $a,b\in H$, þá er $a-b\in H$.
  \end{itemize}
\end{ath}

\section{Rásaðar grúpur}

Rifjum upp að alveg í upphafi sönnuðum við: Ef $A$ er mengi af heilum tölum,
$A\neq \emptyset$ og fyrir öll $a,b\in A$ er $a-b\in A$, þá er til nákvæmlega
ein náttúrleg tala $m$ þ.a. $A=m\Z$. Þetta má nú orða svo:
\begin{setn}[Mikilvæg setning]
  Hlutgrúpurnar í samlagningargrúpunni $\Z$ eru nákvæmlega mengin $m\Z$, þar
  sem $m\in \N$ (ath. að $0\in \N$).
\end{setn}
\begin{daemi}
  Látum \[ Z_m := \{ 0,1,\dots, m-1 \} \]
  þar sem $m\in \N, m\ge 2$. Við skilgreindum samlagningu á $Z_m$ og gerðum
  það að samlagningargrúpu. Við getum líka skilgreint margföldun á $Z_m$ með
  \[ a\odot b := ab\bmod m . \]
  Þessi reikniaðgerð gerir $Z_m$ að víxlinni hálfgrúpu með hlutleysu $1$,
  höfum
  \[ (a\odot b)\odot c = a\odot (b\odot c) = (a\cdot b \cdot c)\mod m. \]
  Grúpan af umhverfanlegu stökunum er táknuð með
  \[\mathcal U (m) . \]
  Hvaða stök hafa umhverfu? Við höfum
  \[ a\in \mathcal U(m) \quad \text{ þ.þ.a.a. } \quad \ssd(a,m) = 1 . \]
  Hvað þýðir það að $a$ hafi umhverfu? Það þýðir að til sé $x$ þ.a. $ax \equiv
  1\; (\bmod\; m)$ og það þýðir aftur að til eru $x$ og $y$ þ.a. $1-ax = my$,
  þ.e. $1 = ax + my$, sem þýðir að $\ssd(a,m) = 1$.
\end{daemi}
\begin{skilgr}
  Látum $G$ vera grúpu með hlutleysu $e$ og skrifum aðgerðina sem margföldun.
  Fyrir $a\in G$ og $n\in \N$ skilgreinum við \emph{veldið}\index{veldi}
  $a^n$ með þrepun þannig að
  \begin{align*}
    a^0 &{:=} e,\\
    a^{n+1} &{:=} a^n a.
  \end{align*}
  Setjum líka
  \[ a^{-n} := (a^{-1})^n \]
  og höfum þá skilgreint $a^n$ fyrir öll $n\in \Z$. Ef aðgerðin í $G$ er
  skrifuð sem samlagning, þá skrifum við $na$ í stað $a^n$; skilgreiningin
  verður þá
  \begin{align*}
    0_\N a &{:=} 0_G \\
    (n+1) a &{:=} na + a \\
    (-n) a &{:=} n(-a).
  \end{align*}
\end{skilgr}
%%
%% 10. september 2009
%%
Látum $G$ vera grúpu og $a\in G$. Þá gildir 
  \begin{align*}
    a^{n+m} &= a^n\cdot a^m \\
    a^{nm} &= \left( a^n \right)^m
  \end{align*}
\begin{ath} [Viðvörun]
  Almennt er $\left( ab \right)^n$ \emph{ekki} jafnt $a^nb^n$. Ef það gildir
  fyrir $n=2$, þ.e. $a^2 b^2 = \left( a^b \right)^2 = abab$, þá getum við stytt
  $a$ frá vinstri og $b$ frá hægri og fáum \[
  ab = ba.
  \]
  Ef hins vegar $ab = ba$, þá er auðvelt að sjá með þrepun að $(ab)^n = a^nb^n$.
  Sér í lagi gildir þetta fyrir öll stök í víxlgrúpum.
\end{ath}
\begin{skilgr}
  Látum $a$ vera stak í grúpu $G$. Setjum \[
  \langle a \rangle := \left\{ a^n : n\in \Z \right\}.
  \]
\end{skilgr}
\begin{setn}
  Mengið $\langle a \rangle$ er hlutgrúpa í $G$, og raunar minnsta hlutgrúpa í
  $G$ sem inniheldur $a$.
\end{setn}
\begin{sonnun}
  Það er ljóst að $a\in \langle a \rangle$, svo að $\langle a\rangle \neq
  \emptyset$. Ef $x,y\in \langle a\rangle$ þá má skrifa $x = a^n, y=a^m$ þar sem
  $n,m\in \Z$ og þá er $xy^{-1} = a^{n-m}\in \langle a \rangle$. Þá er
  $\langle a \rangle$ hlutgrúpa. Ef $H$ er hlutgrúpa í $G$ þ.a. $a\in H$, þá er
  ljóst með þrepun að $a^n\in H$ fyrir öll $n\in\N$ og $a^{-n}=\left(
  a^{-1}
  \right)^{n}\in H$ fyrir öll $n\in\N$, svo að $\langle a \rangle \subset H$.
\end{sonnun}
\begin{skilgr}
  Segjum að $\langle a \rangle$ sé hlutgrúpan í $G$ sem stakið
  $a$ \emph{spannar}\index{spann!staks í grúpu}\index{grúpa!spann staks}.
  Látum almennar $A$ vera hlutmengi í grúpu $G$. Þá er til minnsta hlutgrúpa í
  $G$ sem inniheldur $A$ sem hlutmengi; við getum skilgreint hana sem sniðmengið
  af öllum hlutgrúpum sem innihalda $A$ sem hlutmengi. Við táknum þessa grúpu
  með\[
  \langle A\rangle
  \]
  og segjum að $\langle A\rangle$ sé \emph{hlutgrúpan í $G$ sem $A$
  spannar}\index{spann!hlutmengis í grúpu}\index{grúpa!spann hlutmengis}.
\end{skilgr}
\begin{setn}
  Grúpan $\langle A \rangle$ er mengi allra margfelda $a_1 a_2 \cdots a_r$, þar
  sem $a_1,\dots,a_r \in A\cup A^{-1}$, þar sem $A^{-1} := \left\{ a^{-1}:a\in A
  \right\}$.
\end{setn}
\begin{skilgr}
  Ef $A$ er endanlegt, $A=\left\{ a_1,\dots,a_n \right\}$, þá skrifum við
  $\langle A \rangle = \langle a_1,\dots,a_n \rangle. $
\end{skilgr}
\begin{ath}
  Höfum $\langle a \rangle = \langle \left\{ a \right\} \rangle$ fyrir
  $a\in G$.
\end{ath}
\begin{skilgr}
  (1) Við segjum að grúpa $G$ sé \emph{rásuð}\index{rásuð
  grúpa}\index{grúpa!rásuð} ef til er stak $a\in G$ þ.a.
  $G= \langle a \rangle$.

  (2) Við segjum að grúpa $G$ sé \emph{endanlega spönnuð}\index{endanlega
  spönnuð grúpa}\index{grúpa!endanlega spönnuð} ef til eru endanlega
  mörg $a_1,\dots,a_n$ þ.a. $G=\langle a_1,\dots,a_n \rangle$.
\end{skilgr}
\begin{ath}
  Látum $G$ vera grúpu með hlutleysu $e$. Þá eru $\left\{ e \right\}$ og
  $G$ hlutgrúpur í $G$; $\left\{ e \right\}$ er minnsta hlutgrúpan en $G$ sú
  stærsta. Hlutgrúpan í $G$ er \emph{eiginleg}\index{eiginleg
  hlutgrúpa}\index{hlutgrúpa!eiginleg} ef hún er ekki $G$ sjálft, þ.e.
  hún er eiginlegt hlutmengi í $G$. Á ensku er grúpa kölluð \emph{trivial} ef
  hún hefur bara eitt stak; slík grúpa kallast
  \emph{örgrúpa}\index{zorgrupa@\"orgrúpa}\index{grúpa!zorgrupa@\"orgrúpa}.

  Á ensku er fjöldatala grúpu kölluð \emph{order} grúpunnar; við notum bara
  \emph{fjöldatala}\index{fjöldatala}. En það sem á ensku er kallað 
  \emph{order of an element in a group} köllum við
  \emph{raðstig}\index{raðstig}.
\end{ath}
\begin{skilgr}
  \emph{Raðstig}\index{raðstig} staks $a$ í grúpu $G$ er fjöldatala mengisins
  $\langle a\rangle$.
\end{skilgr}

\begin{setn}
  Látum $a$ vera stak í grúpu $G$. Stakið $a$ hefur endanlegt raðstig þá og
  því aðeins að til sé náttúrleg tala $n$ þannig að $n\ge 1$ og $a^n = e$,
  þar sem $e$ er hlutleysan, og raðstigið er þá minnsta slíka talan $n$.
\end{setn}

\begin{sonnun}
  Athugum fyrst: Ef vörpunin $\varphi:\Z\to G, \varphi(n):= a^n$ er eintæk,
  þá er raðstigið óendanlegt. Ef hins vegar til eru heilar tölur $n,m$ þ.a.
  $n\neq m$ og $a^{n} = a^m$, þá má g.r.f. að $n < m$, fyrir $k:= m-n$ er þá
  $a^k = a^m \cdot a^{-n} = e$ og $k\ge 1$.
  Gerum nú á hinn bókinn gráð fyrir að til sé tala $k\ge 1$ þannig að $a^k =
  e$ og látum $k$ vera \emph{minnstu} slíka tölu. Ef nú $n\in \Z$, þá getum
  við skrifað\[
  n = kq + r
  \]
  þar sem $q,r\in \Z$ og $0\le r < k$. Þá er\[
  a^n = a^{kq+ r} = a^{kq} a^r = \left( a^k \right)^q a^r = e^q a^r = a^r.
  \]
  Því er\[
  \langle a \rangle = \left\{ a^0 = e, a^1 = a,\dots, a^{k-1} \right\}
  \]
  sem hefur nákvæmlega $r$ stök, því að stökin $a^0, a^1, \dots, a^{k-1}$ eru
  öll ólík; annars væru til ólíkar tölur $\left( i,j \right)\in \left\{
  0,\dots,k-1
  \right\}$ þ.a. $i\le j$ og $a^i = a^j$, en þá er $1\le j-i \le k-1$ og
  $a^{j-i} = a^j\left( a^j \right)^{-1} = e$ í mótsögn við skilgreiningu á
  $k$.
\end{sonnun}

\begin{ath}
  Sönnunin sýnir: Ef raðstig staksins $a$ er endanleg tala $k$, þá er\[
  \langle a \rangle = \left\{ a^0, a^1, \dots, a^{k-1} \right\}.
  \]
\end{ath}

Athugum betur vörpunina $\varphi:\Z\to G, \varphi(n) := a^n$. Hún fullnægir
skilyrðinu
\begin{equation}
  \varphi(n+m) = \varphi(n)\varphi(m)
  \label{eq:phi_skilyrdi}
\end{equation}
fyrir öll $n,m$; þetta er bara reglan $a^{n+m} = a^n a^m$. Af
\eqref{eq:phi_skilyrdi} leiðir $\varphi(n)=\varphi(n+0) =
\varphi(n)\varphi(0)$, svo að $\varphi(0) = e$; ef
$\varphi(n)=\varphi(m) = e$, þá er\[
\varphi(n-m) = \varphi(n)\varphi(-m) = \varphi(n)\varphi(m)^{-1} = e\cdot
e^{-1} = e.
\]
 Þetta þýðir að 
\[
H:= \left\{ n\in \Z: \varphi(n)=e \right\}
\]
er hlutgrúpa í samlagningargrúpunni $\Z$. Hún er af gerðinni $k\Z$, þar sem
$k\in \N$. Ef $k=0$, þ.e. ekkert veldi $a^n$ fyrir $n\neq 0$ er $e$, þá sýnir
röksemdafærslan í síðustu setningu að vörpunin $\varphi$ er eintæk, og þá er
raðstig $a$ óendanlegt. Ef $k\ge 1$, þá er $k$ minnsta tala þ.a. $k\ge 1$ og
$a^k = e$, svo að $k$ er raðstigið.

Vörpunin $\varphi$ er dæmi um \emph{grúpumótun}.

\chapter{Grúpumótanir}
\section{Eiginleikar grúpumótana}
\begin{skilgr}
  Látum $G_1,G_2$ vera grúpur.
  \emph{Grúpumótun}\index{grúpumótun}\index{grúpa!grúpumótun}
  $\varphi:G_1\to G_2$ er vörpun þannig að
  \begin{equation}
    \label{eq:grupumotun}
    \varphi(xy) = \varphi(x)\varphi(y) 
  \end{equation}
  fyrir öll $x,y\in G_1$.
\end{skilgr}
\begin{ath}
  Ef aðgerðin í $G_1$ er skrifuð sem samlagning, þá verður jafnan
  \eqref{eq:grupumotun} að 
 \[
 \varphi(x+y) = \varphi(x)\cdot\varphi(y).
 \]
 Eins, ef aðgerðin í $G_1$ er skrifuð sem $\cdot$ og sú í $G_2$ er skrifuð sem
 $+$ þá verður hún að\[
 \varphi(x\cdot y) = \varphi(x)+\varphi(y)
 \]
 o.s.frv.
\end{ath}
\begin{daemi}
 (1) Vigurrúm yfir $\R$ eða $\C$ er grúpa m.t.t. samlagningar, og línuleg
 vörpun er grúpumótun.

 (2) Látum $\R^*_+ := \left\{ x\in\R: x>0 \right\}$; þetta er hlutgrúpa í
 $\R^*:=\R\setminus \left\{ 0 \right\}$ m.t.t. margföldunar. Varpanirnar
 \begin{equation*}
   \exp:\R\to\R^*_+, \quad \exp x := \sum_{n\ge 0}\frac{x^n}{n!}
 \end{equation*}
 \begin{equation*}
   \log: \R^*_+ \to \R
 \end{equation*}
 eru grúpumótanir. Hér lítum við á $\R$ sem grúpu m.t.t. samlagningar en
 $\R^*_+$ m.t.t. margföldunar.
\end{daemi}
%%
%% 16. september 2009
%%
\begin{setn}
  \begin{enumerate}[(1)]
    \item Ef $G$ er grúpa, þá er $\id_G:G\to G, x\mapsto x$ grúpumótun.
    \item Ef $\varphi:G\to H$ og $\psi: H\to K$ eru grúpumótanir, þá er
      samskeytingin $\psi\circ\varphi:G\to K$ grúpumótun.
    \item Ef $\varphi: G\to H$ er gagntæk grúpumótun, þá er andhverfan
      $\varphi^{-1}: H\to G$ líka grúpumótun.
  \end{enumerate}
\end{setn}
\begin{sonnun}
  (1) Augljóst.
  
  (2) Fyrir $a,b\in G$ er 
  \[
    \psi\circ\varphi(ab)
    = \psi(\varphi(ab))
    = \psi(\varphi(a)\varphi(b))
    = \psi(\varphi(a))\psi(\varphi(b))
    = (\psi\circ\varphi(a))(\psi\circ\varphi(b)).
  \]
  
  (3) Ef $c,d\in H$, þá eru $\varphi^{-1}(c), \varphi^{-1}(d)\in G$ og 
  \[
    \varphi(\varphi^{-1}(c)\varphi^{-1}(d))
    = \varphi(\varphi^{-1}(c))\varphi(\varphi^{-1}(d))
    = cd
    = \varphi(\varphi^{-1}(cd))
  \]
  og þar eð $\varphi$ er eintæk fæst
  $\varphi^{-1}(cd)=\varphi^{-1}(c)\varphi^{-1}(d)$.
\end{sonnun}
\begin{setn}
  Látum $\varphi:G_1\to G_2$ vera grúpumótun og $e_k$ vera hlutleysuna í $G_k$
  fyrir $k=1,2$. Þá gildir:
  \begin{itemize}
    \item[(1)] $\varphi(e_1) = e_2$
    \item[(2)] $\varphi(a^{-1}) = \varphi(a)^{-1}$ fyrir öll $a\in G$.
    \item[(3)] $\varphi(a^n)= \varphi(a)^n$ fyrir öll $a\in G$ og $n\in \Z$.
  \end{itemize}
\end{setn}
\begin{sonnun}
  
  (1) Höfum 
  \[
  \varphi(e_1)\varphi(e_1) = \varphi(e_1 e_1) = \varphi(e_1) = e_2\varphi(e_1)
  \]
  styttum $\varphi(e_1)$ frá hægri og fáum $\varphi(e_1) = e_2$.
  
  (2) Höfum $aa^{-1}=a^{-1}a = e_1$. Af (1) leiðir að
  \[
    \varphi(a) \varphi(a^{-1}) = \varphi(a^{-1})\varphi(a) = e_2
  \]
  og það þýðir að $\varphi(a^{-1})$ er umhverfa staksins $\varphi(a)$.
  
  (3) Höfum 
  \[
  \varphi(a^0) = \varphi(e_1) = e_2 = \varphi(a)^0,
  \]
  \[
  \varphi(a^{k+1}) 
  = \varphi(a\cdot a^k) 
  = \varphi(a)\cdot\varphi(a^k)
  \]
  svo að þrepun gefur $\varphi(a^n)=\varphi(a)^n$ fyrir öll $n\in \N$. En skv.
  (2) er þá líka $\varphi(a^{-n}) = \varphi((a^{-1})^n)=\varphi(a^{-1})^n =
  (\varphi(a)^{-1})^n = \varphi(a)^{-n}$ fyrir öll $n\in \N$, svo að
  $\varphi(a^n) = \varphi(a)^n$ fyrir öll $n\in\Z$.
\end{sonnun}

\section{Tenging við heiltölurnar}
\begin{fylgisetn}
  Látum $G$ vera grúpu og $a\in G$. Þá er til nákvæmlega ein grúpumótun
  $\varphi:\Z\to G$ þannig að $\varphi(1) = a$; hún er gefin með
  $\varphi(n)=a^n$ fyrir öll $n$.
\end{fylgisetn}
\begin{ath}
  $\Z$ er hér \emph{samlagningargrúpa} heilu talnanna, þ.a. grúpumótun $\varphi
  :\Z \to G$ er vörpun þ.a. $\varphi(n+m)=\varphi(n)\varphi(m)$ fyrir öll
  $n,m\in \Z$.
\end{ath}
\begin{sonnun}
  [á fylgisetningu]
  Skv. veldareglu er $a^{n+m} = a^n a^m$, svo a vörpun $\varphi_a:\Z\to G,
  \varphi_a(n) := a^n$ er grúpumótun þ.a. $\varphi(1) = a$, þá segir liður (3) í
  síðustu setningu að $\varphi(n) = \varphi(n\cdot 1) = \varphi(1)^n=a^n$ fyrir
  öll $n$.
\end{sonnun}
\begin{fylgisetn}
  Grúpa $G$ er rásuð þ.þ.a.a. til sé átæk grúpumótun $\varphi:\Z\to G$.
\end{fylgisetn}
\begin{sonnun}
  Slík grúpumótun er af gerðinni $\varphi_a:\Z\to G, n\mapsto a^n$; og myndmengi
  hennar er $\{a^n:n\in \Z \}$; höfum $\langle a \rangle = G$ þ.þ.a.a.
  $\varphi_a$ sé átæk.
\end{sonnun}

\begin{daemi}
  Vörpunin 
  \[
    \Z\to Z_m, n\mapsto n\bmod m
  \]
  er átæk grúpumótun, svo að $Z_m$ er rásuð grúpa. \emph{Ath:} Samlagning í
  $Z_m$ var skilgreind með $k\bmod m + j\bmod m = (j+k)\bmod m$.
\end{daemi}

\section{Kjarni, mynd og einsmótanir}
\begin{setn}
  Látum $\varphi: G_1\to G_2$ vera grúpumótun.
  \begin{enumerate}[(1)]
    \item Ef $H$ er hlutgrúpa í $G_1$, þá er myndin $\varphi[H] =
      \{\varphi(h):h\in H\}$ hlutgrúpa í $G_2$.
    \item Ef $K$ er hlutgrúpa í $G_2$, þá er frummyndin $\varphi^{-1}[K] =
      \{x\in G_1: \varphi(x)\in K\}$ hlutgrúpa í $G_1$.
  \end{enumerate}
\end{setn}
\begin{sonnun}
  (1) Látum $e_k$ vera hlutleysuna í $G_k, k=1,2$. Höfum $e_2 = \varphi(e_1)\in
  \varphi[H]$, því að $e_1\in H$, svo að $\varphi[H]\neq \emptyset$. Ef $c,d\in
  \varphi[H]$, þá eru til $a,b\in H$ þ.a. $\varphi(a) = c$ og $\varphi(b) = d$.
  Þá er 
  \[
    cd^{-1} 
    = \varphi(a)\varphi(b)^{-1} 
    = \varphi(a)\varphi(b^{-1}) 
    = \varphi(ab^{-1}) \in \varphi[H],
  \]
  vegna $ab^{-1}\in H$. Því er $\varphi[H]$ hlutgrúpa í $G_2$.
  
  (2) Vegna $\varphi(e_1)=e_2\in K$ er $e_1\in \varphi^{-1}[K]$, svo að
  $\varphi^{-1}[K]\neq \emptyset$. Ef $a,b\in \varphi^{-1}[K]$, þá er
  $\varphi(a)\in K$ og $\varphi(b)\in K$, svo að 
  \[
    \varphi(ab^{-1})=\varphi(a)\varphi(b)^{-1}\in K
  \]
  og því $ab^{-1}\in \varphi^{-1}[K].$
\end{sonnun}

\begin{skilgr}
  Látum $\varphi:G_1\to G_2$ vera grúpumótun. Við köllum
  \[ \im\varphi:=\varphi[G_1]=\{\varphi(g):g\in G_1\} \]
  \emph{mynd}\index{mynd, myndgrúpa} eða
  \emph{myndgrúpu}\index{grúpumótun!mynd, myndgrúpa}
  mótunarinnar $\varphi$, og 
  \[
    \Ker \varphi 
    := \varphi^{-1}[\{e_2\}]
    = \{ x\in G_1 : \varphi(x) = e_2 \}
  \]
  þar sem $e_2$ er hlutleysan í $G_2$, köllum við 
  \emph{kjarna}\index{kjarni}\index{grúpumótun!kjarni}
  grúpumótunarinnar $\varphi$.
\end{skilgr}

\begin{fylgisetn}
  Ef $\varphi:G_1\to G_2$ er grúpumótun, þá er $\im \varphi$ hlutgrúpa í $G_2$
  og $\Ker\varphi$ er hlutgrúpa í $G_1$.
\end{fylgisetn}
\begin{setn}
  Látum $\varphi:G_1\to G_2$ vera grúpumótun.
  \begin{itemize}
    \item [(1)] Vörpunin $\varphi$ er átæk þ.þ.a.a. $\im \varphi = G_2$.
    \item [(2)] Vörpunin $\varphi$ er eintæk þ.þ.a.a. $\Ker \varphi = \{e_1\}$,
      þar sem $e_1$ er hlutleysan í $G_1$.
  \end{itemize}
\end{setn}
\begin{sonnun}
  (1) er nánast skilgreiningin á átækni.
  
  (2) Ef $\varphi$ er eintæk og $x\in \Ker\varphi$, þá er $\varphi(x)=e_2 =
  \varphi(e_1)$, svo að $x=e_1$ og því er $x\in \{e_1\}$. Ljóst er að $e_1\in
  \Ker \varphi$, og því er $\Ker\varphi = \{e_1\}$.
  
  Ef nú á hinn bóginn $\Ker\varphi = \{e_1\}$ og $a,b\in G$ eru stök þ.a.
  $\varphi(a) = \varphi(b)$, þá er $\varphi(ab^{-1}) = \varphi(a)\varphi(b^{-1})
  = \varphi(a)\varphi(b)^{-1} = e_2$ svo að $ab^{-1} \in \Ker\varphi = \{e_1\}$,
  þ.e. $ab^{-1} = e_1$ og þá $a=b$. Því er $\varphi$ eintæk.
\end{sonnun}
\begin{skilgr}
  \emph{Gagntæk} grúpumótun $\varphi:G_1\to G_2$ kallast
  \emph{grúpueinsmótun}\index{grúpueinsmótun}\index{grúpumótun!grúpueinsmótun}
  og við segjum að tvær grúpur $G_1$ og $G_2$ séu
  \emph{einsmóta}\index{einsmóta}\index{einsmóta!grúpur} og skrifum 
  \[
    G_1 \approx G_2 \quad \text{eða}\quad G_1 \cong G_2
  \]
  ef til er einsmótun $\varphi:G_1\to G_2$.
\end{skilgr}
\begin{setn}
  Látum $G_1, G_2, G_3$ vera grúpur.
  \begin{itemize}
    \item [(1)] $G_1\cong G_2$.
    \item [(2)] Ef $G_1\cong G_2$ þá er $G_2\cong G_1$.
    \item [(3)] Ef $G_1\cong G_2$ og $G_2\cong G_3$, þá er $G_1\cong G_3$.
  \end{itemize}
\end{setn}
\begin{sonnun}
  (1) $\id_{G_1}:G_1\to G_1$ er grúpueinsmótun.
  
  (2) Ef $\varphi:G_1\to G_2$ er grúpueinsmótun þá er $\varphi^{-1}:G_2\to G_1$ grúpueinsmótun.
  
  (3) Ef $\varphi:G_1\to G_2$ og $\psi: G_2\to G_3$ eru grúpueinsmótanir, þá er $\varphi\circ\psi:G_1\to G_3$ grúpueinsmótun.
\end{sonnun}
\begin{ath}
  Gerum ráð fyrir að $\varphi: G_1\to G_2$ sé \emph{eintæk} grúpumótun. Þá er
  $\im \varphi$ hlutgrúpa í $G_2$ og vörpunin $\overline \varphi:G_1\to \im \varphi,
  x\mapsto \varphi(x)$ er \emph{gagntæk} og því grúpueinsmótun, $G_1$ er því
  einsmóta hlutgrúpu í $G_2$.
\end{ath}
\begin{daemi}
  Látum $D_n$ vera samhverfugrúpu reglulegs $n$-hyrnings
  (\emph{tvíflötungsgrúpa}\index{grúpa!tvíflötungsgrúpa}\index{tvíflötungsgrúpa})
  og tölusetjum hornpunkta marghyrningsins með
  $v_1,\dots,v_n$. Flutningur $\varphi$ sem tekur marghyrninginn á sjálfan sig
  ákvarðast af því hvernig hornpunktarnir varpast. Finnum gagntæka vörpun
  $\sigma_\varphi:\{1,\dots,n\}\to \{1,\dots,n\}$ þannig að 
  \[
    \sigma_\varphi(k) = j \quad \text{þ.þ.a.a.} \quad \varphi(v_k) = v_j.
  \]
  Nú er ljóst að 
  \[
    \sigma_\varphi \circ \sigma_\psi = \sigma_{\varphi\circ \psi}
  \]
  þetta þýðir að vörpunin
  \[
    \Sigma: D_n \to \mathfrak S_n, \varphi\mapsto \sigma_\varphi
  \]
  er \emph{eintæk grúpumótun}, svo að $D_n$ verður einsmóta hlutgrúpunni 
  \[
    \im \Sigma = \{\sigma_\varphi: \phi\in D_n\}
  \]
  í $\mathfrak S_n$.
\end{daemi}

%%
%% 17. september 2009
%%

\paragraph{Smá mengjafræði.} % (fold)
Látum $\varphi:X\to Y$ vera vörpun milli mengja. Ef $A\subset X$ þá er 
\begin{equation}
  A\subset \varphi^{-1}[\varphi[A]]; \label{eq:formynd_mynd}
\end{equation}
ef nefnilega $a\in A$, þá er $\varphi(a)\in \varphi[A]$, og það þýðir að $a\in
\varphi^{-1}[\varphi[A]]$. Jafnaðarmerki þarf ekki að gilda í
\eqref{eq:formynd_mynd}; en gildir þó ef $\varphi$ er \emph{eintæk} vörpun.  Ef
$B\subset Y$, þá er 
\begin{equation}
  \varphi[\varphi^{-1}[B]] \subset B; \label{eq:mynd_formynd}
\end{equation}
ef nefnilega $x\in \varphi[\varphi^{-1}[B]]$, þá er til $y\in \varphi^{-1}[B]$
þ.a. $x=\varphi(y)$; en $y\in \varphi^{-1}[B]$ þýðir að $\varphi(y)\in B$, svo
að $x\in B$. Jafnaðarmerki þarf ekki að gilda í \eqref{eq:mynd_formynd}; en
gildir þó ef $\varphi$ er \emph{átæk}.  Af \eqref{eq:formynd_mynd} leiðir að
$\varphi[A]\subset\varphi[\varphi^{-1}[\varphi[A]]]$. Notum svo
\eqref{eq:mynd_formynd} á $B:=\varphi[A]$ og fáum
\[
  \varphi[\varphi^{-1}[\varphi[A]]]\subset\varphi[A].
\]
Við fáum því \[
  \varphi[\varphi^{-1}[\varphi[A]]] = \varphi[A]
\]
fyrir öll hlutmengi $A$ í $X$
\\
\emph{Afleiðing:} Látum $\varphi: X\to Y$ vera vörpun og $A_1,A_2$ vera
hlutmengi í $X$. Við höfum $\varphi[A_1]=\varphi[A_2]$ þ.þ.a.a.
$\varphi^{-1}[\varphi[A_1]] = \varphi^{-1}[\varphi[A_2]]$.
% subsubchapter sma_mengjafraedi (end)

\begin{setn}
  Látum $\varphi:G_1\to G_2$ vera grúpumótun með kjarna $K:= \Ker \varphi =
  \{x\in G_1:\varphi(x) = e_2 \}$, $e_2$ hlutleysan í $G_2$. Ef $H$ er hlutgrúpa
  í $G_1$, þá er
  \[
    \varphi^{-1}[\varphi[H]] = HK = KH.
  \]
\end{setn}
\begin{ath}[Upprifjun]
  Fyrir hlutmengi $A,B$ í grúpu $G$ er \[
    AB := \{ ab: a\in A, b\in B\}.
  \]
  Ef aðgerðin í $G$ er skrifuð sem samlagning, þá skrifum við \[
    A+B := \{ a+b: a\in A, b\in B \}.
  \]
\end{ath}
\begin{sonnun}[Sönnun á síðustu setningu]
  Látum $x\in \varphi^{-1}[\varphi[H]]$. Þá er $\varphi(x)\in \varphi[H]$, svo
  að til er $h$ úr $H$ þannig að $\varphi(x) = \varphi(h)$. Látum $k :=
  h^{-1}x$. Þá er \[
  \varphi(k) 
  = \varphi(h)^{-1}\varphi(x)
  = \varphi(h)^{-1}\varphi(h) 
  = e_2,
  \]
  svo að $k\in K$ og $x = hk\in HK$. Þetta sýnir að
  $\varphi^{-1}[\varphi[H]]\subset HK$. Eins er $k_1 := xh^{-1}\in K$, og $x =
  k_1 h \in KH$, svo að $\varphi^{-1}[\varphi[H]]\subset KH$.
  
  Ef nú $x\in HK$, þá eru til $h$ úr $H$ og $k$ úr $K$ þannig að $x = hk$, og þá
  er \[
  \varphi(x)
  = \varphi(h)\varphi(k)
  =\varphi(h)\cdot e_2
  = \varphi(h)
  \in\varphi[H],
  \]
  svo að $x\in \varphi^{-1}[\varphi[H]]$. Þetta sýnir að $HK\subset
  \varphi^{-1}[\varphi[H]]$. Eins sést að $KH\subset \varphi^{-1}[\varphi[K]]$.
\end{sonnun}

\begin{fylgisetn}\label{cor:grupum}
  Látum $\varphi: G\to H$ vera grúpumótun og $H_1, H_2$ vera hlutgrúpur í $G$. Höfum \[
    \varphi[H_1] = \varphi[H_2] \quad \text{þ.þ.a.a.} \quad H_1 K = H_2 K.
  \]
\end{fylgisetn}

\section{Meira um rásaðar grúpur}
Látum nú $G$ vera rásaða grúpu og $a$ vera stak í $G$ þannig að $G=\langle a
\rangle$. Þá höfum við grúpumótun
\[
  \varphi_a : \Z\to G, j\mapsto a^j
\]
Höfum að til er nákvæmlega eitt stak $n\in \N$ þ.a. $\Ker\varphi_a = n\Z$; ef $n
= 0$ þá hefur $a$ óendanlegt raðstig; en ef $n\ge 1$, þá hefur $a$ raðstig $n$
og $|G| = |\langle a \rangle| = n$. Látum $k\in \Z$ og athugum
\[
  \langle a^k \rangle.
\]
Þetta er hlutgrúpa í $\langle a \rangle = G$ og
\[
  \langle a^k \rangle = \varphi_a [k\Z].
\]
Ef nú $k,j\in \Z$, þá segir fylgisetning \ref{cor:grupum} að \[
  \langle a^k \rangle
  = \langle a^j \rangle 
  \quad
  \text{þ.þ.a.a.}
  \quad
  k\Z + n\Z = j\Z + n\Z.
\]
En nú er $k\Z + n\Z = d\Z$ þar sem $d = \ssd(k,n)$. Fáum
\begin{fylgisetn}
  Látum $G$ vera grúpu, $a\in G$ vera stak með raðstig $n\ge 1$ og látum $j,k\in
  \Z$. Við höfum
  \[
    \langle a^k \rangle = \langle a^j \rangle
    \quad
    \text{þ.þ.a.a.}
    \quad
    \ssd(k,n) = \ssd(j,n).
  \]
\end{fylgisetn}
\begin{fylgisetn}
  Látum $a$ vera stak af raðstigi $n\in \N$ í grúpu $G$, $k$ vera heila tölu og
  $d:=\ssd(k,n)$. Þá er 
  \[
    \langle a^k \rangle = \langle a^d \rangle.
  \]
\end{fylgisetn}
\begin{sonnun}
  $d = \ssd(d,k)$, því að $d \mid k$
\end{sonnun}
\begin{fylgisetn}
  Látum $a$ vera stak af raðstigi $n\in \N$ í gúpu $G$, $k$ vera heila tölu. Við
  höfum
 \[
 \langle a^k \rangle  = \langle a \rangle
 \quad \text{þ.þ.a.a.} \quad
 \ssd(k,n) = 1.
 \]
\end{fylgisetn}
\begin{ath}
  Sjáum: Stak $k\in Z_n$, sem var grúpan $\left\{ 0,\dots,n-1 \right\}$ með
  samlagningu $\bmod \, n$ spannar grúpuna $Z_n$ þ.þ.a.a. $\ssd(k,n)=1$, og það
  þýðir að $k\in \mathcal U(n)$, þar sem $\mathcal U(n)$ var grúpan af stökum í
  $\left\{ 0,\dots,n-1 \right\}$ sem hafa umhverfu m.t.t. margföldunar $\bmod\,
  n$.
\end{ath}
\begin{skilgr}
  Látum $n\in \N, n\ge 1$. Við táknum með $\varphi(n)$ fjölda allra talna úr
  $\left\{ 0,\dots,n-1 \right\}$ þannig að $\ssd(k,n) = 1$.

  Ef við setjum $\N^+ := \left\{ n\in \N: n\ge 1 \right\}$, þá fáum við fall
  $\varphi:\N^+\to \N$, $n\mapsto \varphi(n)$, sem kallast \emph{$\varphi$-fall
  Eulers}\index{Euler!$\varphi$-fall}\index{Euler!totient function}\index{totient function}
  (á ensku stundum kallað \emph{Euler's totient function}).
\end{skilgr}
Við sjáum: 
\begin{itemize}
  \item [(1)] Talan $\varphi(n)$ er fjöldatala grúpunnar $\mathcal U(n)$.
  \item [(2)] Talan $\varphi(n)$ er fjöldi staka $a$ í rásaðri grúpu $G$ með
    fjöldatölu $n$ þannig að $a$ spanni alla gúpuna.
\end{itemize}
\begin{setn}
  Látum $a$ vera stak með endanlegt raðstig $n\in \N$ í grúpu $G$ og $k\in \Z$.
  Þá hefur $a^k$ raðstigið $\frac nd$, þar sem $d = \ssd(k,n)$.
\end{setn}
\begin{sonnun}
  Fyrir $j\in \Z$ fæst $a^{kj} = e$, $e$ er hlutleysan í $G$, þ.þ.a.a.
  $kj\in n\Z$, það jafngildir $\frac kd j\in \frac nd \Z$ eða m.ö.o. að
  $\frac nd \mid \frac kd j$. En $\frac nd$ og $\frac kd$ eru ósamþátta, svo að þetta
  jafngildir $\frac nd \mid j$, sem þýðir að $j\in \frac nd \Z$. 
\end{sonnun}
\begin{fylgisetn}
  Raðstig  staks í endanlegri rásaðri grúpu gengur upp í fjöldatölu grúpunnar.
\end{fylgisetn}
\begin{setn}
  Látum $G$ vera rásaða grúpu með fjöldatölu $n\in \N$. Þá er sérhver hlutgrúpa
  í $G$ rásuð og fjöldatala hennar gengur upp í fjöldatölu $G$. Fyrir sérhverja
  tölu $j$ þ.a. $j \mid n$ hefur $G$ \emph{nákvæmlega eina} hlugrúpu með
  fjöldatölu $j$.
\end{setn}
\begin{sonnun}
  Látum $a$ spanna $G$ og athugum $\varphi_a:\Z\to G, j\mapsto a^{j}$. Ef
  $H$ er hlutgrúpa í $G$, þá er $\varphi^{-1}_a[H]$ hlutgrúpa í $\Z$ þ.a.
  $n\Z\subset \varphi^{-1}_a[H]$; og þar sem $\varphi_a$ er átæk er
  $H=\varphi_a[\varphi^{-1}_a[H]]$; hlutgrúpurnar í $G$ eru því nákvæmlega
  grúpurnar $\varphi_a[k\Z]$, þar sem $n\Z\subset k\Z$, þ.e. $k \mid n$. Grúpan
  $\varphi_a[k\Z]$ hefur $\frac nk$ stök; því að hún er spönnuð af $a^k$ sem
  hefur raðstig $\frac nk$, því að $k = \ssd (k,n)$.
\end{sonnun}
Stefnum að merkilegri niðurstöðu:
\begin{setn}
  Grúpa er rásuð þ.þ.a.a. hún sé einsmóta $\Z$ eða einhverri af grúpunum
  $Z_n$, þar sem $n\ge 1$. Sér í lagi eru rásaðar grúpur með sömu fjöldatölu
  einsmóta.
\end{setn}
Mjökum okkur í átt að sönnun:

%%
%% 23. september 2009
%%

\subsection{Ámótanir}
\begin{skilgr}
  Átæk grúpumótun er stundum kölluð
  (grúpu)\emph{ámótun}\index{amotun@ámótun}\index{grúpumótun!amotun@ámótun},
  og eintæk grúpumótun er kölluð
  (grúpu)\emph{ímótun}\index{imotun@ímótun}\index{grúpumótun!imotun@ímótun}.
\end{skilgr}

Eftirfarandi setning er ekki í kennslubókinni eftir Gallian, en skiptir miklu
máli og mun birtast síðar.

\begin{setn}
  [Ámótunarsetningin] \label{setn:amotun}
  Látum $\varphi:G\to G_1$ vera \emph{átæka} grúpumótun og $\psi:G\to G_2$ vera
  grúpumótun. Þá er jafngilt:
  \begin{enumerate}[(i)]
    \item Til er grúpumótun $\chi: G_1\to G_2$ þ.a. $\psi = \chi\circ\varphi$.
    \item $\Ker\varphi\subset \Ker\psi$.
  \end{enumerate}
  Ef öðru (og þar með báðum) af þessum skilyrðum er fullnægt, þá ákvarðast
  vörpunin $\chi$ í skilyrði (i) ótvírætt, og hún er eintæk þ.þ.a.a.
  $\Ker\varphi = \Ker\psi$, en hún er átæk þ.þ.a.a. vörpunin $\psi$ sé átæk.
\end{setn}

\begin{sonnun}
  Ef (i) er fullnægt og $x\in\Ker\varphi$, þ.e. $\varphi(x) = e_1$, þar sem
  $e_1$ er hlutelysan í $G_1$, þá er $\psi(x) = \chi(\varphi(x)) = \chi(e_1) =
  e_2$, þar sem $e_2$ er hlutleysan í $G_2$. Því gildir (ii).

  Ef hins vegar (ii) er fullnægt, þá skilgreinum við vörpun $\chi:G_1\to G_2$
  þannig: Látum $x\in G_1$, þar eð $\varphi$ er átæk er til stak $y\in G$ þannig
  að $\varphi(y) = x$, veljum eitt slík stak $y$ og setjum $\chi(x) := \psi(y)$.
  Við sýnum að vörpunin $\chi$ sé \emph{vel skilgreind}; m.ö.o. að
  skilgreiningin á $\chi(x)$ sé óháð valinu á $y$: Gerum ráð fyrir að einnig
  gildi að $\varphi(y_1) = x$. Þá er 
  \[
  \varphi(y_1 y^{-1}) 
  = \varphi(y_1)\varphi(y^{-1}) 
  = xx^{-1} 
  = e_1,
  \]
  svo að $y_1y^{-1}\in \Ker\varphi\subset \Ker\psi$ og þá
  \[
  e_2 
  = \psi(y_1y^{-1})
  =\psi(y_1)\psi(y^{-1}),
  \]
  svo að $\psi(y_1) = \psi(y)$.

  Sýnum nú að $\chi$ er \emph{grúpumótun:} Látum
  $x_1,x_2\in G_1$, veljum $y_1,y_2\in G$ þ.a.  $\varphi(y_1) = x_1$ og
  $\varphi(y_2) = x_2$. Þá er $\varphi(y_1 y_2) = \varphi(y_1)\varphi(y_2) = x_1
  x_2$, svo að skv. skilgreiningu á $\chi$ er 
  \[
  \chi(x_1 x_2) 
  = \psi(y_1y_2)
  =\psi(y_1)\psi(y_2)
  =\chi(x_1)\chi(x_2).
  \]
  Ef $y\in G$, og
  $x = \varphi(y)$, þá er $\psi(y) = \chi(x) = \chi(\varphi(y))$ og því er
  $\psi=\chi\circ\varphi$. Þar með er skilyrði (i) fullnægt. 
  
  Ef $\chi_1$ er
  önnur grúpumótun $G_1\to G_2$ þ.a. $\psi=\chi_1\circ\varphi$, þá látum við
  $x\in G_1$ og finnum $y$ þ.a. $\varphi(y) = x$. Þá er 
  \[
  \chi_1(x) 
  = \chi_1(\varphi(y)) 
  = (\chi_1\circ\varphi)(y) 
  = \psi(y) 
  = \chi(x)
  \]
  og því er $\chi_1 = \chi$.

  Ef vörpunin $\chi$ er eintæk og $x\in\Ker\psi$, þá er $e_2 = \psi(x) =
  \chi(\varphi(x))$ en við höfum líka $\chi(e_1) = e_2$ og vegna eintækni er
  $\varphi(x) = e_1$, þ.e. $x\in\Ker\varphi$. Þar með er $\Ker\varphi =
  \Ker\psi$. Ef á hinn bóginn $\Ker\varphi = \Ker\psi$ og $x\in G_1$ er þannig
  að $\chi(x) = e_2$, þá veljum við $y\in G$ þannig að $\varphi(y) = x$. Þá er
  $\psi(y) = (\chi\circ\varphi)(y) = \chi(\varphi(y)) = \chi(x) = e_2$, svo að
  $x\in \Ker\psi = \Ker\varphi$ og því er $e_1 = \varphi(y) = x$. Þar með er
  $\Ker\chi = \left\{ e_1 \right\}$, svo $\chi$ er eintæk.

  Þar sem $\varphi$ er átæk er $\varphi[G] = \chi[\varphi[G]] = \chi[G_1]$, þ.e.
  $\im(\varphi) = \im(\psi)$, svo að $\chi$ er átæk þ.þ.a.a. $\psi$ sé átæk.
\end{sonnun}
\begin{fylgisetn}
  Látum $\varphi:G\to G_1$ og $\psi:G\to G_2$ vera átækar grúpumótanir þannig að
  $\Ker\varphi = \Ker\psi$. Þá eru grúpurnar $G_1$ og $G_2$ einsmóta.
\end{fylgisetn}
\begin{sonnun}
  Ámótunarsetningin segir að til sé grúpumótun $\chi:G_1\to G_2$ þ.a. $\psi =
  \chi\circ\varphi$, og hún er eintæk vegna $\Ker\varphi=\Ker\psi$ og átæk þar
  sem $\psi$ er átæk, svo að $\chi$ er einsmótun.
\end{sonnun}
Látum $G$ vera rásaða grúpu. Þá höfum við \emph{átæka} grúpumótun $\varphi:\Z\to
G$. Sáum: $G$ hefur óendanlega fjöldatölu þ.þ.a.a. $\Ker\varphi = 0$ en
endanlega fjöldatölu $n$ þ.þ.a.a. $\Ker\varphi = n\Z$, þar sem $n\in\N$,
$n\ge 1$. Fáum:
\begin{fylgisetn}
  Tvær rásaðar grúpur eru einsmóta þ.þ.a.a. þær hafi sömu fjöldatölu.
\end{fylgisetn}
Með öðrum orðum:
\begin{fylgisetn}
  Látum $G$ vera rásaða grúpu. Ef hún er óendanleg, þá er hún einsmóta
  samlagningargrúpunni $\Z$. Ef hún hefur endanlega fjöldatölu $n$, $n\ge 1$,
  þá er hún einsmóta samlagningargrúpunni $Z_n$, sem var mengið $\left\{
  0,\dots,n-1 \right\}$ með samlagningu $\bmod\;n$.
\end{fylgisetn}

\chapter{Uppstokkunargrúpur}
\section{Upprifjun og Cayley's theorem}
Við höfum skilgreint
uppstokkunargrúpu\index{uppstokkunargrúpa}\index{grúpa!uppstokkunargrúpa}
$\mathfrak S_n$; hún er grúpa allra gagntækra varpana $\sigma:\left\{ 1,\dots,n
\right\}\to\left\{ 1,\dots,n \right\}$, þar sem aðgerðin er samskeyting varpana,
köllum stökin í $\mathfrak S_n$ \emph{uppstokkanir}\index{uppstokkun} mengisins
$\left\{ 1,\dots,n \right\}$.  Fyrir $\sigma\tau\in \mathfrak S_n$ skrifum við 
\[
  \sigma\tau 
  \quad\text{í stað}\quad
  \sigma\circ\tau.
\]
\begin{ath}
  Látum $G$ vera (endanlega) grúpu. Fyrir $x\in G$ athugum við \emph{vinstri}
  hliðrunina
  \begin{equation*}
    v_g : G\to G, v_g(x) := gx.
  \end{equation*}
  Þá er 
  \begin{equation*}
    v_{gh}(x) = (gh)x = g(hx) = v_{g}(v_{h}(x)),
  \end{equation*}
  þ.e. $v_{gh} = v_g\circ v_h$. Sjáum að $v_g\circ v_{g^{-1}} = v_{g^{-1}}\circ
  v_{g} = v_e = \id_G$, þar sem $e$ er hlutleysan í $G$. Því er $v_g:G\to G$
  \emph{gagntæk} og vörpunin $v:G\to\mathfrak S(G), g\mapsto v_{g}$ er
  \emph{grúpumótun}. Hún er augljóslega eintæk: Ef $v_g = v_h$, þá er $g=v_g e =
  v_h e = h$. Þar með sést að $G$ er einsmóta myndgrúpunni $\im(v)$. Sjáum því:
  Sérhver grúpa er einsmóta hlutgrúpu í grúpu $\mathfrak S(X)$ af gagntækum
  vörpunum frá mengi $X$ í sjálft sig.

  Ef $X,Y$ eru mengi og $\varphi:X\to Y$ er gagntæk vörpun, þá fæst
  \emph{grúpueinsmótun}
  \begin{equation*}
    \Phi: \mathfrak S(G)\to \mathfrak S(Y), \sigma\mapsto
    \varphi\circ\sigma\circ\varphi^{-1}
  \end{equation*}
  því að 
  \begin{equation*}
    \Phi(\sigma\circ\tau)
    = \varphi\circ\sigma\circ\tau\varphi^{-1}
    = \varphi\circ\sigma\circ\varphi^{-1}\circ\tau\circ\varphi^{-1}
    = \Phi(\sigma)\circ\Phi(\tau)
  \end{equation*}
  og $\Phi$ hefur andhverfu $\mathfrak(Y)\to\mathfrak(X),
  \tau\mapsto\varphi^{-1}\tau\circ\varphi$. Ef nú $G$ er \emph{endanleg} grúpa,
  þá höfum við gagntæka vörpun $G\to\left\{ 1,\dots,n \right\}$ fyrir eitthvert
  $n\in\N$, $n\ge 1$. Þetta sýnir:
\end{ath}
\begin{setn}
  [Cayley's theorem]
  Sérhver endanleg grúpa er einsmóta hlutgrúpu í $\mathfrak S_n$ fyrir eitthvert
  $n\in\N$, $n\ge 1$.
\end{setn}
\begin{setn}
  Grúpan $\mathfrak S_n$ er endanlegt mengi með fjöldatölu $n!$.
\end{setn}

\section{Brautir og rásir}
\begin{skilgr}
  Látum $\sigma\in\mathfrak S_n$, þ.e. $\sigma$ er gagntæk vörpun $\left\{
  1,\dots,n \right\}\to \left\{ 1,\dots,n \right\}$. Látum $x\in \left\{
  1,\dots,n \right\}$. \emph{Braut}\index{braut}\index{uppstokkun!braut}
  staksins $x$ m.t.t. $\sigma$ er mengið
  \begin{equation*}
    [x]_\sigma := \left\{ \sigma^k(x):k\in \Z \right\}.
  \end{equation*}
  Þetta er hlutmengi í $\left\{ 1,\dots,n \right\}$. Hlutmengi í $\left\{
  1,\dots,n \right\}$ kallast \emph{braut uppstokkunarinnar} $\sigma$ ef það er
  af gerðinni $[x]_\sigma$ fyrir eitthvert $x\in \left\{ 1,\dots,n \right\}$.
\end{skilgr}
%%
%% 24. september 2009
%%
%Látum $\sigma\in\mathfrak S_n$. Við skilgreinum braut staks $x$ úr $\left\{
%1,\dots,n
%\right\}$ m.t.t. $\sigma$ sem 
%\[
%\left[ x \right]_\sigma := \left\{ \sigma^k(x): k\in \Z \right\}.
%\]
%Segjum að hlutmengi í $\left\{ 1,\dots,n \right\}$ sé \emph{braut}
%uppstokkunarinnar $\sigma$ ef það er af gerðinni $[x]_\sigma$. 
Við getum skilgreint jafngildisvensl $\sim$ á menginu $\left\{ 1,\dots,n
\right\}$ þannig að við setjum $x\sim y$ þ.þ.a.a. til sé $k\in \Z$ þannig að $y
= \sigma^k(x)$.  Til að sjá að þetta eru jafngildisvensl:
\begin{enumerate}[(a)]
  \item $x=\sigma^0(x)$, því að $\sigma^0 = \id_{\left\{1,\dots,n \right\}}$
    svo að $x\sim x$.
  \item Ef $x\sim y$ þá er til $k\in \Z$ þannig að $y = \sigma^k(x)$, en þá er
    $x=\sigma^{-k}(y)$, svo að $y\sim x$.
  \item Ef $x\sim y$ og $y\sim z$, þá eru til $k,j\in\Z$ þ.a. $y=\sigma^k(x)$ og
    $z=\sigma^j(y)$. En þá er $z = \sigma^j(\sigma^k(x))=\sigma^{j+k}(x)$, svo
    að $x\sim z$.
\end{enumerate}
Nú er ljóst að $[x]_\sigma = \left\{ y:x\sim y \right\}$ svo að $[x]_{\sigma}$
er jafngildisflokkur $x$ m.t.t. $\sigma$. Fáum:
\begin{setn}
  Brautir uppstokkunarinnar $\sigma$ mynda deildaskiptingu mengisins $\left\{
  1,\dots,n \right\}$.
\end{setn}
Látum $x$ vera \emph{kyrrapunkt} vörpunarinnar $\sigma$; það þýðir að $\sigma(x)
= x$. Þá er $[x]_\sigma = \left\{ x \right\}$. Ef $x$ er ekki kyrrapunktur, þá
hefur mengið $[x]_\sigma$ ólíka punkta, en það er \emph{endanlegt} og inniheldur
alla punkta $\sigma^k(x)$ fyrir öll $k\in\Z$, svo að til eru $k,j\in\Z$ þannig
að $j < k$ og $\sigma^j(x)=\sigma^k(x)$, og þá er $\sigma^{k-j}(x)=x$, sjáum því
að til er tala $m\ge 1$ þ.a. $\sigma^{m}(x) = x$. Nú er ljóst að mengið $\left\{
k\in\Z:\sigma^k(x) = x \right\}$ er \emph{hlutgrúpa} í $\Z$: Ef $m$ er
\emph{minnsta} talan þ.a.  $\sigma^m(x)=x$ og $m\ge 1$, þá er $\left\{
k\in\Z:\sigma^k(x)=x \right\}=m\Z$.  Sjáum þá: Ef við setjum
\[
x_k:=\sigma^{k-1}(x) \quad \forall k=1,\dots,m,
\]
þá eru stökin $x_1,\dots,x_m$ ólík og
\[
[x]_\sigma = \left\{ x_1 = x, x_2,\dots,x_m \right\},
\]
og við höfum
\begin{eqnarray*}
  \sigma(x_k) &=&  x_{k+1} \quad \text{fyrir } k = 1,\dots,m-1, \\
  \sigma(x_m) &=&  x_1.
\end{eqnarray*}
\begin{skilgr}
  Uppstokkun $\alpha$ í $\mathfrak S_n$ kallast
  \emph{rás}\index{rás}\index{uppstokkun!rás} ef hún hefur
  \emph{nákvæmlega eina} braut sem hefur fleiri en eitt stak; köllum hana
  \emph{aðalbraut}\index{aðalbraut}\index{rás!aðalbraut
  rásar}\index{braut!aðalbraut}
  rásarinnar. Ef fjöldatala aðalbrautarinnar er $m$, þá
  segjum við að $\alpha$ sé $m$-rás.

  Tvær rásir $\alpha$ og $\beta$ kallast \emph{sundurlægar}\index{sundurlægar
  rásir}\index{rásir!sundurlægar} ef aðalbrautir þeirra eru sundurlægar.
\end{skilgr}
Ef $\alpha$ er $m$-rás, þá eru til $m$ ólík stök $x_1,\dots,x_m$ í $\left\{
1,\dots,n
\right\}$ þ.a.
\begin{eqnarray*}
  \alpha(x_k) &=& x_{k+1} \quad \text{fyrir } k = 1,\dots,m-1, \\
  \alpha(x_m) &=& x_1, \\
  \alpha(x)   &=& x \quad \text{ef } x\notin\left\{ x_1,\dots,x_m \right\}.
\end{eqnarray*}
Táknum þá $\alpha$ með\[
\left( x_1,\dots,x_m \right)_n \quad\text{eða}\quad \left( x_1,\dots,x_m \right)
\]
og sleppum gjarnan kommunum ef $n < 10$. Í $\mathfrak S_9$ er $(7\,8\,1\,2) =
(7,8,1,2)_9$ rásin
\[
 \alpha =\begin{pmatrix}
   1 & 2 & 3 & 4 & 5 & 6 & 7 & 8 & 9 \\
   2 & 7 & 3 & 4 & 5 & 6 & 8 & 1 & 9
 \end{pmatrix}
\]
Látum nú $\alpha,\beta$ vera \emph{sundurlægar} rásir, $A$ vera aðalbraut
$\alpha$ og $B$ vera aðalbraut $\beta$. Þá er
\[
\alpha\beta(x) =\begin{cases}
  \alpha(x), & \text{ef } x\in A,\\
  \beta(x),  & \text{ef } x\in B,\\
  x,         & \text{ef } x\notin A\cup B.
\end{cases}
\]
Af samhverfuástæðum gildir sama um $\beta\alpha$, svo að
$\alpha\beta=\beta\alpha$. Höfum þá:
\begin{setn}
  Sundurlægar rásir víxlast.
\end{setn}
Látum nú $\sigma\in\mathfrak S_n$ og $A_1,\dots,A_r$ vera upptalningu á þeim
brautum $\sigma$ sem hafa fleiri en eitt stak. Fyrir $k=1,\dots,r$ er þá
$\alpha_k:\left\{ 1,\dots,n \right\}\to\left\{ 1,\dots,n \right\}$ sem
skilgreind er með 
\[
\alpha_k(x) :=\begin{cases}
  \sigma(x), & \text{ef } x\in A_k,\\
  x,         & \text{ef } x\notin A_k
\end{cases}
\]
rás með aðalbraut $A_k$ og við höfum
\[
\sigma = \alpha_1\alpha_2\cdots\alpha_r.
\]
Ef á hinn bóginn $\sigma$ er margfeldi $\sigma = \alpha_1\cdots\alpha_r$ af
rásum sem eru sundurlægar tvær og tvær, þá eru brautir $\sigma$ nákvæmlega
mengin $A_1,\dots,A_r$, þar sem $A_k$ er aðalbraut $\alpha_k$, $k=1,\dots,r$
ásamt einstökunum $\left\{ x \right\}$ þar sem $x\notin A_1\cup\cdots\cup A_r$.
Við höfum 
\[
\sigma | A_k = \alpha_k| A_k
\]
fyrir $k = 1,\dots,r$ og
\[
\sigma | B = \id | B
\]
þar sem $B:=\left\{ 1,\dots,n \right\}\setminus(A_1\cup\cdots\cup A_r)$. Höfum
þá
\begin{setn}
  Sérhverja uppstokkun $\sigma$ má skrifa sem samskeytingu
 \[
 \sigma = \alpha_1\cdots \alpha_r
 \]
 af rásum sem eru sundurlægar tvær og tvær (og víxlast þá tvær og tvær); og
 framsetningin ákvarðast ótvírætt burtséð frá röð.
\end{setn}
\begin{ath}
  $\id_{ \left\{ 1,\dots, n \right\}}$ er samskeyting \emph{tómu fjölskyldunnar}
  af rásum.
\end{ath}
\begin{daemi}
  Við höfum
  \[
  \begin{pmatrix}
    1 & 2 & 3 & 4 & 5 & 6 & 7 & 8 & 9 \\
    9 & 5 & 3 & 8 & 4 & 6 & 1 & 2 & 7
  \end{pmatrix}
  = (1\,9\,7)(2\,5\,4\,8).
  \]
\end{daemi}
\begin{setn}
  Sérhver rás er samskeyting af $2$-rásum, t.d.
 \[
 (x_1\cdots x_r) = (x_1\,x_r)(x_1\,x_{r-1})(x_1\,x_{r-2})\cdots
 (x_1\,x_3)(x_1\,x_2).
 \]
\end{setn}
\begin{setn}
  Sérhver uppstokkun er samskeyting af $2$-rásum.
\end{setn}
\begin{ath}
  [Viðvörun]
  Þessar tvírásir víxlast almennt ekki tvær og tvær, og framsetningin er alls
  ekki ákvörðuð ótvírætt.
\end{ath}

\section{Formerki uppstokkunar}\index{formerki
uppstokkunar}\index{uppstokkun!formerki}
\begin{setn}\label{setn:signum}
  Til er nákvæmlega ein grúpumótun
 \[
 \sign: \mathfrak S_n\to \left\{ 1,-1 \right\}
 \]
 þannig að $\sign\tau = -1$ fyrir sérhverja tvírás. Hún er gefin með
 \[
 \sign\sigma = (-1)^{n-b(\sigma)}
 \]
 þar sem $b(\sigma)$ er fjöldi brauta uppstokkunarinnar $\sigma$
 (\emph{einstökungar meðtaldir!}).
\end{setn}
\begin{fylgisetn}\label{fylgisetn:signum_1}
  Ef $\sigma\in\mathfrak S_n$ og 
 \[
 \sigma = \tau_1\cdots \tau_r
 \]
 þar sem $\tau_1,\dots,\tau_r$ eru $2$-rásir, þá er 
\[
\sign\sigma = \left( -1 \right)^r.
\]
\end{fylgisetn}
\begin{fylgisetn}\label{fylgisetn:signum_2}
  Ef $\sigma\in\mathfrak S_n$ og\[
  \sigma = \tau_1\cdots \tau_r = \tau_1'\cdots \tau_s'
  \]
  þar sem $\tau_1,\dots,\tau_r,\tau_1',\dots,\tau_s'$, þá er
 \[
 r \equiv s \quad (\bmod \,2)
 \]
 þ.e. annaðhvort eru tölurnar $r$ og $s$ báðar jafnar eða báðar oddatölur.
\end{fylgisetn}
\begin{skilgr}
  Uppstokkun $\sigma$ kallast \emph{jafnstæð}\index{uppstokkun!jafnstæð}
  ef $\sign\sigma = 1$ er \emph{oddstæð}\index{uppstokkun!oddstæð}
  ef $\sign\sigma = -1$.
\end{skilgr}
\begin{sonnun}
  [á setningu \ref{setn:signum}]
  Ef slík vörpun $\sign$ er til, þá er hún gefin með formúlunni í fylgisetningu
  \ref{fylgisetn:signum_1} og ákvarðast því ótvírætt ef hún er til.

  Látum $\sigma\in \mathfrak S_n$ og $b := b(\sigma)$ tákna fjölda braut
  uppstokkunarinnar $\sigma$. Þar sem tvírás hefur nákvæmlega $n-1$ brautir.
  Skilgreinum $\sign\sigma$ með formúlunni í setningunni og sjáum að 
 \[
 \sign\tau = (-1)^{n-(n-1)} = -1
 \]
 ef $\tau$ er $2$-rás. Til að sanna að $\sign$ sé grúpumótun sýnum við fyrst: Ef
 $\sigma\in\mathfrak S_n$ og $\tau$ er tvírás, þá er
\[
\sign(\sigma\tau) = -\sign\sigma.
\]
Athugum tvö tilvik: Skrifum $\tau = (u\;v)$.

\emph{1. tilvik:} Stökin $u,v$ eru á sömu braut $\sigma$. Skrifum brautina
$\left\{ x_1,\dots,x_r \right\}$, þar sem $\sigma(x_i) = x_{i+1}$ fyrir
$i=1,\dots,r-1$, $\sigma(r) = x_1$; megum gera ráð fyrir að $n = x_1$ og
$r = x_k$, $2\le k \le r$. Þá hefur $\sigma\tau$ sömu brautir og $\sigma$ nema
hvað brautir $\left\{ x_1,\dots,x_r \right\}$ skiptast í tvennt, nefnilega
$\left\{ x_1,x_{k+1},\dots,x_r \right\}$ og $\left\{ x_2,\dots,x_k \right\}$.

%%
%% 30. september 2009
%%

\emph{2. tilvik:} Skoðum nú tilvikið þegar $\tau = (u\;v)$ þar sem $u,v$ eru í
ólíkum brautum uppstokkunarinnar $\sigma$. Getum gert ráð fyrir að brautirnar
séu $\{x_1,\dots,x_r \}$ og $\{y_1, \dots,y_s\}$, þar sem $\sigma(x_j) =
x_{j+1}$ fyrir $j = 1,\dots,r-1$, $\sigma(x_r) = x_1$, $\sigma(y_k) = y_{k+1}$
fyrir $k = 1,\dots,s-1$, $\sigma(y_k)= y_1$ og við getum tölusett þannig að
$x_r = u$ og $y_s = v$. Brautirnar tvær sameinast í eina braut
uppstokkunarinnar $\sigma\tau$, nefnilega $\{x_1,\dots,x_r,y_1,\dots,y_s\}$
þar sem $\sigma\tau(x_j) = x_{j+1}$ fyrir $j=1,\dots,r-1$, $\sigma\tau(x_r) =
y_1$, $\sigma\tau(y_k) = y_{k+1}$ fyrir $k = 1,\dots,s-1$, $\sigma\tau(y_s) =
x_1$ en aðrar brautir $\sigma\tau$ eru eins og brautir $\sigma$. Brautir
$\sigma\tau$ eru því einni færri en brautir $\sigma$, svo að
$\sign(\sigma\tau) = -\sign(\sigma)$.

Athugum nú að $\id_{\{1,\dots,n\}} $ hefur $n$ brautir sem allar eru
einstökungar, svo að $\sign(\id_{\{1,\dots,n\}}) = (-1)^{n-n} = 1$. Ef nú
$\tau_1,\dots,\tau_s$ eru 2-rásir, þá fæst með þrepun að
\[
\sign(\tau_1\cdots\tau_s)
= \sign(\id_{\{1,\dots,n \}} \tau_1\cdots\tau_s) 
= (-1)^s \sign(\id_{\{1,\dots,n\}}) 
= (-1)^s.
\]
Látum nú $\sigma ,\tau\in\mathfrak S_n$ og skrifum $\tau = \tau_1\cdots\tau_s$
þar sem $\tau_1,\dots,\tau_s$ eru 2-rásir, þá er
\[
  \sign(\sigma\tau) 
  = \sign(\sigma\tau_1\cdots\tau_s)
  = (-1)^s\sign(\sigma)
  = \sign(\sigma)\sign(\tau)
\]
og við höfum sýnt að $\sign:\mathfrak S_n\to \{-1,1\}$ er grúpumótun.
\end{sonnun}
\begin{skilgr}
  Talan $\sign(\sigma)$ kallast \emph{formerki} uppstokkunarinnar $\sigma$.
  Tvírás er líka kölluð \emph{umskipting} (e. \emph{transposition}). Rifjum
  upp að uppstokkun $\sigma$ kallast \emph{jafnstæð} ef $\sign(\sigma) = 1$ en
  \emph{oddstæð} ef $\sign(\sigma) = -1$. Táknum með $\mathfrak A_n$ mengi
  allra \emph{jafnstæðra} uppstokkana í $\mathfrak S_n$.
\end{skilgr}
\begin{setn}
  Mengið $\mathfrak A_n$ er hlutgrúpa í $\mathfrak S_n$ og fyrir $n\ge 2$ er
  fjöldatala þess $\frac 12 n!$.
\end{setn}
\begin{sonnun}
  Höfum $\mathfrak A_n = \Ker(\sign)$, svo að $\mathfrak A_n$ er hlutgrúpa í
  $\mathfrak S_n$. Látum $n\ge 2$ og $\tau$ vera tvírás. Þá er
  \[
  \mathfrak S_n \setminus \mathfrak A_n 
  = \tau\mathfrak A_n 
  := \{\tau\rho: \rho\in \mathfrak A_n\}
  \]
  ef nefnilega $\sigma \in \tau \mathfrak A_n$, $\sigma = \tau\rho$ með
  $\rho\in \mathfrak A_n$, þá er $\sign(\sigma) = \sign(\tau)\cdot \sign(\rho)
  = (-1)\cdot 1 = -1$, svo að $\sigma\in \mathfrak S_n\setminus \mathfrak
  A_n$. Ef hins vegar $\sigma\in\mathfrak S_n\setminus \mathfrak A_n$, þ.e.
  $\sign(\sigma) = -1$, setjum $\rho := \tau\sigma$, þá er $\sigma = \tau^2
  \sigma = \tau\rho$ og $\sign(\rho) = \sign(\tau)\sign(\sigma) = (-1)(-1) =
  1$, svo að $\rho\in \mathfrak A_n$. En $\tau\mathfrak A_n$ hefur jafnmörg
  stök og $\mathfrak A_n$, því að vörpunin $\mathfrak A_n \to \tau\mathfrak
  A_n, \rho\mapsto \tau\rho$ er gagntæk með andhverfu $\tau\mathfrak A_n \to
  \mathfrak A_n, \sigma\mapsto \tau\sigma$ (vegna $\tau^2 =
  \id_{\{1,\dots,n\}}$), þess vegna hafa $\mathfrak A_n$ og $\tau\mathfrak
  A_n$ jafnmörg stök, þau eru sundurlæg, og sammengið $\mathfrak S_n$ hefur
  $n!$ stök, svo að $\mathfrak A_n$ hefur $\frac 12 n!$ stök.
\end{sonnun}
\begin{ath}
  Mengið $\tau\mathfrak A_n$ er svokallað \emph{hjámengi} hlutgrúpunnar
  $\mathfrak A_n$ í $\mathfrak S_n$.
\end{ath}


\chapter{Hjámengi}
\section{Hjámengi}
\begin{skilgr}\index{hjámengi}\index{grúpa!hjámengi}
  Látum $H$ vera hlutgrúpu í grúpu $G$ og $x$ vera stak í $G$. Við köllum mengið
  \begin{equation*}
    xH := \{ xh : h\in H \}
  \end{equation*}
  \emph{vinstra hjámengi} hlutgrúpunnar $H$ gegnum stakið $x$, og mengið
  \begin{equation*}
    Hx := \{ hx : h\in H \}
  \end{equation*}
  kallast \emph{hægra hjámengi} hlutgrúpunnar $H$ gegnum stakið $x$. 
\end{skilgr}
\begin{ath}
  Höfum $x\in xH$ og $x\in Hx$ vegna $x = xe$ og $x=ex$, þar sem $e$ er
  hlutleysan í $G$, og $e\in H$. Ef aðgerðin í $G$ er skrifuð sem samlagning, þá
  eru hjámengin skrifuð
  \[
    x+H = \{ x+h: h\in H\}
  \]
  \[
    H + x = \{ h + x: h\in H\}.
  \]
  Ef grúpan $G$ er víxlin, þá er $xH = Hx$ fyrir öll $x\in G$ og allar
  hlutgrúpur $H$ í $G$. Þá þarf ekki að gera neinn greinamun á hægri og vinstri
  hjámengjum.
\end{ath}
\begin{daemi}
  Látum $G= \R^2$ með samlagningu. Þetta er línulegt rúm yfir $\R$, og línuleg
  hlutrúm eru hlutgrúpur. Sér í lagi er sérhver lína gegnum núllpunktinn
  hlutgrúpa í $\R^2$; látum $H$ vera slíka línu og $x\in\R^2$. Þá er hjámengið
  $x+H$ línan gegnum $x$ samsíða línunni $H$.
\end{daemi}
Látum nú $H$ vera hlutgrúpu í grúpu $G$ og $x\in G$. Fyrir öll $y\in G$ gildir
$y\in xH$ þ.þ.a.a. til sé $h\in H$ þ.a. $y = xh$; og það er jafngilt $x^{-1} y
\in H$ (það er líka jafngilt því að $y^{-1} x = (x^{-1}y)^{-1} \in H$). Skrifum
nú
\begin{equation*}
  x\sim y \quad\text{þ.þ.a.a.}\quad x^{-1} y \in H,
\end{equation*}
þetta eru jafngildisvensl:
\begin{itemize}
  \item [(a)] $x^{-1} x = e\in H$, svo að $x\sim x$.
  \item [(b)] Ef $x\sim y$, þá er $x^{-1}y \in H$, þá er $y^{-1}x =
    (x^{-1}y)^{-1}\in H$, svo að $y\sim x$.
  \item [(c)] Ef $x\sim y$ og $y\sim z$, þá er $x^{-1}y\in H$ og $y^{-1} z\in
    H$, en þá er $x^{-1}z = x^{-1}yy^{-1}z \in H$, svo að $x\sim z$.
\end{itemize}
Höfum sýnt:
\begin{setn}
Vinstra hjámengið $xH$ er jafngildisflokkur staksins $x$ mt.t.
jafngildisvenslanna $\sim$, þar sem 
\begin{equation*}
  x\sim y \quad\text{þ.þ.a.a.} \quad x^{-1}y\in H.
\end{equation*}
Sér í lagi mynda vinstri hjámengin deildaskiptingu mengisins $G$.
\end{setn}
\begin{fylgisetn}
  Höfum $y \in xH$ þ.þ.a.a. $yH = xH$.
\end{fylgisetn}
Höfum líka $yx^{-1}\in H$. Fáum samsvarandi setningu:
\begin{setn}
  Hægra hjámengið $Hx$ er jafngildisflokkur staksins $x$ m.t.t.
  jafngildisvenslanna $\sim_1$, þar sem
  \begin{equation*}
    x\sim_1 y \quad\text{þ.þ.a.a.}\quad yx^{-1}\in H.
  \end{equation*}
  Sér í lagi mynda hægri hjámengin deildaskiptingu mengisins.
\end{setn}
\begin{fylgisetn}
  Höfum $y\in Hx$ þ.þ.a.a. $Hy= Hx$. 
\end{fylgisetn}
Látum $x\in G$, þá fæst gagntæk vörpun
\begin{equation*}
  H\to xH, y\mapsto xy
\end{equation*}
umhverfan er $xH\to H, z\mapsto x^{-1}z$.
\begin{fylgisetn}
    Látum $G$ vera endanlega grúpu og $H$ vera hlutgrúpu í $G$. Þá hafa öll
    vinstri hjámengin $xH$ jafnmörg stök. Ef $h$ er fjöldatala $h$, $g$
    fjöldatala $G$ og $m$ fjöldi hjámengja, þá er $g = mh$. Sér í lagi gildir
    \[
      h \mid g.
    \]
\end{fylgisetn}
%%
%% 1. október 2009
%%
\section{Vísitala og tenging við rásaðar grúpur}
\begin{skilgr}
  Látum $H$ vera hlutgrúpu í grúpu $G$. Við segjum að $H$ hafi \emph{endanlega
  vísitölu}\index{vísitala}\index{grúpa!vísitala}\index{hlutgrúpa!vísitala}
  í $G$ ef vinstri hjámengi grúpunnar $H$ í $G$ eru endanlega mörg, og
  fjöldi þeirra kallast þá vísitala $H$ í $G$ og er táknuð
 \[
  (G:H),
 \]
 í bók er skrifað $|G:H|$.
\end{skilgr}
\begin{ath}
  Vörpunin $\psi:G\to G, \psi(x) := x^{-1}$ varpar $xH$ á $Hx^{-1}$; svo að
  $\psi$ gefur af sér gagntæka vörpun milli mengis allra vinstri hjámengja og
  mengis allra hægri hjámengja; þau eru því jafnmörg.
\end{ath}
Getum nú orðað seinustu setningu öðruvísi: Ef grúpan $G$ er endanleg, þá er
\[
 \#G = (G:H)\#H.
\]
\begin{ath}
  Táknum með $\#G$ fjöldatölu (endanlegs) mengis.
\end{ath}
\begin{fylgisetn}
  Ef $G$ er endanelg grúpa og $H$ er hlutgrúpa í $G$, á er
 \[
 (G:H) = \frac{\#G}{\#H}.
 \]
\end{fylgisetn}
\begin{fylgisetn}
  Sérhver grúpa með fjöldatölu sem er frumtala er rásuð.
\end{fylgisetn}

\begin{sonnun}
  Látum $G$ vera grúpu, $\#G = p$, þar sem $p$ er frumtala. Sér í lagi er
  $p > 1$, svo að til er stak $a$ í $G$ annað en hlutleysan. Þá er
  $\langle a \rangle$ hlutgrúpa með fjöldatölu $\ge 2$ sem gengur upp í
  $p$, svo að $\#\langle a \rangle = p$, og þá $\langle a \rangle = G$.
\end{sonnun}

\begin{fylgisetn}
  Ef $a$ er stak í endanlegri grúpu $G$, þá gengur raðstig þess upp í $\#G$.
\end{fylgisetn}
\begin{sonnun}
  Raðstigið er $\#\langle a \rangle$.
\end{sonnun}
\begin{fylgisetn}
  Ef $G$ er endanleg grúpa með hlutleysu $e$ og $n = \#G$, þá er
 \[
 a^{n} = e.
 \]
\end{fylgisetn}
\begin{sonnun}
  Látum $r$ vera raðstig $a$, þá er til $m$ þ.a. $n = mr$, svo að
 \[
 a^n = \left( a^r \right)^{m} = e^{m} = e.
 \]

 \subsection{Fermat, Fermat-Euler}
\end{sonnun}
\begin{fylgisetn}
  [Litla Fermat-setningin]
  Ef $a$ er heil tala sem er ekki heilt margfeldi af frumtölu $p$, þá er 
 \[
 a^{p-1} \equiv 1 \pmod p.
 \]
\end{fylgisetn}
\begin{sonnun}
  Tölurnar $1,\dots,p-1$ mynda trúpu m.t.t. margföldunar $(\bmod \;p)$.
\end{sonnun}
\begin{fylgisetn}
  [Fermat-Euler]
  Ef $m\ge 1$ og $a\in\Z$ er þannig að $\ssd(a,m) = 1$, þá er
 \[
 a^{\varphi(m)}\equiv 1 \pmod m,
 \]
 hér er $\varphi$ fall Eulers.
\end{fylgisetn}
\begin{sonnun}
  $a\bmod m$ er í grúpu $\mathcal U(m)$ sem hefur margföldun $\bmod p$ sem
  aðgerð, og $\#\mathcal U(m) = \varphi(m)$.
\end{sonnun}

\section{Normlegar hlutgrúpur}
\begin{skilgr}
  Hlutgrúpa $H$ í grúpu $G$ kallast \emph{normleg}\index{normleg
  hlutgrúpa}\index{hlutgrúpa!normleg}\index{grúpa!normleg}
  í $G$ ef $xH = Hx$ fyrir öll $x$ úr $G$.
\end{skilgr}
\begin{ath}
  Skilgreindum almennt $AB := \left\{ ab : a\in A, b\in B \right\}$ fyrir
  hlutmengi $A,B$ í $G$; höfum reikniregluna
 \[
 (AB)C = A(BC) = \left\{ abc : a\in A,b\in B,c\in C \right\}.
 \]
 Höfum líka
 \[
 xH = \{ x \}H 
 \qquad\text{og}\qquad
 Hx = H\{ x \}.
 \]
 Af því leiðir t.d.
 \[
 xH = Hx 
 \qquad \text{þ.þ.a.a.} \qquad
 xHx^{-1} = H.
 \]
 Af $\{ x \}H = H\{ x \}$ leiðir nefnilega
 $\{ x \} H \{ x^{-1}\} = H\{x\}\{x^{-1}\} = H\{e\} = H$, og af $H = \{x\} H \{
 x^{-1} \}$ leiðir $H\{x\} = \{x\} H \{x^{-1}\}\{x\} = xH $.
\end{ath}
\emph{Innskot:} Eftirfarandi skilgreiningu setti Reynir fram í dæmatíma 23.
september, hún á eiginlega betur heima hér í fyrirlestrunum:
\begin{skilgr}
  (i) Látum $G$ vera grúpu. Einsmótun $\phi:G\to G$ kallast
  \emph{sjálfmótun}\index{sjálfmótun} grúpunnar $G$ og mengi allra sjálfmótana
  $G$ er táknað\[
  \Aut G.
  \]
  Þetta er hlutgrúpa í $\mathfrak S(G)$ og kallast
  \emph{sjálfmótanagrúpa}\index{grúpa!sjálfmótanagrúpa}\index{sjálfmótanagrúpa}
  grúpunnar $G$.

  (ii) Skilgreinum fyrir sérhvert $x\in G$ vörpun \[
  \iota_x : G\to G,\quad g\mapsto xgx^{-1}.
  \]
  Þetta er sjálfmótun.\footnotemark Köllum slíkar sjálfmótanir \emph{innri
  sjálfmótanir}\index{innri sjálfmótun}\index{sjálfmótun!innri} grúpunnar $G$.
  Innri sjálfmótanirnar mynda hlutgrúpu í $\Aut G$, hún er táknuð \[
  \Inn(G).
  \]
\end{skilgr}
\footnotetext{Sjá nánar dæmablað 4, dæmi 21}
Höldum nú áfram þaðan sem frá var horfið, en skiljum betur eftirfarandi setningu:
\begin{fylgisetn}
  Hlutgrúpa $H$ í $G$ er normleg í $G$ þ.þ.a.a. 
  \[
  \iota_x\left[ H \right] = H
  \]
  fyrir allar innri sjálfmótanir $\iota_x$ grúpunnar $G$.
\end{fylgisetn}
Athugum að fyrir sérhverja hlutgrúpu $H$ í grúpu $G$ er $HH = H$: Höfum
$H\subset HH$ vegna þess að $h = eh = he$ og $e\in H$; og $HH\subset H$ því að
$H$ er lokað m.t.t. margföldunar.

Látum nú $N$ vera \emph{normlega} hlutgrúpu í $G$. Þá er 
\[
 (xN)(yN) = x(Ny)N = x(yN)N = xyNN = xyN.
\]
Við höfum þá
\[
xNx^{-1}N = xx^{-1}N = eN = N
\]
og líka $x^{-1}NxN = x^{-1}xN = eN = N$. Líka er $xNN = xN$ og 
\[
 N(xN) = (Nx)N = (xN)N = xNN = xN.
\]
Höfum sýnt:
\begin{setn}
  Látum $N$ vera normlega hlutgrúpu í grúpu $N$ og $G/N$ vera mengi (vinstri)
  hjámengja grúpunnar $N$ í $G$. Þá verður $G/N$ að grúpu með reikniaðgerð sem
  gefin er með
 \[
  xN \cdot yN = xyN.
 \]
  Hlutleysan í grúpunni $G/N$ er
 \[
  eN = N,
 \]
  og margföldunarumhverfa $xN$ er $x^{-1}N$. Vörpunin 
 \[
  \pi : G\to G/N
 \]
  er átæk grúpumótun með kjarna $N$.
\end{setn}
\begin{skilgr}
  Köllum grúpu $G/N$, þar sem $N$ er normleg hlutgrúpa í $G$,
  \emph{deildagrúpu}\index{grúpa!deildagrúpa}\index{deildagrúpa}
  af grúpunni $G$, og vörpunina $\pi: G\to G/N$ náttúrlega ofanvarpið.
\end{skilgr}
Eftir var að sanna fullyrðinguna um $\pi$, en 
\[
 \pi(xy) = xyN = xN\cdot yN = \pi(x)\pi(y)
\]
svo að $\pi$ er grúpumótun, og við höfum $x\in \Ker \pi$ þ.þ.a.a. $xN = N$, en
það jafngildir $x\in N$.
\begin{ath}
  Fyrir grúpu $G$ þar sem aðgerðin er skrifuð sem samlagning er aðgerðin á
  $G/N$ gefin með
 \[
  (x+N) + (y+N) = (x+y)+N.
 \]
\end{ath}
\begin{ath}
  [Mikilvægt]
  Allar hlutgrúpur í \emph{víxl}grúpu eru normlegar!
\end{ath}
\begin{daemi}
  Látum $m\in \N, m\ge 1$. Stökin í $\Z / m\Z$ eru $m$ talsins, nefnilega
  hjámengin
 \[
  0 + m\Z = m\Z, 1+m\Z, \dots, (m-1) + m\Z.
 \]
 Höfum
 \[
 k + m\Z = \left\{ x\in \Z : x\equiv k \;(\bmod\;m) \right\}.
 \]
 Náttúrlega ofanvarpið $\pi: \Z \to \Z / m\Z$ er átæk grúpumótun með kjarna
 $m\Z$. Nú höfum við líka aðra átæka grúpumótun $\phi: \Z \to Z_m, k\mapsto
 k\bmod m$, með kjarna $m\Z$.

 Skv. \emph{ámótunarsetningunni} er til nákvæmlega ein grúpumótun $\chi: \Z / m\Z
 \to Z_m$, og hún er \emph{grúpueinsmótun}. Fyrir $m = 0$ er ljóst að $m\Z =
 \left\{ 0 \right\}$ og $\Z/m\Z \cong \Z$. Sjáum: Grúpa er rásuð þ.þ.a.a. hún sé
 einsmóta deildagrúpu af $\Z$.
\[
\xymatrix{
\Z\ar[r]^{\pi}\ar[dr]^{\phi} & \Z/m\Z\ar[d]^{\chi} \\
&Z_m
}
\]
\end{daemi}
%%
%% 7. október 2009
%%
Með öðrum orðum:
\begin{setn}
  Látum $\pi:\Z\to\Z/m\Z$ vera náttúrlega ofanvarpið,
  \begin{equation*}
    \pi(x) = x + m\Z
    \qquad
    \forall x\in\Z
  \end{equation*}
  og $\varphi:\Z\to Z_m$ vera vörpunina sem er gefin með $\varphi(x) = x\bmod
  m$. Þá eru $\pi$ og $\varphi$ átækar grúpumótanir með sama kjarna, nefnilega
  $m\Z$, svo að til er nákvæmlega ein grúpumótun
  \begin{equation*}
    \chi:\Z/m\Z \to Z_m
  \end{equation*}
  þannig að $\varphi = \chi\circ\pi$, þ.e. $x\bmod m = \chi(x+m\Z)$ og hún er
  \emph{einsmótun}.
\end{setn}
\begin{sonnun}
  Þetta er bein afleiðing af \emph{ámótunarsetningunni} (bls.
  \pageref{setn:amotun})
\end{sonnun}

\section{Setningar um einsmótanir}
Önnur afleiðing ámótunarsetningarinnar er:
\begin{setn}
  [Fyrsta einsmótunarsetningin]
  Látum $\varphi:G\to H$ vera grúpumótun, þá er $\Ker\varphi$ normleg hlutgrúpa
  í $G$ og við höfum náttúrlega einsmótun\footnotemark
  \begin{equation*}
    G/\Ker\varphi\to \im(\varphi),
  \end{equation*}
  þ.e. $\im(\varphi) \cong G/\Ker\varphi$.
\end{setn}
\footnotetext{\emph{Náttúrleg einsmótun} er einsmótun sem telst venjulega gefin
og við þurfum ekki að velja neitt af handahófi (gróf lýsing).}
\begin{sonnun}
  Látum $x\in \Ker\varphi$ og $g\in G$. Þá er
  \begin{align*}
    \varphi(gxg^{-1})
    &= \varphi(g)\varphi(x)\varphi(g^{-1}) \\
    &= \varphi(g)e_H\varphi(g^{-1}) \\
    &= \varphi(g)\varphi(g^{-1}) \\
    &= \varphi(gg^{-1}) \\
    &= \varphi(e_G) \\
    &= e_H,
  \end{align*}
  svo að $gxg^{-1}\in \Ker\varphi$, þetta sýnir að $\Ker\varphi$ er normleg
  hlutgrúpa. Látum $\pi:G\to G/\Ker\varphi$ vera náttúrlega ofanvarpið, það
  hefur kjarnann $\Ker\varphi$ og er átæk vörpun. Vörpunin
  $\tilde{\varphi}:G\to\im(\varphi)$, $\tilde{\varphi}(x):=\varphi(x)$ fyrir öll
  $x$, er líka átæk grúpumótun með kjarnann $\Ker\varphi$. Þá er til nákvæmlega
  ein grúpumótun $\chi:G/\Ker\varphi\to \im(\varphi)$ þannig að örvaritið
  \[
  \xymatrix{
  G \ar[r]^{\pi} \ar[dr]_{\tilde{\varphi}} & G/\Ker\varphi\ar[d]^{\chi} \\
  & \im(\varphi)
  }
  \]
  sé víxlið, þ.e. $\chi\circ\pi = \tilde{\varphi}$, og hún er einsmótun skv.
  ámótunarsetningunni.
\end{sonnun}
\begin{ath}
  \emph{Örvarit} er fjölskylda af mengjum ásamt vörpunum á milli þeirra, þeim má
  lýsa með myndum á borð við
  \[
  \xymatrix{
  X \ar[r]^{\varphi} \ar[d]_\alpha 
  & Y \ar[r]^\psi \ar[d]_\beta
  & Z\ar[d]_\gamma
  \\
  U \ar[r]^{\chi}
  & V \ar[r]^\theta
  & T
  }
  \]
  Að svona örvarit sé \emph{víxlið} þýðir: Fyrir sérhver tvö mengi í örvaritinu
  eru allar varpanir frá öðru í hitt, sem fást sem samskeytingar af vörpunum í
  örvaritinu, sama vörpunin.
  Að örvaritið hér að ofan sé víxlið þýðir að 
  \begin{equation*}
    \beta\circ\varphi = \chi\circ\alpha
    \quad\text{og}\quad
    \gamma\circ\psi = \theta\circ\beta
  \end{equation*}
  því af því leiðir að 
  \begin{equation*}
      \gamma \circ \psi  \circ \varphi
    = \theta \circ \beta \circ \varphi
    = \theta \circ \chi  \circ \alpha.
  \end{equation*}
\end{ath}
\begin{setn}
  [Önnur einsmótunarsetningin]
  Látum $K$ vera hlutgrúpu í grúpu $G$ og $N$ vera normlega hlutgrúpu í
  $G$. Þá er
  \begin{equation*}
    K/(K\cap N) \cong KN/N.
  \end{equation*}
\end{setn}
\begin{sonnun}
  Þar sem $N$ er normleg er $kN = Nk$ fyrir öll $k\in K$ og þá $KN = NK$, svo að
  skv. gömlu heimadæmi er $KN$ hlutgrúpa í $G$. Hún inniheldur $N$ sem
  hlutgrúpu, og þar sem $N$ er normleg í $G$ er hún normleg í $KN$, svo að við
  getum myndað deildagrúpuna $KN/N$ og höfum náttúrlega ofanvarpið $\pi: KN \to
  KN/N$. Fáum líka vörpun $\varphi:K\to KN/N, k\mapsto \pi(k)$ (athugum að
  $K\subset KN$), höfum $\varphi = \pi|K$. Þá er $\varphi$ átæk: Látum
  $\zeta\in KN/N$, þá má skrifa $\zeta = xN$, þar sem $x\in KN$, og $x = kn$,
  þar sem $k\in K$ og $n\in \N$. Þá er $nN = N$, svo að 
  \[
  \varphi(x) = kN = k(nN) = (kn)N = xN = \zeta.
  \]
  Nú er $\varphi(k) = N$ þ.þ.a.a. $k\in K\cap N$ ef $k\in K$, svo að
  $\Ker\varphi = K\cap N$. Þá er $K\cap N$ normleg hlutgrúpa í $K$, og við höfum
  átækar varpanir
  \[
  \xymatrix{
  K\ar[r]^\varphi \ar[dr]_{\txt<6pc>{\scriptsize{náttúrlega ofanvarpið}}}
  & KN/N \ar[d] \\
  & K/(K\cap N)
  }
  \]
  með sama kjarna $K\cap N$, svo að $K/K\cap N \cong KN/N$.
\end{sonnun}
\begin{setn}
  [Þriðja einsmótunarsetningin]
  Látum $M$ og $N$ vera normlegar hlutgrúpur í $G$ þannig að $N\subset M$. Þá er
  $(G/N)/(M/N)\cong G/M$.
\end{setn}
\begin{sonnun}
  Athugum að náttúrlegu ofanvörpin $\pi_M:G\to G/M$ og $\pi_N:G\to G/N$ eru
  átækar grúpumótanir og $\Ker\pi_N = N\subset M = \Ker\pi_M$. Skv.
  ámótunarsetningunni er til nákvæmlega ein grúpumótun $\chi:G/N\to G/M$ þ.a.
  örvaritið
  \[
  \xymatrix{
  G\ar[r]^{\pi_N} \ar[dr]^{\pi_M} & G/N \ar[d]^\chi \\
  & G/M
  }
  \]
  sé víxlið, þ.e. $\chi(xN)=xM$ fyrir öll $x\in G$. Höfum þá $\chi(xN) = M$
  þ.þ.a.a. $x\in M$, og það þýðir að $xN\in M/N$. M.ö.o. er
  \begin{equation*}
    \Ker\chi = M/N
  \end{equation*}
  því $M/N$ er normleg hlutgrúpa í $G/N$ og við höfum nú þegar átækar
  grúpumótanir
  \[
  \xymatrix{
  G/N\ar[r]^{\chi} \ar[dr]_{\txt<4pc>{\scriptsize{náttúrlegt ofanvarp}}}
  & G/M \ar[d] \\
  & (G/N)/(M/N) 
  }
  \]
  með sama kjarna, nefnilega $M/N$; því er $G/M\cong (G/N)/(M/N)$.
\end{sonnun}

\chapter{Bein margfeldi af grúpum}

\section{(Ytri) bein margfeldi og beinar summur}
Látum $G_1,\dots,G_n$ vera grúpur. Þá má skilgreina reikniaðgerð á
margfeldismenginu 
\[
G_1\times\cdots \times G_n 
= \left\{ (g_1,\dots,g_n):g_k\in G_k, k=1,\dots,n \right\}
\]
með því að setja
\[
(g_1,\dots,g_n)(h_1,\dots,h_n)
= (g_1 h_1,\dots,g_n h_n),
\]
þetta gerir $G_1\times \cdots \times G_n$ að grúpu, hlutleysan er
$(e_1,\dots,e_n)$ þar sem $e_k$ er hlutleysan í $G_k$ fyrir $k=1,\dots,n$.
Umhverfa staks $g = (g_1,\dots,g_n)$ er $g^{-1} = (g_1^{-1},\dots,g_n^{-1})$,
þar sem $g_k^{-1}$ er umhverfa $g_k$ í $G_k$.
\begin{skilgr}
  Köllum mengið $G_1\times \cdots \times G_n$ með þessari margföldun
  \emph{(ytra) margfeldi}\index{beint margfeldi!ytra}\index{beint margfeldi}
  af grúpunum $G_1,\dots,G_n$ og táknum það með
  \begin{equation*}
    G_1\times\cdots\times G_n
    \quad\text{eða}\quad
    \prod_{k=1}^n G_k.
  \end{equation*}
  \emph{Ef grúpurnar $G_1,\dots,Gn$ eru víxlnar}, þá táknum við það líka með
  \[
  G_1 \oplus\cdots\oplus G_n
  \quad\text{eða}\quad
  \bigoplus_{k=1}^n G_k
  \]
  og köllum það \emph{(ytri) beinu summuna}\index{bein summa}\index{bein summa!ytri}
  af $G_1,\dots,G_n$.
\end{skilgr}
\begin{setn}
  Látum $G_1,\dots,G_n$ vera grúpur, náttúrlegu ofanvörpin
  $\pr_k:G_1\times\cdots\times G_n\to G_k$ eru grúpumótanir og fullnægja
  eftirfarandi: Ef $G$ er grúpa og $f_k:G\to G_k$ er grúpumótun fyrir
  $k=1,\dots,n$, þá er til nákvæmlega ein grúpumótun $f:G\to
  G_1\times\cdots\times G_n$ þannig að örvaritið
  \begin{equation*}
    \xymatrix{
    G\ar[rr]^f \ar[drr]^{f_k} && G_1\times\cdots\times G_n\ar[d]^{\pr_k} \\
    && G_k
    }
  \end{equation*}
  sé víxlið fyrir öll $k=1,\dots,n$. Vörpunin $f$ er gefin með 
  \begin{equation*}
    f(g) = \left( f_1(g),\dots,f_n(g) \right).
  \end{equation*}
\end{setn}
\begin{sonnun}
  Augljóst, það þarf bara að sýna að þetta sé grúpumótun.
\end{sonnun}
%%
%% 8. október 2009
%%
\begin{setn}
  Látum $G_1,\dots, G_n$ vera víxlgrúpur, skrifaðar sem samlagningargrúpur
  og táknum með $0_i$ hlutleysuna í $G_i$, $i=1,\dots,n$. Varpanirnar
  \begin{equation*}
    \ivarp_i:G_i\to G_1\oplus\cdots\oplus G_n,
    \quad
    \ivarp_i(g_i) := \left( 0_1,\dots,0_{i-1},g_i,0_{i+1},\dots,0_n \right)
  \end{equation*}
  eru grúpumótanir.
  Látum
  $f_i : G_i \to G$ vera grúpumótun frá $G_i$ í \emph{víxl}grúpu $G$ fyrir
  $i=1,\dots,n$. 
  Þá er til nákvæmlega ein grúpumótun $f:G_1\oplus\cdots\oplus G_n \to
  G$ þannig að örvaritið 
  \begin{equation*}
  \xymatrix{
    & G & 
    \\
    G_i \ar[ur]^{f_i} \ar [rr]_{\ivarp_i} 
    &
    & {G_1\oplus\cdots\oplus G_n} \ar[ul]_{f}
  }
  \end{equation*}
  sé víxlið fyrir öll $i = 1,\dots, n$, þ.e.
  $f\circ in_i = f_i$. Vörpunin $f$ er gefin með
  \begin{equation*}
    f(x_1,\dots,x_n ) = \sum_{i=1}^{n}f_i(x_i).
  \end{equation*}
\end{setn}
\begin{sonnun}
  Ef til er grúpumótun $f$ þ.a. $f\circ \ivarp_i$ fyrir öll $i$, þá er
  \[
  f(x_1,\dots,x_n)
  = f\left( \sum_{i=1}^{n}\ivarp_i(x_i) \right)
  = \sum_{i=1}^{n}f\circ\ivarp_i(x_i)
  = \sum_{i=1}^{n}f_i(x_i),
  \]
  svo að $f$ ákvarðast ótvírætt og er gefið með formúlunni í setningunni, ef það
  er til. Skilgreinum nú á hinn bóginn vörpun $f$ með formúlunni, þá er
  $f$ grúpumótun því að
  \begin{align*}
    f( (x_1,\dots,x_n)+ (y_1,\dots,y_n) )
    &= f(x_1+y_1,\cdots,x_n+y_n)\\
    &= \sum_{i=1}^{n}f(x_i+y_i) \\
    &= \sum_{i=1}^{n}\left( f_i(x_i) + f_i(y_i) \right) \\
    &= \sum_{i=1}^{n}f_i(x_i) + \sum_{i=1}^{n}f_i(y_i) \\
    &= f(x_1,\dots,x_n) + f(y_1,\dots,y_n).
  \end{align*}
  Auk þess er
  \begin{align*}
    f\circ\ivarp_i(x)
    &= f(0_1,\dots,0_{i-1},x,0_{i+1},\dots,0) \\
    &= f_1(0_1) + \cdots + f_{i-1}(0_{i-1}) + f_i(x) + f_{i+1}(0_{i+1}) + \cdots
    + f_n(0_n) \\
    &= f_i(x)
  \end{align*}
  fyrir öll $x\in G_i$, svo að $f\circ \ivarp_i = f$ fyrir öll $i\in I$.
\end{sonnun}
\begin{ath}
  Við notuðum regluna $\sum_{i=1}^{n}(x_i+y_i) = \sum_{i=1}^{n} x_i +
  \sum_{i=1}^{n}y_i$, sem gildir í víxlgrúpum. Ef aðgerðin er skrifuð sem
  margföldun, þá verður þetta 
\begin{equation*}
  \prod_{i=1}^n (a_i b_i)
  = \left( \prod_{i=1}^n a_i \right)\left( \prod_{i=1}^n b_i \right).
\end{equation*}
Hér er $\prod_{i=1}^n a_i := a_1\cdots a_n$, nánar tiltekið er 
$\prod_{i=1}^0 a_i := e$, þ.e. hlutleysan í $G$, $\prod_{i=1}^{n+1}a_i =
\left(\prod_{i=1}^n\right)a_{n+1}$ þrepunarskilgreining.
\end{ath}
\begin{ath}
  Reglan $\prod_{i=1}^n a_ib_i = \left( \prod_{i=1}^n a_i \right)\left(
  \prod_{i=1}^n b_i \right)$ gildir \emph{alls ekki} í grúpum sem eru ekki
  víxlnar. Fyrir $n=2$ segir hún
  \[
   a_1 b_1 a_2 b_2 = a_1 a_2 b_1 b_2
  \]
  sem jafngildir $b_1 a_2 = a_2 b_1$. Þess vegna gildir síðasta setning ekki
  fyrir óvíxlnar grúpur. 
\end{ath}

\section{Kínverska leifasetningin}
\begin{setn}\index{kínverska leifasetningin}
  [Kínverska leifasetningin]
  Látum $m_1,\dots,m_r$ vera náttúrlegar tölur þ.a. $m_k\ge 1$ fyrir öll
  $k=1,\dots,r$ og $\ssd(m_j,m_k)=1$ ef $j\neq k$. Setjum $m:=m_1\cdots m_r$. Þar
  sem $m_i \mid m$ er til nákvæmlega ein grúpumótun
 \[
  \psi_i : \Z/m\Z\to \Z/m_i\Z
  \quad\text{þ.a.}\quad
  \psi_i(x+m\Z) = x+m_i\Z
 \]
  fyrir öll $x\in \Z$. Þá er vörpunin
 \[
  \psi := (\psi_1,\dots,\psi_n):\Z/m\Z \to \Z/m_1\Z\oplus\cdots\oplus\Z/m_r\Z
 \]
  grúpumótun.
\end{setn}
\begin{sonnun}
  Ákvörðum $\Ker\psi$. Látum $\xi\in\Ker\psi$, $\xi=x+m\Z$ með $x\in\Z$. Að
  $\psi(\xi)= 0$ þýðir að $\psi_i(\xi) = 0$ fyrir öll $i=1,\dots,r$, þ.e.
  \begin{equation*}
    x + m_i\Z
    = m_i \Z
    \qquad\text{þ.e.}\qquad
    x\in m_i\Z
  \end{equation*}
  eða m.ö.o. $m_i\mid x$ fyrir öll $i=1,\dots,r$. Vegna $\ssd(m_j,m_k)=1$ ef
  $j\neq k$ leiðir af því að $m=m_1\cdots m_r \mid x$, þ.e. $x\in m\Z$, og það
  þýðir að $\xi = 0$. Því er $\Ker\psi = \left\{ 0 \right\}$, og það þýðir að
  vörpunin $\psi$ er eintæk. Nú er $\#(\Z/m\Z) = m$ og $\#(\Z/m_1\Z\oplus
  \cdots \oplus \Z/m_r\Z) = m_1\cdots m_r = m$. En eintæk vörpun milli tveggja
  mengja sem hafa jafnmörg stök er gagntæk!
\end{sonnun}
\begin{fylgisetn}
  Látum $m_1,\dots,m_r$ vera eins og í setningu og $x_k \in \left\{
  0,\dots,m_{k} - 1
  \right\}$ fyrir $k=1,\dots,r$. Þá er til nákvæmlega ein tala $x$ í $\left\{
  0,\dots,m-1 \right\}$, þar sem $m:=m_1\cdots m_r$, þannig að 
  \begin{equation*}
    x\bmod m_k = x_k
  \end{equation*}
  fyrir $k = 1,\dots,r$.
\end{fylgisetn}
Varpanirnar $\psi_k$ í kínversku leifasetningunni varðveita \emph{margföldun}:
Við höfum margföldun á $\Z/n\Z,$ $n\ge 1$, þannig að 
\[
(x+n\Z)(y+n\Z) = xy + n\Z.
\]
Þetta samsvarar margföldun $\bmod\;n$. Stökin sem hafa margföldunarumhverfu
mynda grúpu $\mathcal U(n)$.
\begin{setn}
  [Viðbót við kínversku leifasetninguna]
  Vörpunin $\psi$ varpar $\mathcal U(m)$ gagntækt á margfeldið
  \begin{equation*}
    \mathcal U(m_1)\times\cdots\times \mathcal U(m_r).
  \end{equation*}
\end{setn}
\begin{sonnun}
  Ef $x$ hefur margföldunarumhverfu $y$, þá er
  \[
  \psi(xy)
  = (\psi_1(xy),\dots,\psi_r(xy))
  = (\psi_1(x)\psi_1(y),\dots,\psi_r(x)\psi_r(y)).
  \]
  En $\psi(1+m\Z) = (1+m_1\Z,\dots,1+m_r\Z)$, svo að $\psi(y)$ er
  margföldunarumhverfa $\psi(x)$; öfugt, ef
  $\psi(x)=(\psi_1(x),\dots,\psi_r(x))$ hefur margföldunarumhverfu
  $z=(z_1,\dots,z_r)$, þá er til $y$ úr $\Z/m\Z$ þ.a. $\psi(y) = z$, og þá er
  $\psi(xy) = (1+m_1\Z,\dots,1+m_r\Z)$, svo að $xy = 1+m\Z$, þ.e. $x$ hefur
  margföldunarumhverfu $y$.
\end{sonnun}

\subsection{Tenging við $\varphi$-fall Eulers}
\index{Euler!$\varphi$-fall}\index{Euler!totient function}\index{totient function}
Munum að $\varphi(n) = \#\mathcal U(n)$ (með $\varphi$ fyrir $\varphi$-fall Eulers).
\begin{fylgisetn}
  \label{setn:euler_telja}
  Látum $m$ vera náttúrlega tölu $\ge 1$, með frumþáttun $m=p_1^{n_1}\cdots
  p_r^{n_r}$, þar sem $p_1,\dots,p_r$ eru \emph{ólíkar} frumtölur. Þá er
  \begin{equation*}
    \varphi(m) = \varphi(p_1^{n_1})\cdots \varphi(p_r^{n_r}).
  \end{equation*}
\end{fylgisetn}
\begin{ath}
  Stak $k$ í $\left\{ 1,\dots,p^{n}-1 \right\}$, þar sem $p$ er frumtala, er
  ósamþátta $p^n$ þ.þ.a.a. það sé ekki margfeldi af $p$. Margfeldin af
  $p$ í þessu mengi eru $pj$ þar sem $j = 1,\dots,p^{n-1}-1$ og þau eru
  $p^{n-1}-1$ talsins.
  Því er
  \begin{equation*}
    \varphi(p^n) = p^n-p^{n-1} = p^{n-1}(p-1) = p^{n}\left(
    1-\frac{1}{p} \right).
  \end{equation*}
\end{ath}
\begin{fylgisetn}
  Látum $m$ vera eins og í fylgisetningu \ref{setn:euler_telja}. Þá er
  \begin{align*}
    \varphi(m)
    &= (p^{n_1} - p^{n_1-1})\cdots (p^{n_r}-p^{n_r-1}) \\
    &= m\left(1-\frac{1}{p_1}\right)\cdots \left( 1-\frac{1}{p_r} \right).
  \end{align*}
\end{fylgisetn}

%%
%% 14. október 2009
%%

\section{(Innri) bein margfeldi og beinar summur}

\begin{skilgr}
  Látum $H_1,\dots,H_n$ vera hlutgrúpur í grúpu $G$. Við segjum að $G$ sé
  \emph{(innra) beint margfeldi}\index{beint margfeldi}\index{beint margfeldi!innra}
  af hlutgrúpunum
  $H_1,\dots,H_n$ ef vörpunin
  \[
    H_1\times\cdots\times H_n\to G,
    \quad
    (g_1,\dots,g_n)\mapsto g_1\cdots g_n
  \]
  er grúpueinsmótun. Skrifum þá gjarnan
  \[
    G = H_1\times \cdots \times H_n
  \]
  og ef $G$ er \emph{víxlin} þá skrifum við
  \[
    G = H_1\oplus \cdots \oplus H_n
  \]
  og segjum að $G$ sé \emph{(innri) bein summa}\index{bein summa}\index{bein summa!innri}
  af hlutgrúpunum $H_1,\dots,H_n$.
\end{skilgr}
\begin{ath}
  Þetta er ekki sami ritháttur og í bókinni, þar er 
  \[
    H_1\times\cdots\times H_n \cong H_1\oplus\cdots \oplus H_n
  \]
  innri og ytri summa, sem Reyni Axelssyni finnst fráleitt.
\end{ath}
\begin{ath}
  Ef aðgerðin í $G$ er víxlin og skrifuð sem samlagning og $H_1,\dots,H_n$
  eru hlutgrúpur í $G$, þá er vörpunin
  \[
    H_1\oplus \cdots \oplus H_n,
    \quad
    (g_1,\dots,g_n) \mapsto g_1 + \cdots + g_n
  \]
  sjálfkrafa grúpumótun. Það þýðir $G=H_1 \oplus\cdots\oplus H_n$ að sérhvert
  stak $g$ í $G$ megi skrifa með \emph{nákvæmlega einum hætti} sem summu
  $g = g_1+\cdots + g_n$ þar sem $g_k\in H_k$ fyrir öll $k = 1,\dots,n$.
  
  Athugum að ef $G$ er ekki víxlin er ekki sjálfgefið að þetta sé grúpumótun
  því fyrir hlutgrúpur $H,K$ í $G$ fæst ef
  \[
    \varphi: H\times K \to G, 
    \quad
    \varphi(h,k) := hk,
  \]
  að 
  \[
    \varphi((h_1,k_1),(h_2,k_2))
    = \varphi(h_1h_2, k_1k_2)
    = h_1h_2k_1k_2
  \]
  en
  \[
    \varphi(h_1,k_1)\varphi(h_2,k_2) = h_1k_1h_2k_2.
  \]
\end{ath}
\begin{setn}
  Látum $H_1,\dots,H_n$ vera hlutgrúpur í grúpu $G$. Grúpan $G$ er bein innri
  summa af þessum hlutgrúpum $H_1,\dots,H_n$ þ.þ.a.a. eftirfarandi skilyrðum
  sé fullnægt:
  \begin{itemize}
    \item [(i)] Ef $i,j\in \left\{ 1,\dots,n \right\}$, $i\neq j$, þá víxlast
    sérhvert stak úr $H_i$ við sérhvert stak úr $H_j$.
    \item [(ii)] $G = H_1\cdots H_n$, m.ö.o. má skrifa sérhvert stak $g$ úr $G$
    sem margfeldi $g = g_1\cdots g_n$ með $g_k\in H_k$ fyrir $k=1,\dots,n$.
    \item [(iii)] $(H_1\cdots H_{j-1}H_{j+1}\cdots H_n)\cap H_j = \left\{ e
    \right\}$ fyrir öll $j\in \left\{ 1,\dots,n \right\}$.
    
  \end{itemize}
\end{setn}
\begin{sonnun}
  Höfum að (i) jafngildir því að vörpunin
  \[
    \varphi: H_1\times\cdots\times H_n\to G,
    \quad
    (g_1,\dots,g_n)\mapsto g_1\cdots g_n
  \]
  sé grúpumótun: Ef (i) gildir þá er 
  \begin{align*}
    \varphi((g_1,\dots,g_n)(h_1,\dots,h_n))
    &= \varphi(g_1h_1,\dots,g_nh_n)
    \\
    &= g_1h_1\cdots g_nh_n
    \\
    &\overset{\text{\scriptsize(i)}}{=} g_1\cdots g_n \cdot h_1\cdots h_n
    \\
    &= \varphi(g_1,\dots,g_n)\cdot \varphi(h_1,\dots,h_n).
  \end{align*}
  Ef hins vegar (i) gildir ekki og til eru $i,j$, $i< j$, og stök $x$ úr $H_i$ 
  og $y$ úr $H_j$ þannig að $xy \neq yx$ þá er
  \begin{align*}
    & \varphi((e,\dots,e,
    \underset{\underset{i\text{-ta sæti}}{\uparrow}}{y},
    e,\dots,e)
    (e,\dots,e,
    \underset{\underset{j\text{-ta sæti}}{\uparrow}}{x},
    e,\dots,e))
    \\
    &= \varphi(e,\dots,e,x,e,\dots,e,y,e,\dots,e)
    \\
    &= xy
  \end{align*}
  en 
  \begin{align*}
    \varphi(e,\dots,e,y,e,\dots,e)\varphi(e,\dots,x,e,\dots,e)
    = yx
    \neq xy
  \end{align*}
  svo að $\varphi$ er ekki grúpumótun. Ljóst er að (ii) jafngildir því að
  $\varphi$ sé átæk. Skilyrði (iii) segir að $\Ker\varphi = \left\{ e \right\}$.
  Ef $(g_1,\dots,g_n)\in \Ker\varphi$, þ.e. $g_1\cdots g_n = e$, þá er (að
  skilyrði (i) gefnu):
  \begin{equation*}
    g_i^{-1}
    = g_1\cdots g_{i-1}g_{i+1}\cdots g_n
    \in (H_1\cdots H_{i-1}H_{i+1}\cdot H_n)\cap H_i;
  \end{equation*}
  ef (iii) gildir fæst $g_i^{-1}=e$ og þá $g_j = e$ fyrir öll $j=1,\dots,n$.
  Öfugt, ef $\Ker\varphi = \left\{ (e,\dots,e) \right\}$ og 
  $x\in (H_1\cdots H_{i-1}H_{i+1}\cdots H_n)\cap H_i$ þá má skrifa
  $x = g_1\cdots g_{i-1}g_{i+1}\cdots g_n$ með $g_j\in H_j$, $j\neq i$. Þá er
  (að (i) gefnu)
  \begin{align*}
    & \varphi(g_1,\dots,g_{i-1},x^{-1},g_{i+1},\dots,g_n)
    \\
    &= g_1\cdots g_{i-1} x^{-1} g_{i+1} \cdots g_n
    \\
    &= e
  \end{align*}
  og þá
  \[
    (g_1,\cdots,g_{i-1},x^{-1},g_{i+1},g_n)
    = (e,\dots,e),
  \]
  sér í lagi er $x^{-1} = e$ og þá $x = e$. Því er
  \[
    (H_1\cdots H_{j-1} H_{j+1}\cdots H_n)\cap H_j
    = \left\{ e \right\}.
  \]
\end{sonnun}
\begin{ath}
  [Viðvörun!]
  Það er ekki nóg að krefjast í stað skilyrðis (iii) að $H_i\cap H_j = \left\{ e
  \right\}$ fyrir $i\neq j$. Mótdæmi fæst með því að láta $H_1,H_2,H_3$ vera 
  þrjár ólíkar línur gegnum núllpunktinn í $G:=\R^2$: Þá er $H_i+H_j = \R^2$
  ef $i\neq j$ og þá $(H_i+H_j)\cap H_k = H_k$ ef $\left\{ i,j,k \right\}
  = \left\{ 1,2,3 \right\}$.
\end{ath}
Það má þó veikja skilyrði (iii) aðeins:
\begin{setn}
  Í síðustu setningu má í stað skilyrðanna (i) og (iii) setja
  \begin{itemize}
    \item [(i')] Grúpurnar $H_1,\dots,H_n$ eru normlegar í $G$.
    \item [(iii')] $(H_1\cdots H_{j-1})\cap H_j = \left\{ e \right\}$
    fyrir $j=1,\dots,n$
  \end{itemize}
\end{setn}
\begin{sonnun}
  Sönnum með þrepun: Ef $H_1,\dots,H_n$ fullnægja (i') og (iii') þá er 
  $H_1\cdots H_n$ bein summa af $H_1,\dots,H_n$ og normleg hlutgrúpa í $G$. En
  þetta þarf bara að sýna fyrir $n=2$, því að í þrepunarskrefinu þá notum við
  tilvikið $n=2$ á $H_1\cdots H_{j-1}$ og $H_j$. Látum þá $H,K$ vera normlegar
  hlutgrúpur í $G$ þannig að $H\cap K = \left\{ e \right\}$; sýnum að öll stök
  úr $H$ víxlist við öll stök úr $K$: Látum $h\in H, k\in K$. Þá er
  \[
    hkh^{-1}k^{-1} \in H\cap K = \left\{ e \right\}
  \]
  því að $h\in H$ og $kh^{-1}k^{-1}\in H$ af því að $h^{-1}\in H$ og $h$ normleg,
  svo að $hkh^{-1}k^{-1} = h(kh^{-1}k^{-1})\in H$; en líka $hkh^{-1}\in K$ af
  því að $K$ er normleg, svo að $hkh^{-1}k^{-1} = (hkh^{-1})k^{-1}\in K$. En
  þá er $hkh^{-1}k^{-1} = e$ og þá $hk = kh$ Þar með gildir (i) í síðustu
  setningu. Fyrir $n = 2$ eru skilyrði (iii) og (iii') sama skilyrðið.
\end{sonnun}
\begin{ath}
  Höfum nú sýnt að okkar skilgreining á að grúpa sé innra margfeldi af 
  hlutgrúpum sé jafngild skilgreiningunni í kennslubókinni.
\end{ath}


\chapter{Flokkunarsetning fyrir endanlegar víxlgrúpur}
% Meginsetning

\begin{setn}
  Sérhver endanleg víxlgrúpa er einsmóta grúpu af gerðinni
  \[
    \Z/p_1^{r_1}\Z \oplus \cdots \oplus \Z/p_n^{r_n}\Z
  \]
  þar sem $p_1,\dots,p_n$ eru frumtölur og $r_j \ge 1$ fyrir $j = 1,\dots,n$.
  Talnarunan $p_1^{r_1},\dots,p_n^{r_n}$ ákvarðast ótvírætt burtséð frá röð
  (sérhver víxlgrúpa er bein summa af rásuðum víxlgrúpum, gætum svo notað
  kínversku leifasetninguna á það)
\end{setn}
Við sönnum þessa setningu með röð af hjálparsetningum. Skrifum víxlgrúpu sem
samlagningargrúpu, svo að við skrifum $nx$ í stað $x^n$ og $\Z x$ í stað
$\langle x \rangle$ (hlutgrúpan sem er spönnuð af $x$), þ.e.
\[
  \Z x = \left\{ nx:n\in \Z \right\}.
\]
\begin{hjalparsetn}
  Látum $\varphi: G\to H$ vera grúpumótun og $x$ vera stak í $G$ með endanlegt
  raðstig, þá hefur $\varphi(x)$ endanlegt raðstig sem gengur upp í raðstig $x$.
\end{hjalparsetn}
\begin{sonnun}
  Athugum að $G$ og $H$ þurfa ekki að vera víxlnar, svo við skrifum aðgerðir sem
  margföldun. Látum $\psi:\Z \to G$, $n\mapsto x^n$; þá er raðstig $x$ talan
  $j$ þ.a. $\Ker\psi = \Z j$. Nú er
  \[
    \varphi\psi(n)= \varphi(x^n) = (\varphi(x))^n
  \]
  svo að raðstig $\varphi(x)$ er talan $k$ þ.a. $\Ker(\varphi\circ\psi) = \Z k$.
  En ljóst er að $\Ker\psi \subset \Ker(\varphi\circ\psi)$, svo að 
  $\Z j \subset \Z k$, þ.e. $k \mid j$.
\end{sonnun}
%%
%% 15. október 2009
%%
\begin{hjalparsetn}
  Látum $G$ vera endanlega víxlgrúpu með fjöldatölu $n$ og $p$ vera frumtölu sem
  gengur upp í $n$, þá hefur $G$ stak af raðstigi $p$.
\end{hjalparsetn}
\begin{sonnun}
  Notum þrepun yfir $n$. Ljóst ef $n = 1$. Gerum ráð fyrir að $n > 1$. Látum
  $x\in G$, $x\neq 0$, setjum $H:= \Z x$, sem er rásuð; látum $q$ vera frumþátt
  í $\# H$. Þá hefur $H$ stak $y$ af raðstigi $q$. Ef $q = p$, þá þarf ekki að gera
  meira. Annars setjum við
  \[
    G' := G/\Z y,
  \]
  sem hefur fjöldatölu $\frac 1q \#G < \#G$. Skv. þrepunarforsendu hefur
  $G'$ stak af raðstigi $p$, því að $p \mid \frac 1q \#G$. Skrifum það $z+\Z y$. Þá
  er $pz\in\Z y$, svo að $pqz = 0$ í $G$. Nú er raðstig $z$ margfeldi af
  $p$ og því $p$ eða $pq$. Ef það er $p$, þá erum við búin; annars hefur
  $qz$ raðstig $p$.
\end{sonnun}
\begin{hjalparsetn}
  Látum $G$ vera víxlgrúpu með fjöldatölu $mn$, þar sem $\ssd(m,n) = 1$. Setjum
  \begin{equation*}
  H := \left\{ x\in G: mx = 0 \right\},
  \quad
  K := \left\{ x\in G: nx = 0 \right\}.
  \end{equation*}
  Þá er $G = H\oplus K$ og $\#H = m$, $\#K = n$.
\end{hjalparsetn}
\begin{sonnun}
  Til eru $j,k$ úr $\Z$ sem leysa Bézout-jöfnuna $1 = jn + km$. Fyrir $x$ úr
  $G$ er þá
  \begin{equation*}
  x = jnx + kmx \in H+K,
  \end{equation*}
  því að $nx\in H$ og $mx\in K$ vegna $mnx = 0$. Þar með er $H+K = G$. Ef
  $x\in H\cap K$, þá er $mx = nx = 0$, svo að raðstig $x$ gengur upp í $m$ og
  $n$ og er þá $1$ vegna $\ssd(m,n)=1$; svo að $x = 0$. Því er $H\cap K =
  \left\{ 0 \right\}$; og við höfum $G=H\oplus K$.

  Þá er $\#H\cdot \#K = \#G = mn$. Ef $p $ er frumtala sem gengur upp í
  $\#K$, þá er til stak $x$ úr $K$ með raðstig $p$, svo að $p\mid n$ og þá p
  $p\nmid m$. Því hefur $\#K$ engan frumþátt úr $m$. Eins hefur $\#H$ engan
  frumþátt úr $n$, svo að $\#H = m$ og $\#K = n$.
\end{sonnun}
\emph{Afleiðing} (með þrepun): Ef $G$ er víxlgrúpa með fjöldatölu $n$ og
$n = p_1^{\gamma_1}\cdots p_s^{\gamma_s}$ er frumþáttun $n$, þar sem
$p_1,\dots,p_s$ eru ólíkar frumtölur, þá er $G = G_1\oplus \cdots \oplus G_s$,
þ.a. $\#G_j = p_j^{\gamma_j}$ fyrir $j = 1,\dots,s$.
\begin{hjalparsetn}
  Ef $p$ er frumtala og $G$ er víxlgrúpa með fjöldatölu $p^n$, þá er $G$
  annaðhvort rásuð eða $G = H\oplus K$ þar sem $H$ er rásuð hlutgrúpa í
  $G$ af sem stærstri fjöldatölu.
\end{hjalparsetn}
\begin{sonnun}
  Þrepun yfir $n$. Látum $p^{m}$ vera hæsta raðstig staks í $G$, $H$ vera
  hlutgrúpu spannaða af slíku staki. Við höfum þá $p^{m}x=0$ fyrir öll $x$ úr
  $G$.

  \emph{Fyrra tilvik}. Ekkert stak úr $G\setminus H$ hefur raðstig $p$. Athugum
  þá grúpumótunina 
  \begin{equation*}
  \varphi_1:G\to G,\quad x\mapsto px.
  \end{equation*}
  Þá er $\Ker\varphi_1$ mengi allra staka í $G$ af raðstigi $1$ eða $p$, svo að
  $\Ker\varphi_1 \subset H$. En $H$ er rásuð grúpa með fjöldatölu $p^m$ og hefur
  nákvæmlega $p$ stök þ.a. $px = 0$. Þau mynda hlutgrúpuna $H_1 := 
  p^{m-1}H = \left\{ p^{m-1}x:x\in H \right\}$. Þá er $G/H_1\cong pG$, svo að 
  $\left( G:pG \right) = \#H_1 = p$. Ef $m = 1$, þá er $pG = \left\{ 0 \right\}$
  og $G = H$. Annars athugum við grúpumótunina 
  \begin{equation*}
  \varphi_2 : pG \to pG,\quad x\mapsto px.
  \end{equation*}
  Höfum aftur $\Ker\phi_2 \subset H$ og þá $\Ker\phi_2 \subset H_1$; en
  $H_1 = p^{m-1} H\subset p^{m-1} G \subset pG$, svo að $\Ker\phi_2 = H_1$. Því
  fæst $pG/H_1 \cong p^{2} G$, og $\left( pG:p^{2}G \right) = \#H_1 = p$. Ef
  $m = 2$, þá er $p^2 G = \left\{ 0 \right\}$ og $\#G = p^2$, svo að $G = H$. 
  Annars athugum við
  \begin{equation*}
  \varphi_3: p^{2}G\to p^2 G, \quad x\mapsto px;
  \end{equation*}
  o.s.frv. Þetta endar á að við fáum $p^{m-1}G = \left\{ 0 \right\}$ og
  $G = H$.

  \emph{Seinna tilvik}. Til er stak úr $G\setminus H$ sem hefur raðstig
  $p$. Það spannar hlutgrúpu $L$ í $G$ þ.a. $L\cap H = \left\{ 0 \right\}$, því
  að sniðmengið $L\cap H$ er eiginleg hlutgrúpa í $L$ og $\#L = p$. Athugum
  ofanvarpið $\pi:G\to G/L$. Það varpar $H$ gagntækt á rásaða hlutgrúpu
  $\hat{H}$ með stærðstu fjöldatölu í $G/L$. Skv. þrepunarforsendu er til
  hlutgrúpa $\hat{K}$ í $G/L$ þ.a. $G/L = \hat{H}\oplus\hat{K}$. Setjum
  $K := \pi^{-1}[ \hat K ]$ og sýnum að $G = H\oplus K$: Látum
  $x\in G$, skrifum $\pi(x) = \hat h + \hat k$ með $\hat h \in \hat H$ og
  $\hat k \in \hat K$; látum $h$ vera stak í $H$ þ.a. $\pi(h) = \hat h$ og
  setjum $k := x-h$; þá er $\pi(k) = \pi(x) - \hat h = \hat k \in \hat K$, svo
  að $k\in K$ og $x = h+k$; við höfum því að $G = H + K$. Látum $x\in H\cap K$,
  þá er $\pi(x)\in \hat H \cap \hat K = \left\{ 0 \right\}$, svo að $x\in L$ og
  þá $x\in H\cap L = \left\{ 0 \right\}$. Því er $H\cap K = \left\{ 0 \right\}$;
  við höfum því að $G = H\oplus K$.
\end{sonnun}
\begin{fylgisetn}
  Endanleg víxlgrúpa er einsmóta beinni summu af rásuðum grúpum sem hafa
  fjöldatölu sem er veldi af frumtölu.
\end{fylgisetn}
\begin{hjalparsetn}
  Ef $G = H_1\oplus\cdots\oplus H_r = K_1\oplus\cdots\oplus K_s$ þar sem
  $H_j,K_k$ eru rásaðar grúpur með fjöldatölur sem eru (jákvæð) veldi af
  frumtölu $p$, þá er $r = s$ og eftir umröðun er $H_j\cong K_j$ fyrir
  $j = 1,\dots,r$.
\end{hjalparsetn}
\begin{sonnun}
  Athugum $\varphi:G\to G, x\mapsto px$; setjum $G[p]:=\Ker\varphi$. Þá er ljóst
  að 
  \begin{equation*}
  G[p] = H_1[p]\oplus\cdots\oplus H_r[p]\cong K_1[p]\oplus\cdots\oplus K_s[p]
  \end{equation*}
  og $H_j\cong \Z/p\Z$; svo að 
  \[
  G 
  \cong
  \left( \Z/p\Z \right)^r
  \cong
  \left( \Z/p\Z \right)^s,
  \]
  svo að $r = s$. Einnig er 
  \begin{equation*}
  G/G[p]
  \cong pG
  \cong pH_1\oplus\cdots\oplus pH_r
  \cong pK_1\oplus\cdots\oplus pK_r.
  \end{equation*}
  En $\#(pH_j) = \frac{1}{p}\#H_j$ og því hefur $G/G[p]$ færri stök en
  $G$. Getum nú notað þrepun!
\end{sonnun}
%%
%% 21. október 2009
%%
\begin{skilgr}
  \emph{Einföld grúpa}\index{einföld grúpa}\index{grúpa!einföld} er grúpa sem
  hefur enga normlega hlutgrúpu.
\end{skilgr}

\part{Baugar}
\chapter{Baugar}
\section{Baugar, baugamótanir}
\begin{skilgr}
  \emph{Baugur}\index{baugur} er mengi $R$ ásamt tveimur reikniaðgerðum á $R$,
  samlagningu og margföldun, þannig að gildi:
  \begin{enumerate}[(i)]
    \item Mengið $R$ er víxlgrúpa m.t.t. samlagningar.
    \item Margföldunin er tengin.
    \item Um samlagningu og margföldun gilda
      \emph{dreifireglur}\index{dreifireglur}:
    \begin{description}
      \item[Vinstri dreifireglan] $a(b+c) = ab + ac$ fyrir öll $a,b,c\in R$.
      \item[Hægri dreifireglan] $(a+b)c = ac + bc$ fyrir öll $a,b,c\in R$.
    \end{description}
  \end{enumerate}
  Ef margföldunin hefur hlutleysu, þá kallast hún
  \emph{einingarstak}\index{einingarstak} baugsins $R$ og er táknuð $1_R$ eða
  $1$; og baugurinn kallast
  \emph{einbaugur}\index{einbaugur}\index{baugur!einbaugur}. Ef margföldunin
  í baugnum er víxlin, þá kallast baugurinn \emph{víxlinn} eða
  \emph{víxlbaugur}\index{víxlbaugur}\index{baugur!víxlbaugur}.
\end{skilgr}
\begin{ath}
  Nú til dags er venja að skilgreina ávallt bauga sem einbauga.
\end{ath}
\begin{skilgr}
  Látum $S,R$ vera bauga. \emph{Baugamótun}\index{baugamótun} $\varphi:R\to S$
  er vörpun þ.a.
  \[
    \varphi(a+b) = \varphi(a) + \varphi(b)
  \]
  og
  \[
    \varphi(ab) = \varphi(a) + \varphi(b)
  \]
  fyrir öll $a,b$ úr $R$. Ef $R,S$ eru einbaugar, þá er \emph{einbaugamótun}
  \index{einbaugamótun} $\varphi:R\to S$ baugamótun þ.a. $\varphi(1_R)=1_S$.
  Gagntæk baugamótun kallast \emph{baugaeinsmótun}\index{baugaeinsmótun}.
\end{skilgr}
\begin{daemi}
  (1) Heilu tölurnar $\Z$, ræðu tölurnar $\Q$, rauntölurnar $\R$ og
  tvinntölurnar $\C$ eru víxlnir einbaugar m.t.t. venjulegu reikniaðgerðanna.
  Ívörpin $\Z\hookrightarrow\Q,x\mapsto x$, $\Z\hookrightarrow\R$,
  $\Z\hookrightarrow\C$, $\Q\hookrightarrow\R$, $\Q\hookrightarrow\C$,
  $\R\hookrightarrow\C$ eru einbaugamótanir. Náttúrlegu tölurnar $\N$ mynda ekki
  baug (þær eru ekki víxlgrúpa m.t.t. samlagningar).
  
  (2) Ef $R$ er baugur og $X$ er mengi, þá má gera mengið $R^X$ af öllum
  vörpunum $f:X\to R$ að baug með því að setja
  \begin{align*}
      (f+g)(x) &:= f(x) + g(x)
      \\
      (fg)(x)  &:= f(x)g(x)
  \end{align*}
  fyrir öll $x$ úr $X$. Ef $R$ er einbaugur, þá verður $R^X$ einbaugur,
  eingarstakið er fastavörpun með gildið $1_R$. Sértilvik eru mengi allra
  raunfalla $\R^X$ á $X$ og allra tvinnfalla $\C^X$ á $X$.
  
  (3) Á grúpunum $\Z/m\Z$ er vel skilgreind margföldun þannig að
  \[
    [x]_m\cdot [y]_m = [xy]_m
  \]
  fyrir öll $x,y\in\Z$; hér höfum við skrifað $[x]_m := x+ m\Z$; þetta gerir
  $\Z/m\Z$ að einbaug með einingarstakið $[1]_m$.
  
  (4) Látum $R$ vera baug; fjölskylda 
  $(a_{jk})_{ \left\{ 1,\dots,m \right\}\times\left\{ 1,\dots,n \right\} }$
  kallast $m\times n$-fylki yfir $R$ ef $a_{jk}\in R$ fyrir öll $j,k$. Fyrir
  $m\times n$-fylki $A =(a_{jk})$ og $n\times p$-fylki $B = (b_{kl})$ 
  skilgreinum við fylkið $AB$ sem $m\times p$-fylkið $C=(c_{jl})$ þar sem
  \[
    c_{jl} := \sum_{k=1}^n a_{jk} b_{kl}.
  \]
  Táknum með $R^{m\times n}$ mengi $m\times n$-fylkja yfir $R$. Sjáum: Ef
  $m\in \N$, þá gerir þessi margföldun, ásamt samlagningunni
  \[
    A+B := (a_{jk} + b_{jk})
  \]
  mengið $R^{m\times m}$ að baug; ef $R$ er einingarbaugur, þá er 
  $R^{m\times m}$ einbaugur; einingarstak hans er $m\times m$-einingarfylkið
  $I := (\delta_{jk})$, þar sem $\delta_{jk}$ er \emph{Kronecker-táknið}
  \index{Kronecker-táknið} í $R$,
  \[
    \delta_{jk} = \begin{cases}
      1_R & \text{ef } j = k \\
      0_R & \text{ef } j\neq k.
    \end{cases}
  \]
  Þótt $R$ sé víxlinn þarf $R^{m\times m}$ ekki að vera víxlinn ef $m\geq 2$;
  þetta er sér í tilvikið fyrir $\R^{m\times m}$, baug $m\times m$-raunfylkja,
  sem er óvíxlinn baugur ef $m\geq 2$.
\end{daemi}

\section{Reiknireglur}
\begin{setn}
  Látum $R$ vera baug með núllstaki $0_R$.
  \begin{enumerate}[(1)]
    \item Fyrir öll $a$ úr $R$ er
    \[
      0_Ra = a0_R = 0_R.
    \]
    \item (Formerkjareglur)\index{formerkjareglur} Fyrir öll $a,b$ úr $R$ er
    \[
      (-a)b = a(-b) = -ab,
    \]\[
      (-a)(-b) = ab.
    \]
  \end{enumerate}
\end{setn}
\begin{sonnun}
  Þetta er afleiðing af dreifireglunni.
  
  (1) Höfum
  \begin{align*}
    0_R + 0_R a 
    &= 0_R a
    \\
    &= (0_R + 0_R)a
    \\
    &= 0_R a + 0_R a, \tag{dreifiregla}
  \end{align*}
  styttireglan í samlagningargrúpunni gefur $0_R=0_R a$. Eins fæst
  $a0_R = 0_R$ með hinni dreifireglunni (ath. að ef baugurinn er víxlinn
  eru dreifireglurnar jafngildar, en ef ekki þá þurfum við á þeim báðum að
  halda).
  
  (2) Höfum
  \begin{align*}
    ab + (-a)b 
    &= (a+(-a))b \tag{dreifiregla}
    \\
    &= 0_R b
    \\
    &= 0_R,
  \end{align*}
  svo að $(-a)b$ er samlagningarumhverfa $ab$, þ.e. $(-a)b = -ab$. Eins fæst
  að $a(-b) = -ab$ með hinni dreifireglunni. En þá er
  \[
    (-a)(-b) = -a(-b) = -(-ab) = ab.
  \]
\end{sonnun}
\begin{setn}
  [Almenn dreifiregla]\index{almenn dreifiregla}
  Ef $a_1,\dots,a_m,b_1,\dots,b_n\in R$, $R$ er baugur, þá er 
  \[
    \left( \sum_{j=1}^m a_j \right) \left( \sum_{k=1}^{m} \right)
    =
    \sum_{i=1}^{m} \sum_{k=1}^{n} a_j b_k.
  \]
\end{setn}
\begin{sonnun}
  Fæst með þrepun, fyrst yfir $n$, svo yfir $m$.
\end{sonnun}
\begin{setn}
  Látum $R_1,\dots,R_n$ vera bauga; samlagningargrúpan
  \[
    \prod_{k=1}^{n} R_k = R_1\times \cdots \times R_n
  \]
  verður að baug með margfölduninni
  \[
    (a_1,\dots,a_n)(b_1,\dots,b_n)
    := (a_1 b_1,\dots,a_n b_n).
  \]
  Ofanvörpin
  \[
    \pr_k : \prod_{k=1}^{n} R_k \to R_k, 
    (a_1,\dots,a_n) \mapsto a_k
  \]
  verða baugamótanir. Ef $R_1,\dots,R_n$ eru einbaugar, þá er
  $\prod_{k=1}^{n}R_k$ einbaugur með einingarstakið
  $(1_{R_1},1_{R_2},\dots,1_{R_n})$ og ofanvörpin verða baugamótanir.
\end{setn}
\begin{sonnun}
  Augljóst (þarf að fara í gegnum að dreifireglurnar gildi, en það er einfalt).
\end{sonnun}
\begin{setn}
  [Viðbót við kínversku leifasetninguna]
  Látum $m_1,\dots,m_r$ vera náttúrlegar tölur þ.a. $m_k\geq 1$ og
  $\ssd(m_j,m_k)=1$ ef $j\neq k$. Samlagningargrúpueinsmótunin
  \[
    \Z/m_1\cdots m_r\Z\to \Z/m_1\times\cdots\times \Z/m_r\Z,
    [x]_{m_1\cdots m_r} \mapsto ([x]_{m_1},\dots,[x]_{m_r})
  \]
  er í raun \emph{einbaugaeinsmótun}.
\end{setn}
\begin{sonnun}
  Augljóst! (Það var augljóst að hún er baugamótun en var ekki augljóst að hún
  væri gagntæk, en við erum búin að sanna það).
\end{sonnun}
\begin{ath}
  Í kennslubók er þess krafist að í einbaug $R$ sé $1_R\neq 0_R$; þetta er ekki 
  sniðugt.
  
  Athugum hins vegar: Ef $1_R = 0_R$ og $a \in R$, þá er
  \[
    a = 1_R a = 0_R a = 0_R,
  \]
  svo að $R = \left\{ 0_R \right\}$; m.ö.o. hefur $R$ bara eitt stak, sem er 
  bæði núllstak og einingarstak! Slíkur baugur kallast
  \emph{núllbaugur}\index{núllbaugur}\index{baugur!núllbaugur}.
  Við teljum núllbaug vera einbaug en bókin ekki.
\end{ath}

\section{Deilibaugar, heilbaugar og svið}
\begin{skilgr}
  Látum $R$ vera einbaug. Stak $a$ í $R$ kallast \emph{eind}\index{eind} ef það
  hefur margföldunarumhverfu. Skv. gamalli setningu (um hálfgrúpur með 
  hlutleysu) mynda eindirnar grúpu m.t.t. margföldunar; við köllum hana
  \emph{eindagrúpu}\index{eindagrúpa}\index{baugur!eindagrúpa}\index{grúpa!eindagrúpa baugs}
  (eða bara \emph{margföldunargrúpu})\index{margföldunargrúpa}
  baugsins og táknum hana með 
  \[
    R^*
  \]
\end{skilgr}
\begin{daemi}
  $\R^* = \R\setminus\left\{ 0 \right\}$, $\Q^* = \Q\setminus\left\{ 0 \right\}$,
  $\C^* = \C\setminus\left\{ 0 \right\}$, en $\Z^* = \left\{ 1,-1 \right\}$.
  Athugum að grúpan $(\Z/m\Z)^*$ er einsmóta $\mathcal U(m)$, sem var grúpa
  staka úr $\left\{ 0,\dots,m-1 \right\}$ þ.a. $\ssd(k,m) = 1$ með margföldun
  $\bmod\;m$.
\end{daemi}
%%
%% 22. október 2009
%%
\begin{skilgr}
  Við segjum að baugur sé
  \emph{deilibaugur}\index{deilibaugur}\index{baugur!deilibaugur} ef hann er
  einbaugur og $R^* = R\setminus\left\{ 0 \right\}$; þetta þýðir að $1_R\neq0_R$
  og sérhvert stak nema núllstakið hafi margföldunarumhverfu. Víxlinn 
  deilibaugur kallast \emph{svið}\index{svið} (\sout{sumir segja
  \emph{kroppur}}).
\end{skilgr}
\begin{ath}
  Ef $1_R\neq 0_R$ í baug $R$, þá er $0_R\cdot a = 0_R\neq 1_R$ fyrir öll
  $a$ úr $R$ þannig að $0_R$ getur alls ekki haft margföldunarumhverfu.
\end{ath}
\begin{daemi}
  $\Q,\R,\C$ eru svið, en $\Z$ ekki. Eitt einfaldasta dæmið um deilibaug sem er
  ekki víxlinn er \emph{fertölur Hamiltons}\index{Hamilton} (sjá dæmablað).
\end{daemi}
\begin{skilgr}
  Látum $R$ vera baug.
  \emph{Hlutbaugur}\index{hlutbaugur}\index{baugur!hlutbaugur} í $R$ er hlutmengi
  $S$ í $R$ sem er lokað m.t.t. bæði samlagningar og margföldunar og verður 
  baugur m.t.t. aðgerðanna sem þannig fást á $S$.
  
  \emph{Hluteinbaugur}\index{hluteinbaugur} í einbaug $R$ er hlutbaugur
  $S$ í $R$ sem er einbaugur og þannig að $1_S = 1_R$.
\end{skilgr}
\begin{ath}
  $\left\{ 0 \right\}$ er hlutbaugur í $\Z$ og einbaugur, en ekki hluteinbaugur
  í $\Z$; annað dæmi: $\left\{ 0 \right\}\times\Z$ er hlutbaugur í $\Z\times\Z$
  og einbaugur, \emph{líka} í skilningi bókarinnar, með hlutleysu $(0,1)$; en 
  ekki hluteinbaugur í $\Z\times\Z$, því að hlutleysan í $\Z\times\Z$ er 
  $(1,1)$.
\end{ath}
\begin{skilgr}
  Látum $R$ vera baug. 
  \begin{enumerate}[(1)]
    \item Við segjum að stak $a$ í $R$ sé \emph{núlldeilir frá vinstri}
      \index{núlldeilir!frá vinstri} ef til er stak $b$ í $R$ þ.a. $b\neq 0$
      en $ab = 0$.
    \item Við segjum að stak $a$ í $R$ sé \emph{núlldeilir frá hægri}
      \index{núlldeilir!frá hægri} ef til er stak $b$ í $R$ þ.a. $b\neq R$
      en $ba = 0$.
    \item Við segjum að stak $a$ í $R$ sé \emph{styttanlegt frá vinstri}
      \index{styttanlegt!frá vinstri} ef $ab = ac$ leiðir til $b=c$ fyrir
      öll $b,c$ úr $R$.
    \item Við segjum að stak $a$ í $R$ sé \emph{styttanlegt frá hægri}
      \index{styttanlegt!frá hægri} ef $ba = ca$ leiðir til $b=c$ fyrir öll 
      $b,c$ úr $R$.
  \end{enumerate}
  Ef baugurinn er \emph{víxlinn}, þá er enginn munur á núlldeilum frá hægri
  og frá vinstri, og við tölum einfaldlega um \emph{núlldeila}
  \index{núlldeilir}; eins er stak styttanlegt frá vinstri þ.þ.a.a. það sé 
  styttanlegt frá hægri; og við köllum það einfaldlega \emph{styttanlegt}
  \index{styttanlegt}.
\end{skilgr}
\begin{ath}
  Skv. þessari skilgreiningu er núllstakið núlldeilir þ.þ.a.a. $R$ sé ekki
  núllbaugur (skv. bók telst $0$ ekki vera núlldeilir).
\end{ath}
\begin{setn}
  Stak $a$ í baug er styttanlegt frá vinstri þ.þ.a.a. það sé ekki núlldeilir
  frá vinstri; það er styttanlegt frá hægri þ.þ.a.a. það sé ekki núlldeilir
  frá hægri.
\end{setn}
\begin{sonnun}
  Sönnum fyrri fullyrðinguna. Ef $a$ er styttanlegt frá vinstri og $b$ er stak
  þ.a. $ab = 0$, þá er $ab = 0 = a\cdot 0$ og þá $b = 0$. Því er $a$ ekki
  núlldeilir frá vinstri. Ef hins vegar $a$ er ekki núlldeilir frá vinstri og
  $ab = ac$, þá er $a(b-c) = ab - ac = 0$ og þar sem $a$ er ekki núlldeilir
  frá vinstri er $b - c = 0$ og því $b = c$. 
\end{sonnun}
\begin{ath} 
  \emph{Eind} er alltaf styttanleg, bæði frá hægri og frá vinstri. Í deilibaug
  er núllstakið því einni núlldeilirinn (hvort sem er frá hægri eða frá
  vinstri).
\end{ath}
\begin{skilgr}
  \emph{Heilbaugur}\index{heilbaugur}\index{baugur!heilbaugur} er víxlinn
  einbaugur sem er ekki núllbaugur þannig að öll stök nema núllstakið séu
  styttanleg, þ.e. núllstakið er eini núlldeilirinn.
\end{skilgr}
\begin{daemi}
  (1) Sérhvert svið er heilbaugur.
  
  (2) Baugur heilu talnanna $\Z$ er heilbaugur.
\end{daemi}
Látum $R$ vera baug, $a\in R$ og athugum vörpunina $v_a:R\to R, v_a(x):=ax$.
Að $a$ sé styttanlegt \emph{frá vinstri} þýðir að vörpunin $v_a$ sé
\emph{eintæk}. Ef $R$ er \emph{víxlinn} einbaugur og vörpunin $v_a$ er
\emph{átæk}, þá er til stak $b$ í $R$ þannig að $v_a(b) = 1$, þ.e. $ab = 1$,
það þýðir að $a$ sé eind, þ.e. hafi margföldunarumhverfu.

Ef nú $R$ er \emph{endanlegt} mengi, þá er vörpun $R\to R$ átæk
þ.þ.a.a. hún sé eintæk . Fáum:
\begin{setn}
  Endanlegur heilbaugur er svið.
\end{setn}
\begin{daemi}
  Látum $m\geq 1$. Fyrir $m = 1$ er $\Z/m\Z = \Z/\Z = \left\{ 0 \right\}$
  núllbaugur, sem er einbaugur en ekki heilbaugur. Fyrir $m\geq 2$ er
  $\Z/m\Z$ heilbaugur þ.þ.a.a. ekki séu til stök í baugnum sem eru ekki núll 
  en margfeldi þeirra sé núll; það þýðir að fyrir öll $x,y\in \Z$ þ.a.
  $m \mid xy$ gildi $m\mid x$ eða $m\mid y$. Og þetta þýðir að $m$ sé frumtala. Sjáum:
  $\Z/m\Z$ er heilbaugur þ.þ.a.a. $m$ sé frumtala, og þá er $\Z/m\Z$
  \emph{svið}.
\end{daemi}
\begin{ath}
  Að $R$ sé heilbaugur þýðir að $R$ sé víxlinn einbaugur þ.a. $1_R\neq 0_R$ og
  eftirfarandi skilyrði sé fullnægt:
  \begin{quote}
    Ef $a,b\in R$ og $ab = 0$, þá er $a = 0$ eða $b = 0$.
  \end{quote}
\end{ath}
\begin{skilgr}
  \emph{Kennitala baugs}\index{kennitala}\index{baugur!kennitala} $R$ er minnsta
  jákvæða heila talan $n$ þannig að $na = 0$ fyrir öll $a$ úr $R$ \emph{ef} slík
  tala er til; annars er kennitalan $0$. Kennitalan er táknuð með 
  \[
    \kennitala R.
  \]
\end{skilgr}
Fyrir einbauga er þetta einfaldara:
\begin{setn}
  Látum $R$ vera einbaug. Þá er til nákvæmlega ein einbaugamótun 
  $\varphi: \Z \to R$; hún er gefin með 
  \[
    \varphi(n) = n\cdot 1_R.
  \]
  Látum $\varphi$ vera þessa einbaugamótun. Þá er $\kennitala R$ talan $n$ þ.a.
  \[
    \Ker\varphi = n\Z.
  \]
  M.ö.o. er $\kennitala R$ raðstig $1_R$ í samlagningargrúpunni $R$ ef það er
  endanlegt, en $0$ ef það er óendanlegt.
\end{setn}
\begin{sonnun}
  Athugum: Ef $n\cdot 1_R = 0$, þá er
  \[
    na 
    = n\cdot (1_R a) 
    = (n1_R)a 
    = 0a
    = 0
  \]
  fyrir öll $a$ úr $R$. Öfugt, ef $na = 0$ fyrir öll $a$ úr $R$, þá er sér í
  lagi $n 1_R = 0$.
\end{sonnun}

%%
%% 28. október 2009
%%

\chapter{Íðul}
\index{idal@íðal}
\begin{skilgr}
  Látum $R$ vera baug. Hlutmengi $\mathfrak A$ í $R$ kallast
  \begin{enumerate} [(i)]
    \item \emph{Íðal frá vinstri (vinstra íðal)}\index{idal@íðal!frá vinstri} ef það
      er hlutgrúpa í samlagningargrúpu $R$ og $ra \in\mathfrak A$ fyrir öll
      $r\in R$ og $a\in\mathfrak A$.
    \item \emph{Íðal frá hægri (hægra íðal)}\index{idal@íðal!frá hægri} ef það er
      hlutgrúpa í samlagningargrúpunni $R$ og $ar\in\mathfrak A$ fyrir öll
      $r\in R$ og $a\in\mathfrak A$.
    \item \emph{Tvíhliða íðal}\index{idal@íðal!tvíhliða} ef það er íðal frá vinstri
      og frá hægri.
  \end{enumerate}
\end{skilgr}
\begin{ath}
  Að $\mathfrak A$ sé vinstra íðal í einbaug $R$ er jafngilt því að gildi:
  \begin{enumerate}[(i)]
    \item $0\in\mathfrak A$.
    \item Ef $a,b\in\mathfrak A$, þá er $a+b\in\mathfrak A$.
    \item Ef $r\in R$ og $a\in\mathfrak A$ þá er $ra\in\mathfrak A$.
  \end{enumerate}
  Ástæðan er að af (iii) leiðir að fyrir öll $a\in\mathfrak A$ er $-a =
  (-1)a\in\mathfrak A$.
\end{ath}

\section{Deildabaugar}
\begin{setn}
  Látum $R$ vera baug og $\mathfrak A$ vera hlutgrúpu í samlagningargrúpunni
  $R$. Þá er jafngilt:
  \begin{enumerate}[(i)]
    \item $\mathfrak A$ er tvíhliða íðal í $R$.
    \item Skilgreina má margföldun á deildagrúpunni $R/\mathfrak A$ þannig að\[
      (a + \mathfrak A)\cdot (b+\mathfrak A) = ab + \mathfrak A
      \]
      fyrir öll $a,b\in R$.
  \end{enumerate}
  Ef þessum skilyrðum er fullnægt, þá verður $R/\mathfrak A$ að baug með þessari
  margföldun úr (ii), og $R/\mathfrak A$ verður einbaugur ef $R$ er einbaugur.
\end{setn}
\begin{sonnun}
  (i)$\Rightarrow$(ii). Sýna þarf: Ef $\mathfrak A$ er tvíhliða íðal og
  $a + \mathfrak A = c + \mathfrak A$ og $b + \mathfrak A = d + \mathfrak A$, þá
  er \[
  ab + \mathfrak A = cd + \mathfrak A
  \]
  því þá er þetta vel skilgreind margföldun. En $a + \mathfrak A = c + \mathfrak
  A$ þýðir að $a - c \in\mathfrak A$, og $b + \mathfrak A = c + \mathfrak A$
  þýðir að $(b-d)\in \mathfrak A$. En þá er\[
  ad - cd 
  = \underbrace{(a-c)}_{\in \mathfrak A}\underbrace{b}_{\in R}
  + \underbrace{c}_{\in R}\underbrace{(b-d)}_{\in\mathfrak A}
  \in \mathfrak A
  \]
  af því að $\mathfrak A$ er \emph{tvíhliða íðal}. Þetta sýnir að margföldunin í
  (ii) er vel skilgreind. Gerum nú ráð fyrir að (ii) gildi. Sýnum fyrst að
  $R/\mathfrak A$ verður að baug: Vitum að $R/\mathfrak A$ jer víxlgrúpa með
  tilliti til samlagningar. Nú fæst
  \begin{align*}
    (a+\mathfrak A)( (b+\mathfrak A)(c+ \mathfrak A))
    &= (a+\mathfrak A)(bc+\mathfrak A)
    \\
    &= a(bc) + \mathfrak A
    \\
    &= (ab)c + \mathfrak A
    \\
    &= (ab + \mathfrak A)(c+\mathfrak A)
    \\
    &= ( (a+\mathfrak A)(b+\mathfrak A))(c+\mathfrak A)
  \end{align*}
  svo að tengireglan gildir. Einnig er
  \begin{align*}
    (a+\mathfrak A)( (b + \mathfrak A)+(c+\mathfrak A))
    &= (a+\mathfrak A)( (b+c)+\mathfrak A)
    \\
    &= a(b+c)+\mathfrak A
    \\
    &= (ab + ac) + \mathfrak A
    \\
    &= (ab + \mathfrak A) + (ac + \mathfrak A)
    \\
    &= (a+\mathfrak A)(b + \mathfrak A) + (a+\mathfrak A)(c + \mathfrak A)
  \end{align*}
  svo að vinstri dreifireglan gildir. Eins sést að hægri dreifireglan gildir. Ef
  $R$ er einbaugur, þá er
  \begin{align*}
    (1_R +\mathfrak A)(a+\mathfrak A) 
    &= 1_Ra + \mathfrak A = a +\mathfrak A
    \\
    (a+\mathfrak A)(1_R + \mathfrak A)
    &= a1_R + \mathfrak A = a +\mathfrak A
  \end{align*}
  svo að $1_R + \mathfrak A$ er einingarstak í $R/\mathfrak A$. Þá gildir sér í
  lagi að
  \begin{align*}
    ra + \mathfrak A 
    &= (r + \mathfrak A)(a + \mathfrak A) = 0_{R/\mathfrak A}
    \\
    ar + \mathfrak A
    &= (a + \mathfrak A)(r + \mathfrak A) = 0_{R/\mathfrak A}
  \end{align*}
  svo að $ra,ar\in\mathfrak A$ og því er $\mathfrak A$ tvíhliða íðal í
  $R$; þar með gildir (ii)$\Rightarrow$(i).
\end{sonnun}
\begin{skilgr}
  Látum $\mathfrak A$ vera tvíhliða íðal í baug $R$, við köllum bauginn
  $R/\mathfrak A$ með margföldun úr síðustu setningu
  \emph{deildabaug}\index{deildabaugur}\index{baugur!deildabaugur} af
  $R$ með tilliti til $\mathfrak A$.
\end{skilgr}
\begin{ath}
  Látum $\mathfrak A$ vera tvíhliða íðal í $R$. Skilyrði (ii) segir að
  náttúrlega ofanvarpið\[
  \pi: R\to R/\mathfrak A,
  \quad
  r\mapsto r+\mathfrak A
  \]
  sé \emph{baugamótun}. Ef $R$ er einbaugur, þá er $\pi(1_R) = 1_R+\mathfrak A$
  \emph{einingarstakið} í $R/\mathfrak A$, svo að $\pi$ er þá
  \emph{einbaugamótun}.
\end{ath}
\begin{ath}
  Fyrir víxlbauga er enginn munur á hægri og vinstri íðulum, og þau eru þá
  tvíhliða; tölum þá einfaldlega um \emph{íðul}\index{idal@íðal}.
\end{ath}
\begin{setn}
  Látum $\varphi: R\to S$ vera [ein]baugamótun. Þá er $\Ker\varphi$ tvíhliða
  íðal í $R$ og $\im\varphi$ er hlut[ein]baugur í $S$; grúpueinsmótunin
  $\chi$ sem gerir örvaritið\[
  \xymatrix{
  & R/\Ker\varphi \ar[dd]^\chi 
  \\
  R \ar[ur]^\pi \ar[dr]_{\tilde\varphi} &
  \\
  & \im\varphi
  }
  \]
  víxlið, þar sem $\pi$ er náttúrlega ofanvarpið og $\tilde\varphi:
  R\to\im\varphi, x\mapsto \varphi(x)$, er [ein]baugaeinsmótun.
\end{setn}
\begin{sonnun}
  Ef nú $x,y\in R/\Ker\varphi$, þá eru til $a,b\in R$ þannig að $x =
  a+\Ker\varphi = \pi(a)$ og $y = b+\Ker\varphi = \pi(b)$; þá er
  \begin{align*}
    \chi(xy)
    &= \chi(\pi(a)\pi(b))
    \\
    &= \chi(\pi(ab))
    \\
    &= \tilde{\varphi}(ab)
    \\
    &= \varphi(ab)
    \\
    &= \varphi(a)\varphi(b)
    \\
    &= \tilde{\varphi}(a)\tilde{\varphi}(b)
    \\
    &= \chi(\pi(a))\chi(\pi(b))
    \\
    &= \chi(x)\chi(y)
  \end{align*}
  svo að $\chi$ er baugamótun. Ef $\varphi$ er einbaugamótun, þá er\[
  1_S = \varphi(1_R)\in\im\varphi,
  \]
  svo að $\im\varphi$ er hluteinbaugur í $S$ og\[
  \chi(1_{R/\Ker\varphi}) 
  = \chi(\pi(1_R))
  = \varphi(1_R)
  = 1_S
  = 1_{\im\varphi}.
  \]
\end{sonnun}
\begin{ath}
  Ef $\mathfrak A_1,\dots,\mathfrak A_n$ eru vinstri [hægri, tvíhliða] íðul í
  baug $R$, þá er hlutgrúpan\[
  \mathfrak A_1 + \cdots + \mathfrak A_n
  = \left\{ a_1 + \cdots + a_n : a_k\in\mathfrak A_k 
  \text{ fyrir } k = 1,\dots,n \right\}
  \]
  vinstra [hægra, tvíhliða] íðal í $R$. Einnig er\[
  \mathfrak A_1 \cap \cdots \cap \mathfrak A_n
  \]
  vinstra [hægra, tvíhliða] íðal í $R$. Látum $R$ vera \emph{einbaug}, fyrir
  $a\in R$ er\[
  Ra := \left\{ ra : r\in R \right\}
  \]
  vinstra íðal í $R$, og\[
  aR := \left\{ ar : r\in R \right\}
  \]
  hægra íðal í $R$. Ef $a_1,\dots,a_n\in R$, þá er\[
  Ra_1 + \cdots + Ra_n
  \]
  minnsta vinstra íðal í sem inniheldur $a_1,\dots,a_n$; köllum það
  \emph{vinstra íðalið sem $a_1,\dots,a_n$ spanna}. Eins er\[
  a_1 R + \cdots + a_n R
  \]
  minnsta hægra íðalið sem inniheldur $a_1,\dots,a_n$, köllum það
  \emph{hægra íðalið sem $a_1,\dots,a_n$ spanna}.
  Köllum svona íðul \emph{endanlega spönnuð}\index{íðal!endanlega spannað}.
\end{ath}

\section{Höfuðíðalbaugar}
\begin{setn}
  Látum $R$ vera [ein]baug. Þá er
 \[
 Z(R) := \left\{ a\in R : ab = ba \quad \forall b\in R\right\}
 \]
 víxlinn hlut[ein]baugur í $R$.
\end{setn}
\begin{sonnun}
  Ljóst er að $0_R\in Z(R)$; fyrir $a_1,a_2\in Z(R)$ er\[
  (a_1 + a_2)b 
  = a_1 b + a_2 b 
  = ba_1 + ba_2
  = b(a_1 + a_2)
  \]
  fyrir öll $b\in R$, svo að $a_1 + a_2 \in Z(R)$; annað sést með svipuðum
  hætti.
\end{sonnun}
\begin{skilgr}
  Köllum $Z(R)$ \emph{miðju}\index{miðja}\index{miðja!baugs}\index{baugur!miðja}
  baugsins $R$.
\end{skilgr}
Ef $a\in Z(R)$, þá er $Ra = aR$, svo að $Ra$ er tvíhliða íðal.
\begin{skilgr}
  Íðal $\mathfrak A$ í baug $R$ kallast
  \emph{höfuðíðal}\index{höfuðíðal}\index{íðal!höfuðíðal} ef til
  er stak $a\in Z(R)$ þannig að\[
  \mathfrak A = Ra = aR.
  \]
\end{skilgr}
\begin{skilgr}
  Baugur $R$ kallast
  \emph{höfuðíðalbaugur}\index{höfuðíðalbaugur}\index{baugur!höfuðíðalbaugur}
  ef hann er víxlinn einbaugur, $1_R\neq 0_R$ og sérhvert íðal í $R$ er
  höfuðíðal.
\end{skilgr}
\begin{daemi}
  Við þekkjum allar hlutgrúpurnar í $\Z$; það eru mengin $m\Z$ þar sem $m\in\N$,
  en þessar hlutgrúpur eru allar íðul í $\Z$. Sér í lagi er $\Z$
  \emph{höfuðíðalbaugur}. Deildabaugarnir eru baugarnir $\Z/m\Z,m\in\N$.
\end{daemi}
%%
%% 29. október 2009
%%
\begin{ath}
  Í skilgreiningu á höfuðíðalbaug er venja að krefjast þess einnig að hann hafi
  enga núlldeila nema núllstakið.
\end{ath}
Rifjum upp að \emph{heilbaugur} var víxlinn einbaugur sem var ekki núllbaugur og
hefur enga núlldeila nema núllstakið. Þá má orða skilgreininguna þannig:
\begin{skilgr}
  [endurbætt]
  \emph{Höfuðíðalbaugur}\index{höfuðíðalbaugur} er heilbaugur þannig að sérhvert
  íðal hans sé höfuðíðal.
\end{skilgr}
\begin{ath}
  Samkvæmt skilgreiningu bókar er íðal í $R$ hlutbaugur í $R$, en oftast ekki
  \emph{hluteinbaugur}:
  Ef $\mathfrak A$ er íðal (frá vinstri, hægri eða tvíhliða) og $1_R\in\mathfrak
  A$, þá er $\mathfrak A = R$. Ef t.d.  $\mathfrak A$ er íðal frá vinstri, þá
  er \[
   r = r\cdot 1_R\in\mathfrak A
  \]
  fyrir öll $r\in R$ ef $1_R\in\mathfrak A$.
\end{ath}
\begin{setn}
  Látum $\mathfrak A$ vera tvíhliða íðal í baug $R$. Íðölin (frá vinstri, hægri
  eða tvíhliða) í deildabaugnum $R/\mathfrak A$ eru mengin 
  $\mathfrak B/\mathfrak A$, þar sem $\mathfrak B$ er íðal (frá vinstri, hægri
  eða tvíhliða) í $R$ þ.a. $\mathfrak A\subset\mathfrak B$.
\end{setn}
\begin{sonnun}
  Látum $\pi:R\to R/\mathfrak A$ vera náttúrlega ofanvarpið og $\mathfrak C$
  vera íðal af einhverri gerð í $R/\mathfrak A$; þá er $\mathfrak B:=
  \pi^{-1}[\mathfrak C]$ íðal í $R$ af sömu gerð,  $\mathfrak A\subset \mathfrak
  B$, og þar sem $\pi$ er átæk er $\mathfrak C = \pi[\pi^{-1}[\mathfrak C]] =
  \pi[\mathfrak B] = \mathfrak B / \mathfrak A$. Augljóst er að fyrir íðal
  $\mathfrak B$ þ.a. $\mathfrak A\subset \mathfrak B$ er $\mathfrak B /\mathfrak
  A$ íðal í $R/\mathfrak A$ af sömu gerð.
\end{sonnun}
\begin{ath}
  Ef $\varphi:R\to S$ er baugamótun og $\mathfrak C$ er íðal (af einhverri gerð)
  í $S$, þá er frummyndin $\varphi^{-1}[\mathfrak C]$ íðal í $R$ af sömu gerð;
  segjum að $\mathfrak C$ sé íðal frá vinstri og $r\in R$,
  $a\in\varphi^{-1}[\mathfrak C]$, þá er $\varphi(ra) =
  \varphi(r)\varphi(a)\in\mathfrak C$, því að $\varphi(r)\in S$ og
  $\varphi(a)\in\mathfrak C$, svo að $ra\in\varphi^{-1}[\mathfrak C]$. Ef hins
  vegar $\mathfrak A$ er íðal í $R$, þá þarf myndin $\varphi[\mathfrak A]$ ekki
  að vera íðal í $S$, \emph{nema} vörpunin $\varphi$ sé átæk.
\end{ath}
\begin{setn}
  Einbaugur $R$ er deilibaugur þ.þ.a.a. $R$ sé ekki núllbaugurinn og $R$ hafi
  engin vinstri íðul nema $\left\{ 0 \right\}$ og $R$.
\end{setn}
\begin{sonnun}
  Ef $R$ er deilibaugur og $\mathfrak A$ er íðal í $R$ þannig að $\mathfrak
  A\neq\left\{ 0 \right\}$, þá inniheldur $\mathfrak A$ stak $a$ þ.a. $a\neq 0$,
  og $a$ hefur umhverfu $b$; en þá er $1_R = ba\in\mathfrak A$ og þá $\mathfrak
  A = R$.

  Öfugt ef $R\neq \left\{ 0 \right\}$ og hefur engin vinstri íðul nema
  $\left\{ 0 \right\}$ og $R$, látum $a\in R$, $a\neq 0$; þá er $Ra$ vinstra
  íðal þannig að $a\in Ra$, svo að $Ra \neq \left\{ 0 \right\}$ og því
  $Ra = R$; sér í lagi er $1_R \in Ra$, svo að til er $b\in R$ þannig að
  $ba = 1_R$; en þá er $b\neq 0$, því að $0_R\cdot a = 0_R \neq 1_R$; og sama
  röksemdafærsla sýnir að til er $c\in R$ þannig að $cb = 1_R$. En þá er\[
  c = c1_R = c(ba) = (cb)a = 1_R a = a,
  \]
  svo að $ab = ba = 1_R$ og $b$ er því umhverfa fyrir $a$.
\end{sonnun}
\begin{ath}
  Eins sést: $R$ er deilibaugur þ.þ.a.a. $R\neq \left\{ 0 \right\}$ og
  $R$ hafi engin \emph{hægri} íðul nema $\left\{ 0 \right\}$ og $R$. Það er
  \emph{ekki nóg} að $R$ hafi engin \emph{tvíhliða} íðul nema $\left\{ 0
  \right\}$ og $R$; $R$ getur samt haft fullt af hægri og vinstri íðulum.
\end{ath}
\begin{fylgisetn}
  \emph{Víxlinn} einbaugur $K$ er svið þ.þ.a.a. hann sé ekki núllbaugur og hafi
  engin íðul nema $\left\{ 0 \right\}$ og $K$.
\end{fylgisetn}
\begin{ath}
  Skrifum $\left\{ 0 \right\} = K\cdot 0$ og $K = K\cdot 1$, svo að $\left\{
  0 \right\}, K$ eru alltaf höfuðíðul, svo svið er höfuðíðalbaugur.
\end{ath}

\section{Háíðul og frumíðul}
\begin{skilgr}
  \emph{Háíðal}\index{háíðal}\index{íðal!háíðal}
  [frá vinstri, frá hægri, tvíhliða] er eiginlegt
  íðal [frá vinstri, frá hægri, tvíhliða] sem er ekki innihaldið í stærra
  eiginlegu íðali [frá vinstri, frá hægri, tvíhliða].
\end{skilgr}
\begin{ath}
  Notum hugtakið aðallega fyrir víxlbauga og getum þá talað um "`háíðal"' án
  þess að taka fram tegundina. En
\end{ath}
\begin{setn}
  Tvíhliða íðal $\mathfrak A$ er háíðal \emph{frá vinstri} þ.þ.a.a. $R/\mathfrak
  A$ sé deilibaugur (og þá er það háíðal frá hægri).
\end{setn}
\begin{sonnun}
  Því að vinstri íðulin í $R/\mathfrak A$ eru íðulin $\mathfrak B/\mathfrak A$
  þar sem $\mathfrak B$ er vinstra íðal í $R$; og $\mathfrak B/\mathfrak A$ er
  eiginlegt í $R /\mathfrak A$ þ.þ.a.a. $\mathfrak B$ sé eiginlegt í $R$.
\end{sonnun}
\begin{fylgisetn}
  Látum $R$ vera víxlinn einbaug. Íðal $\mathfrak A$ í $R$ er háíðal þ.þ.a.a.
  $R/\mathfrak A$ sé svið.
\end{fylgisetn}
\begin{skilgr}
  Látum $R$ vera víxlinn einbaug. Íðal $\mathfrak A$ í $R$ kallast
  \emph{frumíðal}\index{frumíðal}\index{íðal!frumíðal} ef það er
  \emph{eiginlegt} og eftirfarandi skilyrði er fullnægt:
  \begin{quote}
    Ef $a,b\in R$ og $ab \in \mathfrak A$, þá er $a \in \mathfrak A$ eða $b\in
    \mathfrak A$.
  \end{quote}
\end{skilgr}
\begin{ath}
  $a\in \mathfrak A$ jafngildir $\pi(a) = 0$, þar sem $\pi:R\to R/\mathfrak A$
  er ofanvarpið. Því er ljóst:
\end{ath}
\begin{setn}
  Íðal í víxlnum einbaug er frumíðal þ.þ.a.a. $R/\mathfrak A$ sé heilbaugur.
\end{setn}
\begin{ath}
  Þar sem svið er heilbaugur er sérhvert háíðal frumíðal, en ekki öfugt.
\end{ath}
\begin{daemi}
  Núllíðalið $\Z\cdot 0$ í $\Z$ er frumíðal. Fyrir $m\in\N,m\geq 1$ er jafngilt:
  \begin{enumerate}[(i)]
    \item $m\Z$ er frumíðal,
    \item $m\Z$ er háíðal, 
    \item $m$ er frumtala. 
  \end{enumerate}
  því að $\Z/m\Z$ er heilbaugur þ.þ.a.a. $m$ sé frumtala; og endanlegur
  heilbaugur er svið.
\end{daemi}
\begin{ath}
  Hluteinbaugur $R$ í sviði $K$ er heilbaugur, því stak $a$ í $R$ þannig að
  $a\neq 0$ hefur umhverfu í $K$ og er því styttanlegt í $K$ og þar með líka í
  $R$.
\end{ath}

\section{Brotasvið}
\begin{skilgr}
  Látum $R$ vera heilbaug.
  \emph{Brotasvið}\index{brotasvið}\index{svið!brotasvið} fyrir $R$ er svið
  $K$ þannig að $R$ sé hluteinbaugur í $K$ og sérhvert stak í $K$ megi skrifa
  sem $ab^{-1}$ þar sem $a,b\in R$ og $b\neq 0$ (umhverfan er tekin í $K$).
\end{skilgr}
\begin{setn}
  Sérhver heilbaugur hefur brotasvið sem ákvarðast ótvírætt burtséð frá
  einsmótun.
\end{setn}
\begin{sonnun}
  Setjum $M := R\times(R\setminus \left\{ 0 \right\})$. Skilgreinum vensl á
  $M$ með\[
  (a,b) \sim (c,d) 
  \quad\text{þ.þ.a.a.}\quad
  ad = bc.
  \]
  Þetta eru jafngildisvensl: $(a,b)\sim(a,b)$ vegna $ab = ab$; ef
  $(a,b)\sim(c,d)$, þ.e. $ad = bc$, þá er $cb = da$ og því $(c,d)\sim(a,b)$, því
  að $R$ er víxlinn; ef $(a,b)\sim(c,d)$ og $(c,d)\sim(e,f)$, þá er $ad = bc$ og
  $cf = de$ og því\[
  afd = adf = bcf = bde = bed
  \]
  en $d$ er styttanlegt, svo að $af = de$ og því $(a,b)\sim(e,f)$. Táknum
  jafngildisflokk $(a,b)$ með $\frac ab$. Nú má skilgreina samlagningu og
  margföldun á mengi jafngildisflokkanna þannig að\[
  \frac ab + \frac cd = \frac{ad + bc}{bd},
  \qquad
  \frac ab \cdot \frac cd = \frac{ac}{bd}.
  \]
%%
%% 4. nóvember 2009
%%
  Til að sjá að þetta sé skynsamleg skilgreining þarf að sýna: Ef $\frac ab =
  \frac ef$ og $\frac cd = \frac gh$, þá er\[
  \frac{ad + bc}{bd} = \frac{eh + fg}{fh}
  \qquad\text{og}\qquad
  \frac{ac}{bd} = \frac{eg}{fh}
  \]
  En forsendan þýðir að $af = be$ og $ch = dg$, svo að\[
  (ad + bc)fh = adfh+bcfh = afdh+bfch=bedh+bfdg=bd(eh+fg)
  \]
  sem þýðir að $\frac{ad+bc}{bd} = \frac{eh+fg}{fh}$; og\[
  acfh = afch = bedg = bdeg
  \]
  sem þýðir að $\frac{ac}{bd} = \frac{eg}{fh}$. Nú er auðvelt að sjá að þessar
  reikniaðgerðir gera $K$ að víxlbaug með einingarstaki $\frac 11$ og núllstaki
  $\frac 01$; til dæmis sést tengireglan fyrir samlagningu þannig: 
  \begin{align*}
    \left( \frac ab + \frac cd \right) + \frac ef
    &= \frac{ad+bc}{bd}+\frac ef \\
    &= \frac{(ad + bc)f+bde}{bdf} \\
    &= \frac{adf + bcf + bde}{bdf} \\
    &= \frac{adf + b(cf + de)}{bdf} \\
    &= \frac ab +\frac{cf+de}{df} \\
    &= \frac ab + \left( \frac cd + \frac ef \right),
  \end{align*}
  aðrar reglur sjást á svipaðan hátt. Þar eð $R$ er heilbaugur er ljóst að
  $\frac ab = \frac 01$ þ.þ.a.a. $1a = b0$, þ.e. $a = 0$. Ef nú $\frac ab \neq
  \frac 01$, þá er $\frac ba \in K$ og\[
  \frac ab \cdot \frac ba = \frac{ab}{ba} = \frac{ab}{ab} = \frac 11.
  \]
  Þetta þýðir að $K$ er \emph{svið}. Nú höfum við vörpun $\varphi:R\to
  K,a\mapsto\frac a1$. Hún er eintæk baugamótun og við "`samsömum $R$ við mynd
  sína í $K$"' og skrifum $a$ í stað $\frac a1$; þannig lítum við á $R$ sem
  hlutmengi í $K$. Strangt tiltekið þýðir þetta eftirfarandi: Við tökum
  $\im\varphi$ út úr $K$ og setjum $R$ inn í staðinn, þ.e.a.s. við myndum
  mengið\[
  K' := R\cup (K\setminus \im\varphi).
  \]
  Við fáum gagntæka vörpun $\psi:K'\to K$ með því að setja\[
  \psi(x) := 
  \begin{cases}
    \varphi(x), &\text{ef $x\in R$},\\
    x,          &\text{ef $x\in K\setminus\im\varphi$}.
  \end{cases}
  \]
  Við notum þessa vörpun til að flytja aðgerðirnar af $K$ yfir á $K'$, þannig að
  við setjum\[
  x+y := \psi^{-1}(\psi(x) + \psi(y)),
  \]\[
  xy := \psi^{-1}(\psi(x)\psi(y)).
  \]
  Þar eð $\psi$ er baugamótun gefur þetta okkur aðgerðirnar á $R$ fyrir
  $x,y\in R$; þannig verður $R$ hlutbaugur í $K'$. Leyfum okkur svo að skrifa
  $a$ jöfnum höndum og $\frac a1$ lítum á $K'$ og $K$ sem sama hlutinn. Nú er
  ljóst: Stak $a\in R$ þ.a. $a\neq 0$, hefur margföldunarumhverfu í $K$,
  nefnilega $a^{-1} = \frac 1a$. Fyrir $a,b\in R$, $b\neq 0$ er\[
  \frac ab = \frac a1 \cdot \frac 1b = ab^{-1}.
  \]
  Við höfum því sýnt:
  \begin{quote}
    Sérhver heilbaugur hefur brotasvið.
  \end{quote}
\end{sonnun}
\begin{ath}
  Ef $R=\Z$, þá er $K=\Q$.
\end{ath}
\begin{ath}
  Einnig er hægt að smíða heilu tölurnar út frá $\N$ með því að sýna að
  $(a,b)\sim(c,d)$ þ.þ.a.a. $a+d = c+b$ myndi jafngildisvensl á $\N\times\N$.
\end{ath}
\begin{ath}
  Látum $R$ vera víxlinn einbaug. Hlutmengi $S$ í $R$ kallast
  \emph{margfeldið}\index{margfeldið}\footnote{hér notað sem lýsingarorð} ef
  $1\in S$ og $ab\in S$ fyrir öll $a,b\in S$. Ef $R$ er heilbaugur og $S$ er
  margfeldið hlutmengi í $R$ þ.a. $0\neq S$ og $K$ er brotasvið $R$, þá er\[
  S^{-1}R := \left\{ \frac as : a\in R, s\in S \right\}
  \]
  hluteinbaugur í $K$ sem inniheldur $R$, því að $0=\frac 01\in S^{-1}R$; ef
  $x,y \in S^{-1}R$, þá má skrifa $x=\frac as$ og $y=\frac bt$ með $s,t\in S$ og
  þá er\[
  x+y = \frac as + \frac bt = \frac{at + bs}{st}\in S^{-1}R
  \]
  og 
  $$ xy = \frac{ab}{st}\in S^{-1}R. $$
  Þetta gefur okkur ýmis dæmi um bauga, t.d. hlutbauga í $\Q$ með því að velja
  margfeldið hlutmengi í $\Z$.
\end{ath}
\begin{daemi}
  (1) Ef $a$ er stak í víxlnum einbaug $R$, þá er $\left\{ a^n:n\in\N \right\}$
  margfeldið hlutmengi í $R$; almennar ef $a_1,\dots,a_n\in R$, þá er $\left\{
  a_1^{n_1}\cdots a_r^{n_r}:n_1,\dots,n_r \in\N\right\}$ margfeldið hlutmengi í
  $R$.

  (2) Ef $p$ er frumtala, þá er $\Z/p\Z = \left\{ n\in\Z:p\nmid n \right\}$
  margfeldið hlutmengi í $\Z$. Almennar gildir: Látum $R$ vera víxlinn einbaug.
  Íðal $\mathfrak I$ í $R$ er frumíðal í $R$ þ.þ.a.a. fyllimengið
  $R\setminus\mathfrak I$ sé margfeldið hlutmengi í $R$.
\end{daemi}

\chapter{Margliður}
\section{Margliður, margliðubaugar}
\begin{skilgr}
  Látum $R$ vera víxlbaug og athugum mengið $R[X]$ af öllum runum
  $(a_k)_{k\in\N}$ af stökum í $R$ þannig að mengið $\left\{ k\in\N:a_k\neq 0
  \right\}$ sé endanlegt; m.ö.o. þannig að til sé tala $n\in\N$ þ.a. $a_k = 0$
  fyrir öll $k>n$. Skilgreinum samlagningu og margföldun á $R[X]$ sem hér segir:
  \begin{align*}
    (a_k)_{k\in\N} + (b_k)_{k\in\N}
    &:= (a_k+b_k)_{k\in\N},
    \\
    (a_k)_{k\in\N}\cdot (b_k)_{k\in\N}
    &:= (c_k)_{k\in\N},
  \end{align*}
  þar sem\[
  c_k := \sum_{j=0}^{k}a_j b_{k-j}.
  \]
\end{skilgr}
Til að sýna að þetta sé vel skilgreint athugum við: Ef $a_k = 0$ fyrir $k>n$ og
$b_k = 0$ fyrir $k>m$, þá er $a_k + b_k =0$ fyrir $k > \max\left\{ n,m \right\}$
og $c_k = 0$ fyrir $k > n+m$, því þá gildir um öll $j\in\left\{ 0,\dots,k
\right\}$ að $j>n$ eða $k-j>m$ og þá $a_j b_{k-j}=0$. Þar með er summan og
margfeldið í $R[X]$ aftur í $R[X]$ og einfaldir (en langir) reikningar sýna að
þessar aðgerðir gera $R[X]$ að baug, köllum hann
\textbf{\emph{margliðubauginn}}\index{margliðubaugur}\index{baugur!margliðubaugur}
(í einni óákveðinni stærð) yfir $R$. 

Gerum nú ávallt ráð fyrir að $R$ sé víxlinn \emph{einbaugur}. Þá hefur
$R[X]$ einingarstak\[
1 = 1_{R[X]} = (\delta_{0k})_{k\in\N} = (1,0,0,\dots).
\]
Athugum nú stakið\[
x := (\delta_{1k})_{k\in\N} = (0,1,0,0,\dots).
\]
Við höfum $x^2 = (c_k)_{k\in\N}$, þar sem $c_k =
\sum_{j=0}^{k}\delta_{1j}\delta_{1,k-j}=\delta_{2k}$ svo\[
x^2 = (\delta_{2k})_{k\in\N} = \left( 0,0,1,0,0,\dots \right).
\]
Með sama hætti fæst með þrepun:\[
x^n = (\delta_{nk}n\in\N) = (0,0,\dots,0,1,0,0,\dots),
\]
þar sem stakið $1$ er í $n$-ta sæti.

%%
%% 5. nóvember 2009
%%

Táknuðum "`óákveðnu stærðina"' með "`x"', við skulum heldur nota upphafsstaf
"`X"'. Þannig er 
\[
X = (\delta_{1k})_{k\in\N} = (0,1,0,0,\dots).
\]

Vörpunin $\varphi:R\to R[X], a\mapsto (a\delta_{0k})_{k\in\N} = (a,0,0,0,\dots)
$ er einbaugamótun og eintæk; við getum notað hana til að samsama $R$ við mynd
sína í $R[X]$; það þýðir að við skrifum "`$a$"' í stað
"`$(a\delta_{0k})_{k\in\N}$"', við köllum margliður af þessu tagi
\emph{fastar} margliður\index{fastar margliður}. Nú sjáum við að fyrir $a\in R$
er 
\[
aX^n = (0,\dots,0,a,0,\dots)
\]
með $a$ í $n$-ta sæti. Látum nú $P=(a_n)_{n\in\N}$ vera einhverja margliðu í
$R[X]$ og $n$ vera þannig að $a_k = 0$ fyrir $k>n$. Þá er
\begin{align*}
  \sum_{k=0}^{n}a_kX^k 
  &= (a_0,0,0,\dots) + (0,a_1,0,\dots) + \cdots +
     (0,\dots,0,a_n,0,\dots)
  \\
  &= (a_0,a_1,a_2,\dots,a_n,0,0,\dots).
\end{align*}
M.ö.o. er\[
P = \sum_{k=0}^{n}a_k X^k.
\]
Skrifum hér eftir margliður þannig. Þessi framsetning ákvarðast ótvírætt purtséð
frá að sleppa má liðum $a_k X^k$ þannig að $a_k = 0$ (eða bæta þeim við). Látum
nú $P = (a_k)_{k\in\N}$ vera margliðu og setjum 
\index{stig}\index{margliða!stig}\[
\stig P := \sup \left\{ n\in\N:a_n\neq 0 \right\}.
\]
Þetta þýðir: Ef $P\neq 0$, þá er $\stig P$ stærsta tala $k$ þannig að $a_k\neq
0$, en stig núllmargliðunnar er $-\infty$. Ef $n = \stig P\in\N$, þá kallast
$a_n$ \emph{forystustuðull}\index{forystustuðull}\index{margliða!forystustuðull}
margliðunnar $P$, en $a_0$ kallast
\emph{fastastuðull}\index{fastastuðull}\index{marglið!fastastuðull} hennar. 

\section{Deiling í margliðubaugum}
\begin{setn}
  Látum $P,Q\in R[X]$. Þá er\[
  \stig(P+Q) \leq \max\left\{ \stig P,\stig Q \right\}
  \]
  og \[
  \stig(P\cdot Q) \leq \stig P + \stig Q
  \]
  og í seinni jöfnunni gildir jafnaðarmerki ef forystustuðlar $P$ og $Q$ eru
  ekki núlldeilar í $R$.
\end{setn}
\begin{sonnun}
  Höfðum séð að fyrir $P = (a_k)_{k\in\N}$, $Q=(b_k)_{k\in\N}$ gildir $a_k+b_k =
  0$ fyrir $k > \max\left\{ \stig P,\stig Q \right\}$ og $c_k :=
  \sum_{j=0}^{k}a_j b_{k-j}$ er $0$ fyrir $k > n+m$ ef $n = \stig P$
  og $m = \stig Q$; þetta sýnir ójöfnurnar. Athugum nú að $c_{n+m}=a_n b_m$ þar
  sem $a_n$ er forystustuðull $P$ og $b_m$ er forystuðull $Q$; þá er $a_n\neq 0$
  og $b_m\neq 0$ og við fáum $c_{n+m}\neq 0$ nema $a_n,b_m$ séu núlldeilar í
  $R$.
\end{sonnun}
Sér í lagi gildir jafnan
\[
 \stig(PQ) = \stig P + \stig Q
\]
alltaf ef $R$ er \emph{heilbaugur} og þá sér í lagi ef $R$ er svið. Fáum:
\begin{fylgisetn}
  Ef $R$ er heilbaugur þá er $R[X]$ líka heilbaugur.
\end{fylgisetn}
\begin{setn}
  [Um deilingu með afgangi í $R\lbrack X\rbrack$ ]
  Látum $P$ vera margliðu í $R[X]$ þannig að $P\neq 0$ og $P$ sé \emph{eind} í
  $R$. Fyrir sérhverja margliðu $F$ í $R[X]$ eru til margliður $Q$ og $G$
  þannig að\[
  F = PQ + G 
  \qquad\text{og}\qquad 
  \stig G < \stig P;
  \]
  margliðurnar $P,Q$ ákvarðast ótvírætt af þessum skilyrðum.
\end{setn}
\begin{sonnun}
  Fyrir tilvist notum við þrepun yfir $m:=\stig F$. Látum $n:=\stig P\geq 0$;
  ljóst ef $m < n$, því þá má taka $Q=0$ og $G=F$. Gerum því ráð fyrir að
  $m\geq n$, skrifum $F = \sum_{k=0}^{m} a_k X^k$ og $P = \sum_{k=0}^{n} c_k
  X^k$; þá hafa $F$ og $c_n^{-1} a_m X^{m-n} P$ bæði stig $m$ og sama
  forystustuðul $a_m$, svo að $F_1 := F - c_n^{-1} a_m X^{m-n} P$ er af stigi
  $< m$. Skv. þrepunarforsendu má skrifa $F_1 = PQ_1 + G$, þar sem $\stig G <
  \stig P$, en þá er\[
  F = F_1 + c_n^{-1} a_m X^{m-n} P = PQ + G,
  \]
  þar sem $Q := Q_1 + c_n^{-1}a_m X^{m-n}$.
  
  \emph{Ótvíræðni:} Ef $F = PQ + G = PQ_1 + G_1$ þar sem $G,G_1$ hafa minna stig
  en $P$, en þá er\[
  P(Q-Q_1) = G_1 - G
  \]
  og þá\[
  \stig P + \stig(Q-Q_1) = \stig(G_1-G) < \stig P
  \]
  því að forystustuðull $P$ er eind og því ekki núlldeilir í $R$. Þetta stenst
  ekki nema $\stig(Q-Q_1) = \stig (G_1-G)=-\infty$, þ.e. $Q=Q_1$ og $G = G_1$.
\end{sonnun}
\begin{ath}
  Héðan í frá verður \textbf{aðeins talað um einbauga}. Ef einhvers staðar
  stendur \emph{baugur} í stað \emph{einbaugur}, þá er óhætt að gera ráð fyrir
  að sá baugur sé samt sem áður einbaugur.
\end{ath}

\section{Margliðuföll}
\begin{skilgr}
  Látum $R$ vera hluteinbaug í einbaug $S$ og látum $c\in S$. Fyrir margliðu
  $P = \sum_{k=0}^{n} a_k X^k$ í $R[X]$ setjum við\[
  P(c) := \sum_{k=0}^{n} a_k c^k\in S.
  \]
  Sér í lagi er $P(c)$ skilgreint fyrir öll $c\in R$, og við fáum vörpun\[
  \tilde{P}:R\to R, c\mapsto P(c)
  \]
  sem við köllum \emph{margliðufallið}\index{margliðufall} sem margliðan
  $P$ skilgreinir.
\end{skilgr}
\begin{ath}
  Almennt ákvarðast margliða \emph{ekki} af margliðufallinu. T.d. hefur
  $\Z/2\Z[X]$ óendanlega mörg stök, en aðeins eru til fjórar varpanir $\Z/2\Z\to
  \Z/2\Z$.
\end{ath}
\begin{setn}
  [Reiknireglur]
  Ef $R$ er hluteinbaugur í $S$ og $c\in S$, þá er\[
  (P+Q)(c) = P(c) + Q(c)
  \]\[
  (PQ)(c) = P(c)\cdot Q(c)
  \]
  fyrir öll $P,Q\in R[X]$.
\end{setn}
Fáum:
\begin{setn}
  Látum $R$ vera hluteinbaug í $S$ og $c\in S$. Þá er til nákvæmlega ein
  baugamótun $\psi:R[X]\to S$ þannig að $\psi(r)=r$ fyrir öll $r\in R$ og
  $\psi(X) = c$; hún er gefin með\[
  \psi(P) = P(c).
  \]
\end{setn}
\begin{sonnun}
  Ef $\psi$ er slík baugamótun og $P = \sum_{k=0}^{n}a_k X^k \in R[X]$, þá er\[
  \psi(P)
  = \Psi\left( \sum_{k=0}^{n}a_k X^k \right)
  = \sum_{k=0}^{n}\psi(a_k)\psi(X)^k
  = \sum_{k=0}^{n} a_k c^k.
  \]
  Öfugt segja reiknireglurnar að $\psi$ sé baugamótun.
\end{sonnun}
\begin{skilgr}
  Skrifum \[
  R[c] := \im\psi = \left\{ P(c):P\in R[X] \right\}.
  \]
  Þá er $R[c]$ hlutbaugur í $S$, einsmóta $R[x]/\Ker\psi$.
\end{skilgr}

\section{Núllstöðvar margliða}
\begin{skilgr}
  Látum $R$ vera hluteinbaug í $S$ og $P\in R[X]$. Segjum að stak $c\in S$ sé
  \emph{núllstöð}\index{núllstöð}\index{margliða!núllstöð} margliðunnar $P$ ef
  $P(c) = 0$.
\end{skilgr}
\begin{daemi}
  Margliðan $X^2 + 1\in\R[X]$ hefur ekki núllstöð í $\R$ en hún hefur núllstöð í
  stærri baugnum $\C$.
\end{daemi}
%%
%% 11. nóvember
%%
Gerum nú ráð fyrir að $R$ sé einbaugur og að $P\in R[X]$. Látum $a\in R$. Deilum
$X-a$ upp í $P$ með afgangi og skrifum\[
P = (X-a)Q + b
\]
þar sem $\stig(b) < 1$, þ.e. $b$ er föst margliða sem við höfum samsamað við
stökin í $R$. Setjum nú $a$ inn í margliðuna og fáum $P(a) = b$. Sjáum:
\begin{setn}
  Ef $R$ er einbaugur, $P\in R[X]$ og $a\in R$, þá gildir: Stakið $a$ er
  núllstöð margliðunnar $P$ þ.þ.a.a. skrifa megi \[ P = (X-a) Q \] þar sem
  $Q\in R[X]$. Ef $P(a) = 0$ og $P\neq 0$, þá er $\stig(Q) = \stig(P)-1$.
\end{setn}
Athugum að stuðullinn við $X$ í $X-a$ er $1$, þ.e. eind, svo
$\stig(P)=\stig(X-a)+\stig(Q)=1+\stig(Q)$. Með þrepun fæst:
\begin{setn}
  Ef $R$ er heilbaugur og $P\in R[X]$ hefur ólíkar núllstöðvar $a_1,\dots,a_r\in
  R$, $P\neq 0$, þá má skrifa \[
  P = (X-a_1)\cdots(X-a_r)Q
  \]
  þar sem $Q$ er margliða af stigi $\stig(P)-r$.
\end{setn}
\begin{sonnun}
  Tilviki $r=1$ er síðasta setning. Látum $P$ hafa ólíkar núllstöðvar
  $a_1,\dots,a_{r+1}$. Skv. þrepunarforsendu má skrifa \[
  P = (x-a_1)\cdots(x-a_r)Q_q
  \]
  þar sem $Q_1$ er margliða af stigi $\stig(P)-r$. Nú er \[
  0 = P(a_{r+1})=(a_{r+1}-a_1)\cdots (a_{r+1}-a_r)Q_1(a_{r+1}).
  \]
  Höfum $a_{r+1}-a_k\neq 0$ fyrir $k=1,\dots,r$;  þar eð $R$ er
  \emph{heilbaugur} er $Q_1(a_r+1)=0$, svo að skrifa má $Q_1=(X-a_{r+1})Q$, þar
  sem $\stig(Q)=\stig(Q_1)-1 = \stig(P)-\stig(r+1)$; og $P =
  (X-a_1)\cdots(X-a_{r+1})Q$.
\end{sonnun}
\begin{fylgisetn}
  Látum $R$ vera heilbaug og $P$ vera margliðu í $R[X]$ af stigi $n$; þá hefur
  $P$ í mesta lagi $n$ ólíkar núllstöðvar í $R$.
\end{fylgisetn}
\begin{fylgisetn}
  Ef $R$ er \emph{óendanlegur heilbaugur}, þá ákvarðast margliða $P$ af
  margliðufallinu $\tilde P: R\to R,a\mapsto P(a)$; m.ö.o. gildir: Ef $P,Q\in
  R[X]$ og $\tilde P = \tilde Q$, þá er $P=Q$.
\end{fylgisetn}
\begin{sonnun}
  $P-Q$ hefur óendanlega margar núllstöðvar, svo að $\stig(P-Q)=-\infty$, þ.e.
  $P-Q=0$.
\end{sonnun}
\begin{ath}
  Látum $R$ vera heilbaug, $P\in R[X]$ og $a$ vera núllstöð $P$; skrifum
  $P=(X-a)Q_1$. Ef $a$ er líka núllstöð $Q_1$, þá má skrifa $Q_1 = (X-a)Q_2$,
  þ.e. $P=(X-a)^2Q_2$. Þessu má halda áfram: Ef $P\neq 0$, þá tekur það enda og
  við fáum \[
  P = (X-a)^r Q_r, \qquad Q_r(a)\neq 0.
  \]
\end{ath}
Setjum þá að $r$ sé \emph{margfeldni}\index{margfeldni!núllstöðvar}
núllstöðvarinnar $a$ (í margliðunni P).
\begin{setn}
  Látum $K$ vera svið. Þá er $K[X]$ höfuðíðalbaugur.
\end{setn}
\begin{sonnun}
  Látum $\mathfrak A$ vera íðal í $K[X]$ þ.a. $\mathfrak A\neq \left\{ 0
  \right\}$ (núllíðalið er alltaf höfuðíðal). Veljum margliðu $P\neq 0$ í
  $\mathfrak A$ af lægsta hugsanlega stigi. Látum nú $F\in \mathfrak A$, þar eð
  $K$ er svið er forystustuðull margliðunnar $P$ eind, svo að við getum skrifað
  $F=PQ+G$, þar sem $\stig G<\stig(P)$. En $G=F-PQ\in\mathfrak A$; svo að
  $G=0$ skv. skilgreiningu á $P$. En það þýðir að $F\in P\cdot K[X]$. Ljóslega
  er $P\cdot K[X]\subset \mathfrak A$, svo að $\mathfrak A=PK[X]$.
\end{sonnun}
\begin{ath}
  Þetta gildir ekki ef $K$ er bara heilbaugur; t.d. er $\Z[X]$ ekki
  höfuðíðalbaugur. Sýnum að $2\Z[X]+X\Z[X]$ er ekki höfuðíðal: Ef margliðurnar
  $2$ og $X$ eru ekki margfeldi af sömu margliðu í $\Z[X]$, þá verður sú
  margliða að vera föst, segjum fastinn $a$, en þá gengur $a$ upp í
  forystustuðli $X$, sem er $1$, svo $a=1$ eða $a=-1$. En $1$ er ekki í
  $2\Z[X]+X\Z[X]$; ef $1=2P_1+XP_2$ með $P_1,P_2\in\Z[X]$, þá gengur talan
  $2$ upp í fastastuðli hægri hliðar jöfnunnar en ekki vinstri hliðar.
\end{ath}

\section{Evklíðskir baugar}
\begin{skilgr}
  \emph{Evklíðskur baugur}\index{evklíðskur baugur}\index{baugur!evklíðskur}
  er heilbaugur $R$ þannig að
  til sé vörpun $d:R\setminus\left\{ 0 \right\}\to\N$ sem fullnægir eftirfarandi
  skilyrði:
  \begin{quote}
    Ef $a,b\in R\setminus \left\{ 0 \right\}$, þá eru til stök $q,r\in R$ þ.a.
    $a = bq+r$ og annaðhvort er $r=0$ eða $d(r)<d(b)$.
  \end{quote}
\end{skilgr}
\begin{daemi}
  (1) Baugurinn $\Z$ er evklíðskur með $d(x):=|x|$ fyrir $x\in\Z\setminus
  \left\{ 0 \right\}$. Baugurinn $K[X]$, þar sem $K$ er svið, er evklíðskur með
  $d(P):=\stig(P)$, $P\in K[X]\setminus\left\{ 0 \right\}$.

  (2) Baugurinn $\Z[i]$. Við höfum $i^2=-1$, svo að $i$ er tvinntala sem er
  núllstöð margliðunnar $X^2+1$, sem er í $\Z[i]$; sérhverja margliðu $P$ í
  $\Z[X]$ má skrifa með nákvæmlega einum hætti sem $P=(X^2 + 1)Q+a+bX$, með
  $a,b\in\Z$ og það þýðir að sérhvert stak í $\Z[i]$ má skrifa með nákvæmlega
  einum hætti sem $a+bi$, þar sem $a,b\in\Z$, þ.e.\[
  \Z[i] = \left\{ a+bi : a,b\in\Z \right\}.
  \]
  Þetta er hlutbaugur í $\C$ og kallast
  \emph{Gauss-heiltalnabaugurinn}\index{Gauss-heiltalnabaugurinn}\index{baugur!Gauss-heiltalnabaugurinn} og stök hans
  \emph{Gauss-heiltölur}\index{Gauss-heiltölur}. Þessi baugur er evklíðskur
  baugur með $d(z):=|z|^2=z\overline z$ fyrir $z\in\Z[i]\setminus\left\{ 0
  \right\}$. Athugum að fyrir tvinntölur $z=x+iy$ með $x,y\in\R$ eru til heilar
  tölur $m,n$ þ.a.  $|x-n|\leq \frac 12$ og $|y-m|\leq \frac 12$; þá er\[
  u := n+im \in \Z[i],
  \qquad
  |z-u|^2 =(x-u)^2 + (y-u)^2 \leq \frac 14 + \frac 14 < 1.
  \]
  Látum nú $a,b\in \Z[i]\setminus\left\{ 0 \right\}$. Finnum $q\in\Z[i]$ þannig
  að $|\frac ab - q|^2 < 1$; fyrir $r=a-bq$ er þá 
  \[
  |r|^2=|a-bq|^2=|b|^2|\frac ab-q|^2<|b|^2
  \]
  og $a = bq+r$.
\end{daemi}
\begin{setn}
  Evklíðskur baugur er höfuðíðalbaugur.
\end{setn}
\begin{sonnun}
  Látum $R$ vera evklíðskan baug með samsvarandi falli $d:R\setminus\left\{ 0
  \right\}\to\N$. Látum $\mathfrak A$ vera íðal í $R$, $\mathfrak A\neq 0$,
  látum $b$ vera stak í $\mathfrak A\setminus\left\{ 0 \right\}$ þ.a. $d(b)$ sé
  sem minnst. Fyrir $a\in\mathfrak A$ eru þá til $q,r\in R$ þannig að $a=bq+r$
  og $r=0$ eða $d(r)<d(b)$. En $r=a-bq\in\mathfrak A$, svo að $d(r)<d(b)$ kemur
  ekki til greina vegna skilgreiningar á $b$, svo að $r=0$ og því $a=bq$; fáum
  $\mathfrak A = bR$.
\end{sonnun}
\begin{ath}
  Í bók er þess krafist að $d$ fullnægi líka öðru skilyrði, nefnilega\[
  d(a) \leq d(ab)
  \]
  fyrir öll $a,b\in R\setminus\left\{ 0 \right\}$. Þetta er gagnlegur
  eiginleiki, hann leyfir að finna eindirnar í $R$; ef við höfum hann, þá er\[
  R^* = \left\{ a\in R\setminus\left\{ 0 \right\}: d(a)=d(1) \right\}.
  \]
\end{ath}

%%
%% 12. nóvember
%%

\chapter{Þáttun í heilbaugum}

\section{Tengd stök, þættanleiki og frumstök}
\begin{skilgr}
  (1) Látum $R$ vera heilbaug og $a,b\in R$. Við segjum að $a$ \emph{gangi upp
  í $b$ í $R$}\index{ganga upp í} og skrifum\[
  a \mid b
  \]
  ef til er stak $c\in R$ þ.a. $b=ac$.

  (2) Við segjum að stök $a,b\in R$ séu \emph{tengd}\index{tengd stök} ef
  \[ a\mid b \text{ og } b\mid a . \]
\end{skilgr}
\begin{ath}
  Að stak $u\in R$ sé eind þýðir að $u\mid 1$.
\end{ath}
\begin{setn}
  Látum $R$ vera heilbaug.
  \begin{enumerate}[(1)]
    \item Við höfum $a\mid b$ þ.þ.a.a. $bR\subset aR$.
    \item Stak $u$ í $R$ er eind þ.þ.a.a. það gangi upp í sérhverju staki í
      $R$.
    \item Fyrir öll $a$ úr $R$ er $a\mid 0$. Ef $0\mid a$, þá er $a=0$.
    \item Stök $a,b\in R$ eru tengd þ.þ.a.a. til sé eind $u$ þ.a. $a=ub$.
    \item Fyrir öll $a$ úr $R$ er $a\mid a$.
    \item Ef $a,b,c\in R$ og $a\mid b$ og $b\mid c$, þá $a\mid c$. 
  \end{enumerate}
\end{setn}
\begin{sonnun}
  Atriði (1), (2), (3) eru augljós.

  (4) Ef $a = ub$, þar sem $u$ er eind, þá gildir $b\mid a$; líka er $b=u^{-1}a$ svo
  að $a\mid b$. Ef hins vegar bæði $a\mid b$ og $b\mid a$, þá sjáum við fyrst að $a=0$
  þ.þ.a.a. $b=0$ skv. (3). Gerum því ráð fyrir að $a,b\neq 0$. Nú er til $u$
  þannig að $b=ua$ og $v$ þ.a. $a = vb$. En þá er $a=vb=vua$, og þar sem $R$ er
  heilbaugur fæst $1=vu$, þ.e. $u$ er eind.

  (5) Er ljóst: $a=1a$.
  
  (6) Ef $a\mid b$ og $b\mid c$, þá eru til $x,y$ þ.a. $b=ax$ og $c=by$, og þá er
  $c=by=a(xy)$, svo að $a\mid c$.
\end{sonnun}
\begin{skilgr}
  (1) Við segjum að stak $p$ í heilbaug $R$ sé
  \emph{óþættanlegt}\index{oþættanlegt@óþættanlegt} ef það er hvorki núll né eind og ekki er
  hægt að skrifa $p$ sem margfeldi $p=ab$ nema annaðhvort $a$ eða $b$ sé eind.

  (2) Við segjum að stak $p$ í heilbaug $R$ sé \emph{frumstak}\index{frumstak}
  ef það er hvorki núll né eind og fyrir öll $a,b\in R$ þ.a. $p\mid ab$ gildir
  annaðhvort $p\mid a$ eða $p\mid b$.
\end{skilgr}
\begin{ath}
  (1) Ef $p$ er frumstak í baug $R$ og $a_1,\dots,a_r\in R$ og
  $p\mid a_1\cdots a_r$, þá er $r\geq 1$ og til er $k\in\left\{ 1,\dots,r \right\}$
  þannig að $p\mid a_k$. Þetta sjáum við með þrepun út frá skilgreiningunni.

  (2) Sérhvert frumstak er óþættanlegt. Ef $p$ er frumstak og til eru $a,b$ í
  $R$ þannig að $p=ab$, þá gengur $p$ upp í $a$ eða $b$; sejgum t.d. að $p\mid a$.
  Þá megum við skrifa $a = pu$, þar sem $u\in R$ og við fáum $p=ab=pub$; en
  $p\neq 0$ svo að $1=ub$; en þá er $b$ eind.

  Hins vegar þarf óþættanlegt stak í baug ekki að vera frumstak!
\end{ath}
\begin{setn}
  Stak $p$ í heilbaug $R$ þannig að $p\neq 0$ er frumstak þ.þ.a.a. $pR$ sé
  frumíðal.
\end{setn}
\begin{sonnun}
  Að $pR$ sé frumíðal þýðir að $pR\neq R$ og
  \begin{quote}
    Ef $a,b\in pR$, þá er $a\in pR$ eða $b\in pR$.
  \end{quote}
  Skilyrðið $pR\neq R$ þýðir að $p$ er ekki eind; seinna skilyrðið segir: Ef
  $p\mid ab$, þá $p\mid a$ eða $p\mid b$.
\end{sonnun}

\section{Þáttabaugar}
\begin{skilgr}
  Baugur $R$ kallast
  \emph{þáttabaugur}\index{zþáttabaugur@þáttabaugur}\index{baugur!zþáttabaugur@þáttabaugur}
  ef hann er heilbaugur og sérhvert stak $a$ í $R$ þ.a. $a\neq 0$ má skrifa sem
  margfeldi\[
  a = u \prod_{k=1}^{r}p_k
  \]
  þar sem $u$ er eind og $p_1,\dots,p_r$ eru frumstök. Köllum þessa framsetningu
  \emph{frumþáttun}\index{frumþáttun} staksins $a$ og $p_1,\dots,p_r$
  \emph{frumþætti}\index{frumþættir} þess.
\end{skilgr}
\begin{ath}
  (1) Við leyfum að $r=0$ með því venjulega samkomulagi að $\prod_{k=1}^r = 1$
  ef $r=0$; stökin sem skrifa má þannig með $r=0$ eru þá nákvæmlega eindirnar í
  $R$.

  (2) Látum $u$ vera eind í $R$. Ljóst er að stak $p$ í $R$ er frumstak í $R$
  þ.þ.a.a. $up$ sé frumstak í $R$; ef nú $a\in R, a\neq 0$,\[
  a = u\prod_{k=1}^r p_k,
  \]
  og $u_1,\dots,u_r$ eru eindir, þá má setja $q_k:=up_k$ og $v:=uu_1^{-1}\cdots
  u_r^{-1}$; og þá er líka\[
  a = v \prod_{k=1}^r q_k.
  \]
  Frumþáttunin ákvarðast almennt ekki ótvírætt. En við höfum:
\end{ath}
\begin{setn}
  Frumþáttun staks í þáttabaug ákvarðast ótvírætt burtséð frá röð og tengslum;
  þetta þýðir: Ef\[
  a = u\prod_{k=1}^r p_k = v\prod_{j=1}^s q_j
  \]
  þar sem $u,v$ eru eindir í $R$ og $p_1,\dots,p_r,q_1,\dots,q_s$ eru frumstök í
  $R$, þá er $r=s$ og til er gagntæk vörpun $\sigma:\left\{ 1,\dots,r
  \right\}\to\left\{ 1,\dots,r \right\}$ þ.a. $p_k$ sé tengt $q_{\sigma(k)}$
  fyrir öll $k=1,\dots,r$.
\end{setn}
\begin{sonnun}
  Þrepum yfir $r$. Ljóst ef $r=0$ (eða $r=1$). Látum þá $r\geq 1$. Höfum þá að
  $p_r\mid \prod_{j=1}^s q_j$ og þar sem $p_r$ er frumstak  er til $j$ þ.a.
  $p_r\mid q_j$; með því að umraða $q_1,\dots,q_s$ má g.r.f. að $p_r\mid q_s$. Skrifum
  $q_s=wp_r$. Þar sem $q_s$ er frumstak er það óþættanlegt og $p_r$ er ekki
  eind, svo $w$ er eind. Nú fæst\[
  u\prod_{k=1}^r p_k = uwp_k \prod_{j=1}^{s-1}q_j
  \]
  og þá\[
  u\prod_{k=1}^{r-1} p_k = uw \prod_{j=1}^{s-1}q_j.
  \]
  Skv. þrepunarforsendu er $r-1=s-1$, og við getum umraðað
  $q_1,\dots,q_{r-1}$ þ.a. $p_k$ sé tengt $q_k$ fyrir $k=1,\dots,r-1$.
\end{sonnun}
\begin{fylgisetn}
  Í þáttabaug er sérhvert óþættanlegt stak frumstak.
\end{fylgisetn}
\begin{sonnun}
  Í frumþáttun óþættanlegs staks getur bara verið einn frumþáttur.
\end{sonnun}
\begin{setn}
  Í höfuðíðalbaug er sérhvert frumíðal háíðal og sérhvert óþættanlegt stak
  frumstak.
\end{setn}
\begin{sonnun}
  Látum $p$ vera óþættanlegt stak. Gerum ráð fyrir að $\mathfrak A$ sé eiginlegt
  íðal í $R$ þ.a. $pR\subset\mathfrak A$. Skrifum $\mathfrak A = bR$ þar sem
  $b\in R$. Þá gildir $b\mid p$, svo að $p=bu$  þar sem $u\in R$; nú er $b$ ekki
  eind, því að $pR$ er eiginlegt íðal, svo að $u$ verður að vera eind; en þá er
  ljóst að $bR=buR=pR$. Þar með er $pR$ háíðal, og háíðal er frumíðal, svo að
  $p$ er frumstak.

  Ef svo $\mathfrak P$ er frumíðal, þá er $\mathfrak P = pR$, þar sem $p$ er
  frumstak, þar með óþættanlegt, svo að $\mathfrak P$ er háíðal.
\end{sonnun}
%%
%% 18. nóvember 2009
%%
\begin{setn}
  Sérhver höfuðíðalbaugur er þáttabaugur.
\end{setn}
\begin{sonnun}
  Það nægir að sýna að sérhvert stak í höfuðíðalbaug $R$, sem er ekki eind, megi
  skrifa sem margfeldi af \emph{óþættanlegum} stökum, því að þau eru frumstök
  skv. síðustu setningu. Látum því $a\in R$, $a$ ekki eind. Ef $a$ er
  óþættanlegt, þá þarf ekki að gera meira. Annars má skrifa $a=a_1b_1$ þar sem
  hvorki $a_1$ né $b_1$ er eind. Þá er $a_1\mid a$ og $Ra\subsetneq Ra_1$ (athugum
  að fyrir stök $a,b$ í heilbaug $R$ gildir $Ra=Rb$ þ.þ.a.a. $a\mid b$ og
  $b\mid a$,
  þ.e. $a$ og $b$ eru tengd). Ef $a_1,b_1$ eru óþættanleg, þá þarf ekki meira;
  annars má t.d. þátta $b_1 = a_2 b_2$ og skrifa $a=a_1a_2b_2$ þar sem
  $Ra_1\subsetneq Ra_2$. Annaðhvort endar þetta á að við höfum skrifað
  $a$ sem margfeldi af endanlega mörgum stökum, eða við fáum óendanlega runu
  $(a_k)_{k\in\N}$ af stökum í $R$ þ.a. \[
  Ra\subsetneq Ra_1 \subsetneq Ra_2 \subsetneq Ra_3 \subsetneq \cdots
  \]
  En þá er $\mathfrak A=\bigcup_{k\in\N} Ra_k$ íðal í $R$. Nú er $R$
  höfuðíðalbaugur, svo að til er stak $c$ úr $\mathfrak A$ þannig að $\mathfrak
  A = cR$. En þá er $c$ stak í einhverju íðalanna $Ra_k$, svo að 
  $Rc\subset Ra_k \subset Ra_j \subset Rc$, svo  $Ra_k = Ra_j$, fyrir öll
  $j\geq k$, sem er \emph{mótsögn}.
\end{sonnun}

\section{Þáttun í margliðubaugum}
\begin{fylgisetn}
  (1) Ef $K$ er svið, þá er $K[X]$ þáttabaugur.

  (2) Gauss-talnabaugurinn $\Z[i]$ er þáttabaugur.
\end{fylgisetn}
\begin{ath}
  (1) Sáum að í höfuðíðalbaug er sérhvert óþættanlegt stak frumstak. Þetta
  gildir almennar í öllum þáttabaugum: Ef $p$ er óþættanlegt og hefur frumþáttun
  $p=up_1\cdots p_r$, þar sem $u$ er eind og $p_1,\dots,p_r$ frumstök, þá er
  nauðsynlega $r=1$ og $p=up_1$, svo að $p$ er tengt frumstaki og því frumstak.

  (2) Ef $K$ er svið og $P$ er óþættanleg margliða í $K[X]$, þá er $P$ frumstak
  í $K[X]$ og $\langle P\rangle = PK[X]$ er háíðal, svo að $K[X]/PK[X]$ er svið.
  Athugum: Ef $\stig P$ er 2 eða 3, og $P$ er þættanlegt, þá er annar þátturinn
  af stigi $1$ og hefur því núllstöð, svo að $P$ hefur núllstöð. Sjáum því að
  fyrir margliður af stigi 2 eða 3 gildir: $K[X]/PK[X]$ er svið þ.þ.a.a.
  $P$ hafi enga núllstöð.
\end{ath}
Fáum athyglisverða niðurstöðu:
\begin{fylgisetn}
  Látum $K$ vera svið og $P\in K[X]$, $\stig P\geq 1$. Þá er til svið $L$ þ.a.
  $K$ er hlutsvið í $L$ og $P$ hafi núllstöð í $L$.
\end{fylgisetn}
\begin{sonnun}
  Látum $P_1$ vera einn frumþátt margliðunnar $P$ í $K[X]$. Það nægir að sýna að
  $P_1$ hafi núllstöð í stærra sviði; hún er þá líka núllstöð fyrir $P$. Skv.
  athugasemd er $L_1 := K[X]/P_1K[X]$ svið. Látum $c := X+P_1 K[X]\in L_1$. Þá
  er $P_1(c)=P_1(X)+P_1K[X]=0$.\footnotemark Nú er $\varphi:K\to
  L_1,\varphi(a):=a+P_1K[X]$ sviðamótun (þ.e. einbaugamótun milli sviða) og því
  eintæk, því $\Ker\varphi\neq K$ vegna $1\notin\Ker\varphi$ og $K$ hefur engin
  íðul nema $\left\{ 0 \right\}$ og $K$, svo $\Ker\varphi=\left\{ 0
  \right\}$. Þá má nota $\varphi$ til að samsama $K$ við myndina $\im\varphi$,
  sem er hlutsvið í $L_1$; fáum þannig svið $L$ sem hefur $K$ sem hlutsvið þ.a.
  $P_1$ hafi núllstöð í $L$.
\end{sonnun}
\footnotetext{Að þessi jafna (nánar tiltekið fyrra jafnaðarmerkið)
gildi er ekki alveg augljóst, sjá nánar tilsvarandi sönnun bls. 354-355 í
kennslubók.}
\begin{daemi}
  $X^2+1$ hefur ekki núllstöð í $\R$, svo að $\R[X]/(X^2+1)\R[X]$ er svið þar
  sem $X^2+1$ hefur núllstöð. Athugum: Höfum vörpun $\varphi:\R[X]\to\C,P\mapsto
  P(i)$, og mynd hennar er $\R[i]=\C$ og $\Ker\varphi=(X^2+1)\R[X]$: Ef
  $P\in\Ker\varphi$, þ.e. $P(i)=0$, þá gengur $X-i$ upp í $P$ í $\C[X]$; en þar
  sem $P$ hefur rauntalnastuðla gengur $X+i$ líka upp í $P$, svo að
  $X^2+1 = (X-i)(X+i)$ gengur upp í $P$, þ.e. $P\in(X^2+1)\R[X]$. Því er
  \[
  \C\cong\im\varphi/\Ker\varphi = \R[X]/(X^2+1)\R[X].
  \]
  
  Ef tvinntölurnar væru óþekktar, þá mætti nota þetta til að skilgreina þær út
  frá $\R$.
\end{daemi}
\begin{ath}
  Ef við skrifum $a\sim b$ til að tákna að $a$ og $b$ séu tengd, þá eru þetta
  jafngildisvensl; köllum jafngildisflokkana
  \emph{tenglsaflokka}\index{tengslaflokkur}. Í baugnum $\Z$ er tengslaflokkur
  tölunnar $a$ mengið $\left\{ a,-a \right\}$; hann inniheldur nákvæmlega eina
  náttúrlega tölu. Láum $R$ vera þáttabaug og $P$ vera mengi sem inniheldur
  nákvæmlega eitt stak úr hverjum tengslaflokki \emph{frumstaka} í $R$, og engin
  önnur stök. Þá getum við kallað $P$ \emph{fulltrúamengi fyrir
  frumstökin}\index{fulltrúamengi frumstaka}\index{frumstök} í $R$. Mengi
  frumtalnanna er fulltrúamengi fyrir frumstökin í $\Z$. 
  
  Látum $K$ vera svið; eindirnar í $K[X]$ eru margliður af stigi $0$; það eru
  föstu margliðurnar nema núllmargliðan, m.ö.o. eindirnar í $K$ (ef við samsömum
  stak í $K$ við tilsvarandi fasta margliðu í $K[X]$). Nú er ljóst: Sérhver
  tengslaflokkur margliðu $P\neq 0$ inniheldur nákvæmlega eina margliðu þ.a.
  forystustuðullinn sé $1$.
\end{ath}
\begin{skilgr}
  Margliða er sögð vera \emph{stöðluð}\index{stöðluð
  margliða}\index{margliða!stöðluð} ef hún er ekki núllmargliðan og
  forystustuðull hennar er $1$.
\end{skilgr}
Óþættanlegar staðlaðar margliður mynda fulltrúamengi fyrir frumstökin í
$K[X]$ ef $K$ er svið.

Ef við höfum fulltrúamengi $P$ fyrir frumstöki í þáttabaug $R$ þá má þátta
sérhvert stak $a\neq 0$ í $R$ sem $a=up_1\cdots p_r$ með $p_1,\dots,p_r\in P$,
$u$ eind; þessi þáttun ákvarðast ótvírætt burtséð frá röð. Almennt er engin
kerfisbundin aðferð til að búa til svona fulltrúamengi\index{fulltrúamengi}, þau
eru alltaf til skv. frumsendu um val. Í $\Z[i]$ er fyrir $a\neq 0$
tengslaflokkurinn $\left\{ a,-a,ia,-ia \right\}$ og allar aðferðir til að velja
eitt stak framar öðrum eru ekki sjálfgefnar.
\begin{skilgr}
  Látum $R$ vera heilbaug og $a_1,\dots,a_n\in R$. Við segjum að stak $d\in R$
  sé \emph{stærsti samdeilir}\index{stærsti samdeilir} stakanna $a_1,\dots,a_n$
  ef
  \begin{enumerate}[(i)]
    \item $d\mid a_k$ fyrir $k=1,\dots,n$ og
    \item ef $c\in R$, $c\mid a_k$ fyrir $k=1,\dots,n$, þá $c\mid d$. 
  \end{enumerate}
  Segjum að stak $m\in R$ sé \emph{minnsta samfeldi}\index{minnsta samfeldi}
  $a_1,\dots,a_n$ ef
  \begin{enumerate}[(i')]
    \item $a_k\mid m$ fyrir $k=1,\dots,n$ og
    \item ef $c\in R$, $a_k\mid c$ fyrir $k=1,\dots,n$, þá $m\mid c$. 
  \end{enumerate}
\end{skilgr}
\begin{ath}
  Almennt þarf stærstur samdeilir ekki að vera til; en ef hann er til, þá
  ákvarðast tengslaflokkur hans ótvírætt; segjum að hann ákvarðist ótvírætt
  burtséð frá tengslum. Sama gildir um minnsta samfeldi. Í þáttabaug eru stærsti
  samdeilir og minnsta samfeldi alltaf til: Látum $a_1,\dots,a_m\in
  R\setminus\left\{ 0 \right\}$, skrifum frumþáttun $a_k$ sem 
  \[
  a_k = u_k p_1^{n(k,1)}\cdots p_r^{n(k,r)}
  \]
  með $n(k,m)\geq 0$; getum valið sömu $p_1,\dots,p_r$ með því að leyfa $0$-ta
  veldi frumþáttar. Þá er 
  \[
  d := p_1^{i_1}\cdots p_r^{i_r}
  \]
  þar sem $i_k = \min\left\{ n(k,1),\dots,n(k,r) \right\}$, stærsti samdeilir,
  og 
  \[
  m := p_1^{j_1}\cdots p_r^{j_r}
  \]
  þar sem $j_k := \max\left\{ n(k,1),\dots,n(k,r) \right\}$ minnsta samfelldi.
\end{ath}
\begin{setn}
  Látum $R$ vera höfuðíðalbaug og $a_1,\dots,a_m\in R$. Stak $d$ í $R$ er
  stærstur samdeilir $a_1,\dots,a_m$ þ.þ.a.a. \[
  dR = a_1R + \cdots + a_m R.
  \]
  Stak $g$ í $R$ er minnsta samfeldi $a_1,\dots,a_m$ þ.þ.a.a. \[
  gR = a_1 R\cap \cdots \cap a_m.
  \]
\end{setn}
\begin{sonnun}
  Eins og fyrir heilar tölur!
\end{sonnun}
\begin{ath}
  Ef $b$ er stærstur samdeilir $a_1$ og $a_2$ og $d$ er stærstur samdeilir
  $b$ og $a_3$, þá er $d$ stærstur samdeilir $a_1,a_2,a_3$. Það dugar því
  að geta fundið stærstan samdeili \emph{tveggja} staka. Í \emph{evklíðskum}
  baug má finna stærstan samdeili tveggja staka með \emph{reikniriti
  Evklíðs}\index{Evklíðs!reiknirit} alveg eins og fyrir heilar tölur.
\end{ath}
\begin{ath}
  Sanna má: Ef $U$ er svæði í $\C$ (þ.e. opið og samanhangandi hlutmengi),
  og $R=O(U)$ er baugur allra fágaðra falla á $U$, þá er $R$ ekki
  þáttabaugur, en sérhver (endanleg) fjölskylda af stökum  í $R$ hefur
  stærstan samdeili.
\end{ath}
\begin{skilgr}
  Látum $R$ vera þáttabaug og $F = \sum_{k=0}^n a_k X^k \in R[X]$. Stærstur
  samdeilir stuðlanna $a_0,\dots,a_n$ kallast \emph{innihald}\index{innihald
  margliðu}\index{margliða!innihald} margliðunnar $F$. Margliðan $F$ kallast
  \emph{frumstæð}\index{frumstæð margliða}\index{margliða!frumstæð} ef
  $1$ er innihald hennar.
\end{skilgr}
\begin{ath}
  (1) Innihald margliðu í $R[X]$ er skilgreint ótvírætt \emph{burtséð frá tengslum}.

  (2) Ef $F$ er margliða í $R[X]$, $R$ þáttabaugur, þá má skrifa $F=cF_1$ þar
  sem $F_1$ er frumstæð og $c$ er innihald $F$; öfugt ef $F=cF_1$, þar sem $c\in
  R$ og $F_1$ er frumstæð, þá er $c$ innihald $F$.
\end{ath}
\begin{ath}
  Látum $\phi: R\to S$ vera baugamótun, þá fæst baugamótun
  \[ \hat\phi : R[X] \to S[X] \]
  þannig að
  \[ \phi
  \left(
    \sum_{k=0}^n a_k X^k
  \right)
  := \sum_{k=0}^n \phi(a_k)X^k.
  \]
  Sér í lagi ef $\mathfrak A$ er íðal í $R$, þá fæst baugamótun
  \[
  R[X]\to R/\mathfrak A[X],
  \qquad
  \sum_{k=0}^n a_kX^k \mapsto \sum_{k=0}^n[a_k]X^k,
  \]
  þar sem $[a_k] := a_k + \mathfrak A$ er náttúrlega ofanvarpið af $a_k$.
\end{ath}
\begin{hjalparsetn}
  [Gauss]
  Ef $R$ er þáttabaugur og $F,G$ eru frumstæðar margliður í $R[X]$, þá er
  margfeldið $FG$ frumstæð margliða.
\end{hjalparsetn}
\begin{fylgisetn}
  Ef $F,G\in R[X]$, $R$ þáttabaugur, $c$ er innihald $F$ og $d$ er innihald $G$,
  þá er $cd$ innihald $FG$.
\end{fylgisetn}
\begin{sonnun}
  [Sönnun á hjálparsetningu]
  Gerum ráð fyrir að $FG$ sé ekki frumstæð; þá er til frumstak $p$ í $R$ sem
  gengur upp í öllum stuðlum $FG$. Táknum með $[P]$ mynd margliðu $P$ í $R[X]$ í
  baugnum $R/pR[X]$.  Við höfum þá $0 = [FG] = [F]\cdot[G]$; nú er $R/pR$ svið
  svo að $R/pR[X]$ er heilbaugur og því er $[F]=0$ eða $[G]=0$; segjum að
  $[F]=0$. En það þýðir að $p$ gengur upp í öllum stuðlum $F$, og þá er hún ekki
  frumstæð!
\end{sonnun}
\begin{sonnun}
  [Sönnun á fylgisetningu]
  Skrifum $F=cF_1,G=dG_1$ þar sem $F_1,G_1$ eru frumstæðar, þá er $FG=cdF_1G_1$
  og $F_1G_1$ er frumstæð, og þá er $cd$ innihald $FG$.
\end{sonnun}
\begin{hjalparsetn}
  Ef $R$ er þáttabaugur, $K$ er brotasvið hans og $P$ er margliða í $R[X]$ sem
  er óþættanleg í $R[X]$, þá er hún óþættanleg í $K[X]$.
\end{hjalparsetn}
\begin{sonnun}
  Sýnum: Ef $P$ þáttast í $K[X], P=FG$ með $F,G\in K[X]$, $\stig F\geq 1,\stig
  G\geq 1$, þá þáttast $P$ í $R[X]$. Stuðlarnir í $F$ og $G$ eru brot $\frac ab$
  þar sem $a,b\in R$; látum $a$ vera margfeldi allra nefnaranna sem koma fyrir í
  öllum stuðlunum; þá er $a\neq 0$ og $aG\in R[X]$. Skrifum nú $aF = cF_1$ og
  $aG=dG_1$, þar sem $F_1$ og $G_1$ eru frumstæðar margliður í $R[X]$. Þá er
  $a^2 FG = cd F_1 G_1$; skv. Gauss er $F_1 G_1$ frumstæð. En nú er $FG = P =
  sP_1$ þar sem $P_1$ er frumstæð, $a^2sP_1 = cd F_1 G_1$; og þá er $a^2s$ tengt
  $cd$ í $R$, þ.e. $cd = a^2 r$, þar sem $r$ er tengt $s$ í $R$. Þ.e. $a^2 P =
  a^2 r F_1 G_1$, svo að $P = (rF_1)G_1$, sem er þáttun í $R[X]$. 
\end{sonnun}
\begin{ath}
  Margliða getur þáttast í $R[X]$, en verið óþættanleg í $K[X]$; t.d. er $2X+2 =
  2(X+1)\in\Z[X]$, þá eru $2,X+1$ ekki eindir í $\Z[X]$; svo að $2X+2$ er
  þáttanlegt í $\Z[X]$, en ekki í $\Q[X]$, því $2$ er eind í $\Q$ og því í
  $\Q[X]$. Hins vegar gildir
\end{ath}
\begin{setn}
  Látum $R$ vera þáttabaug, $P$ vera \emph{frumstæða} margliðu í $R[X]$. Þá er jafngilt:
  \begin{enumerate}[(i)]
  \item $P$ er óþáttanleg í $R[X]$,
  \item $P$ er óþáttanleg í $K[X]$,
  \item $P$ er frumstak í $K[X]$,
  \item $P$ er frumstak í $R[X]$.
  \end{enumerate}
\end{setn}
\begin{sonnun}
  (i)$\Rightarrow$(ii) er sértilfelli af síðustu hjálparsetningu.
  (ii)$\Rightarrow$(iii) er þekkt, því $K[X]$ er höfuðíðalbaugur.
  (iv)$\Rightarrow$(i) er líka þekkt. Eftir stendur (iii)$\Rightarrow$(iv):
  Gerum ráð fyrir að $P$ sé frumstak í $K[X]$. Látum $F,G\in R[X]$ vera þ.a.
  $P\mid FG$ í $R[X]$. Þá er líka $P\mid FG$ í $K[X]$, svo að $P\mid F$ eða
  $P\mid G$, segjum $P\mid F$ í $K[X]$. Þá má skrifa $F = PQ$, þar sem $Q\in
  K[X]$. Nú er til stak $a$ í $R$, $a\neq 0$, þ.a. $aQ\in R[X]$. Skrifum þá $aQ
  = cQ_1$, þar sem $Q_1$ er frumstæð margliða í $R[X]$. Höfum þá $aF = P\cdot
  aQ= cPQ_1$. En skv. Gauss er $PQ_1$ frumstæð, svo að $a\mid c$ (því að skrifa
  má $F = dF_1$, $F_1$ frumstæð, og þá $adF_1 = cPQ_1$, svo að $c$ er tengt
  $ad$). Skrifum $c = ab$; höfum þá $aF = abPQ_1$ og vegna $a\neq 0$ er $F =
  P\cdot bQ_1$, $bQ_1\in R[X]$ svo að $P$ gengur upp í $F$ í $R[X]$.
\end{sonnun}
\begin{fylgisetn}
  Látum $R$ vera þáttabaug. Frumstökin í $R[X]$ eru frumstökin í $R$ (sem við
  lítum á sem fastar margliður) og óþættanlegu frumstæðu margliðurnar í $R[X]$.
\end{fylgisetn}
\begin{setn}
  Ef $R$ er þáttabaugur, þá er $R[X]$ þáttabaugur.
\end{setn}
\begin{sonnun}
  Látum $F\in R[X]$, $F\neq 0$, skrifum $F = cF_1$ þar sem $c$ er innihald $F$,
  við getum þáttað það í frumþætti í $R$; og það nægir að sýna að $F_1$ sé
  margfeldi af frumstæðum margliðum í $R[X]$. Þrepum yfir $\stig F$; ljóst ef
  $\stig F = 0$ eða $1$. G.r.f. að $\stig F\geq 2$. Ef $F$ er óþættanlegt er
  ekkert að sýna; annars þáttum við $F = P_1Q_1$, þar sem $P_1Q_1$ eru ekki
  eindir; þær eru frumstæðar margliður skv. Gauss; og þær hafa þá minna stig en
  $F$; skv. þrepunarforsendu er hvor um sig margfeldi af óþættanlegum frumstæðum
  margliðum.
\end{sonnun}
%%
%% 25. nóvember 2009
%%
% Af hverju þetta aftur?
%\begin{fylgisetn}
%  Látum $R$ vera þáttabaug. Frumstökin í $R[X]$ eru frumstökin í $R$ (sem við
%  lítum á sem fastar margliður) og óþættanlegu margliðurnar í $R[X]$.
%\end{fylgisetn}
%\begin{setn}
%  Ef $R$ er þáttabaugur, þá er $R[X]$ þáttabaugur.
%\end{setn}
%\begin{sonnun}
%  Látum $F\in R[X]$, $F\neq 0$, skrifum $F=cF_1$ þar sem $c$ er innihald $F$;
%  við getum þáttað það í frumþætti í $R$; og það nægir að sýna að $F_1$ sé
%  margfeldi af frumstæðum margliðum í $R[X]$. Þrepum yfir $\stig F$: Þetta er
%  ljóst ef $\stig (F) = 0$ eða $\stig(F) = 1$. Gerum ráð fyrir að $\stig(F)\geq
%  2$. Ef $F$ er óþættanlegt þarf ekkert að sýna, annars þáttum við $F = P_1 Q_1$
%  þar sem $P_1, Q_1$ eru ekki eindir; þær eru frumstæðar margliður skv. Gauss,
%  og þær hafa þá minna stig en $F$; skv. þrepunarforsendu er hvor um sig
%  margfeldi af óþættanlegum margliðum.
%\end{sonnun}
\begin{setn}
  [Eisenstein]\index{Eisenstein}
  Látum $F\in R[X]$, $F=\sum_{k=0}^n a_k X^k$ og gerum ráð fyrir að $p$ sé
  frumstak í $R$ þannig að $p\nmid a_n$, $p\mid a_j$ fyrir $j=0,\dots,n-1$ og
  $p^2 \nmid a_0$, þá er $F$ óþættanleg í $R[X]$.
\end{setn}
Sönnun bíður í bili.
\begin{skilgr}
  Látum $R$ vera víxlinn (ein)baug og skrifum $R[X_1]$ í stað $R[X]$; þá getum
  við myndað margliðubauginn \[
  R[X_1,X_2] := (R[X_1])[X_2] = R[X_1][X_2]
  \]
  yfir $R[X_1]$; og almennar má skilgreina
  \emph{margliðubauginn $R[X_1,\dots,X_n]$ yfir $R$ með $n$ óákveðnum
  stærðum}\index{margliðubaugur!með $n$ óákveðnum
  stærðum}\index{baugur!margliðubaugur}
  með þrepun þannig að \[
  R[X_1,\dots,X_n,X_{n+1}] := R[X_1,\dots,X_n][X_{n+1}].
  \]
\end{skilgr}
Fáum nú:
\begin{setn}
  Ef $R$ er þáttabaugur þá er margliðubaugurinn $R[X_1,\dots,X_n]$ líka
  þáttabaugur.
\end{setn}
\begin{sonnun}
  Augljóst með þrepun.
\end{sonnun}
\begin{ath}
  Sér í lagi er $\Z[X_1,\dots,X_n]$ þáttabaugur og fyrir sérhvert svið $K$ er
  $K[X_1,\dots,X_n]$ þáttabaugur. Fyrir $n\geq 2$ eru þetta ekki
  höfuðíðalbaugar (ath. að svið er þáttabaugur sem hefur engin frumstök)!
\end{ath}
Þessir þáttabaugar eru mikilvægir í algebru og algebrulegri rúmfræði.
\begin{ath}
  Notum stundum aðra bókstafi í stað $X_1,\dots,X_n$, t.d. skrifum við oft
  $K[X,Y]$ í stað $K[X_1,X_2]$.
\end{ath}
Eigum enn eftir að sanna setningu Eisensteins, til þess notum við
hjálparsetningu:
\begin{hjalparsetn}
  Látum $R$ vera víxlinn einbaug, $\mathfrak A$ vera íðal í $R$. Ef $\pi: R\to
  R/\mathfrak A$ er ofanvarpið og 
  \[
  \hat\pi:R[X]\to R/\mathfrak A[X],
  \quad
  \sum a_k X^k \mapsto \sum \pi(a_k)X^k
  \]
  er vörpunin sem $\pi$ gefur af sér og $P$ er margliða í $R[X]$ þ.a.
  $\hat\pi(P)$ sé ekki margfeldi af margliðum af minna stigi en $P$ og
  forystustuðull $P$ sé ekki í $\mathfrak A$, þá er $P$ ekki margfeldi
  margliða af minna stigi. 
\end{hjalparsetn}
\begin{sonnun}
  Ef $P = Q_1 Q_2$ með $\stig(Q_1),\stig(Q_2)\leq \stig(P)$,  þá er $\hat\pi(P)
  = \hat\pi(Q_1)\hat\pi(Q_2)$, $\stig(\hat\pi(Q_j)) <
  \stig(P)=\stig(\hat\pi(P))$ (þar sem forystustuðull $P$ er ekki $\mathfrak A$
  varpast hann ekki í núll, svo stigið er það sama).
\end{sonnun}
\begin{sonnun}
  [Sönnun á Eisenstein]
  Látum $\pi : R\to R/pR$ vera ofanvarpið og $\hat\pi : R[X]\to R/pR[X]$ vera
  samsvarandi vörpun. Vegna $p\nmid a_n$ er $\stig(\hat\pi(P)) = \stig(P) = n$.
  Vegna $p\mid a_j$ fyrir $j < n$ er $\hat\pi(P) = \pi(a_n)X^n$; ef $P$ væri
  þættanlegt í $K[X]$, þá mætti skrifa $P=Q_1Q_2$, þar sem $Q_1,Q_2$ hafa minna
  stig en $P$, segjum $Q_1 = \sum_{k=0}^m b_k X^k$, $Q_2 = \sum_{k=0}^l c_k
  X^k$, þar sem $m,l < n$ og $m+l = n$ (þetta er hægt því við gætum tekið
  innihald úr $P = cP_1$ og þáttað $P_1$). En þá er \[
  \hat\pi(Q_1) \hat\pi(Q_2) = \hat\pi(P) = \pi(a_n) X^n,
  \]
  og $R/pR[X]$ er þáttabaugur (því þetta er svið) þannig að $\hat\pi(Q_1) = d_1
  X^m$, $\hat\pi (Q_2) = d_2 X^l$ fyrir einhver stök $d_1,d_2\in R/pR$. En það
  þýðir að $p \mid b_0$ og $p\mid c_0$; en $a_0 = b_0 c_0$, svo að $p^2\mid a_0$
  í mótsögn við forsendu. 
\end{sonnun}
Af hjálparsetningunni að ofan leiðir líka
\begin{setn}
  Ef $P\in\Z[X]$ og $p$ er frumtala, $\pi:\Z\to\Z/p\Z$ er náttúrlega ofanvarpið
  og $\hat\pi(P)$ er óþættanleg margliða í $\Z/p\Z[X]$, en af sama stigi og $P$,
  þá er $P$ óþættanleg í $\Q[X]$. 
\end{setn}
\begin{daemi}
  (1) Látum $p$ vera frumtölu. Þá er margliðan \[
  F := X^{p-1} + X^{p-2} + \cdots + X + 1
  \]
  óþættanleg í $\Q[X]$: Athugum að\[
  X^p - 1 = (X-1)(X^{p-1} + \cdots + X + 1).
  \]
  Ef $F$ væri þáttanleg, þá væri $F(X+1)$ það líka, en við höfum
  \begin{align*}
    XF(X+1) 
    &= (X+1)^p - 1 \\
    &= X^p + \binom p1 X^{p-1} + \binom p2 X^{p-2} + \cdots + \binom p1 X\\
    &= X\left( X^{p-1} + \sum_{k=1}^{p-1}\binom pk X^{p-k-1}  \right)
  \end{align*}
  svo að \[
  F(X+1) = X^{p-1} \sum_{k=1}^{p-1}\binom pk X^{p-k-1}.
  \]
  En við vitum að fyrir frumtölu $p$ gildir $p\mid \binom pk$ fyrir
  $k=1,\dots,p-1$ og $p^2\nmid\binom p1 = p$ (vitum að í baug með kennitölu $p$,
  sem er frumtala, gildir að $(x+y)^p = x^p + y^p$, sem var sannað á heimadæmi);
  svo skv. Eisenstein er $F(X+1)$ og þar með $F$ óþættanleg í $\Q[X]$.

  (2) Um Eisenstein: Sýnum að margliðan \[
  P := X^2 + Y^2 + 1
  \]
  er óþættanleg í $\C[X,Y]$: Lítum á $\C[X,Y]$ sem $\C[X][Y]$; getum skrifað \[
  P = Y^2 + (X-i)(X+i).
  \]
  Nú er $(X_i)$ frumstak í $\C[X]$, gengur ekki upp í forystustuðul $P$ (sem
  margliðu í $Y$), en það gengur upp í fastastuðulinn $(X-i)(X+i)$, en ekki
  tvisvar! Þar með er $P$ óþættanleg skv. Eisenstein.
\end{daemi}
\begin{ath}
  Undirstöðusetning algebrunnar segir að sérhver margliða í $\C[X]$ hafi
  núllstöð í $\C$ ef hún er ekki föst. Af því leiðir að frumþáttun í $\C[X]$ er
  af gerðinni 
  \[
  P = u(X-c_1)(X-c_2)\cdots(X-c_n)
  \]
  þar sem $u\in\C\setminus\left\{ 0 \right\}$ og $c_1,\dots,c_n\in\C$. Stöðluðu
  margliðurnar $X-c$ með $c\in\C$ mynda fulltrúamengi fyrir frumstökin í $\C$. 
  
  \emph{Af þessu leiðir:} Margliður í $\R[X]$ má þátta í línulegar margliður í
  $\C[X]$; ef núllstöð $c\in\C$ er ekki í $\R$, þá er $\bar c$ líka núllstöð, og
  við getum skrifað margliðu $(X-c)(X-\bar c)P$, þar sem $P\in\C[X]$; \[
  (X-c)(X-\bar c) = X^2 - 2\re c + |c|^2 \in \R[X],
  \]
  svo að $P$ fæst með því að deila margliðu í $\R[X]$ upp í upphaflegu
  margliðuna, þar með er $P\in\R[X]$. Þrepun gefur að sérhverja margliðu í
  $\R[X]$ má skrifa sem
  \[
  u P_1\cdots P_r
  \]
  þar sem $u\in\R^* = \R\setminus\left\{ 0 \right\}$ og $P_j$ er annaðhvort af
  gerðinni $X-c$ með $c\in\R$ eða margliða $X^2 + bX+c$ af öðru stigi án
  núllstöðva í $\R$, sem þýðir að $b^2 - 4c < 0$. Þessar margliður $X-c$ og
  $X^2-bX+c,b^2-4c<0$, mynda því fulltrúamengi fyrir frumstökin í $\R[X]$. 
\end{ath}
\emph{Chiroff}-setningar í bókinni eru undirstöðusetningar grúpufræði, sem væri
áhugavert að skoða, en vegna tímaskorts lítum við heldur á efni næsta kafla.

\chapter{Frumþáttun í Gauss-talnabaugnum $\Z[i]$}

Höfum vörpun \[
N:\Z[i]\to\N, 
\quad
z=x+iy \mapsto |z|^2 = z\bar z = x^2 + y^2,
\]
um öll $z,w\in\Z[i]$ gildir \[
N(zw) = N(z)N(w)
\]
og við vitum að $N$ gerir $\Z[i]$ að evklíðskum baug, þar með er hann
höfuðíðalbaugur og þáttabaugur. Við segjum að náttúrleg tala $n$ sé
\emph{summa tveggja ferninga}\index{summa ferninga} ef til eru $x,y\in\Z$ þannig
að $n = x^2 + y^2$; jafngilt er að til sé $z\in\Z[i]$ þannig að $n = N(z)$.

Ætlum að skoða þáttun í þessum baug og sjáum þá nákvæmlega hvaða heilar tölur eru
summa tveggja ferninga!

%%
%% 26. nóvember 2009
%%

Sjáum: Ef $z\mid w$ í $\Z[i]$, þá gildir $N(z)\mid
N(w)$ í $\N$. Af því leiðir: Ef $N(z)=p$ er frumtala, þá er $z$ frumstak í
$\Z[i]$, annars mætti skrifa $z=uv$ þar sem $u,v\in\Z[i]\setminus\left\{0
\right\}$ og $N(u)\neq 1$; og þá fengist $p = N(u)N(v)$ með $N(u),N(v)\geq 2$,
sem fær ekki staðist.

Viljum ákvarða frumstökin í $\Z[i]$: Látum $\pi$ vera frumstak í $\Z[i]$, þá
höfum við $\pi\mid\pi\overline\pi=N(\pi)\geq 2$; nú má þætta $N(\pi)$ í frumþætti í
$\N$; svo að $\pi$ verður að ganga upp í einum frumþættinum svo að til er
frumtala $p$ þ.a. $\pi\mid p$ í $\Z[i]$. Það getur ekki verið til nema ein slík
frumtala $p$, ef $p,q$ væru frumtölur þ.a. $p\neq q$ og $\pi\mid p$, $\pi\mid q$
í $\Z[i]$, þá mætti finna heilar tölur $k,j$ sem leysa Bézout-jöfnuna $1=pq+qj$;
og við fengjum þá að $\pi\mid 1$ í $\Z[i]$, sem er fráleitt. Við fáum þá öll
frumstök í $\Z[i]$ með því að finna frumþætti venjulegra frumtalna í baugnum
$\Z[i]$. Athugum þrjú tilvik:

\emph{1. tilvik, $p=2$:} Við höfum 
\[
2 = (-i)(1+i)^2
\]
og þetta er frumþáttun í $\Z[i]$, því að $-i$ er eind, og $N(1+i)=2$ er
frumtala, svo að $1+i$ er frumstak í $\Z[i]$.

\emph{2. tilvik, $p\equiv 3\pmod{4}$}: Gerum ráð fyrir að $\pi$ sé frumstak í
$\Z[i]$ þ.a. $\pi\mid p$. Þá fæst $N(\pi)\mid N(p) = p^2$; nú er $N(\pi)\neq 1$,
svo að $N(\pi)$ er annaðhvort $p$ eða $p^2$. Skrifum $\pi=x+iy$, þá er
$N(\pi)=x^2+y^2$ og þetta getur aldrei verið jafnt $p$, því að $x^2\equiv
0\pmod 4$ ef $x^2\equiv 1\pmod 4$ ef $x$ er oddatala, svo að $x^2+y^2 \equiv
0,1$ eða $2\pmod 4$, og því er $x^2+y^2 \neq p$. Þar með er $N(\pi)=p^2=N(p)$.
En þá eru $\pi$ og $p$ tengd í $\Z[i]$: Ef við skrifum $p = u\pi$, þá fæst
$N(p)=N(u)N(\pi)$, svo að $N(u)=1$ og því er $u$ eind í $\Z[i]$. Þetta þýðir að
$p$ er frumstak í $\Z[i]$.
\begin{ath}
  Sýndum: Frumtala $p$ þ.a. $p\equiv 3\pmod 4$ er ekki summa tveggja ferninga.
\end{ath}

\emph{3. tilvik, $p\equiv 1\pmod 4$}: Skrifum þá $p=2n+1$, þar sem $n$ er
\emph{jöfn} tala. Þá er
\begin{align*}
  (p-1)!
  &= (2n)! \\
  &= n!(p-1)(p-2)\cdots (p-n) \\
  &\equiv n! (-1)(-2)\cdots(-n) \\
  &\equiv n! (-1)^n \cdot n! \\
  &\equiv (n!)^2 \pmod p.
\end{align*}
Nú segir setning Wilsons að $(p-1)!\equiv -1 \pmod p$. Því er fyrir $x:=n!$ 
\[ p\mid x^2 + 1 = (x-i)(x+i) .\] 
Þá fæst líka $\pi \mid (x-i)(x+i)$ svo $\pi \mid x-i$ eða $\pi \mid x+i$ fyrir
frumstak $\pi\in\Z[i]$ sem gengur upp í $p$. Hins vegar er ljóst að $p$ gengur
hvorki upp í $x+i$ né $x-i$ í $\Z[i]$. Ef $x+i=p(u+vi)=pu+pvi$ með $u,v\in\Z$ þá
fengist $p\mid 1$ í $\N$, sem er fráleitt! Það þýðir að fyrir frumstak $\pi$ í
$\Z[i]$ sem gengur uppí $p$ í $\Z[i]$ geta $\pi$ og $p$ ekki verið tengd í
$\Z[i]$; þá er $N(\pi)\neq N(p) = p^2$, en $N(\pi)\mid p^2$, svo að $N(\pi) =
p$. En þá er líka $N(\overline \pi)=p$, svo að $\overline\pi$ er frumstak í $\Z[i]$ og 
\[ p = N(\pi) = \pi\overline\pi, \]
og þetta er frumþáttun í $\Z[i]$. Stökin $\pi$ og $\overline \pi$ geta ekki verið
tengd í $\Z[i]$: Skrifum $\pi = x+yi$; ef $\pi=\overline \pi$, þá væri $y=0$ og
$p=N(\pi)=x^2$, sem er fráleitt. Ef $\pi = -\overline \pi$, þá væri $x=0$ og
$p=N(\pi)=y^2$. Ef $\pi=i\overline \pi=-y+ix$, þá væri $p=N(\pi)=2x^2$, sama fæst ef
$\pi = -i\overline\pi$.
\begin{ath}
  Sjáum: Ef $p$ er frumtala þ.a. $p\equiv 1\pmod 4$, þá er $p$ summa tveggja
  teninga.
\end{ath}
Fáum:
\begin{setn}
  Frumstökin í $\Z[i]$ eru (burstéð frá tengslum) eftirfarandi
  \begin{enumerate}[(i)]
    \item Talan $1+i$,
    \item sérhver frumtala $p$ þ.a. $p\equiv 3\pmod 4$,
    \item fyrir sérhverja frumtölu $p$ þ.a. $p\equiv 1\pmod 4$ tvö ótengd
      frumstök $\pi,\overline\pi$ þ.a. $\pi\overline\pi = p$.
  \end{enumerate}
\end{setn}
Athugum að $2=1^2+1^2$, svo að $2$ er summa tveggja ferninga. Sjáum:
\begin{setn}
  Frumtala $p$ er summa tveggja ferninga þ.þ.a.a $p=2$ eða $p\equiv 1\pmod 4$.
\end{setn}
\qed
\begin{setn}
  Náttúrleg tala $n$ þ.a. $n\geq 1$ er summa tveggja ferninga þ.þ.a.a
  veldisvísir allra frumþátta $p$ í frumþáttun tölunnar $n$ (í $\N$) þannig að
  $p\equiv 3\pmod 4$ sé jöfn tala.
\end{setn}
\begin{sonnun}
  Gerum ráð fyrir að $n=x^+y^2$ með $x,y\in\Z$. Þá má skrifa frumþáttun staksins
  $x+iy$ í $\Z[i]$ sem\[
  x+iy
  = u(1+i)^h
    \pi_1^{j_1} \overline\pi_1^{k_1}\cdots \pi_r^{j_r}\overline\pi_r^{k_r} 
    q_1^{l_1}\cdots q_s^{l_s}
  \]
  þar sem $u$ er eind í $\Z[i]$ þ.a. $u\overline u=1$; fyrir $\rho=1,\dots,r$ er
  $\pi_\rho$ frumstak í $\Z[i]$ þ.a. $\pi_\rho\overline\pi_\rho = p_\rho$ þar sem
  $p_\rho$ er frumtala þ.a. $p_\rho\equiv 1\pmod 4$ og fyrir $\sigma=1,\dots,s$
  er $q_\sigma$ frumtala þ.a. $q_\sigma \equiv 3\pmod 4$. En þá er 
  \begin{align*}
    n
    &= (x+iy)\overline{(x+iy)}
    \\
    &= 2^h p_1^{j_1+k_1}\cdots p_r^{j_1+k_1} q_1^{2l_1}\cdots q_s^{2l_s}.
  \end{align*}
  \emph{Öfugt,} ef skrifa má 
  \[
  n = 2^h p_1^{j_1} \cdots p_r^{j_r}\cdot q_1^{2l_1}\cdots q_s^{2l_s}
  \]
  þar sem $p_\rho$ eru frumtölur þ.a. $p_\rho \equiv 1\pmod 4$ fyrir
  $p=1,\dots,r$ og $q_\sigma \equiv 3\pmod 4$ fyrir $\sigma = 1,\dots,s$, þá
  setjum við 
  \[
  z := (1+i)^2 \pi_1^{j_1}\cdots\pi_r^{j_r} q_1^{l_1}\cdots q_s^{l_s}
  \]
  þar sem $\pi_\rho$ er frumstak í $\Z[i]$ þ.a. $\pi_\rho\overline\pi_\rho =
  p_\rho$ fyrir $\rho = 1,\dots,r$; fáum 
  \begin{align*}
    N(z) 
    &= N(1+i)^h 
       N(\pi_1)^{j_1} \cdots N(\pi_r)^{j_r}
       N(q_1)^{l_1}\cdots N(q_s)^{l_s}
    \\
    &= 2^h p_1^{j_1}\cdots p_r^{j_r}
       q_1^{2l_1}\cdots q_s^{2l_s};
  \end{align*}
  svo að 
  \[
  n = N(z) = x^2 + y^2
  \]
  ef $z = x+iy$ með $x,y\in\Z$.
\end{sonnun}

\printindex
\end{document}
