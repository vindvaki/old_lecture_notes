\documentclass[a4paper,icelandic,11pt]{book}
\usepackage[T1]{fontenc}
\usepackage[utf8]{inputenc}
\usepackage{babel}

\usepackage{lmodern}

\usepackage{enumerate}
\usepackage{makeidx}
\makeindex

\usepackage{hyperref}
\hypersetup{bookmarks=true}

\usepackage{amsmath}
\usepackage{amssymb}

%% Theorem
\usepackage{amsthm}
\theoremstyle{plain}      \newtheorem{setn}{Setning}[chapter]
                          \newtheorem{lemma}[setn]{Hjálparsetning}
                          \newtheorem{fylgi}[setn]{Fylgisetning}
\theoremstyle{definition} \newtheorem{skilgr}[setn]{Skilgreining}
                          \newtheorem{daemi}[setn]{Dæmi}
\theoremstyle{remark}     \newtheorem*{ath}{Athugasemd}

%% Ný tákn
\newcommand{\R}{\mathbb R}
\newcommand{\C}{\mathbb C}
\newcommand{\Q}{\mathbb Q}
\newcommand{\N}{\mathbb N}
\newcommand{\Z}{\mathbb Z}
\DeclareMathOperator{\id}{id}
\DeclareMathOperator{\Ker}{Ker}
\DeclareMathOperator{\im}{Im}
\DeclareMathOperator{\re}{Re}

\title{\textbf{Mál- og tegurfræði}\\Fyrirlestrar Reynis Axelssonar}
\author{Hörður Freyr Yngvason}
\date{}

\begin{document}

\maketitle
\tableofcontents

Engin ákveðin kennslubók. Til er texti á íslensku eftir Jón Ragnar
Stefánsson sem er víst mjög ítarlegur. Á ensku kallast efni
námskeiðsins \emph{measure and integration}.

\chapter{Inngangur}
%%
\marginpar{10. jan.}
%%
Hvað er \emph{rúmmál} (flatarmál, lengd)?

\subsection{Byrjum á lengd}
Við erum sammála um að \emph{lengd}\index{lengd} bilanna
\[
\left[a,b\right],
\left]a,b\right[,
\left[a,b\right[,
\left]a,b\right]
\]
sé $b-a$; hér eru $a,b\in\R$, $a<b$. Þá er eðlilegt að segja: Ef
$a<b<c<d$, þá sé lengd mengisins
\[
\left]a,b\right[ \cup \left]c,d\right[
\]
talan $(b-a)+(d-c)$. Látum $U$ vera opið hlutmengi í $\R$. Þá er $U$
sammengi samhengisþátta sinna sem eru allir opin bil,
$U=\bigcup_{k\in{K}}I_{k}$ þar sem $I_{k}$ eru opin bil,
$I_{k}\cap{I_{j}}\ne\emptyset$ ef $k\ne j$, og þau eru
\emph{teljanlega mörg}: Hvert þeirra inniheldur ræða tölu, þær eru
ólíkar og teljanlega margar. Þá er kannski eðlilegt að skilgreina
lengd mengisins $U$ sem summuna
\[
\lambda(U)=\sum_{k\in K}\lambda(I_{k}),
\]
þar sem $\lambda(I_{k})$ er lengd bilsins $I_{k}$. Hvað ef $U$ er ekki
opið?  Þá þarf $U$ ekki að vera sammengi af bilum \emph{nema} við
leyfum einstökunga. Við höfum alltaf $U=\bigcup_{x\in U}\{x\}$. En
eðlilegt er að segja að $\{x\}$ hafi lengd 0; og þá er $\sum_{x\in U}
\lambda(\{x\}) = 0$.
\begin{ath}
  Um summur: Lengd bils getur verið $+\infty$, t.d. bilsins
  $[0,+\infty[$. Setjum
  \[
  \tilde{\R} :={\R}\cup\{-\infty,+\infty\}
  \]
  og köllum þetta \emph{útvíkkuðu rauntalnalínuna}.\index{utvikkada
    rauntalnalinan@útvíkkaða rauntalnalínan}
\end{ath}
Athugum fjölskyldu $(a_k)_{k\in K}$ í $\tilde{\R}$ þ.a. $a_{k}\ge 0$.
\emph{Fyrst}: Ef $K$ er endanlegt og $a_{k}\in\R$ fyrir öll $k$, þá er
$\sum_{k\in K}a_{k}$ vel skilgreind rauntala, nefnilega
\[
\sum_{k\in K}a_{k}:=\sum_{j=1}^{n}a_{k_{j}}
\]
þar sem $k_{1},\dots,k_{n}$ er upptalning mengisins $K$;
skilgreiningin er óháð upptalningunni. Ef eitt eða fleiri stök $a_{k}$
eru $+\infty$, hin í $\R$, þá er eðlilegt að segja
\[
\sum_{k\in K}a_{k} := +\infty.
\]
Ef eitt eða fleiri eru $-\infty$, hin í $\R$, þá er $\sum_{k\in{K}}
a_{k}:=-\infty$. Ef sum eru $+\infty$, önnur $-\infty$, þá er summan
ekki skilgreind.  Ef $a_{k}\ge 0$ fyrir öll $k$ í $K$ og $K$
óendanlegt þá er
\[
\sum_{k\in K}a_{k}
:= \sup\left\{\sum_{k\in J}a_{k}:J\subset K,J\;\text{endanlegt} \right\}
\in[0,+\infty].
\]
Þetta er alltaf vel skilgreint. Höfum:
\begin{setn}
  Ef $(a_{k})_{k\in K}$ er fjölskylda af rauntölum, $a_{k}\ge 0$ fyrir
  öll $k\in K$, og $\sum_{k\in K}a_{k}<+\infty$, þá er mengið
  \[
  K':=\{k\in K:a_{k}\ne 0\}
  \]
  teljanlegt.
\end{setn}
\begin{proof}
  Mengið $K_{n}:=\{k\in K : a_{k}\ge\frac 1{n+1}\}$ fyrir $n\in\N$ er
  endanlegt; annars væri summan óendanleg, og
  $K'=\bigcup_{n\in\N}K_{n}$ er teljanlegt sammengi af endanlegum
  mengjum.
\end{proof}
Hvaða eiginleika ætti ``lengd'' að hafa? Við viljum
\begin{enumerate}[(i)]
\item Ef $A\subset\R$, þá er $0\le\lambda(A)\le+\infty$.
\item Lengd breytist ekki ef við hliðrum menginu til,
  þ.e.
  \[
  \lambda(A+r)=\lambda(A)
  \]
  fyrir öll $r$ úr $\R$, þar sem $A+r:=\{a+r:a\in A\}$.
\item $\lambda([0,1[)=1$.
\item Ef $(A_{k})_{k\in\N}$ er runa af mengjum sem eru
  sundurlæg tvö og tvö, þ.e. $A_{k}\cap A_{j}=\emptyset$ ef $j\ne k$,
  þá er
  \[
  \lambda(\bigcup_{k=0}^{+\infty}A_{k})=\sum_{k=0}^{+\infty}\lambda(A_{k}).
  \]
\end{enumerate}
Táknum með $\mathcal P(X)$ mengi allra hlutmengja í gefnu mengi $X$.
\begin{setn}
  Ekki er til vörpun $\lambda: \mathcal{P}(\R)\to\tilde{\R}$ sem
  fullnægir skilyrðum (i-iv) að ofan.
\end{setn}
\begin{proof}
  Gerum ráð fyrir að slík vörpun sé til. Skilgreinum jafngildisvensl
  $\sim$ á $\R$ með
  \[
  x\sim y \quad\text{þ.þ.a.a.}\quad x-y\in\Q.
  \]
  Jafngildisflokkur staksins $x$ er
  mengið
  \[
  x+\Q = \{x+r :r\in \Q\}.
  \]
  Það er ljóst að $x+\Q$ er þétt í $\R$ svo að
  $(x+\Q)\cap[0,1[\ne\emptyset$. Fyrir hvern jafngildisflokk $x+\Q$
  veljum við eitt stak úr $(x+\Q)\cap[0,1[$; þau mynda mengi $N$;
  höfum þá $N\subset[0,1]$, $\bigcup_{x\in N}(x+\Q)=\R$ og fyrir
  $x,y\in N, x\ne y$, er $(x+\Q)\cap(y+\Q)=\emptyset$. Nú er
  \begin{align*}
    \R
    &= \bigcup_{x\in N}(x+\Q) \\
    &= \{x+r : x\in N, r\in\Q\} \\
    &= \bigcup_{r\in\Q}(N+r).
  \end{align*}
  Setjum
  \[
  N_{r}:=(N+r)\cap[0,1[.
  \]
  Þá er
  \[
  [0,1[
  = \R\cap[0,1[
  = (\bigcup_{r\in\Q}(N+r))\cap[0,1[
  = \bigcup_{r\in\Q} N_{r}.
  \]
  En $N_{r}\cap[0,1[=\emptyset$ nema $r\in[-1,1[$. Því
  er
  \[
  [0,1[
  =\bigcup_{r\in[0,1[\cap\Q}(N_{r-1}\cup N_{r})
  =\bigcup_{r\in[0,1[\cap\Q}M_{r}
  \]
  þar sem $M_{r}:=N_{r-1}\cup N_{r}$. Held fram: Ef
  $r,s\in[0,1[\cap\Q$ og $r\ne s$, þá er $M_{r}\cap
  M_{s}=\emptyset$. Fjögur tilvik:
  \begin{enumerate}[(1)]
  \item Ef $x\in N_{r-1}$ og $x\in N_{s}$, þá eru til $n,m\in N$
    þ.a. $x=n+r-1=m+s$; þá er $n-m=s-r+1\in\Q$, svo að $n\sim m$; en
    þá er $n=m$, því að $N$ hefur bara eitt stak úr hverjum
    jafngildisflokki, svo að $s=r-1$, sem er fráleitt fyrir
    $r,s\in[0,1[$.
  \item Ef $x\in N_{r}$ og $N_{s}$, þá eru til $n,m\in\N$
    þ.a. $x=n+r=m+s$; þá er $n-m=s-r\in\Q$, svo að $n=m$ og þá $r=s$ í
    mótsögn við að $r\ne s$.
  \item[(3-4)] Tilfelli (3), $x\in N_{r}$ og $x\in N_{s-1}$, og (4),
    $x\in N_{r-1}$ og $x\in N_{s-1}$, eru eins. Þá segir síðara
    skilyrðið á $\lambda$ að
    \[
    \lambda([0,1[)=\sum_{r\in[0,1[\cap\Q}\lambda(M_{r}).
    \]
    En ljóslega er
    \[
    N_{r-1}\cap N_{r}=\emptyset,
    \quad
    (N_{r-1}+1)\cap N_{r}=\emptyset
    \]
    svo að
    \begin{align*}
      \lambda(M_{r})
      &= \lambda(N_{r-1})+\lambda(N_{r})\\
      &= \lambda(N_{r-1}+1)+\lambda(N_{r}) \\
      &= \lambda((N_{r-1}+1)\cup N_{r}) \\
      &= \lambda(N+r) \\
      &= \lambda(N).
    \end{align*}
    Ef $\lambda(N)=0$, þá
    fæst
    \[
    1
    =\lambda([0,1[)
    =\sum_{r\in[0,1[\cap\Q}\lambda(M_{r})
    =0;
    \]
    ef $\lambda(N)>0$, þá fæst
    \[
    1
    =\lambda([0,1[)
    =\sum_{r\in[0,1[\cap\Q}\lambda(M_{r})
    =+\infty.
    \]
    Í báðum tilvikum fæst mótsögn!
  \end{enumerate}
\end{proof}
%%
\marginpar{11. jan.}
%%
Hvað veldur? Er hugsanlegt að skilyrði (iv) sé of sterkt og við ættum
bara að fara fram á að
\[
\lambda(A_{1}\cup\cdots\cup A_{n})
= \lambda(A_{1})+\cdots+\lambda(A_{n})
\]
hvenær sem $A_{1},\dots,A_{n}$ eru hlutmengi í $\R$ og sundurlæg tvö
og tvö?

\emph{Nei}, þetta er ekki ástæðan. Að vísu má sýna fram á tilvist
(en ekki ótvíræðni) slíks falls í einni vídd, en ekki í víddum
$\ge3$.
\begin{setn}
  \label{setn:enginvorpun}
  Látum $n\ge 3$. Þá er ekki til vörpun
  $\mu:\mathcal{P}(\R^{n})\to\tilde\R$ þannig að eftirfarandi
  skilyrðum sé fullnægt:
  \begin{enumerate}[(i)]
  \item $0\le\mu(A)\le+\infty$.
  \item Ef $A,B$ eru \emph{samsniða} hlutmengi í $\R^{n}$, þá er
    $\mu(A)=\mu(B)$.
  \item $\mu(A_{1}\cup\cdots\cup A_{n})=\mu(A_{1})+\cdots+\mu(A_{n})$
    ef $A_{1},\dots,A_{n}$ eru hlutmengi í $\R^{n}$, sundurlæg tvö og
    tvö.
  \item $\mu(Q)=1$ ef $Q$ er kassi með hliðalengdir 1.
  \end{enumerate}
\end{setn}
\begin{skilgr}[samsniða, flutningur]
  Við segjum að hlutmengi $A,B$ í $\R^{n}$ séu
  \emph{samsniða}\index{samsnida@samsniða} eða \emph{eins}\index{eins}
  ef til er flutningur $T:\R^{n}\to\R^{n}$ þ.a. $T(A)=B$.

  \emph{Flutningur}\index{flutningur} er vörpun sem varðveitir
  fjarlægðir:
  \[
  \|T(x) - T(y)\| = \|x-y\| \qquad \forall\,x,y\in\R^{n}
  \]
\end{skilgr}
\begin{ath}
  Sýna má að flutningur er af gerðinni
  \[
  T(x) = A(x)+b
  \]
  þar sem $A$ er þverstöðluð línuleg vörpun, sem þýðir að línur
  fylkisins fyrir $A$ í venjulega grunninum mynda venjulegan
  þverstæðan grunn af einingarvigrum.
\end{ath}
Setning \ref{setn:enginvorpun} er afleiðing af:
\begin{setn}[Banach-Tarski-þversögnin]
  \index{Banach-Tarski-zthversognin@Banach-Tarski-þversögnin} Látum $A$
  og $B$ vera takmörkuð hlutmengi í $\R^{n}$, $n\ge 3$, þannig að $A$
  og $B$ hafi innri punkta, þ.e. $\overset{\circ}A\ne\emptyset$,
  $\overset{\circ}B\ne\emptyset$, þ.e. bæðin mengin innihalda opna
  kúlu. Þá er til tala $m$ og hlutmengi $A_{1},\dots,A_{m}$ í $\R^{n}$
  þannig að gildi:
  \begin{enumerate}[(i)]
  \item $A=A_{1}\cup\cdots\cup A_{m}$, $B=B_{1}\cup\cdots\cup B_{m}$.
  \item $A_{1},\dots,A_{m}$ eru sundurlæg tvö og tvö.
  \item $B_{1},\dots,B_{m}$ eru sundurlæg tvö og tvö.
  \item $A_{k}$ er samsniða $B_{k}$ fyrir $k=1,\dots,m$.
  \end{enumerate}
  M.ö.o.: Við getum skipt $A$ upp í endanlega mörg mengi, fært þau til
  og raðað þeim saman upp á nýtt þannig að við fáum mengið $B$.
\end{setn}
Sönnun Banach-Tarski-setningarinnar notar frumsendu um val. Sönnunin
verður ekki tekin fyrir að svo stöddu.
\begin{ath} [Niðurstaða]
  Sum hlutmengi í $\R^{n}$ eru of flókin til að nokkur leið sé að
  segja að þau hafi eitthvert rúmmál: Ekki bara að við getum ekki
  reiknað rúmmálið út; sú forsenda að mengið hafi eitthvert tiltekið
  rúmmál leiðir til mótsagnar!
\end{ath}
Við verðum að láta okkur nægja að bara sum hlutmengi í $\R^{n}$ hafi
rúmmál. Önnur ástæða þess hvernig við ætlum að taka á mál- og
heildunarfræðum kemur úr
\emph{líkindafræði}\index{likindafrzaedi@líkindafræði}:

Við gerum ráð fyrir að við höfum eitthvert tilviljanakennt fyrirbæri,
t.d. það að kasta upp teningi, og höfum gefið eitthvert mengi $\Omega$
af hugsanlegum útkomum, t.d. $\{1,2,3,4,5,6\}$ fyrir hliðarnar sem
geta komið upp á teningi. Hlutmengi í $\Omega$ kallast
\emph{atburður}\index{atburdur@atburður}; t.d. er $\{2,4,6\}$ sá
atburður að upp komi jöfn tala og $\{3,6\}$ sá atburður að upp komi
margfeldi af 3. Viljum úthluta sérhverjum atburði $A$ tilteknum
\emph{líkindum}\index{likindi@líkindi} $\lambda(A)$ sem er tala á
bilinu $\left[0,1\right]$. Viljum að
\begin{equation}
\lambda(A\cup B) = \lambda(A)+\lambda(B)
\quad\text{ef}\quad
A\cap B=\emptyset.\label{eq:likur_sammengi}
\end{equation}
Ef við höfum ``ósvikinn tening'', þá $\lambda(\{x\})=\frac16$ fyrir
öll $x$ úr $\{1,\dots,6\}$, og af \eqref{eq:likur_sammengi} leiðir
að
\[
\lambda(A) = \frac 16 \#A
\]
þar sem $\#A$ er fjöldatala $A$. Viljum líka leyfa óendanleg mengi af
hugsanlegum útkomum og lendum þá í sama vanda og áður að ekki er ljóst
að allir atburðir hafi ``hugsanleg líkindi''; t.d. ef við skjótum
örmjórri ör úr mikilli fjarlægð á skífu, þá eru líkindin á að örin
lendi á tilteknu svæði nokkurn veginn í hlutfalli við flatarmál
svæðisins.

Enn önnur ástæða fyrir almenna framsetningu: Kannski viljum við ekki
reikna út rúmmál rúmmynda, heldur t.d. \emph{massa} þeirra, miðað við
að við höfum tiltekna efnisdreifingu í rúminu. Þá þurfa samsniða
rúmmyndir ekki að hafa sama massa, en við búumst við að áfram eigi
reglan
\[
\mu(A\cup B) = \mu(A)+\mu(B)
\quad\text{ef}\quad
A\cap B = \emptyset
\]
og væntanlega líka samsvarandi reglu fyrir teljanlega mörg mengi.


\chapter{Mál og málrúm}

Við hugsum okkur að við höfum gefið eitthvert fast mengi $X$ og ætlum
að athuga hlutmengi í $X$. Við táknum með
\[
\mathcal P(X)
\]
\index{veldismengi}mengi allra hlutmengja í $X$. Fyrir $A\subset X$
skrifum við
\[
A^{C}:=X\setminus A= \{x\in X : x\notin A\}.
\]
Við hugsum okkur að sum mengi í $X$ séu
\emph{mælanleg}\index{mzaelanleg@mælanleg!mengi} og önnur ekki;
gerum jafnan ráð fyrir að þau myndi svokallaða
\emph{$\sigma$-algebru}\index{sigma-algebra@$\sigma$-algebra}:
\begin{skilgr}[mengjaalgebra og $\sigma$-algebra]
  Látum $X$ vera mengi. Mengi $\mathcal A$ af hlutmengjum í $X$
  kallast \emph{mengjaalgebra}\index{mengjaalgebra} (eða bara
  \emph{algebra}\index{algebra}) á $X$ ef eftirfarandi þremur
  skilyrðum er fullnægt:
  \begin{enumerate}[(i)]
  \item $\emptyset\in\mathcal A$.
  \item Ef $A\in\mathcal A$, þá er $A^{C}\in\mathcal A$.
  \item Ef $A,B\in\mathcal A$, þá er $A\cup B\in\mathcal A$.
  \end{enumerate}
  Við segjum að $\mathcal A$ sé \emph{$\sigma$-algebra} ef það
  fullnægir (i) og (ii), en í stað (iii) kemur sterkara skilyrðið:
  \begin{enumerate}
  \item [(iii')] Ef $(A_{k})_{k\in\N}$ er runa af stökum í
    $\mathcal{A}$, þá er $\bigcup_{k\in\N}A_{k}\in\mathcal A$;
    þ.e. sammengi teljanlega margra mengja sem eru stök í $\mathcal A$
    er stak í $\mathcal A$.
  \end{enumerate}
\end{skilgr}
\begin{ath}
  Af (i) og (ii) leiðir að $X=\emptyset^{C}\in\mathcal A$. Mengið
  $\{\emptyset,X\}$ er $\sigma$-algebra á $X$ og er innihaldið í
  sérhverri $\sigma$-algebru á $X$; það er því
  \emph{minnsta}\index{sigma-algebra@$\sigma$-algebra!minnsta}
  $\sigma$-algebran á $X$. Eins er $\mathcal P(X)$
  \emph{stærsta}\index{sigma-algebra@$\sigma$-algebra!stzrsta@stærsta}
  $\sigma$-algebran á $X$.
\end{ath}
\begin{setn}
  \begin{enumerate}[(1)]
  \item Ef $\mathcal A$ er algebra á $X$ og $A,B\in\mathcal A$, þá er
    \[
    A\cap B,A\setminus B\in A.
    \]
  \item Ef $\mathcal A$ er $\sigma$-algebra á $X$ og $(A_{k})_{k\in\N}$
    er runa af stökum í $\mathcal A$, þá er
    \[
    \bigcap_{k\in\N} A_{k}\in A.
    \]
  \end{enumerate}
\end{setn}
\begin{proof}
  (1) Höfum
  \[
  A\cap B = (A^{C}\cup B^{C})^{C}
  \]
  (de-Morgan-formúla) og
  \[
  A\setminus B = A\cap B^{C}.
  \]
  
  (2) Eins höfum við de-Morgan-formúlu
  \[
  \bigcap_{k\in\N} A_{k} = \left(\bigcup_{k\in\N} A_{k}^{C}\right).
  \]
\end{proof}
\begin{setn}
  Fyrir sérhvert mengi $\mathcal C$ af hlutmengjum $X$ er til
  \emph{minnsta} $\sigma$-algebra $\mathcal A$
  þ.a. $\mathcal{C}\subset\mathcal{A}$.
\end{setn}
\begin{proof}
  Setjum
  \[
  \mathcal A
  := \bigcap \{
    \mathcal B : 
      \text{$\mathcal B$ er $\sigma$-algebra og $\mathcal C\subset \mathcal B$} 
    \}
  \]
  Athugum að mengið $\{ \mathcal B : \text{$\mathcal B$ er
    $\sigma$-algebra og $\mathcal C\subset \mathcal B$} \}$ er ekki
  tómt, því það hefur stakið $\mathcal P(X)$. Ljóst er að sniðmengi af
  $\sigma$-algebrum er $\sigma$-algebra.
\end{proof}
\begin{skilgr}[Borel-$\sigma$-algebran]
  \index{Borel-sigma-algebran@Borel-$\sigma$-algebran}
  
  Látum $X$ vera firðrúm (eða grannrúm). Minnsta $\sigma$-algebra á
  $X$ sem inniheldur öll opin mengi í $X$ kallast
  \emph{Borel-$\sigma$-algebran} á $X$.
\end{skilgr}
Sér í lagi má tala um Borel-$\sigma$-algebruna á $\R$ og almennar á
$\R^{n}$.
%%
\marginpar{14. jan.}
%%
Táknum með
\[
\mathcal B_{X}
\]
Borel-$\sigma$-algebruna á firðrúmi
(grannrúmi) $X$; köllum stökin í $\mathcal B_{X}$
\emph{Borel-mengi}\index{Borel-mengi}.
\begin{ath}
  Sérhvert opið mengi og sérhvert lokað mengi í $X$ er Borel-mengi.
  Sniðmengi af teljanlega mörgum opnum mengjum er líka Borel-mengi,
  slík mengi kallast
  \emph{$G_{\delta}$-mengi}\index{G-delta-mengi@$G_{\delta}$-mengi}. Sammengi
  af teljanlega mörgum lokuðum mengjum kallast
  \emph{$F_{\sigma}$-mengi}\index{F-sigma-mengi@$F_{\sigma}$-mengi};
  það er líka Borel-mengi.
\end{ath}
\begin{setn}
  Borel-$\sigma$-algebran $\mathcal B_{\R}$ á $\R$ er spönnuð af
  hverju fyrir sig af eftirfarandi mengjum:
  \begin{enumerate}[(i)]
  \item Mengi allra takmarkaðra opinna bila $\left]a,b\right[$, þar
    sem $a,b\in\R$ og $a<b$.
  \item Mengi allra takmarkaðra lokaðra bila $\left[a,b\right]$, þar
    sem $a,b\in\R$ og $a<b$.
  \item Mengi allra takmarkaðra hálfopinna bila $\left[a,b\right[$,
    þar sem $a,b\in\R$ og $a<b$.
  \item Mengi allra takmarkaðra hálfopinna bila $\left]a,b\right]$,
    þar sem $a,b\in\R$ og $a<b$.
  \item Mengi allra ótakmarkaðra opinna bila $\left]a,+\infty\right[$,
    þar sem $a\in\R$.
  \item Mengi allra ótakmarkaðra opinna bila $\left]-\infty,a\right[$,
    þar sem $a\in\R$.
  \item Mengi allra ótakmarkaðra lokaðra bila
    $\left[a,+\infty\right[$, þar sem $a\in\R$.
  \item Mengi allra ótakmarkaðra lokaðra bila
    $\left]-\infty,a\right]$, þar sem $a\in\R$.
  \end{enumerate}
\end{setn}
\begin{proof}
  Ljóst að opið bil er sammengi teljanlegra margra lokaðra bila; lokað
  bil er sniðmengi teljanlega margra opinna bila; höfum
  $\left]-\infty,b\right[\cap\left]a,+\infty\right[=\left]a,b\right[$
  ef $a<b$ o.s.frv. Sérhvert opið mengi er sammengi teljanlega margra
  opinna bila.
\end{proof}
\begin{skilgr}
  Segjum að fjölsklda $(A_{i})_{i\in I}$ af mengjum sé
  \emph{sundurlæg}\index{sundurlzg@sunduræg fjölskylda} ef
  $A_{j}\cap{A_{k}}=\emptyset$ fyrir öll $j,k\in I$ þ.a. $j\ne k$,
  þ.e. ef mengin í fjölskyldunni eru sundurlæg tvö og tvö.
\end{skilgr}
\begin{ath}
  Til að ganga úr skugga um að mengi $\mathcal A$ af hlutmengjum í $X$
  sé $\sigma$-algebra nægir að sýna tvennt:
  \begin{enumerate}[(1)]
  \item $\mathcal A$ er algebra á $X$.
  \item Ef $(A_{k})_{k\in\N}$ er \emph{sundurlæg} runa í $\mathcal A$,
    þá er $\bigcup_{k\in\N}A_{k}\in\mathcal A$.
  \end{enumerate}
  Ef þá $(B_{k})_{k\in\N}$ er þá einhver runa í $\mathcal A$, þá eru
  skv. (1) mengin $A_{0}:=B_{0}$ og
  $A_{k}:=B_{k}\setminus\bigcup_{j=0}^{k-1}B_{j}$ fyrir $k\ge 1$ öll í
  $\mathcal A$; og runan $(A_{k})_{k\in\N}$ er sundurlæg og
  $\bigcup{B_{k}}=\bigcup{A_{k}}\in\mathcal A$ skv. (2).
\end{ath}
\begin{ath}
  Skilyrðið að fjölskylda $(A_{i})_{i\in I}$ sé sundurlæg er \emph{miklu
  sterkara} en skilyrðið $\bigcap_{i\in I}A_{i}=\emptyset$.
\end{ath}
\begin{skilgr}[mál]
  \begin{enumerate}[(1)]
  \item Látum $X$ vera mengi og $\mathcal A$ vera $\sigma$-algebru á
    $X$. \emph{Mál}\index{mal@mál} á $\mathcal A$ er vörpun
    $\mu:\mathcal A\to\left[0,+\infty\right]$ þannig að eftirfarandi
    skilyrðum sé fullnægt:
    \begin{enumerate}[(i)]
    \item $\mu(\emptyset)=0$.
    \item Ef $(A_{k})_{k\in\N}$ er sundurlæg runa af hlutmengjum í
      $X$ sem eru stök í $\mathcal A$, þá er
      \[
      \mu(\bigcup_{k\in\N}A_{k})
      = \sum_{k\in\N}\mu(A_{k}).
      \]
    \end{enumerate}
  \item \emph{Mælanlegt rúm}\index{mzaelanleg@mælanleg!rum@rúm} er
    tvennd $(X,\mathcal A)$ þar sem $X$ er mengi og $\mathcal A$ er
    $\sigma$-algebra á $X$; við köllum stökin í $\mathcal A$
    \emph{mælanleg
      hlutmengi}\index{mzaelanleg@mælanleg!hlutmengi} í
    mælanlega rúminu (eða einfaldlega í
    $X$). \emph{Málrúm}\index{malrum@málrúm} er þrennd
    $(X,\mathcal{A},\mu)$, þar sem $(X,\mathcal A)$ er mælanlegt rúm
    og $\mu$ er mál á $\mathcal A$.
  \item Við segjum að málrúm $(X,\mathcal A, \mu)$ sé \emph{endanlegt}
    ef
    $\mu(X)<+\infty$. \emph{Líkindarúm}\index{likindarum@líkindarúm}
    er málrúm $(X,\mathcal A,\mu)$ þannig að $\mu(X)=1$. Mælanlegu
    mengin í líkindarúmi eru kölluð
    \emph{atburðir}\index{atburdur@atburður} og málið er kallað
    \emph{líkindamál}\index{likindamal@líkindamál}.
  \end{enumerate}
\end{skilgr}
\begin{daemi}
  [Dæmi um mál]
  \begin{enumerate}[(1)]
  \item Látum $X$ vera mengi og skilgreinum
    $\mu:\mathcal{P}(X)\to\left[0,+\infty\right]$ með
    \[
    \mu(A) :=
    \begin{cases}
      \#A     & \text{ef $A$ er endanlegt}, \\
      +\infty & \text{ef $A$ er óendanlegt}.
    \end{cases}
    \]
    Þá er $\mu$ mál á $\mathcal P(X)$, sem kallast
    \emph{talningarmálið}\index{talningarmalid@talningarmálið} á $X$.
  \item Látum $X$ vera mengi, $a$ tiltekinn punkt í $X$. Skilgreinum
    $\delta_{a}:\mathcal P(X)\to\left[0,+\infty\right]$ með
    \[
    \delta_{a}(A) :=
    \begin{cases}
      1 & \text{ef }a\in A,\\
      0 & \text{ef }a\notin A.
    \end{cases}
    \]
    Þetta er mál á $\mathcal P(X)$, kallað
    \emph{Dirac-málið}\index{Dirac-malid@Dirac-málið} í punktinum $a$.
  \end{enumerate}
\end{daemi}
\begin{setn}
  Látum $(X,\mathcal A,\mu)$ vera málrúm.
  \begin{enumerate}[(1)]
  \item Ef $A,B\in\mathcal A$ og $A\subset B$, þá er
    \[
    \mu(A)\le\mu(B).
    \]
  \item Ef $(A_{k})_{k\in\N}$ er runa af stökum í $\mathcal A$, þá er
    \[
    \mu\left(\bigcup_{k\in\N}A_{k}\right)
    \le \sum_{k\in\N}\mu(A_{k}).
    \]
  \item Ef $(A_{k})_{k\in\N}$ er vaxandi runa af stökum í
    $\mathcal{A}$, þ.e. $A_{k}\subset A_{k+1}$ fyrir öll $k$, þá er
    \[
    \mu\left(\bigcup_{k\in\N}A_{k}\right)
    = \lim_{k\to+\infty}\mu(A_{k}).
    \]
  \item Ef $(A_{k})_{k\in\N}$ er fallandi runa af stökum í
    $\mathcal{A}$ þ.e. $A_{k+1}\subset A_{k}$ fyrir öll $k\in\N$, og
    til er $k\in\N$ þ.a. $\mu(A_{k})<+\infty$, þá er
    \[
    \mu\left(\bigcap_{j\in\N}A_{j}\right)
    = \lim_{j\to+\infty}\mu(A_{j}).
    \]
  \end{enumerate}
\end{setn}
\begin{proof}
  (1) Mengið $B$ er sammengi endanlegu mengjanna $A$ og
  $B\setminus{A}$, svo að
  \[
  \mu(B) = \mu(A)+\mu(B\setminus A)\ge \mu(A),
  \]
  því að $\mu(B\setminus A)\ge 0$.

  (2) Skilgreinum $(B_{k})_{k\in\N}$ með $B_{0}:=A_{0}$ og
  $B_{k+1}:=A_{k+1}\setminus\bigcup_{j=0}^{k}A_{j}$; þá er runan
  $(B_{k})_{k\in\N}$ sundurlæg, $B_{k}\in \mathcal A$ og
  $B_{k}\subset{A_{k}}$, svo að
  \[
  \mu
  \left(
    \bigcup_{k\in\N}A_{k}
  \right)
  = \mu
  \left(
    \bigcup_{k\in\N}B_{k}
  \right)
  = \sum_{k\in\N}\mu(B_{k})
  \le \sum_{k\in\N}\mu(A_{k})
  .
  \]
  
  (3) Setjum $A_{-1}:=\emptyset$. Þá er runan
  $(A_{k}\setminus{A_{k-1}})_{k\in\N}$ sundurlæg, svo að
  \begin{eqnarray*}
    \mu \left( \bigcup_{k\in\N}A_{k} \right)
    &=& \mu \left( \bigcup_{k\in\N}A_{k}\setminus A_{k-1} \right) \\
    &=& \sum_{k\ni\N}\mu(A_{k}\setminus A_{k-1}) \\
    &=& \lim_{k\to+\infty}\sum_{j=0}^{k}\mu(A_{j}\setminus A_{j-1}) \\
    &=& \lim_{k\to+\infty}\mu \left(
      \bigcup_{j=0}^{k}(A_{j}\setminus A_{j-1})
    \right) \\
    &=& \lim_{k\to+\infty}\mu(A_{k}).
  \end{eqnarray*}

  (4) Setjum $B_{j}:=A_{k}\setminus A_{j+1}$. Þá er $(B_{j})$ vaxandi
  runa af mengjum sem eru stök í $\mathcal A$. Við höfum
  $\bigcap_{j\in\N}A_{k+j}\subset{A_{k}}$, svo að 
  \begin{eqnarray*}
    \mu(A_{k})-\mu\left(\bigcap_{j\in\N}A_{k+j}\right)
    &=& \mu\left(A_{k}\setminus\bigcap_{j\in\N}A_{k+j}\right) \\
    &=& \mu\left(\bigcup_{j\in\N}(A_{k}\setminus A_{k+j})\right) \\
    &=& \mu\left(\bigcup_{j\in\N}B_{j}\right) \\
    &{\underset{(3)}{=}}& \lim_{j\to+\infty}\mu(B_{j}) \\
    &=& \lim_{j\to+\infty}\mu(A_{k}\setminus A_{k+j}) \\
    &=& \mu(A_{k})-\lim_{j\to+\infty}\mu(A_{k+j}),
  \end{eqnarray*}
  svo að
  \[
  \lim_{j\to+\infty}\mu(A_{j})
  = \lim_{j\to+\infty}\mu(A_{k+j})
  = \mu\left(\bigcap_{j\in\N}A_{k+j}\right)
  = \mu\left(\bigcap_{j\in\N}A_{j}\right).
  \]
\end{proof}
\begin{ath}
  Forsendan $\mu(A_{k})<+\infty$ í (4) að ofan er nauðsynleg: Höfum
  $\mu(\left[k,+\infty\right[)=+\infty$ en
  $\bigcap_{k\in\N}\left[k,+\infty\right[=\emptyset$.
\end{ath}
%%
\marginpar{17. jan.}
%%
\begin{skilgr}
  \begin{enumerate}[(1)]
  \item Látum $(X,\mathcal A,\mu)$ vera málrúm. Stak $A$ í
    $\mathcal{A}$ kallast \emph{núllmengi}\index{nullmengi@núllmengi}
    ef $\mu(A)=0$.
  \item Látum $\mathbf{E}(x)$ vera einhverja fullyrðingu um punkta $x$
    í málrúmi $(X,\mathcal A,\mu)$. Við segjum að $\mathbf{E}(x)$
    gildi \emph{næstum
      allsstaðar}\index{nzstum@næstum!allsstaðar} ef mengið
    \[
    \{x\in X : \mathbf{E}(x)\text{ er ekki satt}\}
    \]
    er núllmengi. Ef $(X,\mathcal A,\mu)$ er líkindarúm, þá notum við
    \emph{næstum örugglega}\index{nzstum@næstum!zrugglega@örugglega}
    um það sama.
  \item Við segjum að málrúmið $(X,\mathcal A,\mu)$ sé
    \emph{fullkomið}\index{fullkomid@fullkomið!malrum@málrúm}\index{malrum@málrúm!fullkomid@fullkomið}
    ef sérhvert hlutmengi í núllmengi er mælanlegt (og þá nauðsynlega
    núllmengi).
  \end{enumerate}
\end{skilgr}
\begin{ath}
  Þetta þýðir t.d.: Ef $X$ er bæði málrúm og firðrúm og $f:X\to\R$ er
  fall, þá getum við sagt að $f$ sé samfellt næstum allsstaðar ef
  mengi ósamfelldnipunktanna er núllmengi.
\end{ath}
Ein leið til að búa til mál er að byrja á \emph{ytra máli} með aðferð
sem er kennd við
Carathéodory\index{Caratheodory@Carathéodory!adferd@aðferð}:
\begin{skilgr}
  Látum $X$ vera mengi. \emph{Ytra mál}\index{ytra mal@ytra
    mál}\index{mal@mál!ytra} á menginu $X$ er vörpun
  $\mu^{*}:\mathcal{P}(X)\to\left[0,+\infty\right]$ þannig að gildi:
  \begin{enumerate}[(i)]
  \item $\mu^{*}(\emptyset)=0$.
  \item Ef $A\subset B$ þá er $\mu^{*}(A)\le\mu^{*}(B)$.
  \item Ef $(A_{k})_{k\in\N}$ er runa af hlutmengjum í $X$, þá er
    \[
    \mu^{*}(\bigcup_{k\in\N}A_{k})
    \le \sum_{k\in\N}\mu^{*}(A_{k}).
    \]
  \end{enumerate}
\end{skilgr}
\begin{skilgr}
  Látum $\mu^{*}$ vera ytra mál á mengi $X$. Við segjum að hlutmengi
  $A$ í $X$ sé
  \emph{mælanlegt}\index{mzaelanleg@mælanleg!hlutmengi!m.t.t. ytra
    mals@m.t.t. ytra máls} m.t.t. $\mu^{*}$ ef fyrir sérhvert
  hlutmengi $T$ í $X$ gildir
  \[
  \mu^{*}(T)
  = \mu^{*}(T\cap A) + \mu^{*}(T\cap A^{C}).
  \]
\end{skilgr}
\begin{ath}
  Við höfum alltaf
  $\mu^{*}(T)\le\mu^{*}(T\cap{A})+\mu^{*}(T\cap{A^{C}})$. Til að sýna
  að $A$ sé mælanlegt þurfum við bara að sýna að
  \[
  \mu^{*}(T) \ge \mu^{*}(T\cap A) + \mu^{*}(T\cap A^{C})
  \]
  fyrir öll $T$ þ.a. $\mu^{*}(T)<+\infty$ (því ef $\mu^{*}(T)=+\infty$
  er ójafnan augljós).
\end{ath}
\begin{setn}
  [Carathéodory]\index{Caratheodory@Carathéodory!setning}

  Látum $\mu^{*}$ vera ytra mál á mengi $X$. Þá mynda mælanlegu mengin
  m.t.t. $\mu^{*}$ $\sigma$-algebru $\mathcal A$ og einskorðunin
  $\mu=\mu^{*}|\mathcal{A}$ er fullkomið mál.
\end{setn}
\begin{proof}
  Látum $A$ og $B$ vera mælanleg mengi m.t.t. $\mu^{*}$. Fyrir
  $T\subset{X}$ er þá
  \begin{align*}
    \mu^{*}(T)
    &= \mu^{*}(T\cap A) + \mu^{*}(T\cap A^{C})
    \\
    &= \mu^{*}(T\cap A)
    + \mu^{*}(T\cap A^{C}\cap B)
    + \mu^{*}(T\cap A^{C}\cap B^{C})
    \\
    &\ge \mu^{*}((T\cap A)\cup(T\cap A^{C}\cap B))
    + \mu^{*}(T\cap (A\cup B)^{C})
    \\
    &= \mu^{*}(T\cap(A\cup B)) + \mu^{*}(T\cap(A\cup B)^{C})
  \end{align*}
  því að
  \[
  (T\cap A)\cup(T\cap(A^{C}\cap B))
  = T\cap(A\cup(B\setminus A))
  = T\cap(A\cup B),
  \]
  svo að $A\cup B$ er mælanlegt. Vegna $\mu^{*}(\emptyset)=0$ er ljóst
  að $\emptyset$ er mælanlegt; og skilyrðið á mælanlegt mengi í
  skilgreiningunni er samhverft í $A$ og $A^{C}$, svo að $A^{C}$ er
  mælanlegt ef $A$ er mælanlegt. Höfum þá sýnt að mælanlegu mengin
  mynda algebru.

  Látum $A,B$ vera hlutmengi í $X$ þ.a. $A\cap B=\emptyset$ og $A$ sé
  mælanlegt. Þá er
  \begin{align*}
    \mu^{*}(A\cup B)
    &= \mu^{*}((A\cup B)\cap A) + \mu^{*}((A\cup B)\cap A^{C}) \\
    &= \mu^{*}(A) + \mu^{*}(B)
  \end{align*}
  því að $(A\cup B)\cap A = A$ og
  $(A\cup{B})\cap{A^{C}}=B\setminus{A}=B$ vegna
  $A\cap{B}=\emptyset$. Af því leiðir með þrepun að fyrir sundurlæga
  fjölskyldu $A_{1},\dots,A_{n}$ af mælanlegum mengjum er
  \[
  \mu^*(\bigcup_{i=1}^{n}A_{i}) = \sum_{i=1}^{n}\mu^{*}(A_{i}).
  \]
  Til að sýna að $\mathcal A$ sé $\sigma$-algebra nægir að sýna að
  sammengi \emph{sundurlægrar} runu af mælanlegum mengjum sé
  mælanlegt; auk þess viljum við að ytra mál þess sé summan af ytri
  málum mengjanna í rununni. Látum þá $(A_{k})_{k\in\N}$ vera
  sundurlæga runu af mælanlegum mengjum; setjum
  $B_{k}:=\bigcup_{j=0}^{k}A_{j}$ og
  $B:=\bigcup_{k\in\N}B_{k}=\bigcup_{k\in\N}A_{k}$. Ef nú $T$ er
  eitthvert hlutmengi í $X$, þá er
  \begin{align*}
    \mu^{*}(T)
    &= \mu^{*}(T\cap B_{k}) + \mu^{*}(T\cap B_{k}^{C}) \\
    &\ge \sum_{j=0}^{k}\mu^{*}(T\cap A_{j}) + \mu^{*}(T\cap B^{C}).
  \end{align*}
  (Ójöfnumerkið röstutt seinna). Látum nú $k$ stefna á $+\infty$ og
  fáum
  \begin{align*}
    \mu^{*}(T)
    &\ge \sum_{j=0}^{+\infty}\mu^{*}(T\cap A_{j})+\mu^{*}(T\cap B^{C})
    \\
    &\ge \mu^{*}(T\cap B) + \mu^{*}(T\cap B^{C})
    \\
    &\ge \mu^{*}(T)
  \end{align*}
  svo að jafnaðarmerki gildir allsstaðar! Því er
  $B=\bigcup_{k\in\N}A_{k}$ mælanlegt og
  \[
  \mu^{*}(\bigcup_{k\in\N}A_{k})
  = \sum_{k\in\N}\mu^{*}(A_{k}).
  \]
  Látum nú $A$ vera hlutmengi í mælanlegu núllmengi. Þá er
  $\mu^{*}(A)=0$ og því
  \begin{align*}
    \mu^{*}(T)
    &\le \mu^{*}(T\cap A) + \mu^{*}(T\cap A^{C}) \\
    &= 0 + \mu^{*}(T\cap A^{C}) \\
    &\le \mu^{*}(T)
  \end{align*}
  svo að jafnaðarmerkin gilda og $A$ er mælanlegt. Þar með er málið
  fullkomið.
\end{proof}
\begin{setn}
  Látum $\mathcal E$ vera hlutmengi í $\mathcal P(X)$
  þ.a. $\emptyset\in\mathcal E$ og
  $v:\mathcal{E}:\to\left[0,+\infty\right]$ vera fall
  þ.a. $v(\emptyset)=0$. Skilgreinum vörpun
  $\mu^{*}:\mathcal{P}(X)\to\left[0,+\infty\right]$ með
  \[
  \mu^{*}(A) := \inf \{
  \sum_{k\in\N} v(A_{k})
  : (A_{k})\text{ er runa í }\mathcal E
  \text{ og }
  A\subset\bigcup_{k\in\N} A_{k}
  \}.
  \]
  Þá er $\mu^{*}$ ytra mál á $X$.
\end{setn}
%%
\marginpar{21. jan.}
%%
Það vantaði rökstuðning fyrir seinni ójöfnu. Látum $(A_{k})_{k\in\N}$
vera sundurlæga runu af $\mu^{*}$-mælanlegum mengjum og setjum
$B_{k}:=\bigcup_{j=0}^{k}A_{j}$, þá gildir fyrir sérhvert hlutmengi
$T$ í $X$ að
\begin{align*}
  \mu^{*}(T\cap B_{k})
  &= \mu^{*}(T\cap B_{k}\cap A_{k})
  + \mu^{*}(T\cap B_{k}\cap{A_{k}^{C}})
  \\
  &=\mu^{*}(T\cap A_{k}) + \mu^{*}(T\cap B_{k-1}),
\end{align*}
þrepun gefur $\mu^{*}(T\cap
B_{k})=\sum_{j=0}^{k}\mu^{*}(T\cap{A_{j}})$. Þá er
\begin{align*}
  \mu^{*}(T)
  &= \mu^{*}(T\cap B_{k}) + \mu^{*}(T\cap B_{k}^{C})\\
  &\ge \sum_{j=0}^{k}\mu^{*}(T\cap A_{j}) + \mu^{*}(T\cap B^{C})
\end{align*}
þar sem $B:=\bigcup_{k\in\N}B_{k}=\bigcup_{k\in\N}\supset B_{k}$.
\begin{ath}
  Af sönnuninni leiðir: Ef $(A_{k})_{k\in\N}$ er sundurlæg runa af
  mengjum sem eru mælanleg m.t.t. ytra máls $\mu^{*}$, þá gildir fyrir
  öll hlutmengi $T$ í $X$ að
  \[
  \mu^{*}(T\cap\bigcup_{k\in\N}A_{k})
  = \sum_{k=0}^{\infty}\mu^{*}(T\cap A_{k}).
  \]
\end{ath}
\begin{setn}
  Látum $\mathcal E$ vera mengi af hlutmengjum í mengi $X$ og
  $v:\mathcal E\to\left[0,+\infty\right]$ vera vörpun
  þ.a. $\emptyset\in\mathcal E$ og $v(\emptyset) = 0$. Setjum
  \[
  \mu^{*}(T)
  := \inf
  \{
    \sum_{k=0}^{+\infty}v(Q_{k}) : (Q_{k})_{k\in\N}
    \text{ er runa í $\mathcal E$ og }T\subset\bigcup_{k\in\N}Q_{k}
  \}
  \]
  Þá er $\mu^{*}$ ytra mál a $X$.
\end{setn}
\begin{ath}
  Ef ekki er til runa $(Q_{k})$ í $\mathcal E$
  þ.a. $T\subset\bigcup_{k\in\N}Q_{k}$ þá er $\mu(T)=+\infty$ vegna
  $\inf\emptyset = +\infty$.
\end{ath}
\begin{proof}
  Ljóst er að $\mu^{*}(\emptyset)=0$ og fyrir $T,S$ þ.a. $T\subset S$
  er $\mu^{*}(T)\le\mu^{*}(S)$, því að runa $(Q_{k})$ sem þekur $S$
  þekur líka $T$. Látum nú $(A_{k})_{k\in\N}$ vera runu af hlutmengjum
  í $X$; viljum sýna að
  $\mu^{*}(\bigcup_{k\in\N}A_{k})\le\sum_{k=0}^{+\infty}\mu^{*}(A_{k})$. Þetta
  er ljóst ef einhver stærðanna $\mu^{*}(A_{k})$ er $+\infty$, svo að
  við megum gera ráð fyrir að $\mu^{*}(A_{k})<+\infty$ fyrir öll
  $k$. Fyrir hvert $k\in\N$ er þá til runa $(Q_{k_{j}})_{j\in\N}$ í
  $\mathcal E$ þannig að
  $\sum_{j=0}^{+\infty}v(A_{k_{j}})\le\mu^{*}(A_{k})+\frac{\varepsilon}{2^{k+1}}$
  þar sem $\varepsilon$ er fyrirfram gefin tala, $\varepsilon > 0$. Þá
  er $(Q_{kj})_{(k,j)\in\N\times\N}$ teljanleg fjölskylda af stökum í
  $\mathcal E$ þannig að
  $\bigcup_{k\in\N}A_{k}\subset\bigcup_{(k,j)\in\N\times N}Q_{kj}$ og
  \begin{align*}
    \sum_{(k,j)\N\times\N}v(Q_{kj})
    &= \sum_{k=0}^{+\infty}\sum_{j=0}^{+\infty}v(Q_{kj}) \\
    &\le
    \sum_{k=0}^{+\infty}(\mu^{*}(A_{k})+\frac\varepsilon{2^{k+1}}) \\
    &= \sum_{k=0}^{+\infty}\mu^{*}(A_{k})
    + \sum_{k=0}^{+\infty}\frac\varepsilon{2^{k+1}} \\
    &= \sum_{k=0}^{+\infty}\mu^{*}(A_{k}) + \varepsilon.
  \end{align*}
  Þar eð $\varepsilon$ má vera hvaða jákvæða tala sem vera skal fæst
  niðurstaðan.
\end{proof}
\begin{skilgr}
  \emph{Kassi}\index{kassi} í $\R^{n}$ er mengjamargfeldi
  \[
  Q = I_{1}\times I_{2}\times\cdots\times I_{n},
  \]
  þar sem $I_{1},\dots,I_{n}$ eru takmörkuð bil í
  $\R$. \emph{Rúmmál}\index{rummal@rúmmál} kassans $Q$ er talan
  \[
  v(A) := (b_{1}-a_{1})(b_{2}-a_{2})\cdots(b_{n}-a_{n})
  \]
  þar sem $a_{j}$ er neðri endapunktur og $b_{j}$ efri endapunktur
  $I_{j}$ fyrir $j=1,\dots,n$.
\end{skilgr}
\begin{ath}
  Ef öll bilin eru lokuð, þá fáum við lokaðan kassa; ef þau eru öll
  opin, þá fáum við opinn kassa. Bilin mega vera úrkynjuð, þ.e. það má
  gilda $a_{j}=b_{j}$ fyrir eitthvert $j$; þannig er tóma mengið kassi
  (\emph{tómi kassinn}) og $v(\emptyset)=0$.
\end{ath}
Kassi í $\mathbb R$ er bara takmarkað bil.
\begin{skilgr}
  Látum $\mathcal Q$ vera mengi allra kassa í $\R^{n}$ og
  $v:\mathcal{Q}\to\left[0,+\infty\right]$ vera rúmmálið. Ytra málið á
  $\R^{n}$ sem $v$ gefur af sér skv. síðustu setningu kallast
  \emph{ytra mál Lebesgues}\index{Lebesgue!ytra mal@ytra
    mál}\index{ytra mal@ytra mál!Lebesgue} og er táknað
  $\lambda_{n}^{*} $eða $\lambda^{*}$. Hlutmengi í $\R^{n}$ er
  \emph{Lebesgue-mælanlegt}\index{Lebesgue!mzaelanlegt@mælanlegt} ef
  það er mælanlegt m.t.t. $\lambda_{n}^{*}$. Táknum með $\lambda_{n}$
  eða $\lambda$ samsvarandi mál og með
  \[
  \mathcal M_{\lambda_{n}}
  \quad\text{eða}\quad
  \mathcal M_{\lambda}
  \]
  $\sigma$-algebru allra Lebesgue-mælanlegra mengja; málið
  $\lambda:\mathcal M_{\lambda}\to\left[0,+\infty\right]$ kallast
  \emph{Lebesgue-málið}\index{Lebesgue!mal@mál} á $\R^{n}$.
\end{skilgr}
\begin{setn}
  Borel-$\sigma$-algebran á $\R^{n}$ er spönnuð af sérhverju
  eftirtalinna mengja:
  \begin{enumerate}[(i)]
  \item Mengi allra opinna kassa.
  \item Mengi allra lokaðra kassa.
  \item Mengi allra kassa.
  \item Mengi allra opinna hálfrúma af gerðinni
    $H^{-}_{j,a}:=\{(x_{1},\dots,x_{n})\in\R^{n}:x_{j}<a\}$, þar sem
    $j\in\{0,1,\dots,n\}$ og $a\in\R$.
  \item Mengi allra opinna hálfrúma af gerðinni
    $H^{+}_{j,a}:=\{(x_{1},\dots,x_{n})\in\R^{n}:x_{j}>a\}$, þar sem
    $j\in\{0,1,\dots,n\}$ og $a\in\R$.
  \item Mengi allra lokaðra hálfrúma af gerðinni
    $\bar{H}^{-}_{j,a}:=\{(x_{1},\dots,x_{n})\in\R^{n}:x_{j}\le a\}$,
    þar sem $j\in\{0,1,\dots,n\}$ og $a\in\R$.
  \item Mengi allra lokaðra hálfrúma af gerðinni
    $\bar{H}^{+}_{j,a}:=\{(x_{1},\dots,x_{n})\in\R^{n}:x_{j}\ge a\}$,
    þar sem $j\in\{0,1,\dots,n\}$ og $a\in\R$.
  \end{enumerate}
\end{setn}
\begin{proof}
  Sérhvert opið mengi er sammengi af teljanlega mörgum opnum kössum
  svo að mengi opinna kassa spannar $\mathcal B_{\R^{n}}$. Opinn
  kassi er sammengi teljanlega margra lokaðra kassa; og kassi er
  sniðmengi endanlega margra hálfrúma eins og í (iv)-(vii).
\end{proof}

\begin{setn}
  Sérhvert Borel-mengi í $\R^{n}$ er Lebesgue-mælanlegt.
\end{setn}
\begin{proof}
  Látum $A:=\bar H_{j,a} = \{(x_{1},\dots,x_{n}):x_{j}\le a\}$. Það
  nægir að sýna að $A$ sé mælanlegt. Þurfum að sýna: Ef
  $T\subset\R^{n}$ og $\lambda^{*}(T)<+\infty$, þá er
  \begin{equation}
    \label{eq:lebesgue-ojafna}
    \lambda^{*}(T)\ge \lambda^{*}(T\cap A) + \lambda^{*}(T\cap A^{C}).
  \end{equation}

  Látum $\varepsilon>0$ vera gefið; þá er til runa $(Q_{k})_{k\in\N}$
  af kössum þ.a. $T\subset\bigcup_{k\in\N}Q_{k}$ og
  $\sum_{k=0}^{+\infty}v(Q_{k})\le\lambda^{*}(T)+\varepsilon$. Nú
  Skiptir $A$ hverjum kassa $Q_{k}$ í tvo kassa (hugsanlega tóma),
  nefnilega
  \[
  Q^{-}_{k} = A\cap Q_{k}
  \quad\text{og}\quad
  Q^{+}_{k} = A^{C}\cap Q_{k}
  \]
  og ljóst er að $v(Q_{k})=v(Q_{k}^{-})+v(Q_{k}^{+})$. En þá er
  $T\cap{A}\subset\bigcup_{k\in\N}Q^{+}_{k}$,
  $T\cap{A}^{C}\subset\bigcup_{k\in\N}Q_{k}^{-}$, svo að
  \begin{align*}
    \lambda^{*}(T\cap A) + \lambda^{*}(T\cap A^{C})
    &\le \sum_{k\in\N} v(Q^{-}_{k}) + \sum_{k\in\N} v(Q^{+}_{k}) \\
    &= \sum_{k\in\N}v(Q_{k}) \\
    &\le \lambda^{*}(T)+\varepsilon.
  \end{align*}
  Látum $\varepsilon\to 0$ og fáum niðurstöðuna
  \eqref{eq:lebesgue-ojafna}.
\end{proof}
%%
\marginpar{24. jan.}
%%
\begin{lemma}
  Látum $Q$ vera kassa í $\R^{n}$ og $(Q_{k})_{k\in\N}$ vera runu af
  kössum þannig að $Q\subset\bigcup_{k\in\N}Q_{k}$. Þá er
  $v(Q)\le\sum_{k=0}^{+\infty}v(Q_{k})$.
\end{lemma}
\begin{proof}
  Niðurstaðan er augljós ef $v(Q)=0$, svo að við megum gera ráð fyrir
  að $v(Q)>0$.

  Sýnum fyrst: Ef $Q$ er lokaður kassi og $(K_{i})_{i\in A}$ er
  endanleg fjölskylda af lokuðum kössum sem eru hlutmengi í $Q$ o
  ghafa enga sameiginlega innri punkta tveir og tveir, þá er
  $\sum_{i\in I}v(K_{i})\le v(Q)$. Veljum skiptingu kassans $Q$ (sem
  fæst með því að skipta sérhverju bili sem $Q$ er margfeldi af)
  þannig að sérhver hornpunktur einhvers af kössunum $K_{i}$ sé
  skiptipunktur skiptingarinnar. Látum $L_{1},\dots,L_{r}$ vera
  hlutkassa skiptingarinnar. Þá er sérhver af kössunum $K_{i}$
  sammengi einhverra kassanna $L_1,\dots,L_r$. Látum
  \[
  B_{i} := \{  j\in\{1,\dots,r\} : L_{j}\subset K_{j}  \},
  \]
  þá er $K_{i} = \bigcup_{j\in B_{i}}L_{i}$ þar sem kassarnir $K_{i}$
  hafa enga sameiginlega innri punkta tveir og tveir eru mengin
  $B_{i}$ sundurlæg tvö og tvö, og
  $\bigcup_{i\in{I}}B_{i}\subset\{1,\dots,r\}$. Þá fæst
  \[
  \sum_{i\in I}v(K_{i})
  = \sum_{i\in I}\sum_{j\in B_{i}}v(L_{j})
  = \sum_{j\in\bigcup B_{i}} v(L_{j})
  \le \sum_{j=1}^{r}v(L_{j})
  = v(Q).
  \]
  
  Látum nú $Q$, $(Q_{k})$ vera eins og í HS, $v(Q)>0$. Látum
  $\varepsilon>0$ vera gefið. Þá má (augljóslega) finna lokaðan kassa
  $Q'$ þ.a. $Q'\subset Q$ og $v(Q)\le\frac 12\varepsilon+v(Q')$. Eins
  má finna opinn kassa $Q_{k}'$ þ.a. $Q_{k}\subset Q_{k}'$ og
  $v(Q_{k}')\le 2^{-k-2}\varepsilon+v(Q_{k})$. Þá er
  $Q'\subset{Q}\subset\bigcup_{k\in\N}Q_{k}\subset\bigcup_{k\in\N}Q_{k}'$,
  svo að $(Q_{k}')_{k\in\N}$ er opin þakning \emph{þjappaða} kassans
  $Q'$ og hefur því \emph{Lebesgue-tölu} $\delta>0$; þ.e. tala
  $\delta>0$ þ.a. sérhvert hlutmengi í $Q$ sem hefur þvermál minna en
  $\delta$ er hlutmengi í einhverju mengjanna $Q_{k}'$ í opnu
  þakningunni. Finnum skiptingu kassans $Q'$  í hlutkassa
  $K_{1},\dots,K_{s}$ sem hafa lalir þvermál minna en $\delta$. Þá má
  fyrir sérhvert $j$ úr $\{1,\dots,s\}$  finna tölu $\phi(j)$ úr $\N$
  þ.a. $K_{j}\subset Q_{\phi(j)}$, höfum þá vörpun
  $\phi:\{1,\dots,s\}\to\N$; fyrir $k$ úr $\N$ setjum við
  \[
  A_{k} := \phi^{-1}[k] = \{j\in\{1,\dots,s\} : \phi(j) = k \}.
  \]
  Þetta eru endanleg mengi, sundurlæg tvö og tvö, og
  $\bigcup_{k\in\N}A_{k}=\{1,\dots,s\}$. Nú er $(K_{j})_{j\in A_{k}}$
  endanleg fjölskylda af kössum sem eru hlutmengi í $Q_{k}^{-1}$ og
  hafa enga sameiginlega innri punkta tveir og tveir; skv. ofansögðu
  er
  \[
  \sum_{j\in A_{k}}v(K_{j})
  \le v(\overline Q_{k}')
  = v(Q_{k}').
  \]
  Þá er
  \begin{align*}
    v(Q')
    = \sum_{j=1}^{s}v(K_{j})
    &= \sum_{k=0}^{+\infty}\sum_{j\in A_{k}} v(K_{j})
    \\
    &\le \sum_{k=0}^{+\infty} v(Q_{k}')
    \\
    &\le \sum_{k=0}^{+\infty}\left(
      2^{-k-2}\varepsilon+v(Q_{k})
    \right)
    \\
    &= \frac 12\varepsilon + \sum_{k=0}^{+\infty}v(Q_{k}).
  \end{align*}
\end{proof}
Fáum:
\begin{setn}
  Sérhver kassi $Q$ er Lebesgue-mælanlegur og $\lambda(Q)=v(Q)$.
\end{setn}
\begin{proof}
  Ljóst er að $Q$ er Borel-mengi og því Lebesgue-mælanlegt: Kassi er
  sniðmengi af hálfrúmum sem mega hvort sem er vera opin eða
  lokuð. Skv. skilgreiningu er þá
  \begin{align*}
    \lambda(Q)
    &= \lambda^{*}(Q)
    \\
    &= \inf\{\sum_{k=0}^{+\infty}v(Q_{k}) : (Q_{k})\text{ runa af
      kössum sem þekja }Q \}
  \end{align*}
  svo að HS segir að $v(Q)\le\lambda^{*}(Q)=\lambda(Q)$. En nú er
  $Q=\bigcup_{k\in\N}Q_{k}$, þar sem $Q_{0}:=Q$ og $Q_{k}:=\emptyset$
  ef $k\ge 1$. Því er $\lambda^{*}(Q)\le\sum_{k=0}^{+\infty}v(Q_{k})=v(Q)$.
\end{proof}
\begin{ath}
  Við skilgreindum ytra málið $\lambda^{*}$ sem ytra málið sem
  $v:\mathcal{Q}\to\left[0,+\infty\right]$ gefur af sér, þar sem
  $\mathcal{Q}$ er mengi \emph{allra} kassa. Við hefðum eins getað
  tekið mengið $\mathcal{Q}_{1}$ af öllum lokuðum kössum eða mengið
  $\mathcal{Q}_{2}$ af öllum opnum kössum. Látum $\lambda_{1}^{*}$
  vera ytra málið sem $v|\mathcal{Q}_{1}$ gefur af sér og
  $\lambda_{2}^{*}$ vera málið sem $v|\mathcal{Q}_{2}$ gefur af
  sér. Þá er ljóst að
  \[
  \lambda^{*}\le\lambda_{1}^{*}
  \quad\text{og}\quad
  \lambda^{*}\le\lambda_{2}^{*}.
  \]
  Ef $A\subset\R^{n}$ og $(Q_{k})_{k\in\N}$ er runa af kössum sem
  þekur $A$, þá er $(\overline Q_{k})_{k\in\N}$ runa af \emph{lokuðum}
  kössum sem þekur $A$, og
  $\sum_{k=0}^{+\infty}v(\overline{Q}_{k})=\sum_{k=0}^{+\infty}v(Q_{k})$,
  svo að $\lambda_{1}^{*}\le\lambda^{*}$ og þá
  $\lambda_{1}^{*}=\lambda^{*}$. Skv. sönnun HS má líka fyrir
  $\varepsilon>0$ og runu $(Q_{k})_{k\in\N}$ af kössum sem þekur $A$
  finna runu $(Q_{k}')_{k\in\N}$ af opnum kössum sem þekur $A$ þ.a.
  \[
  \sum_{k=0}^{+\infty}v(Q_{k}')
  \le \varepsilon + \sum_{k=0}^{+\infty}v(Q_{k}),
  \]
  svo að $\lambda_{2}^{*}(A)\le\varepsilon+\lambda^{*}(A)$. Þar sem
  $\varepsilon>0$ er hvað sem vera skal er
  $\lambda_{2}^{*}\le\lambda^{*}$ og þá $\lambda_{2}^{*}=\lambda^{*}$.
\end{ath}
\begin{setn}
  Ef $\mu$ er mál á Borel-algebrunni $\mathcal{B}_{\R^{n}}$
  þ.a. $\mu(Q)=v(Q)$ fyrir alla kassa, þá er $\mu(U)=\lambda(U)$ fyrir
  öll opin mengi $U$ í $\mathbb R^{n}$.
\end{setn}
\begin{proof}
  Það nægir að sýna að sérhvert opið mengi í $\R^{n}$ sé sammengi
  sundurlægrar fjöskyldu af kössum. Látum $\mathcal C_{k}$ vera mengi
  allra kassa sem eru margfeldi af hálfopnum bilum af gerðinni
  \[
  \left]\frac j{2^{k}},\frac{j+1}{2^{k}}\right], \quad j\in\Z.
  \]
  Fyrir $k\in\N$ er $\mathcal C_{k}$ mengi af kössum sem eru
  sundurlægir tveir og tveir og
  $\bigcup_{Q\in\mathcal{C}_{k}}Q=\R^{n}$. Skilgreinum
  $\mathcal{E}_{k}$ sem hlutmengi í $\mathcal C_{k}$ með þrepun:
  \[
  \mathcal E_{0} := \{ Q\in\mathcal C_{0} : Q\subset U \}
  \]
  \[
  \mathcal E_{k+1} := \{ Q\in\mathcal C_{k+1} : Q\subset U,
  Q\nsubseteq\bigcup_{K\in\mathcal E_{0}\cup\cdots\cup\mathcal E_{k}} K
  \}.
  \]
  Þá er $(Q)_{Q\in\mathcal E}$,
  $\mathcal{E}=\bigcup_{k\in\N}\mathcal{E}_{k}$, sundurlæg fjölskylda
  af kössum þ.a. $U=\bigcup_{Q\in\mathcal{E}} Q$.
  %% 
  \marginpar{28. jan}
  %%
  Höfðum skilgreint mengi $\mathcal C_{k}$ af hálfopnum teningum
  $I_{1}\times\cdots{I_{n}}$ í $\R^{n}$, þar sem
  $I_{1},\dots,I_{n}\in{J_{k}}$, þar sem $J_{k}$ er mengi allra bila
  $\left]\frac{j}{2^{k}},{j+1}{2^{k}}\right[$, $j\in\Z$. Sjáum að
  fyrir sérhvert $k$ er $\R^{n}$ sammengi sundurlægu fjölskyldunnar
  $(Q)_{Q\in\mathcal{C}_{k}}$. Teningarnir $\mathcal{C}_{k+1}$ fást
  með því að skipta teningunum $\mathcal{C}_{k}$ í $2^{n}$ minni
  teninga; sjáum: ef $l\ge k$, $Q_{l}\in\mathcal{C}_{l}$ og
  $Q_{k}\in\mathcal{C}_{k}$, þá er annaðhvort
  $Q_{l}\cap{Q_{k}}=\emptyset$ eða $Q_{l}\in{Q_{k}}$. Fyrir opið mengi
  $U$ skilgreinum við hlutmengi $\mathcal E$ í $\mathcal
  C:=\bigcup_{k\in\N}\mathcal C_{k}$ þannig: $\mathcal E_{0}$ er mengi
  allra teninga $Q$ í $\mathcal C_0$ þ.a. $Q\subset U$. Ef $\mathcal
  E_{0},\dots,\mathcal E_{k}$ eru þekkt, þá er $\mathcal{E}_{k+1}$
  mengi allra $Q$ í $\mathcal C_{k+1}$ þ.a. $Q\subset U$ og
  $Q\nsubseteq\bigcup_{Q\in\mathcal{E}_{0}\cup\cdots\cup\mathcal{E}_{k}}Q$,
  seinna skilyrðið er jafngilt
  $Q\cap\left(\bigcup_{Q\in\mathcal{E}_{0}\cup\cdots\cup\mathcal{E}_{k}}Q\right)\ne\emptyset$. Setjum
  $\mathcal{E}:=\bigcup_{k\in\N}\mathcal{E}_{k}$. Þá er
  $(Q)_{Q\in\mathcal E}=U$: Ef $x\in U$, þá er $x$ stak í nákvæmlega
  einum teningi $Q_{k}$ úr $\mathcal C_{k}$ fyrir sérhvert $k$; ef $k$
  er það stórt að þvermál $Q_{k}$ er minna en fjarlægð $x$ frá
  $\R^{n}\setminus{U}$, þá er $Q_{k}\subset U$. Ef $k$ er
  \emph{minnsta} talan þ.a. $Q_{k}\subset U$, þá er $Q_{k}\in\mathcal
  E_{k}$ og við höfum þá $x\in\bigcup_{Q\in\mathcal E}Q$. E fnú $\mu$
  er mál á $\mathcal{B}_{\R^{n}}$ þ.a. $\mu(Q)=\lambda(Q)$ fyrir alla
  teninga úr $\mathcal C$, þá er
  \begin{align*}
    \mu(U)
    = \mu(\bigcup_{Q\in\mathcal E}Q)
    = \sum_{Q\in\mathcal E}\mu(Q)
    = \sum_{Q\in\mathcal E}\lambda(Q)
    = \lambda(\bigcup_{Q\in\mathcal E}Q)
    = \lambda(U).
  \end{align*}
  Þar em $U$ var hvaða opna mengi sem vera skal fæst $\mu(U)=\lambda(U)$
  fyrir öll opin mengi í $\R^{n}$. 
\end{proof}
\begin{ath}
  Almennar rétt: Ef $\mu, \nu$ eru mál á $\mathcal B_{\R^{n}}$ og
  $\mu(Q)=\nu(Q)$ fyrir alla teninga $Q$ í $\mathcal C$, þá er
  $\mu(U)=\nu(U)$ fyrir öll opin mengi $U$ í $\R^{n}$.
\end{ath}
\begin{setn}
  Látum $A$ vera Lebesgue-mælanlegt hlutmengi í $\R^{n}$. Þá er
  \[
  \lambda(A) = \inf\{\lambda(U) : U\text{ er opið mengi í
  }\R^{n}\text{ og } A\subset U \}
  \]
  og
  \[
  \lambda(A) = \sup\{\lambda(K) : K\text{ er þjappað mengi í
  }\R^{n}\text{ og }K\subset A \}
  \]
\end{setn}
\begin{proof}
  Ef $A\subset U$, $U$ opið, þá er $\lambda(A)\le\lambda(U)$, svo að
  \[
  \lambda(A)\le\inf\{\lambda(U) : U\text{ opið og }A\subset U \}
  \]
  er augljóst. Eins er
  \[
  \lambda(A)\ge\sup\{\lambda(K) : K\text{ þjappað og }K\subset A \}.
  \]
  Látum $\varepsilon>0$ vera gefið. Skv. athugasemd úr síðasta tíma er
  til runa $(Q_{k})_{k\in\N}$ af \emph{opnum} teningum
  þ.a. $A\subset\bigcup_{k\in\N}Q_{k}$ og
  $\sum_{k\in\N}\lambda(Q_{k})\le\lambda^{*}(A)+\varepsilon=\lambda(A)+\varepsilon$. Þá
  er $U:=\bigcup_{k\in\N}Q_{k}$ opið hlutmengi í $\R^{n}$,
  $A\subset{U}$ og $\lambda(U)\le\sum_{k\in\N}\lambda(Q_{k})$, svo að
  $\lambda(U)\le\lambda(A)+\varepsilon$. Þar sem þetta er rétt fyrir
  öll $\varepsilon>0$ fæst
  \[
  \lambda(A)\ge\inf\{\lambda(U):U\text{ opið og }A\subset U\}.
  \]
  Til að sanna að $\lambda(A)\le\sup\{\lambda(K):K\text{ þjappað og
  }K\subset A\}$: Gerum ráð fyrir að $A$ sé takmarkað. Látum þá
  $\mathcal C$ vera þjappað hlutmengi í $\R^{n}$ þannig að
  $A\subset\mathcal C$. Látum $\varepsilon>0$ vera gefið. Finnum
  skv. ofansögðu opið mengi $U$ þ.a. $C\setminus A\subset U$ og
  $\lambda(U)<\lambda(C\setminus{A})+\varepsilon=\lambda(C)-\lambda(A)+\varepsilon$. Setjum
  $K:=C\setminus U$. Þá er $K$ þjappað, $K\subset A$, $C\subset
  K\subset U$ og því $\lambda(C)\le\lambda(K)+\lambda(U)$ og þá
  $\lambda(A)-\varepsilon\le\lambda(C)-\lambda(U)\le\lambda(K)$. Þar
  sem þetta gildir fyrir öll $\varepsilon>0$ er
  $\lambda(A)\le\sup\{\lambda(K) : K\subset A \text{ og }K\text{
    þjappað }\}$. Ef $A$ er ekki takmarkað, þá skrifum við
  $A=\bigcup_{k\in\N}A_{k}$, þar sem $(A_{k})_{k\in\N}$ er vaxandi
  runa af takmrörkuðum Lebesgue-mælanlegum mengjum
  (t.d. $A_{k}:=\{x\in A : \|x\|\le k\})$. Látum $M$ vera einhverja
  takmarkaða rauntölu þ.a. $M<\lambda(A)$. Skv. setningu er
  $\lambda(A)=\lim_{k\to+\infty}\lambda(A_{k})$, svo að til er $k$
  þ.a. $\lambda(A_{k})>M$. Skv. ofansögðu er til þjappað hlutmengi $K$
  í $A_{k}$ þ.a. $\lambda(K)>M$. En þá er $K$ líka þjappað hlutmengi í
  $A$, og við höfum $\sup\{\lambda(K) : K\subset A, K\text{
    þjappað}\}>M$. Þar sem þetta gildir fyrir öll $M<\lambda(A)$ er
  $\sup\{\lambda(K):K\subset A, K\text{ þjappað}\}\ge\lambda(A)$.
\end{proof}
\begin{setn}
  Látum $\mu$ vera mál á $\mathcal{B}_{\R^{n}}$
  þ.a. $\mu(Q)=\lambda(Q)$ fyrri alla kassa $Q$, þá er
  $\mu=\lambda|\mathcal{B}_{\R^{n}}$.
\end{setn}
\begin{ath}
  Það nætir að vita að $\mu(Q)=\lambda(Q)=v(Q)$ fyrir alla teninga í
  mengi $\mathcal{C}$.
\end{ath}
\begin{proof}
  Skv. setningu er $\mu(U)=\lambda(U)$ fyrir öll opin mengi í
  $\R^{n}$. Fyrir Borel-mengi $A$ í $\R^{n}$ og opið mengi $U$
  þ.a. $A\subset U$ er þá $\mu(A)\le\mu(U)=\lambda(U)$. Skv. síðustu
  setningu fæst
  \[
  \mu(A)
  \le\inf\{\lambda(U) : U\text{ opið}, A\subset U \}
  = \lambda(A),
  \]
  þ.e. $\mu(A)\le\lambda(A)$ fyrir öll $A$ úr
  $\mathcal{B}_{\R^{n}}$. Til að sýna ójöfnu í hina áttina gerum við
  fyrst ráð fyrir að $A$ sé takmarkað og finnum takmarkað opið mengi
  $U$ þ.a. $A\subset U$. Þá fæst
  \[
  \mu(U)
  = \mu(A) + \mu(U\setminus A)
  \le \lambda(A) + \lambda(U\setminus A)
  = \lambda(U)
  = \mu(U).
  \]
  En fyrir rauntölu $a,b,c,d$ þ.a. $a\le c,b\le d$ og $a+b=c+d$ fæst
  $a=c$ og $b=d$. Fáum $\mu(A)=\lambda(A)$. Ef $A$ er ekki takmarkað,
  þá er $A$ sammengi sundurlægrar runu $(A_{k})_{k\in\N}$ af
  takmörkuðum Borel-mengjum, og þá er
  \[
  \mu(A)
  = \sum_{k=0}^{+\infty}\mu(A_{k})
  = \sum_{k=0}^{+\infty}\lambda(A_{k})
  = \lambda(A).
  \]
\end{proof}
\begin{skilgr}
  Látum $\mathcal{E}$ vera hlutmengi í $\mathcal{P}(\R^{n})$. Segjum
  að $v:\mathcal{E}\to Y$, þar sem $Y$ er mengi, sé
  \emph{hliðrunaróbreytt}\index{hlidrunarobreytt@hliðrunaróbreytt}
  (eða \emph{hliðrunaróháð}\index{hlidrunarohad@hliðrunaróháð}) ef
  fyrir sérhvert $A$ úr $\mathcal{E}$ og $x$ úr $\R^{n}$ gildir
  \[
  A+x\in\mathcal{E}
  \quad\text{og}\quad
  v(A+x)=v(A).
  \]
  Hér er $A+x := \{a+x : a\in A\}$.
\end{skilgr}
%%
\marginpar{31. jan.}
%%
Getum þá talað um hliðrunaróbreytt mál á $\sigma$-algebrum á $\R^{n}$,
um hliðrunaróbreytt ytri mál á $\R^{n}$; hliðrunaróbreytt innihöld á
algebrum á $\R^{n}$ o.s.frv.
\begin{setn}
  Ytra mál Lebesgues og Lebesgue-málið á $\R^{n}$ eru
  hliðrunaróbreytt.
\end{setn}
\begin{proof}
  Ljóst er að fyrri kassa $Q$ er $Q+x$ líka kassi og $v(Q+x)=v(Q)$. Ef
  $(Q_{k})_{k\in\N}$ er runa af kössum sem þekur hlutmengi $A$ í
  $\R^{n}$, þá er $(Q_{k}+x)_{k\in\N}$ runa af kössum sem þekur $A+x$;
  og $\sum_{k\in\N}v(Q_{k}+x)=\sum_{k\in\N}v(Q_{k})$; öfugt ef
  $(Q_{k}')$ er runa af kössum sem þekur $A+x$, þá er
  $(Q_{k}'-x)_{k\in\N}$ runa af kössum sem þekur $A$ og
  $\sum_{k\in\N}v(Q_{k}'-x)_{k\in\N}=\sum_{k\in\N}v(Q_{k}')$. Þá er
  ljóst að $\lambda^{*}(A+x)=\lambda^{*}(A)$. Nú er $A$ mælanlegt
  þ.þ.a.a. $\lambda^{*}(T)=\lambda^{*}(T\cap{A})+\lambda^{*}(T\cap{A}^{C})$
  fyrir öll $T$. En þá er
  \begin{align*}
    \lambda^{*}(T\cap(A+x)) + \lambda^{*}(T\cap(A+x)^{C})
    &= \lambda^{*}((T-x)\cap A) + \lambda^{*}((T-x)\cap A^{C})
    \\
    &= \lambda^{*}(T-x)
    \\
    &= \lambda^{*}(T),
  \end{align*}
  því að $(A+x)^{C}=A^{C}+x$ fyrir öll $x$, svo að $A+x$ er mælanlegt,
  og þá er $\lambda(A+x)=\lambda^{*}(A+x)=\lambda^{*}(A)=\lambda(A)$.
\end{proof}
Nú sjáum við:
\begin{setn}
  Til er hlutmengi $N$ í $[0,1]$ sem er ekki Lebesgue-mælanlegt.
\end{setn}
\begin{proof}[Uppkast að sönnun.]
  Við sýndum í innganginum að ekki er til hliðrunaróbreytt mál á
  $\sigma$-algebrunni $\mathcal P(X)$ með því að sýna að til væri
  hlutmengi $N$ þannig að til væri sundurlæg teljanleg fjölskylda
  $(M_{r})_{r\in I}$ þ.a.a hvert mengi $M_{r}$ væri sammengi hliðrana
  af $N\cap A$ og $N\cap A^{C}$ fyrir sérhvert bil $A$; þ.a.a öll
  $M_{r}$ yrðu að hafa sama Lebesgue-mál ef $N$ vær i æmlanlegt; en
  auk þess væri $\bigcup_{r\in I}M_{r} = [0,1]$. Ef $N$ væri
  mælanlegt, þá fengist $1=\sum_{r\in I}\lambda(M_{r})$; en summan er
  $0$ ef $\lambda(N)=0$, og $+\infty$ ef $\lambda(N)>0$. Því er $N$
  ekki mælanlegt.
\end{proof}
Rifjum upp að mengið $\mathcal C=\bigcup_{k\in\N}\mathcal C_{k}$ af
teningum, $\mathcal C_{k}$ var mengi allra teninga sem eru margfeldi
af bilum í
$\mathcal{J}_{k}:=\left\{\left]\frac{j}{2^{k}},\frac{j+1}{2^{k}}\right[:j\in\Z\right\}$.
Ef $Q\in\mathcal C_{k}$ þá er $Q$ sammengi af $2^{n}$ teningum úr
$\mathcal C_{k+1}$ sem eru sundurlægir tveir og tveir. Nú er sérhver
teningur í $\mathcal C_{k}$ hliðrun af teningi
$\left]0,\frac{1}{2^{k}}\right]^{n}=\left]0,\frac{1}{2^{k}}\right]\times\cdots\times\left]0,\frac{1}{2^{k}}\right]$.
Ef nú $\mu$ er hliðrunaróbreytt mál í $\mathcal B_{\R^{n}}$, þá hafa
allir kassar $Q_{k}$ í $\mathcal C_{k}$ sama mál, því að þeir eru
allir hliðrun af sama tening; og fyrir kassa $Q_{1}$ í
$\mathcal{C}_{k}$ og $Q_{2}$ í $\mathcal{C}_{k+1}$ gildir því
$\mu(Q_{1})=2^{n}\mu(Q_{2})$. Einföld þrepun: Ef
$c:=\mu(\left]0,1\right]^{n})$, þá er
\[
\mu(Q)=\frac{1}{2^{nk}}c \quad\text{fyrir}\quad Q\in\mathcal C_{k}.
\]
Þetta gildir sér í lagi fyrir Lebesgue-málið; þar sem
$\lambda(\left]0,1\right]^{n})=1$ fæst: Ef $\mu$ er hliðrunaróbreytt
mál á $\mathcal B_{\R^{n}}$ og $0<c<+\infty$, þá er
\[
\frac{1}{c}\mu(Q)=\lambda(Q)
\]
fyrir alla kassa $Q\in\mathcal C$. Setning gefur $\frac 1c\mu=\lambda$
á $\mathcal B_{\R^{n}}$. Höfum sýnt
\begin{setn}
  Ef $\mu$ er hliðrunaróbreytt mál á $\mathcal B_{\R^{n}}$, þá er
  \[
  \mu = c\lambda|\mathcal B_{\R^{n}},
  \]
  þar sem $c := \mu(\left]0,1\right]^{n})$.
\end{setn}
Rifjum upp að mál $\mu:\mathcal A\to\left[0,+\infty\right]$ kallast
\emph{fullkomið}\index{fullkomid@fullkomið!mal@mál}\index{mal@mál!fullkomid@fullkomið}
ef sérhvert hlutmengi í núllmengi er mælanlegt (og því
núllmengi). Vitum að Lebesgue-málið er fullkomið.
\begin{setn}
  Látum $(X,\mathcal A,\mu)$ vera málrúm og $\bar{\mathcal A}_{\mu}$
  vera mengi allra hlutmengja $A$ í $X$ þ.a. til séu mengi $E,F$ sem
  eru stök í $\mathcal A$ þ.a.
  \[
  E\subset A \subset F
  \quad\text{og}\quad
  \mu(F\setminus E) = 0.
  \]
  Þá má skilgreina vörpun
  $\bar\mu:\bar{\mathcal A}_{\mu}\to\left[0,+\infty\right]$ með því að
  setja
  \[
  \mu(A) := \mu(E) = \mu(F)
  \]
  ef $E,F$ eru eins og að framan; þetta gerir
  $\bar\mu:\bar{\mathcal A}_{\mu}\to\left[0,+\infty\right]$ að
  fullkomnu máli.
\end{setn}
\begin{skilgr}
  Köllum $(X,\bar{\mathcal A}_{\mu},\bar\mu)$ fullkomnum málrúmsins
  $(X,\mathcal A,\mu)$.
\end{skilgr}
\begin{ath}
  Sjáum að $A\in\bar{\mathcal A}_{\mu}$ þ.þ.a.a. skrifa megi
  $A=E\cup{B}$, þar sem $E\in\mathcal{A}$ og $B$ er hlutmengi í
  núllmengi úr $\mathcal A$.
\end{ath}
Byrjum á sönnun setningar. Þurfum að sjá að $\bar\mu$ sé vel
skilgreint: Látum $E,F,E_{1},F_{1}\in\mathcal{A}$,
$E\subset{A}\subset{F}$, $E_{1}\subset A\subset F_{1}$
$\mu(F\setminus E)=\mu(F_{1}\setminus E_{1})=0$. Þá er
$\mu(E)\le\mu(F_{1})$, $\mu(E_{1})\le\mu(F)$, því að
$E\subset{F_{1}}$, $E_{1}\subset F$, og
$0=\mu(F\setminus{E})=\mu(F)-\mu(E)$, svo að $\mu(E)=\mu(F)$, eins
er $\mu(E_{1})=\mu(F_{1})$.
%%
%%

\marginpar{4. feb.}

\begin{proof}
  Fyrir mál $\mu$ á $\sigma$-algebru $\mathcal A$ skilgreindum við
  mengi $\overline{\mathcal A}_{\mu}$ og vörpun
  $\overline\mu:\overline{\mathcal{A}}_{\mu}\to[0,+\infty]$ þannig:
  $A\in\overline{\mathcal A_{\mu}}$ þ.þ.a.a. til séu $E,F$ í
  $\mathcal{A}$ þ.a. $E\subset A\subset F$ og
  $\mu(F\setminus{E})=0$. Settum þá $\overline\mu(A)=\mu(E)$. Þetta er
  vel skilgreint: Ef líka gildir $E_{1}\subset{A}\subset{F_{1}}$ og
  $\mu(F_{1}\setminus{E_{1}})=0$, þá er $\mu(E_{1})=\mu(E)$ því að við
  höfum $\mu(F)=\mu(F\setminus{E})+\mu(E)=\mu(E)$, eins er
  $\mu(E_{1})=\mu(F_{1})$. Líka er $E\subset F_{1}$ og
  $E_{1}\subset{F}$, svo að $\mu(E)\le\mu(F_{1})$ og
  $\mu(E_{1})\le\mu(F)$. En þá er
  \[
  \mu(E)\le\mu(F_{1})=\mu(E_{1})\le\mu(F)=\mu(E).
  \]
  Jafnaðarmerki hljóta þá að gilda alls staðar. Sýnum að
  $\overline{\mathcal A}_{\mu}$ er $\sigma$-algebra. Ljóst er að
  $\emptyset\in\overline{\mathcal A}_{\mu}$ og
  $\overline\mu(\emptyset)=0$. Ef $A\in\overline{\mathcal A}_{\mu}$ þá
  látum við $E,F\in\mathcal A$ og $E\subset A\subset F$ og
  $\mu(F\setminus{E})=0$. Þá er $F^{C}\subset{A^{C}}\subset{E^{C}}$ og
  $F^{C},E^{C}\in\mathcal{A}$ og
  \[
  \mu(E^{C}\setminus F^{C}) = \mu(E^{C}\cap F) = \mu(F\setminus E) = 0
  \]
  svo $A^{C}\in\overline{\mathcal A}_{\mu}$. Ef $(A_{k})$ er runa í
  $\overline{\mathcal A}_{\mu}$, þá finnum við $(E_{k})$, $(F_{k})$ í
  $\mathcal A$ þannig að $E_{k}\subset A_{k}\subset F_{k}$ og
  $\mu(F_{k}\setminus E_{k})=0$ fyrir öll $k$. Þá er
  $\bigcup_{k\in\N}E_{k},\bigcup_{k\in\N}F_{k}\in\mathcal{A}$,
  $\bigcup_{k\in\N}E_{k}\subset\bigcup_{k\in\N}A_{k}\subset\bigcup_{k\in\N}F_{k}$
  og
  $(\bigcup_{k\in\N}F_{k})\setminus(\bigcup_{k\in\N}E_{k})\subset\bigcup_{k\in\N}(F_{k}\setminus{E_{k}})$
  svo
  \[
  \mu(\bigcup_{k\in\N}F_{k}\setminus\bigcup_{k\in\N} E_{k})
  \le \mu(\bigcup_{k\in\N}(F_{k}\setminus E_{k}))
  \le \sum_{k\in\N}\mu(F_{k}\setminus E_{k})
  =0.
  \]
  Því er $\bigcup_{k\in\N}A_{k}\in\overline{\mathcal A}_{\mu}$. Þar
  með er $\overline{\mathcal A}_{\mu}$ $\sigma$-algebra. Ef mengin í
  rununni $(A_{k})$ eru sundurlæg tvö og tvö, þá er runan $(E_{k})$
  líka sundurlæg. Því fæst:
  \[
  \overline\mu(\bigcup_{k\in\N}A_{k})
  = \mu(\bigcup_{k\in\N}E_{k})
  = \sum_{k=0}^{+\infty}\mu(E_{k})
  = \sum_{k=0}^{+\infty}\overline\mu(A_{k})
  \]
  svo að $\overline\mu$ er mál á $\overline{\mathcal A}_{\mu}$.
\end{proof}
\begin{skilgr}[Fullkomnun]
  \index{fullkomnun@fullkomnun!máls, málrúms}
  \index{malrum@málrúm!fullkomnun}
  \index{mal@mál!fullkomnun}
  Málrúmið $(X,\overline{\mathcal A}_{\mu},\overline\mu)$ kallast
  \emph{fullkomnun} málrúmsins $(X,\mathcal{A},\mu)$. Segjum líka að
  $\overline{\mathcal{A}}_{\mu}$ sé \emph{fullkomnun} $\mathcal A$
  m.t.t. $\mu$ og $\overline\mu$ sé fullkomnun málsins $\mu$.
\end{skilgr}
\begin{lemma}
  Látum $A$ vera Lebesgue-mælanlegt mengi í $\R^{n}$. Þá eru til
  Borel-mengi $E,F$ í $\R^{n}$ þannig að $E\subset A\subset F$ og
  $\lambda(F\setminus E)=0$. 
\end{lemma}
\begin{proof}
  Gerum fyrst ráð fyrir að $\lambda(A)<+\infty$. Fyrir sérhvert $n$ úr
  $\N$ er þá til þjappað mengi $K_{n}$ og opið mengi $U_{n}$ þannig að
  $K_{n}\subset A\subset U_{n}$ og $\mu(K_{n})>\mu(A)-\frac1{n+1}$ og
  $\mu(U_{n}<\mu(A)+\frac1{n+1}$ (skv. einhverri fyrri
  setningu). Þetta gildir fyrir öll $n$ svo að
  $\lambda(F\setminus{E})=0$. Í almenna tilvikin, þ.e. þegar
  $\lambda(A)$ er ekki nauðsynlega endanlegt, finnum við runu
  $(A_{k})_{k\in\N}$ af Lebesgue-mælanlegum mengjum $A_{k}$ þannig að
  $A=\bigcup_{k\in\N}A_{k}$, $\mu(A_{k})<+\infty$ og Borel-mengi
  $E_{k},F_{k}$ þannig að $E_{k}\subset A_{k}\subset F_{k}$ og
  $\mu(F_{k}\setminus{E_{k}})=0$. Þá er
  $\bigcup_{k\in\N}E_{k}\subset{A}\subset\bigcup_{k\in\N}F_{k}$,
  mengin $\bigcup_{k\in\N}E_{k}$ og $\bigcup_{k\in\N}F_{k}$ eru
  Borel-mengi og
  \[
  \mu(\bigcup_{k\in\N}F_{k}\setminus\bigcup_{k\in\N}E_{k})
  \le \mu(\bigcup_{k\in\N}(F_{k}\setminus{E_{k}}))
  \le \sum_{k\in\N}\mu(F_{k}\setminus E_{k})
  = 0.
  \]
\end{proof}
\begin{setn}
  Málrúmið $(\R^{n},\mathcal M_{\lambda_{n}},\lambda_{n})$ er
  fullkomnun málrúmsins
  $(\R^{n},\mathcal{B}_{\R}^{n},\lambda_{n}|\mathcal{B}_{\R^{n}})$. 
\end{setn}
\begin{proof}
  Skv. hjálparsetningu er sérhvert Lebesgue-mælanlegt mengi $A$ í
  fullkomnun $\mathcal B_{\R^{n}}$
  m.t.t. $\lambda_{n}|\mathcal{B}_{\R^{n}}$ og
  $\overline\lambda(A_{n})=\lambda(A)$ þar sem $\overline\lambda$ er
  fullkmnun $\lambda|\mathcal{B}_{\R^{n}}$. Þá þarf bara að sýna að
  sérhvert mengi í fullkomnun $\mathcal{B}_{\R^{n}}$
  m.t.t. $\lambda|\mathcal{B}_{\R^{n}}$ sé Lebesgue-mælanlegt. En ef
  $A$ er í þeirri fullkomnun, þá eru til Borel-mengi $E,F$
  þ.a. $E\subset A\subset F$ og $\lambda(F\setminus E)=0$. Þá er
  $A\setminus E\subset F\setminus E$ og þar sem Lebesgue-málið er
  fullkomið er $A\setminus E$ Lebesgue-mælanlegt. En þá er
  $A=E\cup(A\setminus E)$ líka Lebesgue-mælanlegt.
\end{proof}
\begin{fylgi}
  Hlutmengi $A$ í $\R^{n}$ er Lebesgue-mælanlegt þ.þ.a.a. til séu
  Borel-mengi $E$ og núllmengi $N$ þ.a. $A=E\cup N$. 
\end{fylgi}
\begin{ath}
  Hlutmengi $N$ í $\R^{n}$ er núllmengi (m.t.t. $\lambda$)
  þ.þ.a.a. $\lambda^{*}(N)=0$, þ.e. fyrir hvert $\varepsilon>0$ er til
  runa $(Q_{k})_{k\in\N}$ af kössum
  þ.a. $N\subset\bigcup_{k\in\N}Q_{k}$ og
  $\sum_{k\in\N}v(Q_{k})<\varepsilon$.
\end{ath}

\chapter{Mælanleg föll}

Við segjum að hlutmengi $U$ í $\tilde\R=\R\cup\{-\infty,+\infty\}$ sé
\emph{grennd um $\infty$}\index{grenndir@grenndir um!rauntölur,
  óendanlegt} ef það inniheldur bil $\left]a,+\infty\right]$ fyrir
eitthvert $a\in\R$, \emph{grennd um $-\infty$} e fþað inniheldur bil
$\left[-\infty,a\right[$, $a\in\R$; \emph{grennd um rauntölu} $x$ ef
það inniheldur opið bil um $x$. Segjum að hlutmengi í $\tilde\R$ sé
\emph{opið} ef það er grennd um sérhvern punkt
sinn. \emph{Borel-algebran} $\mathcal{B}_{\tilde\R}$ er
$\sigma$-algebran sem er spönnuð af opnum hlutmengjum í $\tilde\R$
Auðséð er að $A\in\mathcal{B}_{\tilde\R}$
þ.þ.a.a. $A\cap\R\in\mathcal{B}_{\R}$, þ.e. til er
$B\in\mathcal{B}_{\R}$ þ.a. $A$ sé eitt af mengjunum $B$,
$B\cup\{+\infty\}$, $B\cup\{-\infty\}$, $\{-\infty,+\infty\}\cup B$.
\begin{skilgr}
  \index{mzaelanleg@mælanleg!vörpun}
  \index{mzaelanleg@mælanleg!fzoll@föll}

  Látum $(X,\mathcal A)$ og $(Y,\mathcal B)$ vera mælanleg rúm.
  \begin{enumerate}[(1)]
  \item \emph{Mælanleg vörpun} $f:(X,\mathcal A)\to(Y,\mathcal B)$ er
    vörpun $f:X\to Y$ þannig að $f^{-1}[B]\in\mathcal A$ fyrir öll
    $B\in\mathcal{B}$.
  \item \emph{Mælanlegt raunfall} á $(X,\mathcal A)$ er mælanleg
    vörpun $(X,\mathcal A)\to(\R,\mathcal{B}_{\R})$, \emph{mælanlegt
      tvinnfall} á $(X,\mathcal A)$ er mælanleg vörpun
    $(X,\mathcal{A})\to(\C,\mathcal B_{\C})$.
  \end{enumerate}
\end{skilgr}


%%
%%
\marginpar{7. feb.}
\begin{skilgr}
  \index{mzaelanleg@mælanleg!vzorpun@vörpun}
  Látum $f:X\to Y$ vera vörpun milli mengja og $\mathcal E$ vera mengi
  af hlutmengjum í $Y$. Setjum
  \[
  f^{*}[\mathcal E] := \{ f^{-1}[E] : E\in\mathcal E \}.
  \]
  Þá getum við umorðað skilgreiningu á mælanlegri vörpun þannig:
  Mælanleg vörpun frá mælanlegu rúmi $(X,\mathcal A)$ í mælanlegt rúm
  $(Y,\mathcal B)$ er vörpun $f:X\to Y$ þ.a.
  $f^{*}[\mathcal{B}]\subset{\mathcal A}$.
\end{skilgr}
\begin{setn}
  \begin{enumerate}[(1)]
  \item Ef $(X,\mathcal A)$ er mælanlegt rúm, þá er $\id_{X}:X\to X$
    mælanleg  vörpun $(X,\mathcal A)\to(X,\mathcal A)$.
  \item Ef $f$ er mælanleg vörpun frá $(X,\mathcal A)$ í
    $(Y,\mathcal{B})$, og $g$ er mælanleg vörpun frá $(Y,\mathcal{B})$
    í $(Z,\mathcal{C})$, þá er $g\circ f$ mælanleg vörpun frá
    $(X,\mathcal{A})$ í $(Z,\mathcal{C})$.
  \end{enumerate}
\end{setn}
\begin{proof}
  Höfum $g^{*}[\mathcal C]\subset\mathcal B$, svo að
  \[
  (g\circ{f})^{*}(\mathcal{C})
  = f^{*}(g^{*}(\mathcal{C}))\subset{f^{*}(\mathcal{B})}
  \subset \mathcal A.
  \]
\end{proof}
\begin{lemma}
  Látum $f:X\to Y$ vera vörpun, $\mathcal B$ vera $\sigma$-algebru á
  $Y$ sem er spönnuð af hlutmengi $\mathcal E$ í $\mathcal P(Y)$, þá
  er $f^{*}[\mathcal B]$ $\sigma$-algebra á $X$ spönnuð af
  $f^{*}[\mathcal E]$.
\end{lemma}
\begin{proof}
  Að $f^{*}[\mathcal B]$ sé $\sigma$-algebra er afleiðing af því að
  $f^{-1}[\emptyset]=\emptyset$ og
  \begin{equation}
    \label{eq:hs-1}
    X\setminus f^{-1}[B]
    = f^{-1}[Y\setminus B]
    \quad\text{og}\quad
    f^{-1}[\bigcup_{k\in\N} B_{k}]
    = \bigcup_{k\in\N}f^{-1}[B_{k}]
  \end{equation}
  fyrir öll hlutmengi $B$ í $Y$ og allar runur $(B_{k})_{k\in\N}$ af
  hlutmengjum í $Y$. Látum $\mathcal A$ vera $\sigma$-algebruna á $X$
  sem $f^{*}[\mathcal E]$ spannar, þá er
  $\mathcal{A}\subset{f^{*}[\mathcal{B}]}$, því að
  $f^{*}[\mathcal{B}]$ er $\sigma$-algebra sem inniheldur
  $f^{*}[\mathcal{E}]$. Setjum nú
  \[
  \mathcal F := \{
    B\subset Y : f^{-1}[B]\in\mathcal A
  \}.
  \]
  Af \eqref{eq:hs-1} leiðir að $\mathcal F$ er $\sigma$-algebra á $Y$
  og skv. skilgreiningu á $\mathcal F$ er
  $f^{*}[\mathcal{F}]\subset\mathcal{A}$. En líka er
  $\mathcal{E}\subset\mathcal{F}$, því að
  $f^{*}[\mathcal{E}]\subset\mathcal{A}$, svo að $\mathcal{F}$ er
  $\sigma$-algebra sem inniheldur $\mathcal{E}$, en $\mathcal{B}$ er
  minnsta slík $\sigma$-algebran, svo að
  $\mathcal{B}\subset\mathcal{F}$ og þar með
  \[
  f^{*}[\mathcal{B}]
  \subset f^{*}[\mathcal{F}]
  \subset \mathcal A.
  \]
  Því er $f^{*}[\mathcal B] = \mathcal A$, þ.e. $f^{*}[\mathcal{E}]$
  spannar $f^{*}[\mathcal{B}]$. 
\end{proof}
Sem afleiðingu fáum við
\begin{setn}
  Ef $(X,\mathcal A)$, $(Y,\mathcal B)$ eru mælanleg rúm, $f:X\to{Y}$
  er vörpun, $\mathcal B$ er $\sigma$-algebran spönnuð af hlutmengi
  $\mathcal E$ í $\mathcal P(Y)$ og
  $f^{*}[\mathcal{E}]\subset\mathcal{A}$, þá er $f$ mælanleg vörpun
  frá $(X,\mathcal{A})$ í $(Y,\mathcal{B})$.
\end{setn}
\begin{ath}
  Ef $X$ er mengi, $f:X\to\R$ er vörpun, þá er frummynd bilsins
  $\left[a,+\infty\right[$ mengið $\{x\in X : f(x)\ge a\}$; ef
  $f:X\to\tilde\R$ er vörpun, þá er $\{x\in X : f(x)\ge a\}$ frummynd
  mengisins $\left[a,+\infty\right]$ m.t.t. $f$. Eins fyrir strangar
  ójöfnur. Fáum:
\end{ath}
\begin{setn}
  Látum $(X,\mathcal A)$ vera mælanlegt rúm og $f$ vera fall frá $X$ í
  $\R$ eða $\tilde\R$. Þá er jafngilt:
  \begin{enumerate}[(i)]
  \item Fallið $f$ er mælanlegt.
  \item Fyrir sérhverja rauntölu $a$ er mengið
    $\{x\in{X}:f(x)\ge{a}\}$ mælanlegt.
  \item Fyrir sérhverja rauntölu $a$ er mengið $\{x\in{X}:f(x)>{a}\}$
    mælanlegt.
  \item Fyrir sérhverja rauntölu $a$ er mengið
    $\{x\in{X}:f(x)\le{a}\}$ mælanlegt.
  \item Fyrir sérhverja rauntölu $a$ er mengið $\{x\in{X}:f(x)<{a}\}$
    mælanlegt.
  \end{enumerate}
  Vörpun $f:X\to\tilde\R$ er mælanleg þ.þ.a.a. $f^{-1}[\mathcal{B}]$
  sé mælanlegt fyrir sérhver Borel-mengi í $\R$ \emph{og} mengin
  $f^{-1}[+\infty]$ og $f^{-1}[-\infty]$ séu einnig mælanleg.
\end{setn}
\begin{setn}
  Látum $f:X\to Y$ vera samfellda vörpun milli firðrúma (grannrúma),
  þá er $f$ mælanleg vörpun frá $(X,\mathcal{B}_{X})$ til
  $(Y,\mathcal{B}_{Y})$.
\end{setn}
\begin{proof}
  Látum $\mathcal{T}_{X}$ vera mengi opnu mengjanna í $X$,
  $\mathcal{T}_{Y}$ vera mengi opnu mengjanna í $Y$. Að $f$ sé
  samfelld þýðir að $f^{*}[\mathcal{T}_{Y}]\subset\mathcal{T}_{X}$;
  en $\mathcal{T}_{Y}$ spannar $\mathcal{B}_{Y}$ og $\mathcal{T}_{X}$
  spannar $\mathcal{B}_{X}$, svo að
  $f^{*}[\mathcal{B}_{Y}]\subset\mathcal{B}_{X}$ skv. setningu.
\end{proof}
\begin{ath}
  Látum $\mathcal{T}_{\tilde\R}$ vera mengi opnu hlutmengjanna í
  $\tilde\R=\R\cup\{-\infty,+\infty\}$. Við getum talað um að vörpun
  $f:X\to\tilde\R$ frá firðrúmi $X$ sé
  \emph{samfelld}\index{samfelldni} ef
  $f^{*}[\mathcal{T}_{Y}]\subset\mathcal{T}_{X}$; eins má tala um
  samfelldar varpanir $\tilde\R\to X$ eða
  $\tilde\R\to\tilde\R$. T.d. er vörpunin $f:\tilde\R\to\tilde\R$
  þ.a. $f(x):=x^{2}$, fyrir öll $x$, samfelld; hér er
  \[
  (+\infty)^{2} := +\infty =: (-\infty)^{2}.
  \]
  Þessi vörpun er mælanleg. Eins er vörpun
  $[0,+\infty]\to[0,+\infty]$, $x\mapsto\sqrt{x}$, þar sem
  $\sqrt{+\infty}:=+\infty$, samfelld og því mælanleg.
\end{ath}
\begin{setn}
  Látum $(X,\mathcal A)$ vera mælanlegt rúm og $f,g:X\to\R$ vera
  mælanleg föll og $c\in\R$. Þá eru föllin
  \[
  f+g,       \quad
  cf,        \quad
  fg,        \quad
  \max(f,g), \quad
  \min(f,g), \quad
  |f|
  \]
  mælanleg.
\end{setn}
\begin{ath}
  Hér er $\max(f,g)(x):=\max(f(x),g(x))$ og
  $\min(f,g)(x)=\min(f(x),g(x))$.
\end{ath}
\begin{proof}
  Til að sjá að $f+g$ sé mælanlegt þarf að sjá að
  \[
  A_{a} := \{ x\in X : f(x)+g(x)<a \}
  \]
  sé mælanlegt, ef $a\in\R$ er gefið. En $f(x)+g(x)<a$ jafngildir því
  að $f(x)<a-g(x)$, sem aftur jafngildir því að til sé ræð tala $r$
  þ.a. $f(x)<r<a-g(x)$, þ.e. $f(x)<r$ \emph{og} $g(x)<a-r$. Því er
  \[
  A_{a}
  = \bigcup_{r\in\Q}\left(
    \{ x\in X : f(x) < r \}
    \cap
    \{ x\in X : g(x) < a-r \}
  \right)
  \]
  svo að $A_{a}$ er sammengi teljanlegrar fjölskyldu af mælanlegum
  mengjum og því mælanlegt.

  Fallið $cf$ er samskeyting $f$ og samfellda fallsins
  $\R\to\R,x\mapsto{cx}$ og því mælanlegt.

  Fallið $f^{2}$ er samskeyting $f$ og samfellda fallsins
  $\R\to\R,x\mapsto{x^{2}}$ og því mælanlegt. En þá er líka
  \[
  fg = \frac 14 ((f+g)^{2}-(f-g)^{2})
  \]
  mælanlegt.

  Höfum
  \[
  \{ x\in X : \max(f,g)(x)\le a \}
  = \{ x\in X : f(x)\le a \} \cap \{x\in X : g(x)\le a \},
  \]
  \[
  \{ x\in X : \min(f,g)(x)\le a \}
  = \{ x\in X : f(x)\le a \} \cup \{x\in X : g(x)\le a \},
  \]
  svo að $\max(f,g)$ og $\min(f,g)$ eru mælanleg.

  Loks er $|f|$ samskeyting af $f$ og samfellda fallinu
  $\R\to\R,x\mapsto|x|$ og því mælanlegt.
\end{proof}
\begin{setn}
  Sama setning gildir um föll $f,g\to\tilde\R$ að því gefnu að
  $f+g,fg$ séu skilgreind.
\end{setn}
\begin{ath}
  Getum hér
  \[
  0\cdot(+\infty)
  =(+\infty)\cdot0
  =0\cdot(-\infty)
  =(-\infty)\cdot0
  =0.
  \]
  Eins verður $f+g$ mælanlegt ef við setjum
  \[
  (+\infty)+(-\infty)=(-\infty)+(+\infty)=0.
  \]
\end{ath}
%%
%%
\marginpar{11. feb.}
\begin{skilgr}
  Fyrir fall $f:X\to\tilde\R$ setjum við
  \[
  f^{+} := \max\{f,0\},
  \quad
  f^{-} := \max\{-f,0\} = -\max\{f,0\}.
  \]
  Höfum þá $f = f^{+} - f^{-}$ og $|f| = f^{+} + f^{-}$. 
\end{skilgr}
\begin{fylgi}
  Fall $f:X\to\tilde\R$ á málrúmi $X$ er mælanlegt þ.þ.a.a. bæði
  föllin $f^{+}$ og $f^{-}$ séu mælanleg.
\end{fylgi}
\begin{setn}
  Látum $(f_{k})$ vera runu af mælanlegum föllum
  $f_{n}:X\to\tilde\R$. Þá eru föllin
  \[
  \sup_{k\in\N}f_{k},
  \quad
  \inf_{k\in\N}f_{k},
  \quad
  \limsup_{k\to+\infty}f_{k},
  \quad
  \liminf_{k\to+\infty}f_{k}
  \]
  mælanleg.
\end{setn}
\begin{proof}
  Skrifum $f:=\sup_{k\in\N}f_{k}$ og $g:=\inf_{k\in\N}f_{k}$, þá er
  \[
  \{x\in X : f(x)\le a \}
  = \bigcap_{k\in\N} \{x : f_{k}(x) \le a \}
  \]
  mælanlegt, og eins er
  \[
  \{x\in X : f(x)\ge a \}
  = \bigcap_{k\in\N} \{x : f_{k}(x) \ge a \}
  \]
  mælanlegt. Þá eru
  \[
  \limsup_{k\to+\infty}f_{k}
  = \inf_{k\in\N}\sup_{j\ge k} f_{j},
  \quad
  \liminf_{k\to+\infty}f_{k}
  = \sup_{k\in\N}\inf_{j\ge k} f_{j}
  \]
  líka mælanleg.
\end{proof}
\begin{fylgi}
  Ef $(f_{k})$ er runa af mælanlegum föllum $f_{k}:X\to\tilde\R$ og
  fyrir sérhvert $x\in X$ hefur runan $(f_{k}(x))_{k\in\N}$ markgildi
  í $\tilde\R$, þá er $\lim_{k\to+\infty}f_{k}:X\to\tilde\R$ mælanlegt
  fall.
\end{fylgi}
\begin{proof}
  Þá er
  $
  \lim_{k\to+\infty}f_{k}
  = \limsup_{k\to+\infty}f_{k}
  = \liminf_{k\to+\infty}f_{k}
  $.
\end{proof}
\begin{skilgr}
  Skilgreinum fall $\sigma_{k}:\R\to\left[0,+\infty\right[$ með
  \[
  \sigma_{k}(x)
  :=
  \begin{cases}
    0, & x < 0,\\
    \frac j{2^{k}}, & x\in\left[\frac
      j{2^{k}},\frac{j+1}{2^{k}}\right[, j=0,1,\dots,k2^{k}-1,\\
    k & x\ge k.
  \end{cases}
  \]
\end{skilgr}
\begin{setn}
  Látum $X$ vera mengi og $f:X\to[0,+\infty]$ vera fall. Þá er
  $(\sigma_{k}\circ f)_{k\in\N}$ vaxandi runa af föllum þannig að
  $\lim_{k\to+\infty}\sigma_{k}\circ f = f$; ef $X$ er mælanlegt rúm
  og $f$ er mælanlegt fall, þá eru föllin $\sigma_{k}\circ f$
  mælanleg. Ef $f$ er takmarkað á $X$, þá er $(\sigma_{k}\circ f)$
  samleitin í jöfnum mæli á $X$.
\end{setn}
\begin{proof}
  Ljóst er að $(\sigma_{k})_{k\in\N}$ er vaxandi runa af mælanlegum
  föllum, og
  \[
  \lim_{k\to+\infty}\sigma_{k}(t)
  = t^{+}
  = \max\{t,0\}
  \]
  fyrir öll $t$. Fyrir $x\in X$ fæst þá
  $\lim_{k\to+\infty}\sigma_{k}(f(x))=f(x)$ ef
  $f(x)\in\left[0,+\infty\right[$, en ef $f(x)=+\infty$, þá er
  $\sigma_{k}(f(x))\ge k$ fyrir öll $k$, svo að
  $\lim_{k\to+\infty}(\sigma_{k}\circ f)=+\infty=f(x)$. Ef $f$ er
  mælanlegt, þá er samskeytingin $\sigma_{k}\circ f$ mælanlegt fall
  $f$ fyrir öll $k$. Ef $f$ er takmarkað, $0\le f\le M < +\infty$, þá
  fæst $0\le f(x)-(\sigma_{k}\circ f)(x)\le\frac{1}{2^{k}}$ fyrir $k>M$
  og öll $x$ úr $X$, svo að samleitnin er í jöfnum mæli á $X$.
\end{proof}
\begin{skilgr}
  \begin{enumerate}[(1)]
  \item Látum $X$ vera mengi. Fall $f:X\to\R$ kallast \emph{einfalt
      fall}\index{einfalt fall} ef það tekur bara endanlega mörg
    gildi, þ.e. myndmengið $f[X]$ er endanlegt.
  \item Látum $X$ vera mengi. Fall $f:X\to\R$ kallast
    \emph{kennifall}\index{kennifall} ef það tekur bara gildin $0$ og
    $1$; þá ákvarðast $f$ af menginu
    \[
    A
    := f^{-1}[1]
    = \{ x\in X : f(x) = 1 \},
    \]
    og við köllum $f$ \emph{kennifall
      mengisins}\index{kennifall!mengis} ($A$ í menginu $X$) og táknum
    það með $\chi_{A}$.
  \end{enumerate}
\end{skilgr}
\begin{ath}
  Kennifall mengis $A$ í $X$ er þá gefið með
  \[
  \chi_{A}(X)
  = \begin{cases}
    0, & x\in X\setminus A,\\
    1, & x\in A.
  \end{cases}
  \]
  Ljóst er að fall $f:X\to\R$ er \emph{einfalt} þ.þ.a.a. það megi
  skrifa sem summu
  \begin{equation}
    \label{eq:einfalt-summa}
    f = \sum_{j=1}^{k}a_{j}\chi_{A_{j}}
  \end{equation}
  þar sem $a_{1},\dots,a_{k}\in\R$ og $A_{1},\dots,A_{k}$ eru
  hlutmengi í $X$.  Við getum valið $a_{1},\dots,a_{k}$ ólík og
  ekki-núll, og $A_{1},\dots,A_{k}$ sundurlæg tvö og tvö, og þá
  ákvarðast framsetningin ótvírætt burtséð frá röð, því að þá er
  $a_{1},\dots,a_{k}$ upptalning á $f[X]\setminus\{0\}$ og
  $A_{j}=f^{-1}[a_{j}]$ fyrir $j=1,\dots,k$. Ef $X$ er málrúm og
  \eqref{eq:einfalt-summa} er þannig framsetning á $f$, þá er $f$
  mælanlegt þ.þ.a.a. mengin $A_{1},\dots,A_{k}$ séu öll mælanleg, því
  að fyrir hlutmengi $B$ í $\tilde\R$ er $f^{-1}[B]$ sammengi
  einhverra af mengjunum $A_{1},\dots,A_{k}$. Ljóst er: Ef
  $f=\sum_{j=1}^{k}a_{j}\chi_{A_{j}}$ er einhver framsetning einfalds
  falls og $A_{1},\dots,A_{k}$ eru mælanleg, þá er $f$ mælanlegt. 
\end{ath}
Af ofansögðu leiðir:
\begin{setn}
  Fall $f:X\to[0,+\infty]$ á mælanlegu rúmi er mælanlegt þ.þ.a.a. það
  sé markgildi runu $(u_{k})_{k\in\N}$ af einföldum mælanlegum föllum
  $u_{k}:X\to[0,+\infty]$ í þeim skilningi að $(u_{k}(x))_{k\in\N}$
  hefur markgildið $f(x)$ í $\tilde\R$ fyrir sérhvert $x\in X$.
\end{setn}
\begin{skilgr}
  Látum $(X,{\cal A},\mu)$ vera málrúm og
  $u:X\to\left[0,+\infty\right[$ vera einfalt mælanlegt fall; skrifum
  $u=\sum_{j=1}^{k}a_{j}\chi_{A_{j}}$, þar sem
  $a_{1},\dots,a_{k}\in\left[0,+\infty\right[$ og $A_{1},\dots,A_{k}$
  eru mælanleg hlutmengi í $X$. Setjum
  \[
  \int_{X} u\,d\mu
  := \sum_{j=1}^{k}a_{j}\mu(A_{j}).
  \]
\end{skilgr}
Þurfum að sýna að þessi skilgreining sé óháð framsetningunni.
\begin{ath}
  Hér setjum við $0\cdot(+\infty)=0$.
\end{ath}
Þurfum fyrst að sanna:
%%
%%
\marginpar{14. feb.}
\begin{lemma}
  Látum $(X,\mathcal A,\mu)$ vera málrúm, $f:X\to\R$ vera einfalt
  mælanlegt fall þ.a. $f\ge 0$. Setjum fallið $f$ fram sem línulega
  samantekt
  \[
  f = \sum_{j=1}^{k}a_{j}\chi_{A_{j}}
  \]
  þar sem $A_{1},\dots,A_{k}$ eru mælanleg mengi og $a_{j}\ge 0$ fyrir
  öll $j=1,\dots,m$. Þá er stærðin
  \[
  \sum_{j=1}^{k}a_{j}\mu(A_{j})
  \]
  óháð valinu á þessari framsetningu. Nánar tiltekið, ef
  $A_{k+1},\dots,A_{l}$ eru mælanleg og
  $a_{k+1},\dots,a_{l}\in[0,+\infty]$ eru þannig að
  \[
  \sum_{j=1}^{k}a_{j}\mu(A_{j})
  = \sum_{j=k+1}^{l}a_{j}\mu(A_{j})
  \]
  þá er
  \[
  \sum_{j=1}^{k}a_{j}\mu(A_{j})
  = \sum_{j=k+1}^{l} a_{j}\mu(A_{j}).
  \]
\end{lemma}
\begin{proof}
  Myndum öll mengi af gerðinni $\bigcap_{j=1}^{l}B_{j}$ þar sem
  $B_{j}=A_{j}$ eða $B_{j}=A_{j}^{C}$, og látum
  $C_{1},\dots,C_{2^{l}}$ vera upptalningu þeirra. Ljóst er að
  $C_{1},\dots,C_{2^{l}}$ eru sundurlæg tvö og tvö, því að fyrir tvö
  þeirra er til $j$ þannig að annað sé í $A_{j}$ en hitt í
  $A_{j}^{C}$. Hvert $A_{j}$ er sammengi einhverra af mengjunum
  $C_{i}$. Á $C_{i}$ er $f$ fast. Því má skrifa
  $f=\sum_{i=1}^{2^{l}}c_{i}\chi_{C_{i}}$. Af samhverfuástæðum nægir
  þá að sýna að
  \[
  \sum_{j=1}^{k}a_{j}\mu(A_{j})
  = \sum_{j=1}^{2^{l}}c_{i}\mu(C_{i}).
  \]
  En
  \[
  c_{i}
  = \sum_{\substack{j=1\\C_{i}\subset A_{j}}}^{2^{l}}a_{i}
  \]
  svo að
  \[
  \sum_{j=1}^{k}a_{j}\mu(A_{j})
  = \sum_{j=1}^{k}a_{j}
  \sum_{\substack{j=1\\C_{i}\subset A_{j}}}^{2^{l}} a_{i}\mu(C_{i})
  = \cdots
  = \sum_{j=1}^{2^{l}}c_{i}\mu(C_{i}).
  \]

  \emph{Framhald:} Höfðum skilgreint mengi $C_{1},\dots,C_{2^{l}}$, sundurlæg tvö og
  tvö, þ.a. fyrir sérhvert $j=1,\dots,k $og $i=1,\dots,2^{l}$ er
  $C_{i}$ annaðhvort innihaldið í $A_{j}$ eða í $A_{j}^{C}$ og þannig
  að sérhvert $A_{j}$ er sammengi einhverra af mengjunum
  $C_{i}$. Fyrir $j=1,\dots,k$ og $i=1,\dots,2^{l}$ er þá
  $\chi_{A_{j}}$ fast á $C_{i}$, svo að við getum skrifað
  $f=\sum_{i=1}^{2^{l}}c_{i}\chi_{C_{i}}$ og þá
  \[
  c_{i} = \sum_{\substack{j=1\\C_{i}\subset A_{j}}}^{k}a_{j}.
  \]
  Fáum þá að
  \begin{align*}
    \sum_{i=1}^{2^{l}}c_{i}\mu(C_{i})
    &= \sum_{i=1}^{2^{l}}\sum_{j=1}^{k}a_{j}\mu(C_{i}) \\
    &= \sum_{(i,j)\in\{1,\dots,2^{l}\}\times\{1,\dots,k\}}a_{j}\mu(C_{i}) \\
    &= \sum_{j=1}^{k}\sum_{\substack{i=1\\C_{i}\subset A_{j}}}^{2^{l}}a_{j}\mu(C_{i})\\
    &= \sum_{j=1}^{k}a_{j}\sum_{\substack{i=1\\C_{i}\subset A_{j}}}^{2^{l}}\mu(C_{i})\\
    &= \sum_{j=1}^{k}a_{j}\mu(A_{j}),
  \end{align*}
  því að
  \[
  A_{j}=\bigcup_{\substack{i=1\\C_{i}\subset A_{j}}}^{2^{l}}C_{i}
  \]
  og mengin $C_{i}$ eru sundurlæg tvö og tvö. Sáum síðast að þetta
  nægði.
\end{proof}
\begin{ath}
  Sönnunin gengur óbreytt án forsendunnar $f\ge 0$ og $a_{j}\ge 0$
  fyrir öll $j$, \emph{að því gefnu} að málið $\mu$ sé endanlegt. Ef
  það er ekki endanlegt, þá þarf summan
  $\sum_{j=1}^{k}a_{j}\mu(A_{j})$ ekki að vera vel skilgreind, ef ekki
  er gert ráð fyrir að $a_{1},\dots,a_{k}$ hafi sama formerki, því að
  þá gætu liðirnir $+\infty$ og $-\infty$ komið fyrir í sömu summunni.
\end{ath}
Fyrir mælanlegt einfalt fall $f$ þannig að $f\ge 0$ getum við
skilgreint
\index{heildi}
\[
\int_{X}f\,d\mu := \sum_{j=1}^{k}a_{j}\mu(A_{j}),
\]
þar sem mengin $a_{j},A_{j}$ eru eins og í setningu.
\begin{lemma}
  Látum $(X,\mathcal A,\mu)$ vera málrúm, $v$ vera einfalt mælanlegt
  fall á $X$ og $(u_{k})_{k\in\N}$ vera vaxandi runu af einföldum
  mælanlegum föllum á $X$ þannig að $v\ge 0$ og $u_{k}\ge 0$ fyrir öll
  $k$ og $v\le\lim_{k\to+\infty}u_{k}$. Þá er
  \[
  \int_{X}v\,d\mu \le \lim_{k\to+\infty}\int_{X}u_{k}\,d\mu.
  \]
\end{lemma}
\begin{proof}
  Skrifum $v=\sum_{j=1}^{m}a_{j}\chi_{A_{j}}$, þar sem
  $A_{1},\dots,A_{m}$ eru mælanleg mengi, sundurlæg tvö og tvö, og
  $a_{j}\ge 0$ fyrir $j=1,\dots,m$. Látum $\beta$ vera fasta rauntölu,
  $\beta>1$, og skilgreinum
  \[
  B_{k} := \{ x\in X : \beta u_{k}(x)\ge v(x) \}
  \]
  fyrir $k\in\N$. Þá er $(B_{k})_{k\in\N}$ vaxandi runa af mælanlegum
  hlutmengjum í $X$. Ef $v(x)=0$, þá er $x\in B_{k}$ fyrir öll $k$; en
  ef $v(x)>0$, þá er
  \[
  \lim_{k\to+\infty}\beta u_{k}(x)
  \ge \beta v(x)
  > v(x)
  \]
  svo að $x\in B_{k}$ fyrir öll nógu stór $k$. Því er
  $X=\bigcup_{k\in\N}B_{k}$. Því fæst
  \begin{align*}
    \int_{X}v\,d\mu
    &= \sum_{j=1}^{m}a_{j}\mu(A_{j}) \\
    &= \lim_{k\to+\infty} \sum_{j=1}^{m} a_{j}\mu(A_{j}\cap B_{k}) \\
    &= \lim_{k\to+\infty} \int_{X} v\chi_{B_{k}}\,d\mu \\
    &\le \lim_{k\to+\infty}\beta\int_{X} u_{k}\,d\mu \\
    &= \beta\lim_{k\to+\infty}\int_{X}u_{k}\,d\mu
  \end{align*}
  þar sem þetta gildir fyrir öll $\beta>1$ fæst niðurstaðan.
\end{proof}
\begin{fylgi}
  Látum $(X,\mathcal A,\mu)$ vera málrúm, $f:X\to[0,+\infty]$ vera
  mælanlegt fall, $(u_{k})_{k\in\N}$ vera vaxandi runu af einföldum
  mælanlegum föllum þ.a. $u_{k}\ge 0$ og
  $\lim_{k\to+\infty}u_{k}=f$. Þá er
  \[
  \lim_{k\to+\infty}\int_{X}u_{k}\,d\mu
  = \sup \{ \int_{X} v\,d\mu : v\text{ einfalt, mælanlegt og }
  0 \le v\le f \}.
  \]
  Sér í lagi er markgildið óháð vali á rununni.
\end{fylgi}
\begin{proof}
  ``$\le$'' er augljóst. Til að sjá ``$\ge$'', látum við
  $M<\sup\{\int_{X}v\,d\mu : \dots \}$; þá er til einfalt mælanlegt
  fall $v$ þ.a. $0\le v \le f$ og $\int_{X}v\,d\mu>M$; skv. setningu
  er þá
  \[
  \lim_{k\to+\infty}\int_{X}u_{k}\,d\mu
  \ge\int_{X}v\,d\mu
  > M.
  \]
  Þar sem þetta gildir fyrir öll slík $M$ fæst niðurstaðan.
\end{proof}
\begin{skilgr}
  Látum $(X,\mathcal A,\mu)$ vera málrúm og $f:X\to[0,+\infty]$ vera
  mælanlegt fall. Við skilgreinum \emph{heildi}\index{heildi} fallsins
  $f$ m.t.t. málsins $\mu$ með
  \[
  \int_{X} f\, d\mu
  := \sup \{ \int_{X} v,d\mu : \text{v er einfalt, mælanlegt og }
  0 \le v \le f \}.
  \]
\end{skilgr}
\begin{ath}
  Skv. fylgisetningu er
  \[
  \int_{X} f\, d\mu = \lim_{k\to+\infty} \int_{X} u_{k}\,d\mu
  \]
  fyrir śerhverja \emph{vaxandi} runu $(u_{k})$ af mælanlegum
  einföldum föllum $u_{k}$ þannig að $0\le u_{k}\le f$ og
  $\lim_{k\to+\infty}u_{k}=f$. Slíkar runur eru alltaf til
  skv. setningu.
\end{ath}
\begin{ath}
  Kennifall mælanlegs mengis$A$ hefur vel skilgreint heildi, og
  $\int_{X}\chi_{A}\,d\mu=\mu(A)$.
\end{ath}
\begin{setn}
  Látum $(X,\mathcal A,\mu)$ vera málrúm og $f,g:X\to[0,+\infty]$ vera
  mælanleg föll þ.a. $f\le g$. Þá er
  \[
  \int_{X}f\,d\mu \le \int_{X}g\,d\mu.
  \]
\end{setn}
\begin{proof}
  Augljóst af skilgreiningu.
\end{proof}
\begin{setn}[Beppo-Levi]\index{Beppo-Levi}
  Látum $(f_{k})_{k\in\N}$ vera \emph{vaxandi} runu af mælanlegum
  föllum $f_{k}:X\to[0,+\infty]$, þar sem $(X,\mathcal{A},\mu)$ er
  málrúm. Þá er
  \[
  \lim_{k\to+\infty}\int_{X} f_{k}\,d\mu
  = \int_{X}\lim_{k\to+\infty}f_{k}\,d\mu.
  \]
\end{setn}
\begin{proof}
  Ljóst er að $f_{k}\le\lim_{k\to+\infty}f_{k}$ fyrir öll $k$, svo að
  \[
  \int_{X}f_{k}\,d\mu
  \le\int_{X}\lim_{k\to+\infty}f_{k}\,d\mu
  \]
  fyrir öll $k$ og því
  \[
  \lim_{k\to+\infty}\int_{X}f_{k}\,d\mu
  \le \int_{X}\lim_{k\to+\infty}f_{k}\,d\mu.
  \]
  Setjum $f:=\lim_{k\to+\infty}f_{k}$. Látum $u$ vera einfalt
  mælanlegt fall, $0\le u\le f$ og $\beta>1$ vera fasta
  rauntölu. Setjum $B_{k}:=\{x\in X : \beta f_{k}(x)\ge u(x)\}$. Þá er
  $(B_{k})_{k\in\N}$ vaxandi runa af mælanlegum mengjum og
  $\bigcup_{k\in\N}B_{k}=X$ og $\beta f_{k}\ge u\chi_{B_{k}}$, og
  $u\chi_{B_{k}}$ er einfalt, mælanlegt, og $(u\chi_{B_{k}})_{k\in\N}$
  er vaxandi, $\lim_{k\to+\infty}u\chi_{B_{k}}=u$; svo að
  skv. hjálparsetningu er
  \[
  \int_{X}u\,d\mu
  \le\lim_{k\to+\infty}\int_{X}u\chi_{B_{k}}\,d\mu
  \le\beta\lim_{k\to+\infty}\int_{X} f_{k}\,d\mu.
  \]
  Þetta gildir fyrir öll $\beta>1$, svo að
  \[
  \int_{X}u\,d\mu
  \le \lim_{k\to+\infty}\int_{X} f_{k}\,d\mu.
  \]
  En þá er
  \[
  \int_{X}f\,d\mu
  \le \lim_{k\to+\infty}\int_{X}f_k\,d\mu.
  \]
\end{proof}
Jafngild framsetning:
\begin{setn}
  Látum $(X,\mathcal A,\mu)$ vera málrúm og $(f_{k})_{k\in\N}$ vera
  runu af mælanlegum föllum $f_{k}:X\to[0,+\infty]$. Þá er
  \[
  \sum_{k=0}^{+\infty}\int_{X} f_{k}\,d\mu
  = \int_{X}\sum_{k=0}^{+\infty}f_{k}\,d\mu.
  \]
\end{setn}
%%
%%
\marginpar{18. feb.}
%%
%%
Setning Beppo-Levis er oftast kölluð \emph{setning um einhalla
  samleitni}\index{einhalla samleitni}. Til þess að sjá jafngildi
hennar og síðustu setningar þurfum við:
\begin{setn}
  Látum $f,g:X\to[0,+\infty]$ vera mælanleg föll og $c\in\R,c\ge
  0$. Þá er
  \[
  \int_{X}(f+g)\,d\mu
  = \int_{X} f\,d\mu + \int_{X} g\,d\mu
  \]
  og
  \[
  \int_{X}cf\,d\mu
  = c\int_{X}f\,d\mu.
  \]
\end{setn}
\begin{proof}
  Þetta er augljóst ef föllin $f,g$ eru einföld; í almenna tilvikinu
  veljum við vaxandi runur $(u_{k})$ og $(v_{k})$ af einföldum
  mælanlegum föllum $X\to[0,+\infty]$ þannig að
  $\lim_{k\to+\infty}u_{k}=f$ og $\lim_{k\to+\infty}v_{k}=g$. Þá er
  \begin{align*}
    \lim_{k\to+\infty}\int_{X} (f+g)\,d\mu
    &= \lim_{k\to+\infty}\int_{X}(u_{k}+v_{k})\,d\mu
    \\
    &= \lim_{k\to+\infty}\int_{X}u_{k}+\lim_{k\to+\infty}v_{k}\,d\mu
    \\
    &= \int_{X}f\,d\mu + \int_{X}g\,d\mu
  \end{align*}
  og eins er $\lim_{k\to+\infty}cu_{k}=cf$ og þá
  \begin{align*}
    \int_{X}cf\,d\mu
    &= \lim_{k\to+\infty}\int_{X}cu_{k}\,d\mu \\
    &= c\lim_{k\to+\infty}\int_{X}u_{k}\,d\mu \\
    &= c\int_{X}f\,d\mu.
  \end{align*}
\end{proof}
Þá fáum við: Ef $(f_{k})$ er runa af mælanlegum föllum
$X\to[0,+\infty]$ og $S_{k}=\sum_{j=0}^{k}f_{j}$, þá er $(S_{k})$
vaxandi runa af föllum og setning um einhalla samleitni gefur
\begin{align*}
  \int_{X}\sum_{k=0}^{+\infty}f_{k}\,d\mu
  &= \int_{X}\lim_{k\to+\infty}S_{k}\,d\mu
  \\
  &= \lim_{k\to+\infty}\int_{X}S_{k}\,d\mu
  \\
  &= \lim_{k\to+\infty}\int_{X}\sum_{j=0}^{k}f_{k}\,d\mu
  \\
  &= \lim_{k\to+\infty}\sum_{j=0}^{k}\int_{X}f_{k}\,d\mu
  \\
  &= \sum_{k=0}^{+\infty}\int_{X}f_{k}\,d\mu.
\end{align*}
\begin{daemi}
  Látum $\mu$ vera talningarmálið á $\N$; sérhvert hlutmengi í $\N$ er
  mælanlegt og $\mu(\{k\})=1$. Fyrir sérhvert fall$f:\N\to[0,+\infty]$
  er
  \[
  \int_{\N}f\,d\mu = \sum_{k=0}^{+\infty}f(k).
  \]
  Setningin segir þá: Ef $(a_{jk})_{(j,k)\in\N\times\N}$ er fjölskylda
  af stökum í $[0,+\infty]$, þá er
  \[
  \sum_{j=0}^{+\infty}\sum_{k=0}^{+\infty}a_{jk}
  = \sum_{k=0}^{+\infty}\sum_{j=0}^{+\infty}a_{jk}.
   \]
 \end{daemi}
 \begin{setn}
   [Fatou-hjálparsetningin]\index{Fatou}
   Látum $X$ vera málrúm, $(f_{k})$ vera runu af mælanlegum föllum
   $f_{k}:X\to[0,+\infty]$. Þá er
   \[
   \int_{X}\liminf_{k\to+\infty}f_{k}\,d\mu
   \le \liminf_{k\to+\infty}\int_{X}f_{k}\,d\mu.
   \]
 \end{setn}
 \begin{proof}
   Setjum $g_{k}:=\inf\{f_{j}:j\ge k\}$; þá er $(g_{k})$ vaxandi og
   $\liminf_{k\to+\infty}f_{k}=\lim_{k\to+\infty}g_{k}$; auk þess er
   $g_{k}\le f_{k}$ fyrir öll $k$. Skv. setningu um einhalla samleitni
   er
   \begin{align*}
     \int_{X}\liminf_{k\to+\infty}f_{k}\,d\mu
     &= \int_{X}\lim_{k\to+\infty}g_{k}\,d\mu
     \\
     &= \lim_{k\to+\infty}\int_{X}g_{k}\,d\mu
     \\
     &= \liminf_{k\to+\infty}\int_{X}g_{k}\,d\mu
     \\
     &\le\liminf_{k\to+\infty}\int_{X}f_{k}\,d\mu.
   \end{align*}
 \end{proof}
 \begin{setn}
   Látum $X$ vera málrúm og $f:X\to[0,+\infty]$ vera mælanlegt
   fall. Við höfum $\int_{X}f\,d\mu=0$ þ.þ.a.a. $f=0$ næstum
   allsstaðar.
 \end{setn}
 \begin{proof}
   Þetta er ljóst ef $f$ er einfalt,
   $f=\sum_{k=1}^{m}a_{k}\chi_{A_{k}}$ þar sem $a_{k}>0$. Ef
   $\sum_{k=1}^{m}a_{k}\mu(A_{k})=0$, þá er $a_{k}\mu(A_{k})=0$ og því
   $\mu(A_{k})=0$ fyrir öll $k$, og
   $\{x:f(x)\ne0\}\subset\bigcup_{k=1}^{m}A_{k}$ og
   $\mu(\bigcup_{k=1}^{m}A_{k})\le\sum_{k=1}^{m}\mu(A_{k})=0$. Ef $f$
   er almennt mælanlegt fall og $\int_{X}f\,d\mu=0$, þá er
   $\int_{X}u\,d\mu=0$ fyrir öll einföld föll $u$
   þ.a. $0\le{u}\le{f}$. Ef $(u_{k})$ er vaxandi runa af einföldum
   föllum þ.a. $0\le{u_{k}}\le f$ og $\lim_{k\to+\infty}u_{k}=f$, þá
   er $\{x:f(x)>0\}=\bigcup_{k\in\N}\{x:u_{k}(x)>0\}$ og þetta er þá
   teljanlegt sammengi af núllmengjum og því núllmengi. Öfugt, ef
   $A:=\{x:f(x)>0\}$ er núllmengi, þá er
   $\int_{X}f\,d\mu\le\int_{X}g\,d\mu$, þar sem
   \[
   g(x) :=
   \begin{cases}
     +\infty & x\in A,\\
     0       & x\in X\setminus A,
   \end{cases}
   \]
   og
   $\int_{X}g\,d\mu=\lim_{k\to+\infty}\int{k}\chi_{A}\,d\mu=\lim_{k\to+\infty}k\mu(A)=0$;
   svo að $\int_{X}f\,d\mu=0$.
 \end{proof}

\chapter{Heildanleg föll}

\begin{skilgr}
  Látum $(X,\mathcal A,\mu)$ vera málrúm.
  \begin{enumerate}[(1)]
  \item Raunfall $f:X\to\R$ eða útvíkkað raunfall $f:X\to\tilde\R$
    kallast \emph{heildanlegt}\index{heildanlegt!fall} ef það er
    mælanlegt og
    \[
    \int_{X}f^{+}\,d\mu<+\infty,
    \quad
    \int_{X}f^{-}\,d\mu<+\infty.
    \]
    Skilgreinum þá \emph{heildi}\index{heildi} fallsins $f$
    (m.t.t. $\mu$) sem rauntöluna
    \[
    \int_{X}f\,d\mu
    := \int_{X}f^{+}\,d\mu - \int_{X}f^{-}\,d\mu.
    \]
    Segjum líka að $f$ sé
    \emph{hálfheildanlegt}\index{halfheildanlegt@hálfheildanlegt} ef
    annað af heildunum $\int_{X}f^{+}\,d\mu$ eða $\int_{X}f^{-}\,d\mu$
    er endanlegt og skilgreinum þá heildið $\int_{X}f\,d\mu$ sem stak
    í $\tilde\R$ með sömu formúlu.

  \item Tvinnfall $f:X\to\C$ kallast
    \emph{heildanlegt}\index{heildanlegt!tvinnfall} (m.t.t. $\mu$) ef
    það er mælanlegt og raunföllin $\re f,\im f:X\to\R$ eru bæði
    heildanleg; við setjum þá 
    \[
    \int_{X}f\,d\mu
    := \int_{X}\re f\,d\mu + i \int_{X}\im f\,d\mu.
    \]

  \item Fall $f$ á $X$ kallast \emph{heildanlegt yfir mælanlega
      hlutmengið}\index{heildanlegt!yfir mzaelanlegt hlutmengi@yfir
      mælanlegt hlutmengi} $A$ í $X$ ef $f\chi_{A}$ er heildanlegt;
    setjum þá
    \[
    \int_{A}f\,d\mu = \int_{X}f\chi_{A}\,d\mu.
    \]
  \end{enumerate}
\end{skilgr}
\begin{ath}
  Látum $f:X\to\tilde R$ vera heildanlegt. Þá er
  $\int_{X}f^{+}\,d\mu<+\infty$, $\int_{X}f^{-}\,d\mu<+\infty$ og af
  því leiðir að $\{x\in X : f(x)\in\{-\infty,+\infty\}\}$ er núllmengi
  (sjá heimadæmi). En þá er
  \[
  \int_{X}f\,d\mu = \int_{X}\hat f\,d\mu
  \quad\text{þar sem}\quad
  \hat f(x) =
  \begin{cases}
    f(x), & x\in\R, \\
    0,    & x\in\{-\infty,+\infty\}.
  \end{cases}
  \]
  Með öðrum orðum má breyta heildanlegu útvíkkuðu raunfalli í raunfall
  með því að breyta gildunum aðeins á núllmengi og það breytir ekki
  heildinu. 
\end{ath}
\begin{lemma}
  Látum $f:X\to\tilde\R$ vera (mælanlegt) fall og gerum ráð fyrir að
  til séu mælanleg föll $g,h:X\to[0,+\infty]$ þannig að $f=g-h$ og
  $\int_{X}g\,d\mu<+\infty$, $\int_{X}h\,d\mu<+\infty$. Þá er $f$
  heildanlegt og
  \[
  \int_{X}f\,d\mu = \int_{X}g\,d\mu - \int_{X}h\,d\mu.
  \]
\end{lemma}
\begin{proof}
  Ef $f(x)\ge0$ þá er $f^+(x)=f(x)=g(x)-h(x)\le g(x)$; ef $f(x)<0$, þá
  er $f^{+}(x) = 0\le g(x)$; svo $f^{+}\le g$. Þar með fæst
  $0\le \int_{X}f^{+}\,d\mu<+\infty$, eins er
  $\int_{X}f^{-}\,d\mu<\infty$, svo að $f$ er heildanlegt. Nú er
  $g+f^{-}=h+f^{+}$, svo að skv. setningu er
  \[
  \int_{X}g\,d\mu + \int_{X}f^{-}\,d\mu
  = \int_{X}h\,d\mu + \int_{X}f^{+}\,d\mu
  \]
  svo að
  \[
  \int_{X}f\,d\mu
  = \int_{X}f^{+}\,d\mu - \int_{X}f^{-1}\,d\mu
  = \int_{X}g\,d\mu - \int_{X}h\,d\mu.
  \]
\end{proof}
%%
%%
\marginpar{21. feb.\\- 11. mars}
%%
%%
\begin{ath}
  Látum $(X,{\cal A},\mu)$ vera málrúm og $f$ vera heildanlegt fall á
  $X$, þát táknum við heildið stundum með
  \[
  \int_{X}f(x)\,d\mu(x);
  \]
  þetta er einkum þægilegt þegar $f$ er háð öðrum breitum; ef t.d. $T$
  er mengi, $f:T\times X\to\C$ er fall og $X\to\C,x\mapsto f(t,x)$ er
  heildanlegt fyrir öll $t$ úr $T$, þá má skilgreina $F:T\to\C$ með
  \[
  F(t) := \int_{X} f(t,x)\,d\mu(x).
  \]
  Höfum hér að $\int_{X}f(t,x)\,d\mu(x) := \int_{X}f_{t}\,d\mu$, þar
  sem $f_{t}(x) := f(t,x)$. Ef $(X,\mathcal A,\mu)$ er
  $(\R^{n},\mathcal M_{\lambda_{n}},\lambda_{n})$, þá köllum við
  heildanlegt fall
  \emph{Lebesgue-heildanlegt}\index{heildanlegt!Lebesgue-}\index{Lebesgue!heildanlegt}
  og $\int_{\R^{n}}f\,d\lambda_{n}$ kallast
  \emph{Lebesgue-heildi}\index{Lebesgue!heildi}\index{heildi!Lebesgue-}
  $f$.
\end{ath}
\begin{lemma}
  Látum $f=g-h$, þar sem $g,h:X\to[0,+\infty]$ eru heildanleg föll á
  málrúmi $X$ þ.a. $g-h$ sé vel skilgreint; þá er $f$ heildanlegt og
  \[
  \int_{X}f\,d\mu = \int_{X}g\,d\mu - \int_{X}h\,d\mu.
  \]
\end{lemma}
\begin{setn}
  Látum $(X,\mathcal A,\mu)$ vera málrúm, látum $f,g$ vera heildanleg
  föll á $X$ þ.a. $f+g$ sé vel skilgreint og $c$ vera fasta; þá eru
  $f+g$ og $cf$ heildanleg og
  \[
  \int_{X}(f+g)\,d\mu
  = \int_{X}f\,d\mu + \int_{X}g\,d\mu,
  \quad
  \int_{X}cf\,d\mu
  = c\int_{X}f\,d\mu.
  \]
\end{setn}
\begin{proof}
  Athugum fyrst tilvikið þegar $f,g:X\to\tilde\R$, þá eru
  $f^{+},f^{-1}$,$g^{+}$ og $g^{-}$ heildanleg og $\ge 0$; þá er
  $f+g=(f^{+}+g^{+})-(f^{-}+g^{-})$, og $f^{+}+g^{+}$,$f^{-}+g^{-}$
  eru heildanleg skv. fyrri setningu, sem gefur
  \[
  \int_{X}(f^{+}+g^{+})\,d\mu
  = \int_{X}f^{+}\,d\mu + \int_{X}g^{+}\,d\mu,
  \]
  \[
  \int_{X}(f^{-}+g^{-})\,d\mu
  = \int_{X}f^{-}\,d\mu + \int_{X}g^{-}\,d\mu.
  \]
  Seinasta hjálparsetning gefur þá að $f+g$ er heildanlegt og
  \begin{align*}
    \int_{X}(f+g)\,d\mu
    &= \left(
      \int_{X}f^{+}\,d\mu
      + \int_{X}g^{+}\,d\mu
    \right)
    -
    \left(
      \int_{X}f^{-}\,d\mu
      + \int_{X}g^{-}\,d\mu
    \right)
    \\
    &= \int_{X}f^{+}\,d\mu
    + \int_{X}g^{+}\,d\mu
    - \int_{X}f^{-}\,d\mu
    - \int_{X}g^{-}\,d\mu
    \\
    &= \int_{X}(f^{+}- f^{-})\,d\mu
    + \int_{X}g^{+} - g^{-}) \,d\mu
    \\
    &= \int_{X}f\,d\mu + \int_{X}g\,d\mu.
  \end{align*}
  Ef $c\ge 0$ þá er $cf=cf^{+}-cf^{-}$, $cf^{+}$ og $cf^{-}$ eru
  heildanleg og $\int_{X}cf^{+}\,d\mu=c\int_{X}f^{+}\,d\mu$ og
  $\int_{X}cf^{-}\,d\mu=c\int_{X}f^{-}\,d\mu$, skv. fyrri setningu,
  svo að $cf=cf^{+}-cf^{-}$ er heildanlegt og
  \begin{align*}
    \int_{X}cf\,d\mu
    &= c\int_{X}f^{+}\,d\mu - c\int_{X}f^{-}\,d\mu
    \\
    &= c\int_{X}(f^{+}-f^{-})\,d\mu
    \\
    &= c\int_{X}f\,d\mu.
  \end{align*}
  Ef $c<0$, þá eru föllin $-cf^{+}$ og $-cf^{-}$ heildanleg föll
  $X\to[0,+\infty]$ og niðurstaðan fæst svipað og að ofan. Ef $f$ er
  tvinnfall og $c\in\C$, þá skrifum við $f=g+ih$, og $c=a+ib$, þar sem
  $g,h$ eru raunföll og $c,g\in\R$; þá reiknum við smávegis og notum
  niðurstöður fyrir raunföll.
\end{proof}
\begin{setn}
  G.r.f. að $(X,\mathcal A,\mu)$ sé málrúm, $f$ sé mælanlegt fall á
  $X$ og $g$ sé heildanlegt fall $g:X\to[0,+\infty]$ þ.a. $|f|\le g$
  næstum allsstaðar. Þá er $f$ heildanlegt.
\end{setn}
\begin{proof}
  Fyrir $f:X\to[0,+\infty]$ fæst að $f^{+},f^{-}\le|f|\le g$ næstum
  allsstaðar, svo að
  \[
  \int_{X}f^{+}\,d\mu
  \le \int_{X}g\,d\mu
  < +\infty,
  \quad
  \int_{X}f^{-}\,d\mu
  \le\int_{X}g\,d\mu
  < +\infty;
  \]
  ef nefnilega $A$ er núllmengi þ.a. $|f|\le g$ á $A^{C}$, þá er
  $f^{+}-f^{+}\chi_{A}\le g$ næstum allsstaðar og
  $\int_{X}f^{+}\chi_{A}\,d\mu=0$, svo að
  \[
  \int_{X}f^{+}\,d\mu
  = \int_{X}(f^{+}-f^{+}\chi_{A})\,d\mu
  \le \int_{X} g\,d\mu;
  \]
  eins fyrir $f^{-}$.
\end{proof}
\begin{setn}
  Ef $f,g\to\tilde\R$ eru heildanleg föll og $f = g$ næstum
  allsstaðar, þá er
  \[
  \int_{X}f\,d\mu = \int_{X}g\,d\mu.
  \]
\end{setn}
\begin{proof}
  Sönnunin er einföld.
\end{proof}
\begin{setn}
  Ef $f,g\to\tilde\R$ eru heildanleg föll og $f\le g$, þá er
  \[
  \int_{X}f\,d\mu \le \int_{X}g\,d\mu.
  \]
\end{setn}
\begin{proof}
  Höfum $f^{+}\le g^{+}$ og $f^{-}\ge g^{-}$, svo að
  \[
  \int_{X}f\,d\mu
  = \int_{X}f^{+}\,d\mu -\int_{X}f^{-}\,d\mu
  \le \int_{X} g^{+}\, d\mu - \int_{X}g^{-}\,d\mu
  = \int_{X}g\,d\mu.
  \]
\end{proof}
\begin{setn}
  Látum $(X,\mathcal A,\mu)$ vera málrúm. Fall $f$ á $X$ er
  heildanlegt þ.þ.a.a. það sé mælanlegt og $|f|$ sé heildanlegt, og þá
  er
  \[
  \left| \int_{X} f\,d\mu \right|
  \le \int_{X} |f| \,d\mu.
  \]
\end{setn}
\begin{proof}
  Þetta er ljóst fyrir föll $f:X\to\tilde\R$, því að
  $|f|=f^{+}+f^{-}$. Fyrir tvinnföll $f:X\to\C$ er ljóst að
  \[
  |\re f|,|\im f|
  \le |f|
  \le |\re f| + |\im f|,
  \]
  svo að $f$ er heildanlegt þ.þ.a.a. $\re f$ og $\im f$ séu
  heildanleg, sem er svo jafngilt því að $|f|$ sé heildanlegt. Látum
  $\theta\in\R$ vera þ.a.
  \[
  \left| \int_{X} f\,d\mu \right|
  = e^{i\theta}\int_{X} f\,d\mu
  = \int_{X}e^{i\theta}f\,d\mu,
  \]
  þá er 
  \[
  \left| \int_{X} f\,d\mu \right|
  = \re\int_{X} e^{i\theta} f\,d\mu
  = \int_{X}\re(e^{i\theta}f)\,d\mu
  \le \int_{X}|f|\,d\mu
  \]
  vegna $|e^{i\theta}f|=|f|$.
\end{proof}
\begin{setn}
  [Lebesgue um yfirgnæfða samleitni]
  \index{Lebesgue!um yfirgnæfða samleitni}
  \index{setning!Lebesgue um yfirgnæfða samleitni}
  \index{yfirgnæfð samleitni}
  \index{samleitni!yfirgnæfð-}

  Látum $(X,\mathcal A,\mu)$ vera málrúm, $(f_{k})$ vera runu af
  mælanlegum föllum á $X$ og $g$ vera heildanlegt fall á $X$. Gerum
  ráð fyrir að $f=\lim_{k\to+\infty}f_{k}$ næstum allsstaðar og fyrir
  sérhvert $k$ sé $|f_{k}|\le g$ næstum allsstaðar. Þá eru $f_{k}$ og
  $f$ heildanleg og við höfum
  \[
  \lim_{k\to+\infty}\int_{X}f_{k}\,d\mu
  = \int_{X}f\,d\mu,
  \]
  \[
  \lim_{k\to+\infty}\int_{X}(f_{k}-f)\,d\mu
  = 0.
  \]
\end{setn}
\begin{proof}
  Ljóst er að $f_{k}$ eru heildanleg samkvæmt setningu og $f$ er
  mælanlegt, því það er markgildi af mælanlegum föllum; með því að
  breyta föllunum á núllmengi má g.r.f. að $f_{k},f$ taki endanleg
  gildi og $\lim_{k\to+\infty}f_{k}=f$ allsstaðar. Okkur nægir að
  athuga raunföll $f_{k}$, því að fyrir tvinnföll er $|\re
  f_{k}|\le|f_{k}|\le g$ og $|\im f|\le g$. Athugum að
  $|f_{k}-f|\le|f_{k}|+|f|\le g+|f|$, svo að
  $g_{k}:=g+|f|-|f_{k}-f|\ge 0$, svo að
  skv. \emph{Fatou}-hjálparsetningunni er
  \begin{align*}
    \int_{X}(g+|f|)\,d\mu
    &= \int_{X}\liminf_{k\to+\infty}g_{k}\,d\mu \\
    &\le \liminf_{k\to+\infty}\int_{X}g_{k}\,d\mu \\
    &= \liminf_{k\to+\infty}\left(
      \int_{X}(g+|f|)\,d\mu
      - \int_{X}|f_{k}-f|\,d\mu
    \right) \\
    &= \int_{X}(g+|f|)\,d\mu
    - \limsup_{k\to+\infty}\int_{X}|f_{k}-f|\,d\mu
  \end{align*}
  svo að
  \[
  0
  \le \limsup_{k\to+\infty}\int_{X}|f_{k}-f|\,d\mu
  \le 0
  \]
  og þar með er $\int_{X}|f_{k}-f|\,d\mu = 0$. En þá er líka
  \[
  \left|
    \lim_{k\to+\infty}f_{k}\,d\mu
    - \int_{X}f\,d\mu
  \right|
  \le \int_{X}|f_{k}-f|\,d\mu
  \xrightarrow[k\to+\infty]{}0,
  \]
  svo að
  \[
  \lim_{k\to+\infty}\int_{X}f_{k}\,d\mu
  = \int_{X}f\,d\mu.
  \]
\end{proof}
\begin{ath}
  Almennt hefur $\lim_{k\to+\infty}f_{k}=f$ ekki í för með sér að
  $\lim_{k\to+\infty}\int_{X}f_{k}\,d\mu=\int_{X}f\,d\mu$.
\end{ath}
\begin{daemi}
  Föllin
  \[
  f_{k} := (k+1)\chi_{\left]0,1/(k+1)\right[}
  \]
  eru heildanleg í $\lambda_{1}$ á $\R$ og
  $\lim_{k\to+\infty}f_{k}=0$ allsstaðar, en
  $\int_{\R}f_{k}\,d\lambda=1$ fyrir öll $k$, svo að
  \[
  \lim_{k\to+\infty}\int_{\R}f_{k}\,d\lambda
  \ne \int_{\R}\lim_{k\to+\infty}f_{k}\,d\lambda .
  \]
\end{daemi}
\begin{fylgi}
  Látum $X$ vera málrúm, $(f_{k})_{k\in\N}$ vera runu af heildanlegum
  föllum á $X$ þannig að
  \[
  \sum_{k=0}^{+\infty}\int_{X}|f_{k}|\,d\mu
  < +\infty .
  \]
  Þá er röðin $\sum_{k=0}^{+\infty}f_{k}$ alsamleitin næstum
  allsstaðar og fyrir $f:=\sum_{k=0}^{+\infty}f_{k}$ næstum alsstaðar
  er $f$ heildanlegt og
  \[
  \int_{X}f\,d\mu
  = \sum_{k=0}^{+\infty}\int_{X}f_{k}\,d\mu,
  \]
  \[
  \lim_{k\to+\infty}\int_{X}\left|
    f - \sum_{j=0}^{k}
  \right|\,d\mu
  = 0.
  \]
\end{fylgi}
\begin{proof}
  Fallarunan $(g_{k})$, þar sem $g_{k}:=\sum_{j=0}^{k}|f_{j}|$ er
  vaxandi, ef $g$ er markgildið, þá er $g$ mælanlegt og skv. B.-Levi
  er
  \[
  \int_{X}g\,d\mu
  = \sum_{k=0}^{+\infty}\int_{X}|f_{k}<\,d\mu
  < +\infty,
  \]
  svo að $g$ er heildanlegt og
  \[
  \left| \sum_{j=0}^{k}f_{j} \right|
  \le \sum_{j=0}^{k} |f_{j}|
  \le g,
  \]
  og niðurstaðan er nú bein afleiðing af setningu Lebesgues um
  yfirgnæfða samleitni. 
\end{proof}
\begin{skilgr}
  Látum $f$ vera heildanlegt fall á málrúmi $(X,\mathcal A,\mu)$. Við
  skrifum
  \[
  \|f\|_{1} := \int_{X}|f|\,d\mu.
  \]
  Táknum með $\mathcal L^{1}(X,\mathcal A,\mu,\R)$ mengi heildanlegra
  falla $X\to\R$ og með $\mathcal L^{1}(X,\mathcal A,\mu,\C)$ mengi
  heildanlegra falla $X\to\C$. Leyfum okkur að skrifa $\mathcal
  L^{1}(X,\R)$ eða álíka framsetningu.
\end{skilgr}
Eftirfarandi er þá augljóst:
\begin{setn}
  Mengið $\mathcal L^{1}(X,\mathcal A,\mu,\R)$ er línulegt rúm yfir
  $\R$ og $\mathcal L^{1}(X,\mathcal A,\mu,\C)$ er línulegt rúm yfir
  $\C$. Vörpunin $\mathcal L^{1}(X,\mathcal A,\mu,\R)\to\R,
  f\mapsto\|f\|_{1}$, hefur eftirfarandi eiginleika:
  \begin{enumerate}[(i)]
  \item $\|f\|_{1}\ge 0$; $\|f\|=0$ þ.þ.a.a $f=0$ næstum allsstaðar.
  \item $\|cf\|_{1}=|c|\cdot\|f\|_{1}$ fyrir öll $c\in\C$ og öll $f$
    úr $\mathcal L(X,\mathcal A,\mu,\C)$.
  \item $\|f+g\|_{1}\le\|f\|_{1}+\|g\|_{1}$.
  \end{enumerate}
\end{setn}
\begin{skilgr}
  \emph{Staðall}\index{stadall@staðall} á línulegu rúmi $V$ yfir
  $\mathbb K$ þar sem $\mathbb K\in\{\C,\R\}$ er vörpun
  $V\to\R,x\mapsto\|x\|$, þannig að
  \begin{enumerate}[(i)]
  \item $\|x\| \ge 0$; $\|x\|=0$ þ.þ.a.a. $x=0$.
  \item $\|cx\|=|c|\cdot\|x\|$ fyrir öll $c\in\mathbb K$ og öll
    $x\in{V}$.
  \item $\|x+y\|\le\|x\|+\|y\|$.
  \end{enumerate}
\end{skilgr}
Vörpunin $\mathcal L^{1}\to\R, f\mapsto \|f\|_{1}$ er \emph{ekki
  staðall}, því að $\|f\|_{1}=0$ leyfir okkur aðeins að álykta að
$f=0$ \emph{næstum allsstaðar}. Athugum jafngildisvensl $\sim$ á
$\mathcal L^{1}$ með því að skrifa
\[
f\sim g
\quad\text{þ.þ.a.a.}\quad
f = g \text{ næstum allsstaðar}.
\]
Táknum með $L^{1}$ mengi allra jafngildisflokkanna. Þá verður
$L^{1}$ að línulegu rúmi yfir $\mathbb K$ með því að setja
\begin{align*}
  [f] + [g] &:= [f+g] \\
  c[f] &:= [cf],
\end{align*}
þar sem $[f]$ er jafngildisflokkur $f$ og $c\in\mathbb K$. Með því að
setja
\[
\|\,[f]\,\| := \| f \|_{1}
\]
fæst þá staðall á $L^{1}$.
\begin{ath}
  Hefð er að gera lítinn greinamun á $f$ og $[f]$, við skrifum
  $f\in L^{1}$ ef $f\in\mathcal L^{1}$ o.s.frv. Tökum þó eftir
  því að gildi $f$ í punkti $x$ er venjulega ekki það sama fyrir öll
  föll í $[f]$.
\end{ath}
\begin{fylgi}
  Ef $(f_{k})$ er runa í $L^{1}$ og
  $\sum_{k=0}^{+\infty}\|f_{k}\|_{1}<+\infty$, þá er til $f$ í $L^{1}$
  þannig að
  \[
  \lim_{k\to+\infty}\left\|
    f-\sum_{j=0}^{k}f_{j}
  \right\|
  = 0,
  \]
  m.ö.o.: Alsamleitin röð í $L^{1}$ er samleitin í $L^{1}$.
\end{fylgi}
Hér notum við:
\begin{setn}
  Ef $V$ er staðalrúm með staðal $\|\cdot\|$, þá fæst firð $d$ á $V$
  með því að láta
  \[
  d(x,y) := \| x - y \|.
  \]
\end{setn}
\begin{proof}
  Augljóst.
\end{proof}
\begin{skilgr}
  Röð $\sum_{k=0}^{+\infty}x_{k}$ í stöðluðu rúmi kallast
  \emph{samleitin}\index{samleitin runa}\index{runa!samleitin} ef
  runan $(s_{k})$ þa rsem $s_{k}:=\sum_{j=0}^{k}x_{k}$, er samleitin,
  þ.e. til er $x\in V$ þ.a.
  \[
  \lim_{k\to+\infty}\| x - \sum_{j=0}^{k} x_{j} \| = 0.
  \]
  Hún kallast \emph{alsamleitin}\index{alsamleitin
    runa}\index{runa!alsamleitin} ef
  $\sum_{k=0}^{+\infty}\|x_{k}\|<+\infty$.
\end{skilgr}
\begin{setn}
  Staðlað rúm $V$ er fullkomið firðrúm þ.þ.a.a. alsamleitin röð í $V$
  sé samleitin.
\end{setn}
\begin{proof}
  Sleppt.
\end{proof}
\begin{skilgr}
  Fullkomið staðalrúm kallast
  \emph{Banach-rúm}.\index{Banach-rum@Banach-rúm}\index{fullkomid@fullkomið!firdrum@firðrúm}.
\end{skilgr}
Höfum sýnt að $L^{1}$ er Banach-rúm m.t.t. staðalsins $\|\cdot\|_{1}$
á $L^{1}$. Athugum fleiri afleiðingar af setningu Lebesgues:
\begin{setn}
  Látum $f:[a,b]\to\C$ vera deildanlegt fall þ.a. afleiðan
  $f':[a,b]\to\C$ sé takmarkað fall. Þá er $f'$ Lebesgue-heildanleg
  fall og
  \[
  \int_{a}^{b}f'(x)\,dx = f(b) - f(a).
  \]
\end{setn}
\begin{proof}
  Framlengjum fallið $f$ í deildanlegt fall á öllu $\R$ þannig að $f'$
  sé áfram takmarkað. Segjum t.d. að $|f'|\le M$, þar sem $M$ er föst
  rauntala. Setjum nú
  \[
  g_{k}(x) := \frac{f(x+\frac 1k) - f(x)}{\frac 1k}
  \]
  fyrir öll $x$. Þá er $\lim_{k\to+\infty}g_{k}=f'$, svo að $f'$ er
  mælanlegt, og þar eð það er takmarkað er það heildanlegt yfir
  $[a,b]$. Skv. meðangildissetningu er fyrir gefið $x$ og $k>0$ til
  $\xi$ milli $x$ og $\frac 1k x$ þ.a. $g_{k}(x)=f'(\xi)$. Þetta þýðir
  að $|g_{k}|\le M$ á $[a,b]$. Skv. setningu um yfirgnæfða samleitni
  er
  \[
  \lim_{k\to+\infty}\int_{[a,b]}g_{k}\,d\lambda
  = \int_{[a,b]}f'\,d\lambda.
  \]
  Nú er $f$ samfellt og hefur því stofnfall $F$, svo að
  \begin{align*}
    \int_{a}^{b}g_{k}(x)\,dx
    &= k \int_{a}^{b}\left(
      f(x+1/x) - f(x)
    \right)\,dx
    \\
    &= k(F(b+1/k) - F(b)) - k(F(a+1/k) - F(a))
    \\
    &\xrightarrow[k\to+\infty]{} F'(b)-F'(a)
    = f(b) - f(a),
  \end{align*}
  svo að
  \[
  \int_{[a,b]}f'\,d\lambda 
  = f(b) - f(a).
  \]
\end{proof}
\begin{setn}
  Látum $T$ vera firðrúm, $t_{0}\in T$, $f:T\times X\to\C$ vera fall,
  $(X,\mathcal A,\mu)$ vera málrúm, og g.r.f. að eftirfarandi þremur
  skilyrðum sé fullnægt:
  \begin{enumerate}[(i)]
  \item Fyrir öll $t$ úr $T$ er fallið $X\to\C, x\mapsto f(t,x)$
    heildanlegt.
  \item Fyrir næstum öll $x$ úr $X$ er fallið $T\to\C,t\mapsto f(t,x)$
    samfellt í $t_{0}$.
  \item Til er grennd $U$ um punktinn $t_{0}$ í $T$ og heildanlegt
    fall $g$ á $X$ þ.a. fyrir sérhvert $t$ úr $U$ gildi
    $|f(t,x)|\le{g(x)}$ fyrir næstum öll $x\in X$.
  \end{enumerate}
  Þá er fallið $F:T\to\C$,
  \[
  F(t) := \int_{X}f(t,x)\,d\mu(x)
  \]
  samfellt í $t_{0}$ og $\Phi:T\to L^{1},\phi(t)(x) := f(t,x)$ er
  samfellt í $t_{0}$.
\end{setn}
\begin{proof}
  Þetta er bein afleiðing af setningu Lebesgues eftir smá umritun.
\end{proof}
\begin{setn}
  Látum $I$ vera bil í $\R$, $t_{0}\in I$, $f:I\times X\to\C$ vera
  fall; $X$ málrúm. G.r.f. að eftirfarandi skilyrðum sé fullnægt:
  \begin{enumerate}[(a)]
  \item Fyrir sérhvert $t$ úr $I$ er fallið $x\to\C,x\mapsto f(t,x)$
    heildanlegt.
  \item Fyrir sérhvert $x$ úr $X$ er fallið $I\to\C,t\mapsto f(t,x)$
    deildanlegt í $t_{0}$.
  \item Til er grennd $U$ um $t_{0}$ í $\R$ og heildanlegt fall $g$ á
    $X$ þ.a. fyrir öll $t\in U\cap I$ gildi
    \[
    \left| \frac{f(t_{0},x) - f(t,x)}{t_{0} - t} \right|
    \le g(x)
    \]
    næstum allsstaðar.
  \end{enumerate}
  Þá er fallið $F: I\to\C$, þar sem
  \[
  F(t) := \int_{X} f(t,x)\,d\mu(x)
  \]
  deildanlegt í $t_{0}$; fallið
  $x\mapsto\frac{\partial{f}}{\partial{t}}(t_{0},x)$ er heildanlegt og
  \[
  F'(t_{0})
  = \int_{X}\frac{\partial{f}}{\partial{t}}(t_{0},x)\,d\mu(x).
  \]
\end{setn}
\begin{proof}
  Þetta er bein afleiðing af síðustu setningu.
\end{proof}
\begin{ath}
  Með því að nota meðalgildissetningu fæst að sama niðurstaða gildir
  ef í stað (b) og (c) látum við: Til er grennd $U$ um $t_{0}$
  þ.a. fallið $t\mapsto f(t,x)$ sé deildanlegt á $U\cap I$ fyrir
  sérhvert $x\in X$ og til er heildanlegt fall $g$ á $X$ þ.a.
  \[
  \left|
    \frac{\partial{f}}{\partial{t}}(t,x)
  \right|
  \le g(x),
  \quad \forall\,x\in X.
  \]
\end{ath}
\begin{setn}
  Látum $X$ vera málrúm, $U$ vera opið hlutmengi í $\C$ og
  $f:U\times{X}\to\C$ vera fall þ.a. eftirfarandi þremur skilyrðum sé
  fullnægt:
  \begin{enumerate}[(a)]
  \item Fyrir sérhvert $z$ úr $U$ er fallið $x\to\C,x\mapsto f(z,x)$
    heildanlegt.
  \item Fyrir sérhvert $x$ úr $X$ er fallið $U\to\C,z\mapsto f(z,x)$
    fágað.
  \item Fyrir sérhverja lokaða hringskífu $K$ í $U$ er til heildanlegt
    fall $g_{K}:X\to\R$ þ.a.
    \[
    |f(z,x)| \le g(x)
    \]
    fyrir næstum öll $x$.
  \end{enumerate}
  Þá er fallið $F:U\to\C$ þ.a.
  \[
  F(z) := \int_{X} f(z,x)\,d\mu(x)
  \]
  fágað; fyrir sérhvert $k\in\N$ er fallið
  $X\to\C,x\mapsto\frac{\partial^{k}{f}}{\partial{z}^{k}}(z,x)$
  heildanlegt og
  \[
  F^{(k)}(z)
  = \int_{X}\frac{\partial^{k}{f}}{\partial{z}^{k}}(z,x)\,d\mu(x)
  \]
  fyrir öll $z\in U$.
\end{setn}
\begin{proof}
  Látum $K:=\{ z\in\C : |z-a| \le 2r \} \subset U$; þá er
  \[
  f(z,x)
  = \frac{1}{2\pi{i}}\int_{\alpha_{2r}(a)}\frac{f(\zeta,x)}{\zeta-z}\,d\zeta,
  \quad
  |z-a| < 2r
  \]
  skv. Cauchy-formúlu, þar sem $\alpha_{2r}(a)$ er vegurinn
  $[0,2\pi]\to U,t\mapsto a+2re^{it}$. Þá er fyrir fast $z\in C$ með
  $|z-a|<2r$:
  \[
  = \int_{X}\underbrace{
    \frac{1}{2\pi i}
    \int_{\alpha_{2r}(a)}
    \frac{f(\zeta,x)}{(\zeta-z)(\zeta-w)}\,d\zeta \,d\mu(x)
  }_{=:\varphi_{w}(\zeta,w)}.
  \]
  Látum $(w_{k})$ vera runu í $\{z\in\C : |z-a|<r\}$  þ.a. $w_{k}\ne
  z$, en $\lim_{k\to+\infty}w_{k}=z$. Þá er fallið
  $x\to\varphi_{w_{k}}(z,x)$ mælanlegt og
  $|\varphi_{w_{k}}(z,x)|\le\frac{2}{r}g_{k}(x)$ næstum allsstaðar;
  niðurstaðan fæst nú út frá fyrri setningu.
\end{proof}
\begin{setn}
  Látum $f:\R^{n}\to\C$ vera takmarkað fall sem er núll fyrir utan
  þjappað mengi $K$. Fallið $f$ er \emph{Riemann}-heildanlegt yfir $K$
  þ.þ.a.a. það sé samfellt næstum allsstaðar; ef svo er, þá er það
  líka heildanlegt m.t.t. \emph{Lebesgue}-málsins $\lambda_{n}$ á
  $\R^{n}$, og Lebesgue-heildið er jafnt Riemann-heildinu.
\end{setn}
\begin{proof}
  Sjá stærðfræðigreiningu IIA; viðbótin að Lebesgue- og
  Riemann-heildin séu þau sömu er augljós.
\end{proof}
\begin{ath}
  Fyrir föll $f:\R\to\C$ þ.a. $f$ sé Riemann-heildanlengt yfir
  sérhvert takmarkað bil í $\R$ má \emph{stundum} skilgreina
  \emph{óeiginlegt}\index{heildi!oeiginlegt@óeiginlegt} Riemann-heildi
  \[
  \int_{-\infty}^{+\infty}f(x)\,dx
  := \lim_{r\to+\infty}\int_{0}^{r}f(x)\,dx
  + \lim_{r\to+\infty}\int_{-r}^{0}f(x)\,dx,
  \]
  nefnilega, ef þessi markgildi eru bæði til. Óeiginlega
  Riemann-heildið getur verið til án þes að fallið sé
  Lebesgue-heildanlegt: Ef $f$ er Lebesgue-heildanlegt, þá er $|f|$
  það líka, en óeiginlega heildið $\int_{-\infty}^{+\infty}f(x)\,dx$
  getur verið til án þess að $\int_{-\infty}^{+\infty}|f(x)|\,dx$ sé
  það. En ef $f$ og $|f|$ hafa óeiginleg Riemann-heildi, þá er $f$
  Lebesgue-heildanlegt, og Lebesgue-heildið er það sama og
  Riemann-heildið. 
\end{ath}
\begin{daemi}
  Látum $p$ vera rauntölu, $p>-1$, og athugum
  $f:[0,1]\to\R,f(x)=x^{p}$. Setjum $f_{n}:=f\cdot\chi_{[1/n,1]}$
  fyrir $n\ge 2$. Þá er $(f_{n})$ vaxandi runa sem stefnir á $f$;
  $f_{n}$ er heildanlegt og
  \[
  \int_{0}^{1}f_{n}\,dx
  = \int_{1/n}^{1}x^{p}\,dx
  = \frac{1}{p+1}(1-(1/n)^{p+1})
  \]
  og þar eð $p>-1$ stefnir þetta á $1/(p+1)$ þegar $n\to+\infty$, svo
  að skv. Levi er $f$ heildanlegt og heildið er $1/(p+1)$. Setjum
  \[
  F(p)
  := \int_{0}^{1}x^{p}\,dx
  = \frac 1{p+1}.
  \]
  Sýnum að deilda má $F$ með því að deilda undir heildinu m.t.t. $p$:
  Athugum að
  \[
  \frac{\partial^{n}}{\partial p^{n}}(x^{p})
  = \frac{\partial^{n}}{\partial p^{n}}(e^{p\log x})
  = x^{p}\log^{n}x.
  \]
  Látum $a>b>-1$ vera föst og $n\in\N$. Fyrir öll $p\ge a$ og öll
  $x\in\left]0,1\right]$ er
  \[
  x^{p-b}|\log x|^{n}\le x^{a-b}|\log x|^{n}
  \]
  og $g(x) := x^{a-b}|\log x|^{n}$ er samfellt á  $\left]0,1\right]$
  og vegna $a>b$ er $\lim_{x\to 0}g(x)=0$; þar með er $g$
  Riemann-heildanlengt á $[0,1]$ ef við setjum $g(0):=0$ og því
  Lebesgue-heildanlegt; það er takmarkað af fasta $C=C(a,b,n)$; höfum
  \[
  |x^{p}\log^{n}(x)| \le Cx^{b}, \quad \forall\, p\ge a.
  \]
  Fallið $x\mapsto Cx^{p}$ er heildanlegt á $\left]0,1\right]$. Notum
  setninguna um deildun undir heildi $n$ sinnum og fáum
  \[
  F^{(n)}(p) = \int_{0}^{1}x^{p}\log^{n}x\,dx
  \]
  fyrir öll $p > a$; en $a$ má vera hvaða tala sem vera skal
  þ.a. $a>-1$; svo þetta gildir fyrir öll $p>-1$. Nú er líka auðvelt
  að deilda $F(p)=\frac{1}{1+p}$ beint. Fáum þá
  \[
  \int_{0}^{1}x^{p}\log^{n}x\,dx
  = (-1)^{n}\frac{n!}{(p+1)^{a+1}}
  \]
  fyrir öll $p>-1$ og öll $n$ úr $\N$. Fyrir öll
  $x\in\left]0,1\right]$er
  \[
  \frac{\log x}{1-x}
  = \sum_{k=0}^{+\infty}x^{k}\log x.
  \]
  Hins vegar er óeiginlega Riemann-heildið
  $\int_{0}^{+\infty}\frac{\sin x}x\,dx$ samleitið, þvi'að með
  hlutheildun fæst
  \[
  \int_{1}^{r}\frac{\sin x}x\,dx
  = \left[ -\frac{\cos x}x \right]_{1}^{r}
  - \int_{1}^{r}\frac{\cos x}{x^{2}}\,dx
  \]
  og $|(\cos x)/x|\le 1/x^{2}$, og $1/x^{2}$ er heildanlegt yfir
  $\left[1,+\infty\right[$, svo að
  $\lim_{r\to+\infty}\int_{1}^{r}\frac{\sin x}x\,dx$ er til og þá líka
  \[
  \int_{0}^{+\infty}\frac{\sin x}x\,dx
  := \lim_{r\to+\infty} \int_{0}^{r}\frac{\sin x}x\,dx.
  \]
  Ekki er til einfalt stofnfall fyrir $(\sin x)/x$, en tökum eftir að
  fyrir fast $t>0$ er
  \[
  f(x,t):=e^{-tx}\frac{\sin x}x
  \]
  heildanlegt fall af $x$ yfir $\left[0,+\infty\right[$, því að fyrir
  $x\ge 1$ er $|f(x,t)|\le e^{-tx}$ og $x\mapsto e^{-tx}$ er
  heildanlegt yfir $\left[0,+\infty\right[$. Setjum
  \[
  F(t) := \int_{0}^{+\infty}e^{-tx}\frac{\sin x}x\,dx.
  \]
  Nú höfum við 
  \[
  \frac{\partial f}{\partial t}(x,t) = -e^{-tx}\sin x
  \]
  og fyrir fast $a>0$ og öll $t\in\left[a,+\infty\right[$ að
  \[
  \left|
    \frac{\partial f}{\partial t}(x,t)
  \right|
  \le e^{-ax}.
  \]
  Setning segir að við megum deilda undir heildinu:
  \[
  F'(t) = -\int_{0}^{+\infty}e^{-tx}\sin x\,dx,
  \]
  þetta heildi er auðvelt að reikna, því að $x\mapsto e^{-tx}\sin x$
  hefur stofnfall $x\mapsto -e^{-tx}(t\sin x + \cos x)/(1+t^{2})$, svo
  að
  \[
  F'(t) = -\frac 1{1+t^{2}}
  \]
  og þá $F(t) = C - \arctan x$, þar sem $C$ er fasti. Til að finna $C$
  tökum við eftir að $f_{n}(x):=e^{-nx}\frac{\sin x}x$ gildir
  \[
  \lim_{n\to+\infty}f_{n} = 0,
  \quad |f_{n}(x)|\le e^{-x},
  \quad \text{ef }n\ge 1,
  \]
  svo að skv. setningu um yfirgnæfða samleitni fæst
  \[
  \lim_{n\to+\infty} F(n)
  = \lim_{n\to+\infty}\int_{0}^{+\infty}f_{n}(x)\,dx
  = 0
  \]
  svo að $0=\lim_{n\to -\infty}(C-\arctan n)=C-\pi/2$, svo að
  \[
  F(t) = \frac \pi2 - \arctan x.
  \]
  Viljum nú sýna að $\int_{0}^{+\infty}\frac{\sin x}x\,dx=\lim_{t\to
    0}F(t)$, en \emph{getum ekki} notað setningu um yfirgnæfða
  samleitni, því að $(\sin x)/x$ er ekki heildanlegt. En með
  hlutheildun fæst
  \[
  \int e^{-tx}\frac{\sin x}x\,dx
  = -\frac{e^{-tx}}x \cdot \frac{t\sin x + \cos x}{1+t^{2}}
  - \int \frac{e^{-tx}}{x^{2}}\cdot \frac{t\sin x + \cos x}{1+t^{2}}\,dx
  \]
  þar sem
  \[
  \left|
    e^{-tx}\cdot
    \frac{t\sin x + \cos x}{1+t^{2}}
  \right|
  \le \frac{t+1}{t^{2}+1}
  < 2
  \]
  svo að
  \[
  \left|
    \int_{r}^{+\infty}e^{-tx}
    \frac{\sin x}x\,dx
  \right|
  \le \frac 4r
  \]
  fyrir öll $t\ge 0$. Látum nú $\varepsilon > 0$ vera gefið. Finnum
  $r>0$ þannig að $4/r < \varepsilon/4$. Nú er $(\sin x)/x$
  heildanlegt yfir $[0,r]$ og þar er $f(x,t)$ yfirgnæft af
  $|(\sin{x})/x|$, svo að fyrir fast $r$ er
  \[
  \lim_{t\to 0}
  \int_{0}^{r}e^{-tx}\frac{\sin x}x\,dx
  = \int_{0}^{r}\frac{\sin x}x\,dx.
  \]
  Við getum þá fundið $t>0$ þannig að
  \[
  \left|
    \int_{0}^{r}\frac{\sin x}x\,dx
    - \int_{0}^{r}e^{-tx}\frac{\sin x}x\,dx
  \right|
  < \frac\varepsilon 4,
  \]
  og að auki þ.a.
  \[
  \left|
    \frac \pi 2 - F(t)
  \right|
  < \frac\varepsilon 4.
  \]
  Þar eð $F(t)=\int_{0}^{r}e^{-tx}(\sin x)/x\,dx +
  \int_{r}^{+\infty}e^{-tx}(\sin x)/x\,dx$ fæst
  \begin{align*}
    &\left|
      \int_{0}^{+\infty}\frac{\sin x}x\,dx - \frac\pi 2
    \right|
    \\
    &\le \left|
      \int_{0}^{+\infty}\frac{\sin x}x\,dx
      - \int_{0}^{+\infty}e^{-tx}\frac{\sin x}x\,dx
    \right|
    + \left|F(t)-\frac\pi 2\right|
    \\
    &\le \left|
      \int_{0}^{r} \frac{\sin x}x\,dx 
      - \int_{0}^{r} e^{-tx}\frac{\sin x}x\,dx
    \right|
    \\
    &\quad
    + \left|
      \int_{r}^{+\infty}
      \frac{\sin x}x\,dx 
    \right|
     + \left|
       \int_{r}^{+\infty}
       e^{-tx}\frac{\sin x}x\,dx
    \right|
    + \left| F(t) - \frac\pi2 \right|
    \\
    &\le
    \frac\varepsilon 4
    + \frac\varepsilon 4
    + \frac\varepsilon 4
    + \frac\varepsilon 4
    = \varepsilon.
  \end{align*}
  Þar eð þetta gildir fyrir öll $\varepsilon > 0$ fæst
  \[
  \int_{0}^{+\infty}\frac{\sin x}x\,dx
  = \frac \pi 2.
  \]
\end{daemi}
\emph{Rifjum upp:} Ef $p,q>1$ og $1/p + 1/q = 1$, þá er
\begin{equation}
  \label{eq:young}
  ab \le \frac {a^{p}}p + \frac{b^{q}}q
\end{equation}
fyrir öll $a,b\ge 0$.
\begin{skilgr}
  Látum $f$ vera mælanlegt fall á málrúmi $X$ þ.a. $|f|^{p}$ sé
  heildanlegt, $p>1$ fast. Setjum
  \[
  \|f\|_{p} := \left(\int_{X}|f|^{p}\,d\mu\right)^{1/p}.
  \]
\end{skilgr}
Fáum:
\begin{setn}
  [Hölder]\index{Hoelder@Hölder-ójafna}
  Látum $p,q\in\R>1$, þ.a. $1/p + 1/q = 1$. Fyrir öll mælanleg föll
  $f,g:X\to\C$ á málrúmi $X$ er
  \[
  \int_{X}|fg|\,d\mu
  \le \left(\int_{X}|f|^{p}\right)^{1/p}
  \left(\int_{X}|g|^{q}\right)^{1/q},
  \]
  m.ö.o.
  \[
  \| fg\|_{1 }
  \le \|f\|_{p}\cdot \|g\|_{q}.
  \]
  (Hér er $(+\infty)^{1/p}=(+\infty)^{1/q}:=+\infty$).
\end{setn}
\begin{proof}
  Þetta er augljóst ef $\|f\|_{p}=0$ eða $\|g\|_{q}=0$, því þá er
  $fg=0$ næstum allsstaðar; eins ef $\|f\|_{p}=+\infty$ eða
  $\|g\|_{q}=+\infty$. Getum gert ráð fyrir að
  \[
  a := \frac{|f(x)|}{\|f\|_{p}},
  \quad
  b := \frac{|g(x)|}{\|g\|_{q}}
  \]
  séu vel skilgreindar tölur í $\left[0,+\infty\right[$ fyrir öll
  $x$. Ójafnan \eqref{eq:young} gefur
  \[
  \frac{|f(x)g(x)|}{\|f\|_{p}\cdot\|g\|_{q}}
  \le \frac{|f(x)|^{p}}{p\|f\|_{p}}
  + \frac{|g(x)|^{q}}{q\|g\|_{q}}.
  \]
  Heildum yfir $X$ og fáum
  \[
  \frac{1}{\|f\|_{p}\cdot\|g\|_{q}}
  \int_{X}|fg|\,d\mu
  \le \frac 1p + \frac 1q = 1.
  \]
  Margföldum nú báðar hliðar með $\|f\|_{p}\|g\|_{q}$.
\end{proof}
Fyrir $p=q=2$ fæst
\emph{Cauchy-Schwarz-Bunyakovsky}-ójafnan:\index{Cauchy-Schwarz-Bunyakovski}
Fyrir mælanleg föll $f,g:X\to\C$ er
\[
\int_{X}|fg|\,d\mu
\le \sqrt{\int_{X}|f|^{2}\,d\mu}\cdot \sqrt{\int_{X}|g|^{2}\,d\mu}.
\]
\begin{setn}
  [Minkowski-ójafnan]\index{Minkowski@Minkowski-ójafnan}
  Látum $p\in\R$, $p\ge 1$. Fyrir öll mælanleg föll $f,g:X\to\C$ á
  málrúmi $X$ gildir
  \[
  \| f + g \|_{p}
  \le \|f\|_{p} + \|g\|_{p}.
  \]
\end{setn}
\begin{proof}
  Augljóst ef $p=1$ eða ef vinstri hliðin er $0$. Að öðrum kosti má
  skrifa
  \[
  |f+g|^{p} \le |f+g|^{p-1}(|f|+|g|)
  \]
  svo að
  \begin{align*}
    \int_{X}|f+g|^{p}\,d\mu
    &\le \int_{X}|f| |f+g|^{p-1}\,d\mu
    + \int_{X}|g| |f+g|^{p-1}\,d\mu
    \\
    &\le \|f\|_{p} \|(f+g)^{p-1}\|_{q}
    + \|g\|_{p} \|(f+g)^{p-1}\|_{q}
    \\
    &\le(\|f\|_{p}+\|g\|_{p})
    \left(
      \int_{X}|f+g|^{p}
    \right)^{1/q}
  \end{align*}
  þar sem $q$ er þannig að $1/p+1/q=1$, þ.e. $(p-1)q=p$, svo að
  \[
  \|f+g\|_{p}
  =
  \left(
    \int_{X}|f+g|^{p}\,d\mu
  \right)^{1/p}
  \le
  \left(
    \int_{X}|f+g|^{p}\,d\mu
  \right)^{1-1/q}
  \le \|f\|_{p}+\|g\|_{p}.
  \]
\end{proof}
\begin{skilgr}
  Látum $p\in\R$, $p>1$ og
  \[
  \mathcal L^{p}
  = \mathcal L^{p}(X)
  = \mathcal{L}^{p}(\C, \mathcal A, \mu)
  \]
  vera mengi allra falla $X\to\mathcal C$ þannig að fallið $|f|^{p}$
  sé heildanlegt, þ.e.
  \[
  \int_{X}|f|^{p}\,d\mu < +\infty
  \]
  og táknum með
  \[
  L^{p} = L^{p}(X) = L^{p}(X,\mathcal A, \mu)
  \]
  mengi allra jafngildisflokka falla úr $\mathcal L^{p}$
  m.t.t. jafngildisvenslanna
  \[
  f\sim g
  \quad\text{þ.þ.a.a.}\quad
  f = g\text{ næstum allsstaðar}.
  \]
\end{skilgr}
\begin{ath}
  Skrifum oft $f\in L^{p}$ þegar við meinum $f\in\mathcal L^{p}$;
  þ.e. við notum gjarnan sama bókstaf fyrir fallið og
  jafngildisflokkinn. 
\end{ath}
Fáum nú:
\begin{setn}
  Mengið $L^{p}$ er línulegt rúm yfir $\C$ og vörpunin $L^{p}\to\R,
  f\mapsto\|f\|_{p}$ er staðall á $L^{p}$.
\end{setn}
\begin{ath}
  Staðall á línulegu rúmi er sértilvik af því sem kallast stundum
  \emph{núllfirð}: Látum $G$ vera víxlgrúpu með aðgerð sem við skrifum
  sem samlagningu. \emph{Núllfirð}\index{nullfird@núllfirð} á $G$ er
  vörpun $G\to\R$ þannig að gildi:
  \begin{enumerate}[(1)]
  \item $\|x\|\ge 0$ fyrir öll $x\in G$; og $\|x\|=0$ þ.þ.a.a. $x=0$.
  \item $\|-x\|=\|x\|$ fyrir öll $x$ úr $G$.
  \item $\|x+y\|\le\|x\|+\|y\|$ fyrir öll $x,y\in G$.
  \end{enumerate}
  Núllfirð gefur af sér firð $d$ á $G$ með $d(x,y)=\|x-y\|$; hún er
  \emph{hliðrunartrygg}\index{hlidrunartrygg@hliðrunartrygg}, sem
  þýðir að $d(x+z,y+z)=d(x,y)$ fyrir öll $x,y,z$ í $G$. Sérhver
  hliðrunartrygg firð $d$ á $G$ kemur frá núllfirð, nefnilega
  $\|x\|=d(x,0)$. Segjum að röð $\sum_{k=0}^{+\infty}x_{k}$ af stökum
  í $G$ sé \emph{samleitin} ef runan $(s_{k})$ af hlutsummunum
  $s_{k}:=\sum_{j=0}^{k}a_{j}$ er samleitin m.t.t. firðarinnar; en
  \emph{alsamleitin} ef $\sum_{k=0}^{+\infty}\|x_{k}\|<+\infty$. 
\end{ath}
Höfum þá:
\begin{setn}
  Víxlgrúpa $G$ er fullkomið firðrúm m.t.t. hliðrunartryggrar firðar
  þ.þ.a.a. sérhver alsamleitin röð í $G$ sé samleitin.
\end{setn}
\begin{proof}
  G.r.f. að $G$ sé fullkomið firðrúm og að röð
  $\sum_{k=0}^{+\infty}a_{k}$ sé alsamleitin; fyrir sérhvert
  $\varepsilon > 0$ er til $k_{0}$
  þ.a. $\sum_{k=k_{0}}^{+\infty}\|a_{k}\|<\varepsilon$. Látum $k\ge
  j\ge k_{0}$; þá er
  \[
  \|s_{k}-s_{j}\|
  = \left\|
    \sum_{i=j+1}^{k}a_{i}
  \right\|
  \le \sum_{i=j+1}^{k}\|a_{i}\|
  \le \sum_{i=k_{0}}^{+\infty}\|a_{i}\|
  < \varepsilon,
  \]
  svo að runan $(s_{k})$ af hlutrununum er \emph{Cauchy}-ruan og því
  samleitin. Gerum nú ráð fyrir að sérhver alsamleitin röð í $G$ sé
  samleitni og látum $(u_{k})_{k\in\N}$ vera \emph{Cauchy}-runu í
  $G$. Fyrir sérhvert $i\in\N$ er þá til tala $l_{i}\in\N$
  þ.a. $\|u_{k}-u_{j}\|<\frac 1{2^{i+1}}$ fyrir öll $k,j\ge
  l_{i}$. Búum nú til stranglega vaxandi runu $(k_{i})_{i\in\N}$ með
  þrepun þ.a. $k_{0}=l_{0}$ og
  $k_{i+1}=\max\{k_{i},l_{i},l_{i+1}\}$. Þá er $k_{i},k_{i+1}\ge
  l_{i}$ fyrir öll $i$ og því $\|u_{k_{i+1}} -
  u_{k_{i}}\|\le\frac{1}{2^{i+1}}$. Það þýðir að röðin
  $\sum_{i=0}^{+\infty}(u_{k_{i+1}}-u_{k_{i}})$ er alsamleitin og því
  samleitin. En þetta er kíkisröð; höfum því
  \[
  \sum_{i=0}^{j}(u_{k_{i+1}}-u_{k_{i}}) = u_{k_{j+1}} -u_{k_{0}}
  \]
  svo að runan $(u_{k_{i}})$ er samleitin hlutruna í rununni
  $(u_{k})$. En að Cauchy-runa sem hefur samleitna hlutrunu er sjálf
  samleitin: Látum $\varepsilon > 0$ vera gefið; látum
  $u=\lim_{i\to+\infty}u_{k_{i}}$, finnum $i_{0}$ þ.a. $\|u -
  u_{k_{i}}\|\le\varepsilon / 2$ fyrir $i\ge i_{0}$ og
  $\|u_{k}-u_{j}\| < \varepsilon / 3$ e f$k,j\ge i_{0}$. Fyrir $k\ge
  i_{0}$ veljum við $i > i_{0}$ þ.a. $k_{i} > i_{0}$; þá er
  \[
  \| u - u_{k} \|
  \le \| u - u_{k_{i}} \| + \| u_{k_{i}} - u_{k} \|
  < \frac\varepsilon 3 + \frac\varepsilon 3
  < \varepsilon.
  \]
  Þar með er $\lim_{k\to+\infty}u_{k} = u$.
\end{proof}
\begin{setn}
  Rúmið $L^{p}$ er fullkomið m.t.t. firðarinnar sem staðallinn
  $L^{p}\to\R, f\mapsto \|f\|_{p}$, gefur af sér.
\end{setn}
\begin{proof}
  Samkvæmt ofansögðu nægir að sýna að alsamleitin röð í $L^{p}$ sé
  samleitin. Látum $(f_{k})$ vera runu af tvinnföllum á $X$
  þ.a. $\sum_{k=0}^{+\infty}\|f_{k}\|_{p} < +\infty$. Setjum
  $g_{k}:=\sum_{j=0}^{k}|f_{k}|$ og
  $g:=\lim_{k\to+\infty}g_{k}:X\to[0,+\infty]$. Við höfum
  $\|g_{k}\|_{p}\le\sum_{j=0}^{k}\|f_{j}\|_{p}\le B$, svo að skv. Levi
  er
  \[
  \int_{X}g^{p}\,d\mu
  = \lim_{k\to+\infty}\int_{X}g_{k}^{p}\,d\mu
  = \lim_{k\to+\infty}\| g_{k} \|_{p}^{p}
  \le B^{p}
  < +\infty
  \]
  svo að $g\in L^{p}$. Þá er $g<+\infty$ næstum allsstaðar; svo að
  $\sum_{k=0}^{+\infty}f_{k}$ er alsamleitin næstum allsstaðar. Setjum
  \[
  f :=
  \begin{cases}
    \sum_{k=0}^{+\infty} f_{k}, & g(x) < +\infty, \\
    0, &  g(x) = +\infty.
  \end{cases}
  \]
  Þá er $f:X\to\C$ vel skilgreint fall og $|f|<g$, svo að $f\in L^{p}$
  og fyrir öll $k\in\N$ er
  \[
  \left|
    f - \sum_{j=0}^{k} f_{j}
  \right|^{p}
  \le (g + g)^{p}
  = (2g)^{p}
  \in L^{1}
  \]
  svo að skv. setningu Lebesgues um yfirgnæfða samleitni er
  \[
  \lim_{k\to+\infty} \left\|
    f - \sum_{j=0}^{k} f_{j}
  \right\|^{p}
  = \lim_{k\to+\infty} \int_{X}
  \left|
    f - \sum_{j=0}^{k} f_{j}
  \right|^{p}\,d\mu
  = 0
  \]
  og því
  \[
  \lim_{k\to+\infty}\left\|
    f - \sum_{j=0}^{k} f_{j}
  \right\|^{p}
  = 0,
  \]
  sem þýðir að röðin $\sum_{k=0}^{+\infty}f_{k}$ er samleitin í
  $L^{p}$ með markgildi $f$.
\end{proof}
\begin{ath}
  Skoðum nú sérstaklega tilfellið $p=2$. Skv. Hölder-ójöfnu, sem
  verður að Cauchy-Schwarz ójöfnu í þessu tilfelli, er
  \[
  \int_{X}|fg|\,d\mu
  \le \|f\|_{2}\cdot\|g\|_{2}
  < +\infty
  \]
  fyrir $f,g\in L^{2}$, það þýðir að fallið $f\bar g$ er heildanlegt,
  og viðgetum skilgreint
  \[
  \langle f,g\rangle
  := \int_{X}f\bar g\,d\mu.
  \]
\end{ath}
\begin{setn}
  Þetta skilgreinir \emph{hermískt innfeldi}\index{hermiskt
    innfeldi@hermískt innfeldi}\index{innfeldi!hermiskt@hermískt} á
  rúmið $L^{2}$, það þýðir:
  \begin{enumerate}[(1)]
  \item Vörpunin $L^{2}\times L^{2}\to\C,(f,g)\mapsto\langle
    f,g\rangle$ er \emph{hálflínuleg}, þ.e. línuleg í fyrri
    breytistærðinni en andlínuleg í seinni, þ.e.
    \begin{align*}
      \langle f_{1} + f_{2},g\rangle
      &= \langle f_{1}, g\rangle + \langle f_{2},g\rangle
      \\
      \langle f, g_{1}+g_{2}\rangle
      &= \langle f,g_{1}\rangle + \langle f,g_{2}\rangle
      \\
      \langle cf,g\rangle
      &= c\langle f,g\rangle
      \\
      \langle f,cg\rangle
      &= \bar c\langle f,g\rangle
    \end{align*}
    fyrir öll $f,f_{1},f_{2},g,g_{1},g_{2}$ í $L^{2}$ og $c\in\C$.
  \item Hún er \emph{hermískt samhverf}, þ.e.
    \[
    \langle g,f \rangle = \overline{\langle f,g\rangle}.
    \]

  \item Fyrir öll $f\in L^{2}$ er $\langle f,f\rangle\ge 0$; og
    $\langle f,f\rangle=0$ þ.þ.a.a. $f=0$.
  \end{enumerate}
  Við höfum
  \[
  \| f_{2}\| = \sqrt{\langle f,f\rangle}.
  \]
\end{setn}
\begin{ath}
  Fyrir gefið $\C$-línulegt rúm $V$ ásamt vörpun $V\times V\to\C,
  (x,y)\mapsto\langle x,y\rangle$ þ.a. skilyrðum (1-3) að ofan er
  fullnægt, fæst staðall $V\mapsto\R,x\mapsto\|x\|:=\sqrt{\langle
    x,x\rangle}$; og við fáum \emph{Cauchy-Schwarz} ójöfnu
  \[
  |\langle x,y\rangle| \le\|x\|\cdot \|y\|
  \]
  fyrir öll $x,y\in V$.
\end{ath}
\begin{skilgr}
  \emph{Hilbert-rúm}\index{Hilbert-rum@Hilbert-rúm} er $\C$-línulegt
  rúm með hermísku innfeldi þ.a. firðin sem samsvarandi staðall gefur
  af sér gerir rúmið að fullkomnu firðrúmi.
\end{skilgr}
Höfum:
\begin{setn}
  $L^2$ er Hilbert-rúm.
\end{setn}
\begin{ath}
  Skoðum sérstaklega talningarmálið $\tau$ á $\N$. Þá er rúmið
  \[
  \ell^{2}
  = \ell^{2}_{\C}
  = L^{2}(\N,\mathcal P(\N),\tau)
  \]
  mengi allra runa $(a_{k})_{k\in\N}$ þ.a.
  \[
  \sum_{k=0}^{+\infty}|a_{k}|^{2}
  < +\infty.
  \]
  Fyrir $a = (a_{k})$ og $b = (b_{k})$ er
  \[
  \langle a,b\rangle
  = \sum_{k=0}^{+\infty}a_{k}\bar b_{k}
  \]
  og Cauchy-Schwarz-Bunyakovsky-ójafnan verður
  \[
  \sum_{k=0}^{+\infty} |a_{k} b_{k}|
  \le \sqrt{\sum_{k=0}^{+\infty}|a_{k}^{2}|}
  \cdot \sqrt{\sum_{k=0}^{+\infty}|b_{k}^{2}|}.
  \]
  Eins má athuga talningarmálið á $\Z$, sem gerir mengið af öllum
  tvöföldum runum $(a_{k})_{k\in\Z}$ þ.a.
  \[
  \sum_{-\infty}^{+\infty}|a_{k}|^{2} < +\infty,
  \]
  þetta rúm er líka táknað $\ell_{\C}^{2}$, því að gagntæk vörpun
  $\N\to\Z$ gefur af sér einsmótun $L^{2}$-rúmanna.
\end{ath}
%%
%%
\marginpar{14. mars}
%%
%%

\chapter{Fourier-raðir}

\begin{skilgr}
  \emph{Fourier-röð}\index{Fourier!rod@röð} er fallaröð af
  gerðinni 
  \[
  \frac{a_{0}}{2} + \sum_{n=1}^{+\infty}(a_{n}\cos(nx)+b_{n}\sin(nx))
  \]
  þar sem $a_{n},b_{n}\in\C$. Við segjum ekkert um samleitni
  raðarinnar í bili, heldur lítum á hana sem runu af hlutsummunum
  $(s_{N})_{N\in\N}$, þar sem 
  \[
  s_{N}(x) :=
  \frac{a_{0}}{2}+\sum_{n=1}^{N}(a_{n}\cos(nx)+b_{n}\sin(nx)).
  \]
  Þegar við segjum að röðin sé samleitin í einhverjum skilningi, þá
  meinum við að fallarunan $(s_{N})_{N\in\N}$ sé þannig samleitin.
\end{skilgr}
Við getum líka skrifað 
\[
s_{N}(x) := \sum_{n=-N}^{N}c_{n}e^{inx}
\]
þar sem 
\[
c_{n} = \frac{a_{n}-ib_{n}}{2},
\quad
c_{-n} = \frac{a_{n}+ib_{n}}{2}
\]
fyrir $n\ge 0$ (setjum alltaf $b_{0}=0$). Stuðlarnir $c_{n}$ ákvarða
stuðlana $a_{n}$ og $b_{n}$, við höfum að 
\[
a_{n} = c_{n} + c_{-n},
\quad
b_{n} = i(c_{n}-c_{-n}).
\]
Gerum ráð fyrir að röðin sé samleitin í sérhverjum punkti úr $\R$, þá
skilgreinir hún fall $f$ á $\R$, 
\[
f(x)
:= \sum_{n=-\infty}^{+\infty}c_{n}e^{inx} 
:= \lim_{N\to +\infty}s_{N}(x).
\]
Vegna þess að föllin $x\mapsto e^{inx}$ eru öll lotubundin með lotu
$2\pi$ verður $f$ lotubundið með lotu $2\pi$. Við höfum
\[
f(x + 2\pi) = f(x)
\]
fyrir öll $x\in\R$. Gerum ráð fyrir að $f$ sé heildanlegt yfir
$[-\pi,\pi]$ (m.t.t. $\lambda$) og að við megum skipta um röð á summu
og heildi; t.d. ef samleitnin er í jöfnum mæli á $[-\pi,\pi]$, eða
almennar yfirgnæfð af heildanlegu falli á $[-\pi,\pi]$; þá fáum við að 
\begin{align*}
  \int_{-\pi}^{\pi}f(x)e^{-inx}\,dx
  &= \int_{-\pi}^{\pi}\sum_{k=-\infty}^{+\infty}c_{k}e^{i(k-n)x}\,dx
  \\
  &= \sum_{k=-\infty}^{+\infty} c_{k}\int_{-\pi}^{\pi}e^{i(k-n)x}\,dx
  \\
  &= 2\pi c_{n}
\end{align*}
því að 
\[
\int_{-\pi}^{\pi}e^{imx}\,dx
=
\begin{cases}
  2\pi, & m = 0,\\
  0, & m\in\Z\setminus\{0\}
\end{cases}
\]
því að fyrir $m\ne 0$ hefur $x\mapsto e^{imx}$ stofnfallið
$\frac{1}{im}e^{imx}$, sem tekur sömu gildi í $\pm\pi$. Við höfum 
\[
c_{n}
= \frac{1}{2\pi}\int_{-\pi}^{\pi}f(x)e^{-inx}\,dx.
\]
Látum nú $f$ vera \emph{eitthvert} heildanlegt fall á $[-\pi,\pi]$, þá
er $x\mapsto f(x)e^{-inx}$ líka heildanlegt og við getum skilgreint
stuðla $c_{n}$ með þessari formúlu.
\begin{skilgr}
  Látum $f:\R\to\C$ vera lotubundið fall með lotu $2\pi$ þ.a. $f$ sé
  heildanlegt yfir $[-\pi,\pi]$ m.t.t. \emph{Lebesgue}-málsins á
  $\R$. Talan 
  \[
  \hat f(n) := \frac{1}{2\pi}\int_{-\pi}^{\pi}f(x)e^{-inx}\,dx
  \]
  kallast \emph{$n$-ti Fourier-stuðull}\index{Fourier!studull@stuðull}
  fallsins $f$ og röðin 
  \[
  \sum_{n=+\infty}^{+\infty}\hat f(n)e^{inx}
  \]
  kallast \emph{Fourier-röð}\index{Fourier!rod@röð} fallsins $f$.
\end{skilgr}
\begin{ath}
  Samsvarandi stuðlar $a_{n},b_{n}$ eru gefnir með
  \begin{align*}
    a_{n}
    &=
    \hat f(n) + \hat f(-n)
    = \frac 1\pi\int_{-\pi}^{\pi} f(x)\cos(nx)\,dx,
    \\
    b_{n}
    &= i(\hat f(n) - \hat f(-n))
    = \frac 1\pi\int_{-\pi}^{\pi} f(x)\sin(nx)\,dx.
  \end{align*}
\end{ath}
\begin{ath}
  Fourier-röð falls þarf ekki að vera samleitin í neinum punkti. Fall
  getur ekki ákvarðast ótvírætt af Fourier-röð sinni, því að föll sem
  eru eins næstum allsstaðar hafa sömu Fourier-röð. Hins vegar gildir
\end{ath}
\begin{setn}
  \label{setn:fourier-nas}
  Látum $f,g$ vera lotubundin með lotu $2\pi$ og heildanleg yfir
  $[-\pi,\pi]$. Ef $f,g$ hafa sömu Fourier-stuðla, þá er $f=g$ næstum
  allsstaðar.
\end{setn}
\begin{lemma}
  Látum $f\in L_{\C}^{1}([-\pi,\pi])$ vera þannig að 
  \[
  \int_{-\pi}^{t}f(x)\,dx = 0
  \]
  fyrir öll $t\in[-\pi,\pi]$. Þá er $f=0$ næstum allsstaðar.
\end{lemma}
\begin{proof}
  [Sönnun á hjálparsetningu]
  Fyrir $s,t\in[-\pi,\pi]$, $s<t$, er 
  \[
  \int_{\left]s,t\right[}f\,dx
  = \int_{-\pi}^{t}f\,dx - \int_{-\pi}^{s}f\,dx
  = 0,
  \]
  þar sem sérhvert opið mengi er teljanlegt sammengi af opnum bilum er 
  \[
  \int_{U}f\,dx = 0
  \]
  fyrir öll opin hlutmengi $U$ í $[-\pi,\pi]$. Sýnum að hið sama
  gildir fyrir mælanleg mengi í $[-\pi,\pi]$. Látum
  $A\subset[-\pi,\pi]$ vera mælanlegt og $\varepsilon > 0$, þá er til
  opið mengi $U$ þ.a. $A\subset U$ og $\lambda(U)\le\lambda(A) +
  \varepsilon$. Höfum þá minnkandi runu $(U_{n})$ af opnum mengjum
  þ.a. $A\subset U_{n}$ og
  $\lim_{n\to+\infty}\lambda(U_{n})=\lambda(A)$; því er
  $B=\bigcap_{n\in\N}U_{n}$ mælanlegt; $A\subset B$ og $B\setminus A$
  er núllmengi svo að 
  \[
  \int_{A}f(x)\,dx
  = \int_{B} f(x)\,dx
  = \int_{\R} f\chi_{B}\,dx
  = \lim_{n\to +\infty}\int_{\R} f\chi_{U_{n}}\,dx
  = 0.
  \]
  Með því að leysa upp í raunhluta og þverhluta má gera ráð fyrir að
  $f$ sé raungilt. Setjum $A^{+} := \{x : f(x) > 0\}$, $A^{-} := \{x :
  f(x) <0 \}$. Þá eru $A^{+},A^{-}$ mælanleg og
  $\int_{[-\pi,\pi]}f^{+}\,dx=\int_{A^{+}}f\,dx = 0$; eins er
  $\int_{[-\pi,\pi]}f^{-}\,dx = 0$, svo að 
  \[
  \int_{[-\pi,\pi]}|f|\,dx
  = \int_{X}f^{+}+\int_{X}f^{-}
  = 0,
  \]
  en þá er $f=0$ næstum allsstaðar.
\end{proof}
\begin{proof}
  [Sönnun á setningu \ref{setn:fourier-nas}]
  Okkur nægir að sýna: Ef $f\in\mathcal L^{1}([-\pi,\pi])$ og $\hat
  f(n)=0$ fyrir öll $n\in\Z$, þá er $f=0$ næstum allsstaðar. En ef
  $\hat f(n) = 0$ fyrir öll $n$, þá fæst fyrir
  $s_{n}(x)=\sum_{k=-n}^{n}c_{k}e^{ikx}$, þar sem $c_{k}$ eru fastar,
  að 
  \[
  \int_{-\pi}^{\pi}fs_{n}\,dx
  = \sum_{k=-n}^{n}c_{k}\hat f(-k)
  = 0.
  \]
  Látum $t\in[-\pi,\pi]$. Þá má finna runu $(g_{n})$ af samfelldum
  föllum, línulegum á köflum, þ.a. $0\le g_{n}\le 1$,
  $g(-\pi)=g(\pi)=0$ og $\lim_{n\to +\infty}g_{n} =
  \chi_{\left]-\pi,t\right[}$; þá má framlengja $g$ í lotubundið fall
  á $\R$ með lotu $2\pi$. Skv. hjálparsetningu, sem við sönnum næst,
  þá stefnir Fourier-röð fallsins $(g_{n})$ á $g_{n}$ í jöfnum mæli á
  öllu $\R$; því má finna hlutsummu hennar sem $p_{n}$
  þ.a. $|p_{n}-s_{n}|<\frac 1n$. Þá gildir 
  \[
  f\chi_{\left[-\pi,t\right[}
  = \lim_{n\to +\infty}f\cdot p_{n}
  \]
  á $[-\pi,\pi]$ og $(fp_{n})\le|f|(|g_{n}|+\frac 1n)\le 2|f|$, svo að
  samkvæmt setningu Lebesgue um yfirgnæfða samleitni er 
  \[
  \int_{-\pi}^{t}f\,dx
  = \int_{-\pi}^{\pi}f\chi_{\left]-\pi,t\right[}\,dt
  = \lim_{n\to +\infty}\int_{-\pi}^{\pi}fp_{n}\,dx
  = 0;
  \]
  skv. hjálparsetningu er þá $f=0$ næstum allsstaðar.
\end{proof}
Hér átti eftir að sanna eftirfarandi:
\begin{lemma}
  \label{setn:fourier-lemma}
  Ef $g$ er lotubundið með lotu $2\pi$, samfellt og $g|_{[-\pi,\pi]}$
  er línulegt á köflum, þá stefnir Fourier-röð $g$ á $g$ í jöfnum mæli
  á $\R$.
\end{lemma}
%%
%%
\marginpar{15. mars}
%%
%%
Byrjum hins vegar á því að reikna dæmi:
\begin{daemi}
  Skilgreinum fall $f:\R\to\C$ með þeim skilyrðum að
  \[
  f(x) =
  \begin{cases}
    \frac x2, & -\pi<x<\pi,\\
    0, & x = -\pi
  \end{cases}
  \]
  og $f$ er lotubundið með lotu $2\pi$. Við höfum $\hat f(0)=0$, því
  að $f$ er oddstætt á $\left]-\pi,\pi\right[$ og fyrir $n\ne 0$ er
  \begin{align*}
    \hat f(n)
    &= \frac{1}{2\pi}\int_{-\pi}^{\pi}\frac x2 e^{-inx}\,dx
    \\
    &= \frac{1}{2\pi}\left[
      -\frac{x}{2in}e^{-inx}
    \right]_{-\pi}^{\pi}
    + \frac{1}{2\pi in}\int_{-\pi}^{\pi}e^{-inx}\,dx
    \\
    &= -\frac{1}{4\pi in}(\pi e^{-\pi in} + \pi e^{\pi in}) + 0
    \\
    &= \frac{(-1)^{n+1}}{2in}.
  \end{align*}
  Hlutsummur Fourier-raðarinnar verða
  \[
  s_{N}(x)
  = \sum_{\substack{n=-N\\n\ne 0}}^{N} \frac{(-1)^{n+1}}{2in}e^{inx}
  = \sum_{n=1}^{N}\frac{(-1)^{n+1}}{n}\sin{nx}.
  \]
  Athugum að
  \[
  \frac{1}{in}e^{inx}
  = \int_{0}^{x}e^{int}\,dt + \frac{1}{in}.
  \]
  Nú er
  \[
  \sum_{\substack{n=-N\\n\ne 0}}^{N}\frac{1}{in}=0,
  \]
  svo að við fáum
  \begin{align*}
    s_{N}(x)
    &= \int_{0}^{x}\sum_{\substack{n=-N\\n\ne0}}^{N}
    \frac 12 (-1)^{n+1}e^{int}\,dt
    \\
    &= \frac x2 - \frac 12\int_{0}^{x}
    \sum_{n=-N}^{N}(-1)^{n}e^{int}\,dt.
  \end{align*}
  Fyrir $-\pi<t<\pi$ er
  \begin{align*}
    \sum_{n=-N}^{N}(-1)^{n}e^{int}
    &= \sum_{n=0}^{2N}(-1)^{n-N}e^{i(n-N)t}
    \\
    &= (-1)^{N}e^{-iNt}\sum_{n=0}^{2N}(-e^{it})^{n}
    \\
    &= (-1)^{N}e^{-iNt}\frac{1-(-1)^{2N+1}e^{i(2N+1)t}}{1+e^{it}}
    \\
    &= (-1)^{N}e^{-iNt-\frac{it}2}
    \frac{1+e^{i(2N+1)t}}{e^{-\frac{it}2}+e^{\frac{it}2}}
    \\
    &= (-1)^{N}
    \frac{e^{i(N+\frac12)t}+e^{-i(N+\frac12)t}}{e^{-\frac{it}2}+e^{\frac{it}2}}
    \\
    &= (-1)^{N}\frac{\cos((N+\frac12)t)}{\cos{\frac t2}}.
  \end{align*}
  Látum $0<a<\pi$. Fyrir $|x|\le a$ er
  \begin{align*}
    &\left|\frac x2-s_{N}(x)\right|
    \\
    &\le
    \left|
      \frac12
      \int_{0}^{x}
      \frac{\cos((N+\frac12)t)}{\cos{\frac t2}}
      \,dt
    \right|
    \\
    &\le \frac 12 \left|
      \left[
        \frac{\sin((N+\frac12)t)}{(N+\frac12)\cos\frac t2}
      \right]_{0}^{x}
      + \frac{1/2}{N+1/2}
      \int_{0}^{x}
      \sin((2N+1)t)
      \frac{\sin\frac t2}{\cos^{2}\frac t2}
      \,dt
    \right|
    \\
    &\le \frac{1}{2N+\frac 12}
    \left(
      \frac 1{\cos\frac a2} + \frac a{2\cos^{2}\frac a2}
    \right)
  \end{align*}
  svo að
  \[
  \lim_{N\to+\infty}s_{N}(x)
  = \frac x2
  \]
  í jöfnum mæli á $[-a,a]$ fyrir sérhvert $a$ úr
  $\left]0,\pi\right[$. Einnig er $s_{N}(-\pi)=s_{N}(\pi)=0$ fyrir öll
  $N$, svo að $\lim_{N\to+\infty}s_{N}(x)=f(x)$ fyrir öll $x\in\R$,
  þ.e.
  \[
  f(x)
  = \sum_{\substack{n=-\infty\\n\ne0}}^{+\infty}
  \frac{(-1)^{n+1}}{2in}e^{inx}
  = \sum_{n=1}^{+\infty}\frac{(-1)^{n+1}}n \sin{nx}.
  \]
  Þar sem samleitnin er í jöfnum mæli á $[-a,a]$ fyrir öll $a$ úr
  $\left]0,\pi\right[$, þá má heilda frá $0$ til $x$,
  $x\in\left]-\pi,\pi\right[$, og skipta um rö ðá summu og heildi; svo
  að fyrir $x\in\left]-\pi,\pi\right[$ er
  \begin{align*}
    \frac{x^{2}}4
    &= \int_{0}^{x}\frac t2\,dt
    \\
    &= 
    \frac{(-1)^{n+1}}{2in}
    \int_{0}^{x}e^{int}\,dt
    \\
    &= \sum_{\substack{n=-\infty\\n\ne 0}}^{+\infty}
    \frac{(-1)^{n+1}}{2(in)^{2}}(e^{inx}-1)
    \\
    &= \sum_{\substack{n=-\infty\\n\ne 0}}^{+\infty}
    \frac{(-1)^{n}}{2n^{2}}e^{inx}
    + \sum_{n=1}^{+\infty}\frac{(-1)^{n+1}}{n^{2}}.
  \end{align*}
  En þessi röð er alsamleitin í jöfnum mæli á öllu $\R$ og skilgreinir
  því samfellt fall, svo að jafnan gildir fyrir $x$ á \emph{lokaða}
  bilinu $[-\pi,\pi]$; og við höfum
  \[
  F(x)
  = \sum_{\substack{n=-\infty\\n\ne 0}}^{+\infty}
  \frac{(-1)^{n}}{2n^{2}}e^{inx}
  + \sum_{n=1}^{+\infty}\frac{(-1)^{n+1}}{n^{2}}
  \]
  þar sem að $F$ ákvarðast af því að $F(x)=\frac{x^{2}}4$ fyrir
  $x\in[-\pi,\pi]$ og $F$ er lotubundið með lotu $2\pi$. Setjum
  $x=\pi$; fáum
  \[
  \frac{\pi^{2}}4
  = \sum_{n=1}^{+\infty}\frac{1}{n^{2}}
  + \sum_{n=1}^{+\infty}\frac{(-1)^{n+1}}{n^{2}}.
  \]
  En nú er
  \[
  \sum_{n=1}^{+\infty}\frac{(-1)^{n+1}}{n^{2}}
  = \sum_{n=1}^{+\infty}\frac{1}{n^{2}}
  - 2\sum_{n=1}^{+\infty}\frac{1}{(2n)^{2}}
  = \frac 12 \sum_{n=1}^{+\infty}{1}{n^{2}}.
  \]
  Fáum
  \[
  \sum_{n=1}^{+\infty}\frac{1}{n^{2}}
  = \frac{\pi^{2}}{6},
  \qquad
  \sum_{n=1}^{+\infty}\frac{(-1)^{n+1}}{n^{2}}
  = \frac{\pi^{2}}{12}.
  \]

  Skoðum aftur fallið $f(x)$, sem hefur stökkpunkta í punktum
  $x_{n}:=\pi+2\pi n$, $n\in\Z$. Höfum
  \[
  f(x_{n})
  = \frac 12\left(
    \lim_{x\to x_{n}+} f(x) + \lim_{x\to x_{n}-}f(x)
  \right).
  \]
  Látum nú $s,t\in\left[-\pi,\pi\right[$ og setjum
  \[
  g(x) := f(x-s) - f(x-t).
  \]
  Fallið er þrepafall á $[-\pi,\pi]$ og lotubundið með lotu
  $2\pi$. Fallið $g$ fullnægir áfram skilyrðinu
  \[
  g(x) = \frac 12 \left(
    \lim_{t\to x-}g(t) + \lim_{t\to x+} g(t).
  \right)
  \]
  Auðvelt er að reikna Fourier-röð fallsins $g$ útfrá Fourier-röð
  falsins $f$: Ef $f(x)=\sum_{n=-\infty}^{+\infty}c_{n}e^{inx}$, þá er
  \[
  f(x-s)
  = \sum_{n=-\infty}^{+\infty}c_{n}e^{-ins}e^{inx}.
  \]
  Fourier-röð fallsins $g$ er allsstaðar samleitin, $g$ er markgildi
  hennar, og samleitnin er í jöfnum mæli á sérhverju lokuðu bili sem
  inniheldur engan stökkpunkt fallsins $g$. En sérhvert fall sem fæst
  úr þrepafalli á $[-\pi,\pi[$ með því að framlengja það í lotubundið
  fall með lotu $2\pi$ er línuleg samantekt af föstu falli og
  endanlega mörgum föllum af sama tagi og $g$ (þ.e. fyrir ólík $s,t$),
  \emph{nema} í stökkpunktum. Köllum slíkt fall \emph{lotubundið
    þrepafall}.\index{zrepafall@þrepafall!lotubundid@lotubundið}
\end{daemi}
Fáum
\begin{setn}
  Látum $h$ vera lotubundið þrepafall með lotu $2\pi$. Þá er
  Fourier-röð fallsins $h$ samleitin á öllu $\R$, samleitnin er í
  jöfnum mæli á sérhverju lokuðu bili sem inniheldur ekki stökkpunkt
  fallsins $h$. Röðin stefnir á $h$ í öllum punktum nema
  stökkpunktunum, en í stökkpunkti $x$ stefnir hún á
  \[
  \frac 12\left(
    \lim_{t\to x-} h(t)
    +
    \lim_{t\to x+} h(t)
  \right).
  \]
\end{setn}
%%
%%
\marginpar{19. og 21. mars}
%%
%%
Á sama hátt fæst líka hjálparsetning \ref{setn:fourier-lemma}, sem
sagði að fall sem er lotubundið með lotu $2\pi$, samfellt og línulegt
á köflum, hefur Fourier-röð sem er alsamleitin í jöfnum mæli á öllu
$\R$ og stefnir á fallið í sérhverjum punkti. Þetta þurfti til þess að
ljúka sönnun setningar \ref{setn:fourier-nas}.
\begin{ath}
  Varðandi fyrri skilgreiningu, þá er venja að kalla röð af gerðinni 
  \[
  \frac{a_{0}}{2} + \sum_{n=1}^{+\infty}(a_{n}\cos(nx)+b_{n}\sin(nx))
  \]
  \emph{hornafallaröð}\index{hornafallarod@hornafallaröð}, en nota
  orðið \emph{Fourier-röð}\index{Fourier!rod@röð} einungis um slíkar
  raðir
  \[
  \sum_{n=-\infty}^{+\infty}\hat f(n)e^{inx},
  \]
  sem eru skilgreindar út frá Fourier-stuðlum gefins falls $f$. Síðan
  er flókið verkefni að kanna, hvaða hornafallaraðir eru á annað borð
  Fourier-raðir.
\end{ath}
\begin{ath}
  (1) Hér að ofan þurftum við að nota að skipta mætti um röð á heildi
  og summu fyrir runur sem eru samleitnar í jöfnum mæli á takmörkuðu
  bili. Látum almennar $(X,\mathcal A,\mu)$ vera málrúm
  þ.a. $\mu(X)<+\infty$ og $(f_{k})$ vera runu af heildanlegum föllum
  sem stefnir á fall $f$ í jöfnum mæli á $X$, þá er $f$ heildanlegt og 
  \[
  \lim_{k\to +\infty}\int_{X}f_{k}\,d\mu
  = \int_{X}f\,d\mu.
  \]
  Til að sjá það, athugum við að til er $k_{0}$ þ.a. $|f-f_{k_{0}}\le
  1$ allsstaðar og þá $|f|=|f-f_{k_{0}}+f_{k_{0}}|\le 1 +
  |f_{k_{0}}|$, svo að $f$ er heildanlegt og við höfum 
  \[
  \left|
    \int_{X}f\,d\mu
    - \int_{X} f_{k}\,d\mu
  \right|
  \le \int_{X}|f-f_{k}|\,d\mu
  \le \mu(X)\sup_{x\in X}|f(x) - f_{k}(x)|
  \xrightarrow[k\to+\infty]{} 0.
  \]
  \emph{Ekki má} sleppa forsendunni $\mu(X)<+\infty$: Athugum
  t.d. $f_{k}:\R\to\R$, 
  \[
  f_{k} := \frac{1}{k+1}\chi_{[0,k+1]}.
  \]
  Þá er $\int_{X}f_{k}\,d\lambda = 1$ fyrir öll $k$, en $(f_{k})$
  stefnir á $0$ í jöfnum mæli á $\R$.

  (2) Látum $f$ vera samfellt fall á $\R$ með lotu $2\pi$ og g.r.f. að
  Fourier-röð þess sé samleitin í jöfnum mæli á öllu $\R$. Þá stefnir
  hún allsstaðar á $f$. Til að sjá það setjum við 
  \[
  g(x) := \sum_{n=-\infty}^{+\infty} \hat f(n)e^{inx}.
  \]
  Skv. (1) er þá
  \begin{align*}
    \hat g(n)
    &= \frac{1}{2\pi} \int_{-\pi}^{\pi}g(x)e^{-inx}\,dx
    \\
    &= \frac{1}{2\pi} \int_{-\pi}^{\pi}\sum_{k=-\infty}^{+\infty}
    \hat f(k)e^{i(k-n)x}\,dx
    \\
    &= \sum_{k=-\infty}^{+\infty}\hat f(k)\frac{1}{2\pi}
    \int_{-\pi}^{\pi}e^{i(k-n)x}\,dx
    \\
    &= \sum_{k=-\infty}^{+\infty}f(x)\delta_{n}
    \\
    &= \hat f(n),
  \end{align*}
  fyrir öll $n$. Skv. setningu er $f=g$ næstum allsstaðar. En $f$ og
  $g$ eru bæði samfelld, svo $f=g$.

  (3) Látum $f$ vera mælanlegt fall á málrúmi $X$
  þ.a. $\mu(X)<+\infty$. Skv. Hölder-ójöfnu er fyrir
  $\frac{1}{q}+\frac{1}{p} = 1$ 
  \[
  \left\|
    f
  \right\|_{1}
  = \int_{X}^{}|f|\,d\mu
  = \int_{X}^{}\left|
    f\cdot 1
  \right|\,d\mu
  \le \left\|
    f
  \right\|_{p}
  \cdot \left\|
    1
  \right\|_{q}
  = \left\|
    f
  \right\|_{p}
  \cdot (\mu(X))^{1/q}.
  \]
  Við sjáum að 
  \[
  \mathcal L^{p}(X) \subset\mathcal L^{1}(X)
  \]
  fyrir öll $p\ge 1$. Sér í lagi, ef $f:\R\to\C$ er mælanlegt,
  lotubundið með lotu $2\pi$ og $|f|^{p}$ er heildanlegt yfir
  $[-\pi,\pi]$, þá er $f$ einnig heildanlegt yfir $[-\pi,\pi]$ og
  Fourier-röð $f$ vel skilgreind. \emph{Athugum sérstaklega} tilvikið
  $p=2$. Við táknum með $\mathcal L_{\C}^{2}(\mathbb T)$ mengi allra
  lotubundinna tvinnfalla með lotu $2\pi$ þ.a. $|f|^{2}$ sé
  heildanlegt yfir $[-\pi,\pi[$ og með $L_{\C}^{2}(\mathbb T)$
  samsvarandi \emph{Hilbert-rúm}\index{Hilbert-rum@Hilbert-rúm} af
  jafngildisflokkum $[f]$, þa rsem föll eru jafngild ef þau eru eins
  næstum allsstaðar. Skrifum venjulega $f$ í stað $[f]$. Rithátturinn
  kemur af því að við lítum á föllin $f$ sem föll á deildagrúpunni
  $\mathbb T:=\R/2\pi\Z$: Ef $\phi:\R\to\R/2\pi\Z$ er náttúrlega
  ofanvarpið og $g:\R/2\pi\Z\to\C$ er fall, þá er $f:=g\circ\phi$
  lotubundið með lotu $2\pi$ og öll föll með lotu $2\pi$ fást
  þannig. Til er einsmótun frá $\mathbb U$ til $\mathbb T$. Almennt
  kallast mælanlegt fall á málrúmi $X$ þ.a. $|f|^{2}$ sé heildanlegt
  \emph{ferningsheildanlegt}\index{ferningsheildanlegt fall} fall á
  $X$. Við notum á $L_{\C}^{2}(\mathbb T)$ innfeldið 
  \[
  \langle f, g \rangle
  := \frac{1}{2\pi}\int_{-\pi}^{\pi}f\bar g\,d\lambda
  \]
  og setjum til samræmis 
  \[
  \left\|
    f
  \right\|_{2}
  := \sqrt{\langle f,f\rangle}
  = \frac{1}{\sqrt{2\pi}}\left(
    \int_{-\pi}^{\pi}\left|
      f
    \right|^{2}\,d\lambda
  \right)^{1/2}.
  \]
\end{ath}
Veljum stuðulinn $\frac{1}{2\pi}$ þ.a. gildi: 
\begin{skilgr}
  Fyrir $n\in\Z$ táknum við með 
  \[
  e_{n}:\R\to\C
  \]
  fallið þ.a. $e_{n}(x) = e^{inx}$; höfum þá 
  \[
  \bar e_{n}(x) = e_{-n}(x) = e^{-inx}
  \]
  og 
  \[
  \langle e_{n}, e_{m}\rangle
  = \delta_{n,m}
  = 
  \begin{cases}
    1, & n = m,\\
    0, & n\ne m.
  \end{cases}
  \]
  Fyrir $f\in\mathcal L_{\C}^{2}(\mathbb T)$ er 
  \[
  \hat f(n) = \langle f,e_{n}\rangle
  \]
  og Fourier-röð fallsins $f$ er því
  \[
  \sum_{n=-\infty}^{+\infty}\langle f,e_{n}\rangle e_{n}.
  \]
\end{skilgr}
\begin{ath}
  Við skilgreinum innfeldi og staðal á $\mathcal L^{2}(\mathbb T)$
  m.t.t. málsins $\frac{1}{2\pi}\lambda$, þar sem $\lambda$ er
  Lebegue-málið.
\end{ath}
\begin{setn}
  [Riesz-Fischer]
  \index{Riesz-Fischer}
  
  Látum $(c_{n})_{n\in\Z}$ vera fjölskyldu af tvinntölum þ.a. 
  \[
  \sum_{k\in\Z}^{}|c_{k}|^{2} < +\infty,
  \]
  þ.e. $(c_{n})_{n\in\Z}$ er stak í $\ell^{2}_{\C} = L^{2}(\Z,\mathcal
  P(\Z),\tau)$; þá er röðin 
  \[
  f := \sum_{n\in\Z}^{}c_{n}e_{n}
  \]
  samleitin í $L^{2}_{\C}(\mathbb T)$ og 
  \[
  \hat f(n) = c_{n}
  \]
  fyrir öll $n\in\Z$.
\end{setn}
\begin{proof}
  Setjum 
  \[
  s_{n} := \sum_{k=-n}^{n}c_{k}e_{k}.
  \]
  Fyrir $m>n>0$ er
  \begin{align*}
    \left\|
      s_{n} - s_{m}
    \right\|_{2}^{2}
    &=
    \left\langle
      \sum_{n<|k|\le m} c_{k}e_{k}
      ,
      \sum_{n<|j|\le m} c_{j}e_{j},
    \right\rangle
    \\
    &= \sum_{\substack{n<|k|\le m\\ n<|j|\le m}} c_{k}\bar c_{j}
    \langle e_{k},e_{j}\rangle
    \\
    &= \sum_{\substack{n<|k|\le m\\ n<|j|\le m}} c_{k}\bar c_{j}
    \delta_{kj}
    \\
    &= \sum_{n<|k|\le m}|c_{k}|^{2}
  \end{align*}
  og síðasta summan stefnir á 0 þegar $m,n\longrightarrow+\infty$. Þar
  með er $(s_{n})$ Cauchy-runa í $L_{\C}^{2}(\mathbb T)$ og því
  samleitin; köllum markgildið $f$. Við höfum fyrir $m\ge n$: 
  \[
  \langle s_{m},e_{n}\rangle
  = \sum_{k=-m}^{m}c_{k}\langle e_{k},e_{n}\rangle
  = c_{n}
  \]
  og með Cauchy-Schwarz-ójöfnum fæst
  \begin{align*}
    \left|
      \langle f,e_{n}\rangle
      - \langle s_{m},e_{n}\rangle
    \right|
    &= \left|
      \langle f - s_{m},e_{k}\rangle
    \right|
    \\
    &\le \left\|
      f - s_{m}
    \right\|_{2}
    \cdot \left\|
      e_{m}
    \right\|_{2}
    \\
    &= \left\|
      f-s_{m}
    \right\|_{2}
    \\
    &\xrightarrow[m\to+\infty]{} 0,
  \end{align*}
  svo að $\hat f(n) = \langle f,e_{n}\rangle = c_{n}$ fyrir öll
  $n\in\Z$.
\end{proof}


\begin{lemma}
  Látum $f\in L_{\C}^{2}$ og setjum 
  \[
  s_{n} = \sum_{k=-n}^{n}\langle f,e_{k}\rangle e_{k}.
  \]
  Þá er $\|s_{n}\|_{2}\le\|f\|_{2}$ fyrir öll $n$ og 
  \[
  \lim_{n\to +\infty}\left\|
    f - s_{n}
  \right\|_{2}
  = 0.
  \]
\end{lemma}
\begin{proof}
  Við höfum 
  \begin{align*}
    \langle f,s_{n}\rangle
    &=
    \left\langle
      f, \sum_{k=-n}^{n}\hat f(k)e_{k}
    \right\rangle
    \\
    &= \sum_{k=-n}^{n}\langle f,e_{k}\rangle
    \overline{\hat f(k)}
    \\
    &= \sum_{k=-n}^{n}\left|
      \hat f(k)
    \right|^{2}
  \end{align*}
  og 
  \[
  \langle s_{n},s_{n}\rangle
  =
  \left\langle
    \sum_{k=-n}^{n}\hat f(k)e_{k},
    \sum_{k=-n1}^{n}\hat f(k)e_{k}
  \right\rangle
  = \sum_{k=-n}^{n}\left|
    \hat f(k)
  \right|^{2}
  \]
  svo að 
  \begin{align*}
    0
    &\le
    \left\|
      f - s_{n}
    \right\|_{2}^{2}
    \\
    &=
    \left\langle
      f - s_{n},f - s_{n}
    \right\rangle
    \\
    &= \langle f,f\rangle
    - \langle f,s_{n}\rangle
    - \langle s_{n},f\rangle
    + \langle s_{n},s_{n}\rangle
    \\
    &= \left\|
      f
    \right\|_{2}^{2}
    - \langle f,s_{n}\rangle
    - \overline{\langle s_{n}, f\rangle}
    + \langle s_{n},s_{n}\rangle
    \\
    &= \left\|
      f
    \right\|_{2}^{2} - \sum_{k=-n}^{n} \left|\hat f(k)\right|^{2}
  \end{align*}
  svo að 
  \[
  \sum_{k=-n}^{n}\left|
    \hat f(k)
  \right|^{2}
  = \left\|
    s_{n}
  \right\|_{2}^{2}
  \le \left\|
    f
  \right\|_{2}^{2}.
  \]
  Látum þá $n\longrightarrow +\infty$ og fáum 
  \[
  \sum_{n\in\N}\left|
    \hat f(k)
  \right|^{2}
  \le \left\|
    f
  \right\|_{2}^{2}.
  \]
  Skv. Riesz-Fischer-setningu er til $g$ úr $\mathcal
  L_{C}^{2}(\mathbb T)$ þ.a. $\hat g(n)=\hat f(n)$ fyrir öll $n$ og
  $\lim_{n\to+\infty}\left\| g - s_{n} \right\|_{2} = 0$. En
  skv. setningu er þá $f = g$ næstum allsstaðar, svo að
  $\lim_{n\to+\infty}\left\| f - s_{n} \right\|_{2} = 0$.
\end{proof}
\begin{setn}
  [Parseval-jöfnur]\index{Parseval-jofnur@Parseval-jöfnur}
  Látum $f,g\in\mathcal L_{\C}^{2}(\mathbb T)$. Þá er  
  \[
  \sum_{n=-\infty}^{+\infty}\hat f(n)\overline{\hat g(n)}
  = \frac{1}{2\pi}\int_{-\pi}^{\pi}f\bar g\,d\lambda
  \]
  og 
  \[
  \sum_{n=-\infty}^{+\infty}\left|
    \hat f(n)
  \right|^{2}
  = \frac{1}{2\pi}\int_{-\pi}^{\pi} \left|
    f
  \right|^{2}\,d\lambda.
  \]
  Sér í lagi eru raðirnar samleitnar.
\end{setn}
\begin{proof}
  Setjum $s_{n} := \sum_{k=-n}^{n}\hat f(k)e_k$,
  $t_n:=\sum_{k=-n}^{n}\hat g(k)e_{k}$. Þá er
  \begin{align*}
    \langle s_{n},t_{n}\rangle
    &= \sum_{k=-n}^{n}\sum_{j=-n}^{n}
    \hat f(k)\overline{\hat g(j)}\langle e_{k},e_{j}\rangle
    \\
    &= \sum_{k=-n}^{n}\hat f(k)\overline{\hat g(k)}.
  \end{align*}
  Með Cauchy-Schwarz fæst:
  \begin{align*}
    \left|
      \langle f,g\rangle
      - \sum_{k=-n}^{n} \hat f(k) \overline{\hat g(k)}
    \right|
    &= \left|
      \langle f,g\rangle
      - \langle s_{n}, t_{n}\rangle
    \right|
    \\
    &\le | \langle f - s_{n},g \rangle |
    + | \langle s_{n},g - t_{n}\rangle |
    \\
    &\le \| f - s_{n}\|_{2}
    \cdot \| g\|_{2} \cdot \| s_{n} \|_{2}
    \cdot \| g - t_{n}\|_{2}
    \\
    &\le \| f - s_{n}\|_{2}
    \cdot \| g\|_{2} \cdot \| f \|_{2}
    \cdot \| g - t_{n}\|_{2}
    \\
    &\xrightarrow[n\to+\infty]{}0.
  \end{align*}
  Því er 
  \[
  \langle f,g\rangle
  = \lim_{n\to +\infty}\sum_{k=-n}^{n}\hat f(k)\overline{\hat g(k)},
  \]
  sem gefur fyrri jöfnuna, sú seinni fæst með því að setja $g = f$.
\end{proof}
Tökum saman:
\begin{setn}
  [Meginsetning um Fourier-raðir ferningsheildanlegra falla]
  Fourier-röð falls í $\mathcal L_{\C}^{2}(\mathbb T)$ er samleitin og
  stefnir á $f$ í $L_{\C}^{2}(\mathbb T)$. Vörpunin 
  \[
  T : L_{\C}^{2}(\mathbb T) \to\ell_{\C}^{2},
  [f]\mapsto (\hat f(k))_{k\in\Z}
  \]
  er línuleg og \emph{gagntæk} og varðveitir innfeldi, þ.e. 
  \[
  \langle T(f),T(g)\rangle_{\ell^{2}_{\C}}
  = \langle f,g\rangle_{L^{2}_{\C}(\mathbb T)},
  \]
  m.ö.o. er hún \emph{einsmótun Hilbert-rúma}.
\end{setn}
\begin{ath}
  Látum $f\in\mathcal L_{\C}^{2}(\mathbb T)$ og
  $s_{n}:=\sum_{k=-n}^{n}\hat f(k)e_{k}$. Skv. setningu stefnir
  $s_{n}$ á $f$ í $L_{\C}^{2}(\mathbb T)$,
  þ.e. $\lim_{n\to+\infty}\left\| f - s_{n} \right\|_{2} = 0$. En hvað
  þýðir það? Ef við lítum á sönnunina á að $L_{\C}^{2}(\mathbb T)$ sé
  fullkomið, þá fáum við: Til er hlutruna í $(s_{n})$ sem stefnir á
  $f$ næstum allsstaðar. 
\end{ath}
Það var ekki fyrr en árið 1966 sem sanna tókst:
\begin{setn}
  [Carleson]\index{Carleson}
  Ef $f\in\mathcal L_{\C}^{2}(\mathbb T)$, þá er
  $\lim_{n\to+\infty}s_{n}=f$ næstum allsstaðar.
\end{setn}
Tveimur árum seinna kom alhæfing:
\begin{setn}
  [R. Hunt]\index{Hunt, R.}  Ef $1 < p < +\infty$ og $f\in\mathcal
  L_{\C}^{p}(\mathbb T)$, þá er $\lim_{n\to+\infty}s_{n}=f$ næstum
  allsstaðar.
\end{setn}
Fyrir 1966 var ekki einu sinni vitað hvort þetta gilti fyrir samfelld
föll. Eftirfarandi er oft þægilegt:
\begin{setn}
  Ef $f$ er lotubundið með lotu $2\pi$, samfellt og samfellt
  deildanlegt á köflum, þá er Fourier-röð þess alsamleitin í jöfnum
  mæli á $\R$ og stefnir á $f$ í hverjum punkti.
\end{setn}
\begin{ath}
  Forsendan þýðir að til sé skipting
  $-\pi=a_{0}<a_{1}<\cdots<a_{n}=\pi$ á $[-\pi,\pi]$
  þ.a. $f|_{\left]a_{k-1},a_{k}\right[}$ sé deildanlegt með samfellda
  afleiðu þ.a. $\lim_{x\to a_{k-1}+}f'(x)$ og
  $\lim_{x\to{a_{k}-}}f'(x)$ séu til.
\end{ath}
\begin{proof}
  [Sönnun á setningunni]
  Föllin $f,f'$ eru augljóslega bæði í $\mathcal L^{2}(\mathbb
  T)$. Vegna 
  \[
  \frac{d}{dx}(f(x)e^{-inx})
  = f'(x)e^{-inx} - in\,f(x)e^{-inx}
  \]
  fæst með heildun frá $-\pi$ til $\pi$ að 
  \[
  \hat f'(n)
  - in\,\hat f(n)
  = \left[ f(x) e^{-inx} \right]_{-\pi}^{\pi}
  = 0
  \]
  sem er jafngilt 
  \[
  in\,\hat f(n) = \hat f'(n).
  \]
  Vegna $f'\in\mathcal L_{\C}^{2}(\mathbb T)$ er 
  \[
  \sum_{n=-\infty}^{+\infty}\left|
    \hat f'(n)
  \right|^{2}
  < +\infty.
  \]
  Cauchy-Schwarz fyrir endanlegar summur gefur
  \begin{align*}
    \sum_{k=-n}^{n}\left|\hat f(k)\right|
    &= \left| \hat f(0) \right|
    + \sum_{\substack{k=-n\\k\ne0}}^{n}\frac{1}{k}\left|
      \hat f'(k)
    \right|
    \\
    &\le \left| \hat f(0) \right|
    + \left(
      \sum_{\substack{k=-n\\k\ne0}}^{n}
      \frac{1}{k^{2}}
    \right)^{1/2}
    \left(
      \sum_{\substack{k=-n\\k\ne0}}^{n}
      \left|
        \hat f'(k)
      \right|^{2}
    \right)^{1/2}
    \\
    &\le C < +\infty
  \end{align*}
  þar sem $C$ er fasti óháður $n$, svo að 
  \[
  \sum_{n\in\Z} \left|
    \hat f(n)
  \right| < +\infty.
  \]
\end{proof}
Setningin verður þá afleiðing af:
\begin{setn}
  Ef $f$ er samfellt og $\sum_{n=-\infty}^{+\infty}\left| \hat f(n)
  \right| < +\infty$, þá er 
  \[
  f(x) = \sum_{n=-\infty}^{+\infty}\hat f(n)e^{-inx}
  \]
  fyrir öll $x\in\R$.
\end{setn}
\begin{proof}
  Skilyrðið þýðir að Fourier-röð $f$ er alsamleitin í jöfnum mæli á
  $\R$ og skilgreinir samfellt fall $g$ á $\R$; $g$ hefur sömu
  Fourier-stuðla og $f$, því er $g=f$ næstum allsstaðar; þar eð bæði
  eru samfelld er $f=g$.
\end{proof}
%%
%%
\chapter{Margfeldi málrúma}
\marginpar{25. mars}

Látum $\mathcal A$ vera mengjaalgebru á mengi $X$ og $\mu$ vera innihald
á $\mathcal A$ þannig að gildi
\begin{quote}
  Ef $(A_k)_{k\in\N}$ er sundurlæg runa af mengjum sem eru stök í
  $\mathcal A$ þannig að $\bigcup_{k\in\N}A_k\in\mathcal A$, þá er
  $\mu\left( \bigcup_{k\in\N}A_k \right)=\sum_{k=0}^{+\infty}\mu(A_k)$.
\end{quote}
Þá er $\mu$ oft kallað \emph{formál}\index{formal@formál} á $\mathcal
A$. Sáum í dæmi 12 á dæmablaði 3: Ef $\mu$ er formál og $\mu^*$ er ytra
málið sem $\mu$ skilgreinir, þá er sérhvert stak úr $\mathcal A$
mælanlegt m.t.t. $\mu^*$ og $\mu^*(A)=\mu(A)$ fyrir öll $A$ úr $\mathcal
A$. 

Látum $\mu$ vera formál á algebru $\mathcal A$, $\mathcal A^\sigma$ vera
$\sigma$-algebruna sem $\mathcal A$ spannar, $\mathcal C$ vera
$\sigma$-algebru allra mælanlegra mengja m.t.t. ytra málsins $\mu^*$, þá
sjáum við að $\mathcal A\subset\mathcal C$ og því $\mathcal
A^\sigma\subset\mathcal C$.
\begin{skilgr}
  Látum $\mathcal A$ vera mengi af hlutmengjum í mengi $X$ og
  $\mu:\mathcal A\to[0,+\infty]$ vera vörpun. Við segjum að $X$ sé
  \emph{$\sigma$-endanlegt}\index{sigma-endanlegt@$\sigma$-endanlegt}
  m.t.t. $\mu$ ef til er runa $(A_k)_{k\in\N}$ af stökum í $\mathcal
  A$ þ.a. $\mu(A_k)< +\infty$ fyrir öll $k$ og $X = \bigcup_{k\in\N}
  A_k$.  Málrúm $(X,\mathcal A,\mu)$ kallast
  \emph{$\sigma$-endanlegt}\index{sigma-endanlegt@$\sigma$-endanlegt}
  ef $X$ er $\sigma$-endanlegt m.t.t. $\mu$; eins fyrir formál.
\end{skilgr}
\begin{daemi}
  Lebesgue-málið á $\R^n$ og talningarmálið á $\Z$ eru
  $\sigma$-endanleg en talningarmálið á $\R$ er það ekki.
\end{daemi}
\begin{setn}
  Látum $\mathcal A$ vera mengjaalgebru á $X$ og $\mu:\mathcal
  A\to[0,+\infty]$ vera formál þ.a. $X$ sé $\sigma$-endanlegt m.t.t.
  $\mu$, þá er $\mu^*|\mathcal A^\sigma$ eina málið á $\mathcal
  A^\sigma$ sem einskorðað við $\mathcal A$ gefur $\mu$; og
  $(X,\mathcal C,\mu^*|\mathcal C)$ er fullkomnun $(X,\mathcal
  A^\sigma,\mu^*|\mathcal A^\sigma)$. (Hér er $\mathcal C$
  $\sigma$-algebra mælanlegu mengjanna m.t.t. ytra málsins $\mu^*$ sem
  $\mu$ skilgreinir).
\end{setn}
\begin{proof}
  Látum $\mathcal D$ vera $\mathcal A^\sigma$ eða $\mathcal C$ og $\nu$
  vera mál á $\mathcal D$ þ.a. $\nu|\mathcal A = \mu$. Viljum sýna að
  $\nu = \mu^*|\mathcal D$. Látum $(X_j)_{j\in\N}$ vera vaxandi runu af
  stökum í $\mathcal A$ þ.a. $\mu(X_j) < +\infty$ fyrir öll $j$ og $X =
  \bigcup_{j\in\N} X_j$. Látum $j$ vera fast, $A\in\mathcal D$ og
  $(A_k)_{k\in\N}$ vera vaxandi runu í $\mathcal A$ þ.a. $A\cap
  X_j\subset\bigcup_{k\in\N} A_k$. Þá er 
  \[
  \nu(A\cap X_j)
  \le \nu \left( \bigcup_{k\in\N} A_k \right)
  \le \sum_{k=0}^{+\infty}\nu(A_k)
  = \sum_{k=0}^{+\infty}\mu(A_k)
  \]
  þannig að $\nu(A\cap X_j)\le\mu^*(A\cap X_j)\mu^*(X_j) = \mu(X_j) <
  +\infty$, þetta gildir eins fyrir $A^C$, svo að 
  \[
  \nu(A\cap X_j) + \nu(A^C\cap X_j)
  = \nu(X_j)
  = \mu^*(X_j)
  = \mu^*(A\cap X_j) + \mu^*(A^C\cap X_j).
  \]
  En þá er $\nu(A\cap X_j) = \mu^*(A\cap X_j)$ (ef $a,b,c,d\in\R$, $0\le
  a \le c$ og $0\le b\le d$ og $a+b=c+d$, þá er $a=c$ og $b=d$). En þá
  er 
  \[
  \nu(A)
  = \nu\left( \bigcup_{j\in\N} (A\cap X_j) \right)
  = \lim_{j\to+\infty}\nu(A\cap X_j)
  = \lim_{j\to+\infty}\mu^*(A\cap X_j)
  = \mu^*(A)
  \]
  svo að $\nu = \mu^*|\mathcal D$. Látum nú $\overline{\mathcal
  A^\sigma}$ vera fullkomnun $\mathcal A^\sigma$ m.t.t. $\mu^*|\mathcal
  A^\sigma$. Þar sem $\mathcal C$ er fullkomin $\sigma$-algebra m.t.t.
  $\mu^*|\mathcal C$ er næsta ljóst að $\overline{\mathcal
  A^\sigma}\subset\mathcal C$. Þurfum að sýna að $\mathcal
  C\subset\overline{\mathcal A^\sigma}$ (það þýðir: Látum $A\in\mathcal
  C$, þá eru til mengi $E,F\in\mathcal A^\sigma$ þannig að $E\subset
  A\subset F$ og $\mu^*(F\setminus E) = 0$). Gerum fyrst ráð fyrir að
  $\mu^*(A) < +\infty$. Fyrir sérhvert $n\in\N$ er til runa
  $(A_{nk})_{k\in\N}$ í $\mathcal A$ þ.a. fyrir $B_n :=
  \bigcup_{k\in\N} A_{nk}$ gildi $A\subset B_n$ og 
  \[
  \mu^*(B_n)
  \le \sum_{k=0}^{+\infty}\mu(A_{nk})
  \le \mu^*(A) + \frac{1}{n+1}
  < +\infty.
  \]
  Fyrir $B = \bigcap_{n\in\N}B_n$ er $A\subset B, B\in\mathcal A^\sigma$
  og 
  \[
  \mu^*(A)
  \le \mu^*(B)
  \le \limsup_{n\to+\infty}\mu^*(B_n)
  \le \mu^*(A).
  \]
  Þá er 
  \[
  \mu^*(B\setminus A) 
  = \mu^*(B) - \mu^*(A)
  = 0
  \]
  því að $A,B$ eru bæði mælanleg. En þá er $B\setminus
  A\in\overline{\mathcal A^\sigma}$ og því $A = B(\setminus B\setminus
  A)\in\overline{\mathcal A^\sigma}$. Ef $\mu^*(A) = +\infty$ finnum við
  runu $(X_j)$ í $\mathcal A$ þannig að $X = \bigcup_{j\in\N}X_j$ og
  $\mu(X_j) < +\infty$. Þá er $A\cap X_j \in\overline{\mathcal
  A^\sigma}$ skv. ofansögðu, svo að $A = \bigcup_{j\in\N}(A\cap
  X_j)\in\overline{\mathcal A^\sigma}$.
\end{proof}
Við getum skilgreint margfeldi endanlega margra málŕuma; til að spara
skriftir látum við nægja að taka tvö. Látum $X$ og $Y$ vera mengi, $A$
og $C$ vera hlutmengi í $X$ og $B$ og $D$ vera hlutmengi í $Y$. Þá er 
\[
(A\times B) \cap (C\times D) 
= (A\cap C)\times(B\cap D)
\]
og 
\[
(A\times B)^C
= (X\times B^C)\cup (A^C\times B).
\]
Af þessu leiðir: Ef $\mathcal A$ er mengjaalgebra á $X$ og $\mathcal B$
er mengaalgebra á $Y$, þá mynda öll sammengi 
\[
\bigcup_{j=1}^k (A_j\times B_j)
\]
þa rsem $A_j\in\mathcal A$ og $B_j\in\mathcal B$ mengjaalgebru á
$X\times Y$. Við getum valið $A_j\times B_j$ þannig að þau verði
sundurlæg tvö og tvö. Þetta er ljóst ef $k=1$. Ef $A_1\times
B_1,\dots,A_k\times B_k$ eru sundurlæg tvö og tvö, þá er 
\begin{align*}
  \bigcup_{j=0}^{k+1}(A_j\times B_j)
  &= \bigcup_{j=1}^k ( (A_j\times B_j) \setminus (A_{k+1}\times
  B_{k+1})) \cup (A_k \times B_{k+1})
  \\
  &= \left(
    \bigcup_{j=1}^k 
    ((A_j\cap A_{k+1})\times(B_j\setminus B_{j+1}) 
  \right)
  \\
  &\quad \cup \left( 
    \bigcup_{j=1}^k ( (A_j\setminus A_{k+1})\times B_j)
  \right)
  \cup (A_{k+1}\times B_{k+1}),
\end{align*}
þessi sammengi mynda minnstu mengjaalgebru sem inniheldur margfeldi
$A\times B$, þar sem $A\in\mathcal A$ og $B\in\mathcal B$.
%%
%% 
\marginpar{28. mars}
%%
%%
Sönnuðum síðast:
\begin{lemma}
  Látum $(X,\mathcal A,\mu)$ og $(Y,\mathcal B,\nu)$ vera málrúm og
  $(A_{i}\times B_{i})_{i\in\N}$ og $(C_{j}\times D_{j})_{j\in
    \{1,\dots,k\}}$ vera sundurlægar fjölskyldur
  þ.a. $A_{i},C_{j}\in\mathcal A$ og $B_{i},D_{j}\in\mathcal B$ fyrir
  öll $i$ og $j$ og
  \[
  \bigcup_{i\in\N} A_{i}\times B_{i}
  = \bigcup_{j=1}^{k} C_{j}\times D_{j}.
  \]
  Þá er 
  \[
  \sum_{i=1}^{+\infty}\mu(A_{})\nu(B_{i})
  = \sum_{j=1}^{k}\mu(C_{j})\nu(D_{j}).
  \]
\end{lemma}
\begin{ath}
  Hér, \emph{eins og alltaf}, er $0\cdot(+\infty) = (+\infty)\cdot 0 = 0$.
\end{ath}
\begin{skilgr}
  \begin{enumerate}[(1)]
  \item Látum $(X,\mathcal A)$ og $(Y,\mathcal B)$ vera mælanleg
    rúm. Við táknum með 
    \[
    \mathcal A \otimes \mathcal B
    \]
    $\sigma$-algebruna sem er spönnuð af menginu
    \[
    \mathcal E
    := \{ A\times B : A\in\mathcal A, B\in\mathcal B \}.
    \]
  \item Látum $(X,\mathcal A,\mu)$ og $(Y,\mathcal A,\nu)$ vera
    málrúm, látum $\mathcal E$ vera eins og í (1) og skilgreinum
    vörpun $v:\mathcal E\to[0,+\infty]$ með
    \[
    v(A\times B) := \mu(A)\times\nu(B).
    \]
    Látum $\kappa^{*}$ vera ytra málið á $X\times Y$ sem $v$ gefur af
    sér, þ.e. 
    \[
    \kappa^{*}(E) := \inf
    \left\{
      \sum_{i=0}^{+\infty}\mu(A_{i})\nu(B_{i})
      : A_{i}\in \mathcal A,
      B_{i}\in\mathcal B
      \text{ fyrir öll }i\in\N
    \right\}.
    \]
    Látum $\mathcal C$ vera $\sigma-algebru$ allra
    $\kappa^{*}$-mælanlegra mengja og setjum 
    \[
    \mu\times\nu := \kappa^{*}\mid\mathcal C;
    \]
    við köllum $\mu\times\nu$
    \emph{Carathéodory-margfeldi}\index{Caratheodory@Carathéodory!margfeldi}
    málanna $\mu$ og $\nu$ og $(X\times Y,\mathcal C,\mu\times\nu)$
    \emph{Carathéodory-margfeldi} málrúmanna $(X,\mathcal A,\mu)$ og
    $(Y,\mathcal B,\nu)$.
  \end{enumerate}
\end{skilgr}
\begin{setn}
  Látum $(X\times Y,\mathcal C,\mu\times\nu)$ vera
  Carathéodory-margfeldi málrúmanna $(X,\mathcal A,\mu)$ og
  $(Y,\mathcal B,\nu)$. Þá er 
  \[
  \mathcal A\otimes \mathcal B\subset \mathcal C
  \]
  og 
  \[
  (\mu\times \nu)(A\times B)
  = \mu(A)\nu(B)
  \]
  fyrir öll $A\in\mathcal A$ og $B\in\mathcal B$.
\end{setn}
\begin{proof}
  Höfum séð að minnsta mengjaalgebran sem inniheldur öll mengi
  $A\times B$ með $A\in\mathcal A$ og $B\in\mathcal B$ samanstendur af
  öllum sammengjum 
  \[
  E = \bigcup_{j=1}^{k}A_{j}\times B_{j}
  \]
  þar sem $(A_{j}\times B_{j})_{j\in\{1,\dots,k}$ er \emph{sundurlæg}
  fjölskylda af mengjum þ.a. $A_{j}\in\mathcal A$ og $B_{j}\in\mathcal
  B$. Fyrir slíkt mengi $E$ setjum við 
  \[
  w(E) := \sum_{j=1}^{k}\mu(A_{j})\nu(B_{j}).
  \]
  Hjálparsetningin sýnir í fyrsta lagi að þessi skilgreining er óháð
  valinu á sundurlægu fjölskyldunni $(A_{j}\times
  B_{j})_{j\in\{1,\dots,k\}}$; og í öðru lagi að er $w$ er
  \emph{formál} á mengjaalgebrunni sem $\mathcal E = \{A\times B :
  A\in\mathcal A, B\in\mathcal B \}$ spannar. Þá segir setning að ytra
  maĺið $\kappa^{*}$ sem $w$ skilgreinir fullnægi $\kappa^{*}(E)=w(E)$
  fyrir $E$ úr algebrunni, og að $\mathcal C$ inniheldur
  $\sigma$-algebruna sem algebran spannar, en það er $\mathcal
  A\otimes\mathcal B$.
\end{proof}
\begin{setn}
  Ef málrúm $(X,\mathcal A,\mu)$ og $(Y,\mathcal B,\nu)$ eru
  $\sigma$-endanleg, þá er $\mu\times\nu\mid \mathcal A\otimes
  \mathcal B$ \emph{eina} málið á $\mathcal A\otimes\mathcal B$ þannig
  að $(\mu\times\nu)(A\times B)=\mu(A)\nu(B)$ og
  $\mu\times\nu:\mathcal C\to[0,+\infty]$ er fullkomnun
  $\mu\times\nu\mid\mathcal A\otimes\mathcal B$.
\end{setn}
\begin{proof}
  Þetta er bein afleiðing af síðustu setningu og setningu úr síðasta
  fyrirlestri, því að formálið á $X\times Y$ verður líka
  $\sigma$-endanlegt: Ef $X=\bigcup_{j\in\N} X_{j}$,
  $Y=\bigcup_{j\in\N}Y_{j}$, þar sem $X_{j}\in\mathcal A$
  $Y_{j}\in\mathcal B$, $\mu(X_{j})<+\infty$, $\mu(Y_{j})<+\infty$, þá
  er $X\times Y = \bigcup_{j\in\N}(X_{j}\times Y_{j})$ og
  $v(X_{i}\times Y_{j})=\mu(X_{j})\nu(Y_{j})<+\infty$.
\end{proof}
\begin{ath}
  (1) Sumir höfundar láta $\mu\times\nu$ vera það sem við skrifum um
  $\mu\times\nu\mid\mathcal A\otimes\mathcal B$. Með okkar rithætti er
  ljóst að fyrir Lebesgue-málin $\lambda_{n}$ á $\R^{n}$ og
  $\lambda_{m}$ á $\R^{m}$ er
  \[
  \lambda_{n}\times\lambda_{m} = \lambda_{n+m}.
  \]

  (2) Ef við höfum þrjú $\sigma$-endanleg málrúm $(X_{k},\mathcal
  A_{k},\mu_{k})$, $k=1,2,3$, þá er 
  \[
  (\mathcal A_{1}\otimes\mathcal A_{2})\otimes\mathcal A_{3}
  = \mathcal A_{1}\otimes(\mathcal A_{2}\otimes\mathcal A_{3}),
  \]
  því að hvort tveggja er $\sigma$-algebran spönnuð af mengjum
  $A_{1}\times A_{2}\times A_{3}$ með $A_{k}\in\mathcal A_{k}$ og
  $(\mu_{1}\times\mu_{2})\times\mu_{3} =
  \mu_{1}\times(\mu_{2}\times\mu_{3})$.
\end{ath}
\begin{skilgr}
  Látum $X$ vera mengi. Mengi $\mathcal B$ af hlutmengjum í $X$
  kallast \emph{einhalla (mengja)flokkur}\index{einhalla
    (mengja)flokkur} ef eftirfarandi tveimur ksilyrðum er fullnægt:
  \begin{enumerate}[(i)]
  \item Ef $(B_{k})_{k\in\N}$ er vaxandi runa af mengjum í $\mathcal
    B$, þá er $\bigcup_{k\ni\N}B_{k}\in\mathcal B$.
  \item Ef $(B_{k})_{k\in\N}$ er fallandi runa af mengjum í $\mathcal
    B$, þá er $\bigcap_{k\in\N}B_{k}\in\mathcal B$.
  \end{enumerate}
\end{skilgr}
\begin{lemma}
  [Um einhalla flokka]
  Látum $\mathcal A$ vera mengjaalgebru á mengi $X$. Minnsti einhalla
  flokkur sem inniheldur $\mathcal A$ er $\sigma$-algebran sem
  $\mathcal A$ spannar.
\end{lemma}
\begin{ath}
  Ef $\mathcal E\subset\mathcal P(X)$, þá er til minnsti einhalla
  flokkur sem inniheldur $\mathcal E$, þ.e. 
  \[
  \bigcap\{ \mathcal B : \mathcal B\text{ er einhalla flokkur og
  }\mathcal E \subset \mathcal B \}.
  \]
\end{ath}
\begin{proof}
  [Sönnun á hjálparsetningu]
  Látum $\mathcal A^{\sigma}$ vera $\sigma$-algebruna sem $\mathcal A$
  spannar. Þá er $\mathcal A^{\sigma}$ einhalla flokkur því að sérhver
  $\sigma$-algebra er það ljóslega; því fæst 
  \[
  \mathcal B\subset \mathcal A^{\sigma}
  \]
  þar sem $\mathcal B$ er minnsti einhalla flokkur sem inniheldur
  $\mathcal A$. Fyrir $A\in\mathcal B$ setjum við 
  \[
  \mathcal B_{A} := \{ B\in \mathcal B
  : A\setminus B, B\setminus A, A\cap B\in \mathcal B
  \}.
  \]
  Við höfum $\emptyset,A\in\mathcal B_{A}$, og $\mathcal B_{A}$ er
  einhalla flokkur. Höfum 
  \[
  B\in \mathcal B_{A}
  \quad\text{þ.þ.a.a.}\quad
  A\in\mathcal B.
  \]
  Ef $A\in\mathcal A$, þá er $\mathcal A\subset\mathcal B_{A}$, því að
  $\mathcal A$ er algebra. En skv. skilgreiningu á $\mathcal B$ er þá
  $\mathcal B\subset\mathcal B_{A}$. Þetta sýnir: Ef $A\in\mathcal A$
  og $B\in\mathcal B$, þá er $B\in\mathcal B_{A}$, svo að
  $A\in\mathcal B_{B}$. Þa rmeð er $\mathcal A\subset\mathcal B_{B}$
  fyrir öll $B$ úr $\mathcal B$. En þá er $\mathcal B\subset\mathcal
  B_{B}$ fyrir öll $B$ úr $\mathcal B$, þ.e.
  \begin{quote}
    Ef $A,B\in\mathcal B$, þá er $A\setminus B\in\mathcal B$ og $A\cap
    B\in\mathcal B$.
  \end{quote}
  En þar með er $\mathcal B$ mengjaalgebra. Mengjaalgebra sem er lokuð
  m.t.t. vaxandi sammengja er $\sigma$-algebra. Vegna $\mathcal
  A\subset\mathcal B$ og $\mathcal B$ er $\sigma$-algebra, er
  $\mathcal A^{\sigma}\subset\mathcal B$. Og því er $\mathcal
  A^{\sigma}=\mathcal B$.
\end{proof}
\begin{lemma}
  Látum $(X,\mathcal A)$ og $(Y,\mathcal B)$ vera mælanleg rúm og
  $f:X\times Y\to[0,+\infty]$ vera mælanlegt fall m.t.t. $\mathcal
  A\otimes\mathcal B$. Fyrir sérhvert $x\in X$ er fallið 
  \[
  f_{x} : Y\to[0,+\infty], y\mapsto f(x,y)
  \]
  mælanlegt m.t.t. $\mathcal B$, og fyrir sérhvert $y$ úr $Y$ er
  fallið 
  \[
  f^{y}:X\to[0,+\infty],x\mapsto f(x,y)
  \]
  mælanlegt m.t.t. $\mathcal A$.
\end{lemma}
\begin{proof}
  Vegna samhverfu nægir að sanna fyrri fullyrðinguna. Fyrir $x\in X$
  og $E\subset X\times Y$ setjum við
  \[
  E_{x} := \{ y\in Y : (x,y)\in E \}.
  \]
  Nú er ljóst að 
  \[
  \mathcal E := \{ E\subset X\times Y : E_{x} \in\mathcal B\text{
    fyrir öll }x\in X \}
  \]
  er $\sigma$-algebra þ.a. $A\times B\in\mathcal E$ fyrir öll
  $A\in\mathcal A$ og $B\in\mathcal B$. Því er $\mathcal
  A\otimes\mathcal B\subset\mathcal E$, svo að við höfum: Ef
  $E\in\mathcal A\otimes\mathcal B$, þá er $E_{x}\in\mathcal B$ fyrir
  öll $x\in X$. Ef nú $f:X\times Y\to[0,+\infty]$ er $\mathcal
  A\otimes\mathcal B$-mælanlegt og $a\in[0,+\infty]$, þá er 
  \[
  E := \{ (x,y)\in X\times Y : f(x,y) < a \}
  \in\mathcal A\otimes\mathcal B
  \]
  svo að 
  \[
  \{ y\in Y : f_{x}(y) < a \}
  = E_{x} \in\mathcal B
  \]
  fyrir öll $a\in [0,+\infty]$ og $x\in X$, svo að $f_{x}$ er
  mælanlegt fyrir öll $x\in X$.
\end{proof}
%%
%%
\marginpar{1. apríl}
%%
%%
Látum $(X,\mathcal A,\mu)$ og $(Y,\mathcal B, \nu)$ vera málrúm og
$(X\times Y,\mathcal C, \mu\times\nu)$ vera margfeldi þeirra. Við
setjum
\[
\mu\otimes\nu := \mu\times\nu|\mathcal A\otimes\mathcal B.
\]
Þá er $\mu\times\nu$ fullkomnun málsins $\mu\otimes\nu$. 
\begin{setn}[Tonelli-setning, fyrri gerð]
  \index{Tonelli-setning!fyrri gerd@fyrri gerð}
  Látum $(X,\mathcal A,\mu)$ og $(Y,\mathcal B,\nu)$ vera
  $\sigma$-endanleg málrúm og $f:X\times Y\to[0,+\infty]$ vera
  $\mathcal A\otimes \mathcal B$-mælanlegt. Þá gildir
  \begin{enumerate}[(1)]
  \item Fallið $X\to[0,+\infty],x\mapsto\int_{Y}f_{x}\,d\nu =
    \int_{Y}f(x,y)\,d\nu(y)$ er $\mathcal A$-mælanlegt og fallið
    $Y\to[0,+\infty], y\mapsto\int_{Y}f_{y}\,d\mu =
    \int_{X}f(x,y)\,d\mu(x)$ er $\mathcal B$-mælanlegt.
  \item Við höfum
    \begin{align*}
      \int_{X\times Y}f\,d(\mu\otimes\nu)
      &= \int_{X}\left(\int_{Y}f(x,y)\,d\nu(y)\right)d\mu(x)
      \\
      &= \int_{Y}\left(\int_{X}f(x,y)\,d\mu(x)\right)d\nu(y).
    \end{align*}
  \end{enumerate}
\end{setn}
\begin{proof}
  Látum $\mathcal T$ vera mengi allra $\mathcal A\otimes\mathcal
  B$-mælanlegra falla $f$ þ.a. fullyrðingarnar (1) og (2) séu
  sannar. Af setningu um einhalla samleitni leiðir: Ef
  $(f_{k})_{k\in\N}$ er vaxandi runa í $\mathcal T$, þá er $\lim_{k\to
    +\infty}f_{k}\in\mathcal T$. Því nægir í fyrsta lagi að sanna
  setninguna í því tilviki að $\mu(x) < +\infty, \mu(Y) < +\infty$;
  því að í almenna tilvikinu eru til vaxandi runur $(X_{n})$ af
  $\mathcal A$-mælanlegum mengjum og $(Y_{n})$ af $\mathcal
  B$-mælanlegum mengjum þ.a. $X = \bigcup_{n\in\N}X_{n}$,
  $Y=\bigcup_{n\in\N} Y_{n}$ og $\mu(X_{n}) < +\infty, \nu(Y_{n}) <
  +\infty$ fyrir öll $n$; en þá er $f$ markgildi vaxandi rununnar
  $(f\chi_{X_{n}\times Y_{n}})_{n\in\N}$. Í öðru lagi nægir að athuga
  einföld mælanleg föll, því að sérhvert mælanlegt fall $X\times
  Y\to[0,+\infty]$ er markgildi vaxandi runu af einföldum mælanlegum
  föllum $X\times Y\to[0,+\infty]$. En þar sem heildið er línulegt og
  einfalt mælanlegt er línuleg samantekt af kenniföllum nægir að sanna
  setninguna fyrir kenniföll mælanlegra mengja, þ.e. fyrir
  $f=\chi_{E}$ þar sem $E\in\mathcal A\otimes\mathcal B$. Setjum
  \[
  \mathcal E :=
  \{ E\in \mathcal A\otimes\mathcal B : \chi_{E}\in\mathcal T \}.
  \]
  Ljóst er að fyrir $A\in\mathcal A$ og $B\in\mathcal B$ er $A\times
  B\in\mathcal E$: Fyrir $f=\chi_{A\times B}$ er 
  \[
  \int_{Y}f(x,y)\,d\nu(y)
  = \int_{Y}\chi_{A\times B}(x,y)\,d\nu(y)
  =
  \begin{cases}
    \nu(B) & x\in A, \\
    0      & x\notin A,
  \end{cases}
  \]
  svo að 
  \[
  x\mapsto \int_{Y}f(x,y)\,d\nu(y)
  = \nu(B)\cdot\chi_{A},
  \]
  og
  \begin{align*}
    \int_{X}\int_{Y}
    f(x,y)\,d\nu(y)d\mu(x)
    &= \mu(A)\nu(B)
    \\
    &= (\mu\otimes\nu)(A\times B)
    \\
    &= \int_{X\times Y}\chi_{A\times B}\,d\mu\otimes v.
  \end{align*}
  Eins á hinn veginn. En þá er líka ljóst að $\mathcal E$ inniheldur
  öll sammengi $\bigcup_{j=1}^{k}A_{j}\times B_{j}$ þar sem
  $(A_{j}\times B_{j})_{j\in\{1,\dots,k\}}$ er \emph{sundurlæg}
  fjölskylda af mengjum þ.a. $A_{j}\in\mathcal A$ og $B_{j}\in\mathcal
  B$ fyrir $j=1,\dots,k$. En þau mynda algebru og $\mathcal
  A\otimes\mathcal B$ er minnsta $\sigma$-algebra sem inniheldur þá
  algebru; skv. HS er $\mathcal A\otimes \mathcal B$ minnsti einhalla
  flokkur sem inniheldur þá algebru. Það nægir því að sýna að
  $\mathcal E$ sé einhalla flokkur, því að þá fæst $\mathcal
  A\otimes\mathcal B \subset\mathcal E$. Látum $(E_{k})_{k\in\N}$ vera
  vaxandi runu í $\mathcal E$ og $E:=\bigcup_{k\in\N}E_{k}$, þá er
  $(\chi_{E_{k}})$ vaxandi runa í $\mathcal T$, svo að $\chi_{E} =
  \lim_{k\to +\infty}\chi_{E_{k}}\in\mathcal T$ og því $E\in\mathcal
  E$. En ef $(E_{k})_{k\in\N}$ er fallandi runa í $\mathcal E$ og
  $E:=\bigcap_{k\in\N}E_{k}$, þá er $\chi_{X\times Y}-\chi_{E_{k}}$
  vaxandi runa með markgildi $\chi_{X\times Y}-\chi_{E}$ og þar sem 
  \[
  \int_{X\times Y}\chi_{X\times Y}\,d(\mu\otimes\nu)
  = \int_{X\times Y} 1\cdot d\mu\otimes\nu
  = \mu(X)\cdot\nu(Y)
  < +\infty,
  \]
  svo að $\chi_{E}\in\mathcal T$ og því $E\in\mathcal E$.
\end{proof}
\begin{fylgi}
  Látum $(X,\mathcal A,\mu), (Y,\mathcal B,\nu)$ vera
  $\sigma$-endanleg málrúm og $E$ vera núllmengi
  m.t.t. $\mu\otimes\nu$. Fyrir $\mu$-næstum öll $x\in X$ er $E_{x}=\{
  y\in Y : (x,y)\in E \}$ $\nu$-núllmengi og fyrir $\nu$-næstum öll
  $y\in Y$ er $E^{y} = \{ x\in X : (x,y) \in E \}$ $\mu$-núllmengi.
\end{fylgi}
\begin{proof}
  Notum Tonelli I á $\chi_{E}$; fáum
  \begin{align*}
    0
    &= \int_{X\times Y} \chi_{E}\,d\mu\otimes\nu
    \\
    &= \int_{X}\left(
      \int_{Y} \chi_{E}(x,y)\,d\nu(y)d\mu(x)
    \right)
    \\
    &= \int_{X}\int_{Y}\chi_{E_{x}}(y)\,d\nu(y)d\mu(x)
    \\
    &= \int_{X}\nu(E_{x})\,d\mu(x)
  \end{align*}
  svo að $x\mapsto\nu(E_{x})$ er núll $\mu$-næstum allsstaðar; eins
  fyrir $y\mapsto \mu(E^{y})$.
\end{proof}
\begin{ath}
  Látum $(X,\mathcal A,\mu)$ vera málrúm og $f$ vera fall sem er
  skilgreint næstum allsstaðar á $X$. Það þýðir að til er mælanlegt
  mengi $E$ í $X$ þ.a. $E^{C}$ sé núllmengi og $f$ sé skilgrient á
  $E$. Nú má tala um að $f$ sé mælanlegt eða heildanlegt eftir
  atvikum; það jafngildir því að $f$ hafi mælanlega, eða þá
  heildanlega, framlengingu $g$ á $X$. Heildið $\int_{X}g\,d\mu$, ef
  til er, er óháð framlengingunni og við táknum það einfaldlega með
  $\int_{X}f\,d\mu$.
\end{ath}
\begin{setn}
  [Tonelli-setning, seinni gerð]
  \index{Tonelli-setning!seinni gerd@seinni gerð}
  Látum $(X,\mathcal A,\mu)$ og $(Y,\mathcal B,\nu)$ vera fullkomin
  $\sigma$-endanleg málrúm og $(X\times Y,\mathcal C, \mu\times\nu)$
  vera Carathéodory-margfeldi þeirra. Látum $f:X\times
  Y\to[0,+\infty]$ vera $\mu\times\nu$-mælanlegt fall. Þá gildir:
  \begin{enumerate}[(1)]
  \item Fyrir $\mu$-næstum öll $x\in X$ er fallið
    $f_{x}:Y\to[0,+\infty],y\mapsto f(x,y)$ mælanlegt, og heildið
    $\int_{Y}f(x,y)\,d\nu(y)$ því skilgreint; og fyrir $\nu$-næstum
    öll $y$ úr $Y$ er fallið $f^{y}:X\to[0,+\infty],x\mapsto f(x,y)$
    mælanlegt og heildið $\int_{X}f(x,y)\,d\mu(x)$ því skilgreint.
  \item Vörpunin $x\mapsto\int_{Y}f(x,y)\,d\nu(y)$ sem er skilgreind
    $\mu$-næstum allsstaðar á $X$ er mælanleg og vörpunin
    $y\mapsto\int_{X}f(x,y)\,d\mu(x)$, sem er skilgreind $\nu$-næstum
    allsstaðar á $Y$ er mælanleg.
  \item Við höfum 
    \begin{align*}
      \int_{X\times Y}f\,d\mu\times\nu
      &= \int_{X}\int_{Y}f(x,y)\,d\nu(y) d\mu(x)
      \\
      &= \int_{Y}\int_{X}f(x,y)\,d\mu(x) d\nu(y).
    \end{align*}
  \end{enumerate}
\end{setn}
\begin{proof}
  Málið $\mu\times\nu$ er fullkomnun málsins $\mu\otimes\nu$. Fyrir
  sérhvert $\mu\times\nu$-mælanlegt mengi $E$ eru til
  $\mu\otimes\nu$-mælanleg mengi $F$ og $G$ þ.a. $F\subset E\subset G$
  og $\mu\otimes\nu(G\setminus F) = 0$. Þá er $\chi_{E} = \chi_{F}$
  $\mu\otimes\nu$-næstum allsstaðar, svo að fyrir
  $\mu\times\nu$-mælanlegt einfalt fall $u$ er til
  $\mu\otimes\nu$-mælanlegt einfalt fall $v$ þ.a. $u=v$
  $\mu\otimes\nu$-næstum allsstaðar. En $\mu\times\nu$-mælanlegt fall
  $X\times Y\to[0,+\infty]$ er markgildi vaxandi runu af einföldum
  $\mu\times\nu$-mælanlegum föllum og því markgildi vaxandi runu af
  $\mu\otimes\nu$-mælanlegum einföldum föllum $\mu\otimes\nu$-næstum
  allsstaðar, svo að til er $\mu\otimes\nu$-mælanlegt fall $\tilde
  f:X\times Y\to[0,+\infty]$ þ.a. $f=\tilde f$ $\mu\otimes\nu$-næstum
  allsstaðar. Af fylgisetningu við Tonelli I er fallið $f_{x}$ jafnt
  fallinu $\tilde f_{x}$ $\nu$-næstum allsstaðar; og skv. HS er
  $\tilde f_{x}$ $\nu$-mælanlegt; og þar sem $(Y,\mathcal B,\nu)$ er
  fullkmoið er $f_{x}$ líka $\nu$-mælanlegt. Eins er $\tilde f^{y}$
  $\mu$-mælanlegt, og niðurstaðan fæst með Tonelli I.
\end{proof}
%%
%%
\marginpar{4. apríl}
%%
%%
\begin{fylgi}
  Látum $(X,\mathcal A,\mu)$ og $(Y,\mathcal B,\nu)$ vera fullkomin
  $\sigma$-endanleg málrúm og $f:X\times Y\to\C$ vera
  $\mu\times\nu$-mælanlegt. Ef
  \[
  \int_{X}\left(\int_{Y} |f(x,y)|\,d\nu(y)\right)\,d\mu(x) < +\infty,
  \]
  þá er $f$ heildanlegt m.t.t. $\mu\times\nu$.
\end{fylgi}
\begin{proof}
  Skv. Tonelli II er $\int_{X\times
    Y}|f(x,y)|\,d(\mu\times\nu)<+\infty$ svo að $|f|$ er heildanlegt
  og þá $f$ líka m.t.t. $\mu\times\nu$.
\end{proof}
\begin{setn}
  [Fubini]\index{Fubini-setningin}
  
  Látum $(X,\mathcal A,\mu)$ og $(Y,\mathcal B,\nu)$ vera fullkomin
  $\sigma$-endanleg málrúm og $f$ vera $\mu\times\nu$-heildanlegt fall
  á $X\times Y$. Þá gildir:
  \begin{enumerate}[(1)]
  \item Fyrir $\mu$-næstum öll $x\in X$ er fallið $f_{x}:y\mapsto
    f(x,y)$ $\nu$-heildanlegt og fallið
    $x\mapsto\int_{Y}f(x,y)\,d\nu(y)$, sem er $\mu$-næstum allsstaðar
    skilgreint, er $\mu$-heildanlegt yfir $X$.
  \item Fyrir næstum öll $y$ úr $Y$ er fallið $f^{y}:x\mapsto f(x,y)$
    $\mu$-heildanlegt og fallið $y\mapsto \int_{X}f(x,y)\,d\mu(x)$,
    sem er $\nu$-næstum allsstaðar skilgreint, er $\nu$-heildanlegt
    yfir $Y$.
  \item Við höfum 
    \[
    \int_{X\times Y}f\,d(\mu\times\nu)
    = \int_{X}\int_{Y}f(x,y)\,d\nu(y)d\mu(x)
    = \int_{Y}\int_{X}f(x,y)\,d\mu(x)d\nu(y).
    \]
  \end{enumerate}
\end{setn}
\begin{proof}
  Fyrir fall $f:X\times Y\to\R$ eða $f:X\times Y\to\tilde\R$ notum við
  Tonelli II á $f^{+}$ og $f^{-}$, fyrir fall $f:X\times Y\to\C$ notum
  við þá niðurstöðu á $\re f$ og $\im f$.
\end{proof}
\begin{ath}
  Gefa má dæmi um fall $f:X\times Y\to[0,+\infty]$ þannig að öll
  föllin $f_{x}:Y\to[0,+\infty]$, $f^{y}:X\to[0,+\infty]$ séu
  heildanleg, en $f$ sé ekki mælanlegt.
\end{ath}

\chapter{Tvinnmál}

\begin{skilgr}
  Látum $\mathcal A$ vera $\sigma$-algebru á $X$ og $\mu$ vera fall á
  $\mathcal A$ með gildum í $\tilde\R$ (eða $\R$ eða $\C$). Segjum að
  $\mu$ sé \emph{útvíkkað raunmál}\index{utvikkad raunmal@útvíkkað
    raunmál} (eða \emph{raunmál} eða
  \emph{tvinnmál})\index{raunmal@raunmál}\index{tvinnmal@tvinnmál} ef
  $\mu(\emptyset)=0$ og fyrir sundurlæga runu $(A_{k})$ af mengjum í
  $\mathcal A$ gildi 
  \begin{equation}
    \mu\left(\bigcup_{k\in\N})A_{k}\right)
    = \sum_{k=0}^{+\infty}\mu(A_{k}),
    \label{eq:raunmal}
  \end{equation}
  og við krefjumst þess líka að $\mu$ taki bara annað af gildunum
  $-\infty, +\infty$.
\end{skilgr}
\begin{ath}
  (1) Látum $\mu$ vera raunmál eða tvinnmál í $\mathcal A$ og
  $(A_{k})_{k\in\N}$ vera sundurlæga runu af mengjum í $\mathcal A$, í
  jöfnu \eqref{eq:raunmal} er vinstri hliðin óháð röðinni á mengjunum
  $A_{k}$; við fáum þá að 
  \[
  \sum_{k=0}^{+\infty}\mu(A_{\alpha(k)})
  = \sum_{k=0}^{+\infty}\mu(A_{k})
  \]
  fyrir allar gagntækar varpanir $\alpha:\N\to\N$; það jafngildir því
  að röðin $\sum_{k=0}^{+\infty}\mu(A_{k})$ sé \emph{alsamleitin}.

  (2) Ljóst er að fyrir tvö tvinnmál $\lambda,\mu$ á $\mathcal A$ og
  tvinntölur $a,b$ er $a\lambda+b\mu$, sem er skilgreint með 
  \[
  (a\lambda+b\mu)(A) := a\lambda(A) + b\mu(A),
  \]
  líka tvinnmál; og tvinnmálin mynda $\C$-línulegt rúm. Eins mynda
  raunmálin á $\mathcal A$ $\R$-línulegt rúm.

  (3) Notum orðalagið ,,almennt mál'' ef við erum að tala um útvíkkað
  raunmál, raunmál eða tvinnmál, og ,,jákvætt mál'' fyrir venjulegt
  mál, ef við viljum taka fram að við séum ekki að tala um almennt
  mál.\index{mal@mál!almennt}\index{mal@mál!jakvaett@jákvætt}
\end{ath}
\begin{skilgr}
  Látum $(X,\mathcal A)$ vera mælanlegt rúm og $\mu$ vera tvinnmál á
  $\mathcal A$. Fyrir $A\in\mathcal A$ setjum við  
  \[
  |\mu|(A)
  := \sup\left\{
    \sum_{k=0}^{+\infty} |\mu(A_{k})| : (A_{k}) \text{ er sundurlæg
      runa í } \mathcal A \text{ og } A = \bigcup_{k\in\N}A_{k}
  \right\}.
  \]
\end{skilgr}
\begin{ath}
  Þetta er ritháttur sem viðgengst allsstaðar, en er óheppilegur því
  að $|\mu|(A)$ er \emph{ekki} nauðsynlega sama og $|\mu(A)|$. Við
  höfum augljóslega 
  \[
  |\mu(A)| \le |\mu|(A).
  \]
\end{ath}
\begin{setn}
  Látum $\mu$ vera tvinnmál á $\sigma$-algebru $\mathcal A$. Þá er
  $|\mu|$ (jákvætt) mál á $\mathcal A$. Það er minnsta jákvæða málið
  $\lambda$ þ.a. $|\mu(A)|\le\lambda(A)$ fyrir öll $A$ úr $\mathcal
  A$.
\end{setn}
\begin{proof}
  Ef $\lambda$ er jákvætt mál á $\mathcal a$ og
  $|\mu(A)|\le\lambda(A)$ fyrir öll $A$ og $(A_{k})_{k\in\N}$ er
  sundurlæg runa af mengjum í $\mathcal A$
  þ.a. $\bigcup_{k\in\N}A_{k}=A$, þá er 
  \[
  \sum_{k=0}^{+\infty}|\mu(A_{k})|
  \le \sum_{k=0}^{+\infty}\lambda(A_{k})
  = \lambda(A)
  \]
  svo að
  $|\mu|(A)=\sup\left\{\sum_{k\in\N}|\mu(A_{k})|\right\}\le\lambda(A)$
  fyrir öll $A$, sem við getum skrifað $|\mu|\le\lambda$.

  Eftir er að sýna að $|\mu|$ sé mál. Ljóst er að
  $|\mu|(\emptyset)=0$. Látum $(A_{k})$ vera sundurlæga runu í
  $\mathcal A$ og $A=\bigcup_{k\in\N}A_{k}$. Veljum
  $t_{k}<|\mu|(A_{k})$ fyrir öll $k$. Þá er sundurlæg runa
  $(B_{kj})_{j\in\N}$ þ.a. $A_{k}=\bigcup_{j\in\N}B_{kj}$ og
  $\sum_{j=0}^{+\infty}|\mu(B_{kj})|>t_{k}$. Þá er
  $(B_{kj})_{(k,j)\in\N\times\N}$ teljanleg sundurlæg fjölskylda með
  sammengi $A$, svo að 
  \[
  \sum_{k=0}^{+\infty}t_{k}
  \le \sum_{k,j\in\N} |\mu(B_{kj})|
  \le |\mu|(A).
  \]
  Tökum efra mark af vinstri hlið fyrir allar slíkar fjölskyldur
  $(t_{k})$ og fáum 
  \[
  \sum_{k=0}^{+\infty}|\mu(A_{k})|
  \le |\mu|(A).
  \]
  Ójafnan í hina áttina: Látum $C_{j}$ vera sundurlæga runu þ.a. $A =
  \bigcup_{j\in\N}C_{j}$. Þá er $(A_{k}\cap C_{j})_{j\in\N}$ sundurlæg
  runa með sammengi $A_{k}$, svo að
  \begin{align*}
    \sum_{j=0}^{+\infty}|\mu(C_{j})|
    &= \sum_{j=0}^{+\infty}\left|
      \sum_{k=0}^{+\infty}\mu(C_{j}\cap A_{k})
    \right|
    \\
    &\le \sum_{j=0}^{+\infty}\sum_{k=0}^{+\infty}|\mu(C_{j}\cap{A_{k}})|
    \\
    &= \sum_{k=0}^{+\infty}\sum_{j=0}^{+\infty}|\mu(C_{j}\cap A_{k})|
    \\
    &\le \sum_{k=0}^{+\infty}|\mu|(A_{k}).
  \end{align*}
  Þar sem þetta gildir fyrir allar svona runur $(C_{j})$ fæst 
  \[
  |\mu|(A)
  \le \sum_{k=0}^{+\infty}|\mu|(A_{k}).
  \]
\end{proof}
\begin{lemma}
  Ef $z_{1},\dots,z_{n}\in\C$, þá er til hlutmengi $S$ í
  $\{1,\dots,n\}$ þ.a. 
  \[
  \left|
    \sum_{k\in S} z_{k}
  \right|
  \ge \frac 16 \sum_{k=1}^{n}|z_{k}|.
  \]
\end{lemma}
\begin{proof}
  Skiptum $\C$ upp í fjórðunga með línum $x=\pm y$; einum þeirra,
  segjum $Q = \{ x + iy : |y|\le x \}$ inniheldur stök úr $\{
  z_{1},\dots,z_{n}\}$ þ.a. summa algilda þeirra sé a.m.k. $\frac
  14t$, $t = \sum_{k=1}^{n}|z_{k}|$. Fyrir $z\in\Q$ er $\re z\ge
  \frac{|z|}{\sqrt{2}}$. Ef $S = \{k:z_{k}\in Q\}$ er 
  \[
  \left|
    \sum_{k\in S} z_{k}
  \right|
  \ge \re \sum_{j\in S} z_{j}
  \ge \frac 1{\sqrt 2} \sum_{j\in S} |z_{j}|
  \ge \frac t{4\sqrt 2}
  \ge \frac t6.
  \]
\end{proof}
%%
%%
\marginpar{8. apríl}
%%
%%
\begin{setn}
  Látum $(X,\mathcal A)$ vera mælanlegt rúm og $\mu$ vera tvinnmál á
  $\mathcal A$. Þá er $|\mu|(X)<+\infty$, m.ö.o. er $(X,\mathcal
  A,|\mu|)$ endanlegt málrúm.
\end{setn}
\begin{proof}
  Sýnum fyrst: Ef $A\in\mathcal A$, og $|\mu|(A) = +\infty$, þá má
  skrifa $A=B\cup C$ þar sem $B,C\in\mathcal A$, $B\cap C=\emptyset$,
  $|\mu(B)|\ge 1$ og $|\mu(C)|=+\infty$: Skv. skilgreiningu $|\mu|$ er
  til sundurlæg runa $(A_{k})_{k\in\N}$ í $\mathcal A$
  þ.a. $\bigcup_{k\in\N}A_{k}=A$ og $\sum_{k=0}^{+\infty}|\mu(A_{k}|$
  sé hversu stórt sem vera skal; veljum rununa
  þ.a. $\sum_{k=0}^{+\infty}|\mu(A_{k})>t := 6(1+|\mu(A)|)$. En þá er
  líka til tala $n$ þ.a. $\sum_{k=0}^{n}|\mu(A_{k})|>t$; og skv. HS er
  til hlutmengi $S$ í $\{0,1,\dots,n\}$ þ.a. $\sum_{k\in
    S}|\mu(A_{k})|>\frac t6$. Setjum $B:=\bigcup_{k\in S}A_{k}$. Þá
  er $B\in\mathcal A, B\subset A$ og $|\mu(B)|>\frac t6 \ge 1$. Setjum
  $C:=A\setminus B\in\mathcal A$. Þá er
  $|\mu(C)|=|\mu(A)-\mu(B)|\ge|\mu(B)|-|\mu(A)|>\frac t6 - |\mu(A)| =
  1$. Nú er $|\mu|$ mál, svo að $|\mu|(A)=|\mu|(B)+|\mu|(C)$, og
  $|\mu|(A)=+\infty$, svo að annaðhvort er $|\mu|(B)=+\infty$ eða
  $|\mu|(C)=+\infty$.  Ef $|\mu|(C)<+\infty$, þá skiptum við um nöfn
  og $B$ og $C$ og fáum niðurstöðuna.

  Gerum nú ráð fyrir að $|\mu|(X) = +\infty$. Þá má skrifa
  $X=B_{0}\cup C_{0}$ þar sem $B_{0},C_{0}\in\mathcal A$,
  $|\mu(B_{0})|\ge 1$ og $|\mu|(C)=+\infty$. Með þrepun fást runur
  $(B_{k})$ og $(C_{k})$ af stökum í $\mathcal A$ þannig að 
  \[
  C_{n} = B_{n+1}\cup C_{n+1},
  \quad
  B_{n+1}\cap C_{n+1} = \emptyset,
  \quad
  |\mu(B_{n+1})|\ge 1,
  |\mu(C_{n+1})| = +\infty.
  \]
  Þá er $(B_{k})_{k\in\N}$ sundurlæg runa í $\mathcal A$ og því 
  \[
  \mu\left(\bigcup_{k\in\N}B_{k}\right)
  = \sum_{k=0}^{+\infty}\mu(B_{k}),
  \]
  sér í lagi er þessi síðasta runa (al)samleitin í $\C$; en það er
  fráleitt, því að $|\mu(B_{k})|\ge 1$.
\end{proof}
\begin{ath}
  Af setningunni er ljóst að 
  \[
  \mu\mapsto |\mu|(X)
  \]
  er staðall á línulegu rúmi allra tvinnmála á $\mathcal A$.
\end{ath}

\section{Innskot um samfelld línuleg föll á Hilbert-rúmi}
\begin{lemma}
  Látum $H$ vera staðlað rúm yfir $\R$ eða $\C$ og $L:H\to\R$ eða $\C$
  vera línulega vörpun. Þá eru eftirtalin skilyrði jafngild:
  \begin{enumerate}[(i)]
  \item Vörpunin $L$ er samfelld í $0$.
  \item Til er fasti $C>0$ þ.a. $|L(X)|\le C\|x\|$ fyrir öll $x\in H$.
  \item Vörpunin $L$ er Libschitz-samfelld, þ.e. itl er $C>0$
    þ.a. $|L(x)-L(y)|<C\|x-y\|$ fyrir öll $x,y\in H$.
  \item $L$ er samfelld í sérhverjum punkti.
  \end{enumerate}
\end{lemma}
\begin{proof}
  LJóst er að (iv)$\Rightarrow$(i), (ii)$\Rightarrow$(iii) vegna
  $L(x)-L(y) = L(x-y)$ og að (iii)$\Rightarrow$(iv). Þá þarf bara að
  sanna (i)$\Rightarrow$(ii). Ef vörpunin $L$ er samfelld í $0$, þá er
  til $\delta>0$ þ.a. $|L(X)|<1$ fyrir öll $x\in H$
  þ.a. $|x|<\delta$. Fyrir $x\in H$, $x\ne 0$, er 
  \[
  \left\| \frac{\delta x}{2\|x\|} \right\|
  = \frac{\delta}{2}
  < \delta,
  \]
  svo að 
  \[
  1
  > \left|
    L\left(\frac{\delta x}{2\|x\|}\right)    
  \right|
  = \left|
    \frac{\delta}{2\|x\|} L(x)
  \right|
  = \frac{\delta}{2\|x\|} |L(x)|,
  \]
  svo að 
  \[
  |L(x)| \le C\|x\|,
  \]
  þar sem $C:=\frac 2\delta > 0$. Þetta gildir líka fyrir $x = 0$.
\end{proof}
\begin{lemma}
  Í Hilbert-rúmi (almennar í innfeldisrúmi) gildir
  \emph{samsíðungsójafna}:\index{samsidungsojafna@samsíðungsójafna}
  \[
  \left\|
    x+y
  \right\|^{2}
  + \left\|
    x - y
  \right\|^{2}
  = 
  2 \left\|
    x
  \right\|^{2}
  + 2 \left\|
    y
  \right\|^{2}.
  \]
\end{lemma}
\begin{proof}
  Beinir reikningar: $\left\| x + y \right\|^{2} = \langle
  x+y,x+y\rangle = \dots$
\end{proof}
\begin{lemma}
  \label{lemma:Hilbert-jadar}
  Ef $K$ er lokað línulegt hlutmengi í Hilbert-rúmi og $x\in H$, þá er
  til stak $y\in K$ þ.a. 
  \[
  \left\|
    x-y
  \right\|
  = d(x,K)
  = \inf_{z\in K} \left\|
    x-z
  \right\|.
  \]
\end{lemma}
\begin{proof}
  Látum $(y_{k})$ vera runu í $K$ þ.a. 
  \[
  \lim_{k\to +\infty}\left\|
    x-y_{k}
  \right\|
  = \delta
  := d(x,K).
  \]
  Af samsíðungsójöfnunni leiðir að 
  \begin{align*}
    \left\|
      y_{k}-y_{j}
    \right\|^{2}
    &= 2 \left\|
      x - y_{k}
    \right\|^{2}
    + 2 \left\|
      x - y_{j}
    \right\|^{2}
    - 4 \left\|
      x - \frac 12 (y_{k} + y_{j})
    \right\|^{2}
    \\
    &\le 2 \left\|
      x - y_{k}
    \right\|^{2}
    + 2 \left\|
      x - y_{j}
    \right\|^{2}
    - 4 \delta^{2}
    \\
    &\xrightarrow[j,k\to+\infty]{} 0.
  \end{align*}
  Því er $(y_{k})$ Cauchy-runa og því samleitin. Þar sem $K$ er lokað
  er $y := \lim_{k\to+\infty}y_{k}\in K$, og
  \[
  \left\|
    x - y 
  \right\|
  = \lim_{k\to +\infty} \left\|
    x - y_{k}
  \right\|
  = \delta.
  \]
\end{proof}
\begin{lemma}
  Látum $L$ vera lokað línulegt hlutrúm í Hilbert-rúmi $H$ og 
  \[
  L^{\bot}
  := \{ x\in H : \langle x,y\rangle = 0 \text{ fyrir öll } y\in L \}
  \]
  Þá er $L^{\bot}$ lokað hlutrúm í $H$ og 
  \[
  H = L\oplus L^{\bot}.
  \]
\end{lemma}
\begin{proof}
  Vegna Cauchy-Schwarz-ójöfnunnar er vörpunin $\phi_{y}:H\to\C$ (eða
  $\R$), sem gefin er með $\phi_{y}(x):=\langle x,y\rangle$ fyrir öll
  $x\in H$, línuleg og samfelld, því að $|\phi_{y}(x)|=|\langle
  x,y\rangle| \le \| x\| \cdot \| y\| = C\cdot \| x\|$, þar sem $C :=
  \| y\|$. En þá er 
  \[
  L^{\bot}
  = \bigcap_{y\in L} \Ker\phi_{y}
  \]
  sniðmengi lokaðra línulegra hlutrúma í $H$ og því lokað og línulegt
  hlutrúm í $H$. Látum $x\in H$. Skv. HS \ref{lemma:Hilbert-jadar} er
  til $y$ þ.a. $\left\| x-y \right\| = \delta := \delta(x,L)$. Setjum
  $z := x-y$. Fyrir $y_{1}\in L$ og $c\in\C$ (eða $\R$) er
  $y+cy_{1}\in L$ og því
  \begin{align*}
    0
    &\le \left\|
      x - (y+cy_{1})
    \right\|^{2}
    - \left\|
      x - y
    \right\|
    \\
    &= \left\|
      z - cy_{1}
    \right\|^{2}
    - \left\|
      z
    \right\|^{2}
    \\
    &= \langle z - cy_{1}, z - cy_{1}\rangle
    - \left\|
      z
    \right\|^{2}
    \\
    &= -c \langle y_{1},z\rangle
    -\bar c \langle z,y_{1}\rangle
    +|c|^{2}\cdot \left\|
      y_{1}
    \right\|^{2}.
  \end{align*}
  Látum nú $\lambda\in\R$ og $c = \lambda\langle z,y_{1}\rangle$ og
  fáum 
  \[
  0 \le -2\lambda \left|
    \langle z, y_{1} \rangle
  \right|^{2}
  + \lambda^{2}\left|
    \langle z,y_{1}\rangle^{2}
  \right|
  \cdot \left\|
    y_{1}
  \right\|^{2}
  \]
  fyrir öll $\lambda\in\R$, svo að $\langle z,y_{1} \rangle = 0$. En
  $y_{1}$ var hvaða stak sem vera skal í $L$, svo að $z\in L^{\bot}$
  og $x = y + z\in L+L^{\bot}$. Ef $x\in L\cap L^{\bot}$, þá er
  $\langle x,x\rangle = 0$ og því $\|x\|^{2} = 0$ og því $x = 0$. Því
  er $L\cap L^{\bot} = \{ 0\}$ og $H = L\oplus L^{\bot}$.
\end{proof}
\begin{setn}
  Látum $\phi:H\to\C$ (eða $\R$) vera línulegt og samfellt fall á
  Hilbert-rúmi (yfir $\C$ eða $\R$). Þá er til nákvæmlega eitt stak
  $y$ í $H$ þ.a.
  \[
  \phi(x) = \langle x, y\rangle
  \]
  fyrir öll $x\in H$.
\end{setn}
\begin{proof}
  Ef $\phi = 0$, þá setjum við $y = 0$. Annars er $L :=
  \Ker\phi\ne\{0\}$ og $H = L\oplus L^{\bot}$. Látum $y_{1}\in
  L^{\bot}$, $y_{1}\ne 0$; setjum $y = cy_{1}$, þar sem 
  \[
  c := \overline{\phi(y_{1})} / \| y_{1}\|^{2}.
  \]
  Fyrir $x\in H$ er þá 
  \[
  \phi\left(
    x - \frac{\phi(x)}{\phi(y_{1})} y_{1}
  \right)
  = \phi(x) - \frac{\phi(x)}{\phi(y_{1})}\phi(y_{1})
  = 0
  \]
  þ.e. $x - \frac{\phi(x)}{\phi(y_{1})}y_{1}\in L$, svo að
  \begin{align*}
    0
    &= \left\langle x - \frac{\phi(x)}{\phi(y_{1})}
      y_{1},y\right\rangle
    \\
    &= \langle x,y\rangle
    - \left\langle\frac{\phi(x)}{\phi(y_{1})}y_{1},c
      y_{1}\right\rangle
    \\
    &= \langle x,y \rangle
    \frac{\phi(x)}{\phi(y_{1})}\cdot \frac{\phi(y_{1})}{\|y_{1}\|^{2}}
    \langle y_{1},y_{1}\rangle
    \\
    &= \langle x,y\rangle - \phi(x)
  \end{align*}
  þ.e. $\phi(x) = \langle x,y\rangle$ fyrir öll $x\in H$. Ef líka
  $\phi(x) = \langle x,z\rangle$ fyrir öll $x$ þá er $\langle
  x,y-z\rangle = 0$ fyrir öll $x$, líka fyrir $x = y-z$, svo að
  $\|y-z\|=0$ og $y=z$.
\end{proof}
%%
%%
\marginpar{11. apríl}
%%
%%
[VANTAR]

(3) Segjum að almenn mál $\lambda_{1},\lambda_{2}$ séu
\emph{sundurlæg} og skrifum 
\[
\lambda_{1}\bot \lambda_{2}
\]
ef til eru sundurlæg mengi $A_{1}$ og $A_{2}$ þ.a. $\lambda_{k}$ sé
afmarkað við $A_{k}$ fyrir $k=1,2$.

Einfaldar reglur:
\begin{setn}
  Látum $\mu$ vera jákvætt mál og $\lambda,\lambda_{1},\lambda_{2}$
  vera tvinnmál á sömu $\sigma$-algebru $\mathcal A$. Þá gildir:
  \begin{enumerate}[(1)]
  \item Ef $\lambda$ er afmarkað við mengi $A$, þá er $|\lambda|$ líka
    afmarkað við $A$.
  \item Ef $\lambda_{1}\ll\mu, \lambda_{2}\ll\mu$ og
    $c_{1},c_{2}\in\C$, þá er $c_{1}\lambda_{1}+c_{2}\lambda_{2}\ll\mu$.
  \item Ef $\lambda_{1}\bot\mu,\lambda_{2}\bot\mu$ og
    $c_{1},c_{2}\in\C$, þá er $c_{1}\lambda_{1}+c_{2}\lambda_{2}\bot\mu$.
  \item Ef $\lambda_{1}\ll\mu$ og $\lambda_{2}\bot\mu$, þá er
    $\lambda_{1}\bot\lambda_{2}$.
  \item Ef $\lambda\ll\mu$, þá er $|\lambda\ll\mu$.
  \end{enumerate}
\end{setn}
\begin{fylgi}
  \begin{enumerate}[(1)]
  \item Ef $\lambda_{1}\bot\lambda_{2}$, þá
    $|\lambda_{1}|\bot|\lambda_{2}|$.
  \item Ef $\lambda\ll\mu$ og $\lambda\bot\mu$, þá er $\lambda = 0$.
  \end{enumerate}
\end{fylgi}
\begin{proof}
  Í heimadæmi.
\end{proof}
\begin{setn}\label{setn:lebesgue-radon}
  Látum $(X,\mathcal A,\mu)$ vera $\sigma$-endanlegt mælanlegt rúm og
  $\lambda$ vera tvinnmál á $\mathcal A$.
  \begin{enumerate}[(1)]
  \item\emph{(Lebesgue-sundurliðun)}\index{Lebesgue!sundurlidun@sundurliðun}
    Til eru ótvírætt ákvörðuð tvinnmál $\lambda_{1},\lambda_{2}$ á
    $\mathcal A$. þ.a.
    \[
    \lambda = \lambda_{1} + \lambda_{2},
    \quad
    \lambda_{1} \ll \mu,
    \quad\text{ og }\quad
    \lambda_{2}\bot\mu.
    \]
  \item\emph{(Radon-Níkodým-setning)} Ef $\lambda_{1}\ll\mu$ er til
    ótvírætt ákvarðað fall $h$ úr $L^{1}_{\C}(\mu)$ þ.a. 
    \[
    \lambda_{1}(A) = \int_{A} h\,d\mu
    \]
    fyrir öll $A$ úr $\mathcal A$.
  \end{enumerate}
\end{setn}
\begin{skilgr}
  Látum $\mu$ vera raunmál á $\sigma$-algebru $\mathcal A$. Við setjum 
  \[
  \mu^{+} := \frac 12 (|\mu|+\mu),
  \quad
  \mu^{-} := \frac 12 (|\mu|-\mu)
  \]
  og höfum þá skrifað 
  \[
  \mu = \mu^{+} - \mu^{-}
  \]
  sem mismun tveggja jákvæðra mála; köllum þetta
  \emph{Jordan-liðun}\index{Jordan-lidun@Jordan-liðun} málsins $\mu$.
\end{skilgr}
\begin{proof}
  [Sönnun setningar \ref{setn:lebesgue-radon}]
  Notum Jordan-liðun á $\re\lambda$ og $\im\lambda$ og sjaúm að það
  nægir að sanna setninguna fyrir jákvæð mál $\lambda$. Líka má gera
  ráð fyrir að $\mu(X)<+\infty$; þá má í almenna itlvikinu skrifa $X$
  sem sammengi sundurlægrar runu $(X_{j})_{j\in\N}$ af stökum í
  $\mathcal A$; ef við höfum setninguna fyrir sérhvert $X_{j}$, þá
  ,,límast lausnirnar saman'' í lausnir á $X$.

  Sýnum fyrst ótvíræðni: Í lið (a) látum við $\lambda =
  \lambda_{1}+\lambda_{2}=\nu_{1}+\nu_{2}$, þar sem
  $\lambda_{1},\nu_{1}\ll\mu$ og $\lambda_{2},\nu_{2}\bot\mu$. Þá er
  $\lambda_{1}-\nu_{1}=\nu_{2}-\lambda_{2}$,
  $\lambda_{1}-\nu_{1}\ll\mu$ og $\nu_{2}-\lambda_{2}\bot\mu$ svo að
  skv. reglu er $\lambda_{1}-\nu_{1}=\nu_{2}-\lambda_{2}=0$ og því
  $\lambda_{1}=\nu_{1}$, $\lambda_{2}=\nu_{2}$. Í lið (b) látum við
  $h_{1},h_{2}$ vera föll í $L^{1}_{\C}(\mu)$
  þ.a. $\int_{A}h_{1}\,d\mu = \int_{A}h_{2}\,d\mu$ fyrir öll $A$ úr
  $\mathcal A$. Setjum $g := \re(h_{1} - h_{2})$ og fáum
  $\int_{A}g\,d\mu = 0$ fyrir $A = \{ x : g(x) > 0 \}$, svo að
  $g\chi_{A} = 0$ næstum allsstaðar; eins $g\chi_{B} = 0$ n.a., þar
  sem $B = \{x : g(x)<0\}$. Þar með er $g = 0$ n.a. Eins er
  $\im(h_{1}-h_{2}) = 0$ n.a., svo að $h_{1} = h_{2}$ n.a.,
  þ.e. $h_{1} = h_{2}$ í $L_{\C}^{1}(\mu)$.

  Sýnum næst tilvist: Skrifum $\nu := \lambda+\mu$. Þá er 
  \[
  \int_{X}f\,d\nu
  = \int_{X}f\,d\lambda + int_{X}f\,d\mu
  \]
  fyrir öll mælanleg föll $f:X\to[0,+\infty]$, eins og við sjám fyrst
  fyrir kenniföll, næst fyrir einföld föll og loks fyrir markgildi
  runu af mælanlegum einföldum föllum. Fyrir $f$ úr $L^{2}(X,\mathcal
  A,\nu)$ er 
  \[
  \left|
    \int_{X}f\,d\lambda
  \right|
  \le \int_{X}|f|\,d\lambda
  \le \int_{X}f\,d\nu
  \le \left( \int_{X}|f|^{2}d\nu \right)^{1/2}\cdot C
  \]
  skv. Cauchy-Schwarz, þar sem $C = \left(\int_{X}1\,d\nu\right)^{1/2}
  = \nu(X)^{1/2}$. Þetta þýðir að fallið 
  \[
  L_{\R}^{2}(X,\mathcal A,\nu)\to\R,
  f\mapsto \int_{X} f\,d\lambda
  \]
  er samfellt línulegt fall á Hilbert-rúmi yfir $\R$, skv. setningu er
  þá til ótvírætt ákvarðað fall $g$ úr $L^{2}(X,\mathcal A,\nu)$ þ.a. 
  \[
  \int_{X}f\,d\lambda = \int_{X}fg\,d\nu.
  \]
  Fyrir $\chi_{A}$ með $A\in\mathcal A$ gefur þetta 
  \[
  0 \le \lambda(A)
  = \int_{X}\chi_{A}g\,d\nu
  = \int_{A}g\,d\nu.
  \]
  Vegna $\lambda\le\nu$ fæst 
  \[
  0
  \le \lambda(A)
  \le \int_{A}g\,d\nu
  \le \nu(A)
  = \int_{A}1\,d\nu,
  \]
  þ.e. $\int_{A}(1-g)\,d\nu\ge 0$ fyrir öll $A\in \mathcal A$. Látum
  $A = \{ x\in X : g(x) < 0 \}$ og fáum $0\le\int_{A}g\,d\nu =
  \int_{X}\chi_{A}g\,d\nu$. Svo að $\chi_{A}g = 0$ $\nu$-næstum
  allsstaðar, sem þýðir að $g\ge 0$ næstum allsstaðar. Með sama hætti
  fæst að $1-g\ge 0$ næstum allsstaðar. Með því að breyta $g$ á
  núllmengi má gera ŕað fyrir að $0\le g\l1 1$ allsstaðar. Nú höfum
  við  
  \[
  \int_{X} f\,d\lambda
  = \int_{X} fg\,d\nu
  = \int_{X} fg\,d\lambda + \int_{X} fg\,d\mu,
  \]
  þ.e.
  \begin{equation}
    \label{eq:lebesgue-radon-1}
    \int_{X}(1-g)f\,d\lambda
    = \int_{X}fg\,d\mu
  \end{equation}
  fyrir öll $f$ úr $L^{2}(\nu)$. Setjum 
  \[
  E := \{x\in X : 0\le g(x) < 1 \},
  \quad
  F := \{x\in X : g(x) = 1 \}
  \]
  og skilgreinum mál $\lambda_{1},\lambda_{2}$ á $\mathcal A$ með 
  \[
  \lambda_{1}(A) = \lambda(A\cap E),
  \quad
  \lambda_{2}(A) := \lambda(A\cap F).
  \]
  Höfum þá $\lambda = \lambda_{1}+\lambda_{2}$. TÖkum $f =
  \lambda_{F}$ í \eqref{eq:lebesgue-radon-1} og fáum
  $\lambda(F)=0$. Þar með er $\lambda_{2}\bot\mu$. Látum $A\in\mathcal
  A$ og setjm $f:=(1+g+\cdots+g^{k})\chi_{A}$ inn í
  \eqref{eq:lebesgue-radon-1} og fáum 
  \[
  \int_{A}(1-g^{k+1})\,d\lambda
  = \int_{A}(1+g+\cdots+g^{k})g\,d\mu.
  \]
  Látum $k\to+\infty$ og athugum að
  $\lim_{k\to+\infty}(1-g^{k})=\chi_{E}$; fáum 
  \[
  \lambda_{1}(A) = \lambda(A\cap F) = \int_{A}h\,d\mu
  \]
  þar sem $h : = \lim_{k\to+\infty}(1+\cdots+g^{k})g$ skv. setningu um
  einhalla samleitni. Þar sem $h\ge 0$ og $\lambda(E)=\int_{X}h\,d\mu$
  er $h\in L^{1}(\mu)$, og sönnun er lokið.
\end{proof}

\emph{Radon-Níkodým segir:} Ef $(X,\mathcal A,\mu)$ er
$\sigma$-endanlegt málrúm og $\lambda$ er tvinnmál á $X$
þ.a. $\lambda\ll\mu$, þá er til fall $h$ á $L_{\C}^{1}(X,\mathcal
A,\mu)$ þ.a. 
\[
\lambda(A) = \int_{A}h\,d\mu
\]
fyrir öll $A\in\mathcal A$.
\begin{skilgr}
  Fallið $h$ kallast
  \emph{Radon-Níkodým-afleiða}\index{Radon-Níkodým-afleiða} málsins
  $\lambda$ m.t.t. málsins $\mu$, og er oft táknað 
  \[
  h =: \frac{d\lambda}{d\mu}.
  \]
\end{skilgr}

\printindex
\end{document}