\documentclass[a4paper,icelandic]{book}

%% Use utf-8 encoding for foreign characters
\usepackage[T1]{fontenc}
\usepackage[utf8]{inputenc}
\usepackage{babel}

%% Margins
\usepackage{fullpage}
% \usepackage{geometry}

%% More symbols
\usepackage{amsmath}
\usepackage{amssymb}

%% Theorem
\usepackage{amsthm}

%% Örvarit
\usepackage[all]{xy}

\theoremstyle{definition}
\newtheorem{skilgr}{Skilgreining}[section]
\newtheorem{daemi}{Dæmi}[section]
\newtheorem*{aefing}{Æfing}

\theoremstyle{plain}
\newtheorem{setn}{Setning}[section]
\newtheorem{fylgisetn}{Fylgisetning}[section]
\newtheorem{hjalparsetn}{Hjálparsetning}[section]

\theoremstyle{remark}
\newtheorem*{ath}{Athugasemd}
\newtheorem*{lausn}{Lausn}

%% Ýmis mengi
\newcommand{\R}{\mathbb{R}} % rauntölurnar
\newcommand{\N}{\mathbb{N}} % náttúrlegu tölurnar
\newcommand{\Z}{\mathbb{Z}} % heilu tölurnar
\newcommand{\Q}{\mathbb{Q}} % ræðu tölurnar
\newcommand{\C}{\mathbb{C}} % tvinntölurnar

%% Föll o.fl. til að nota í math mode (verður aldrei skáletrað)
\DeclareMathOperator{\id}{id} % samsemdarvörpunin
\DeclareMathOperator{\innmengi}{int} % innmengi
\DeclareMathOperator{\diam}{diam} % þvermál
\DeclareMathOperator{\supp}{supp} % stoð

\DeclareMathOperator{\rel}{rel} 
\newcommand{\prel}[1]{\,\left(\rel #1\right)}

\DeclareMathOperator{\Ob}{Ob}
\DeclareMathOperator{\Hom}{Hom}

% If you want to generate a toc for each chapter (use with book)
\usepackage{minitoc}

% Meiri stjórn á númeringu
\usepackage{enumerate}

% Atriðisorðaskrá
\usepackage{makeidx}
\makeindex

% This is now the recommended way for checking for PDFLaTeX:
\usepackage{ifpdf}

\ifpdf
\usepackage[pdftex]{graphicx}
\usepackage[pdftex]{hyperref}
\else
\usepackage{graphicx}
\usepackage{hyperref}
\fi

\title{\textbf{Grannfræði}}
\author{%
Kennari: Jón Ingólfur Magnússon\\
Kennslubók: Munkres, James R. \emph{Topology}, 2. útgáfa. 2003. Prentice Hall.
}
\date{Vorönn 2010}

\begin{document}

\maketitle
\tableofcontents
% Snið dagsetninga: dd.mm.yyyy

% 07.01.2010
\part{Almenn grannfræði}

\chapter{Grannrúm og samfelldar varpanir}

\section{Grannrúm}

\begin{skilgr}
  \emph{Grannmynstur}\index{grannmynstur} á mengi $X$ er safn $\mathcal T$ af
  hlutmengjum í $X$ sem uppfylla eftirfarandi skilyrði:
  \begin{enumerate}[(i)]
    \item $\emptyset$ og $X$ eru í $\mathcal T$.
    \item Sammengi mengja í $\mathcal T$ eru í $\mathcal T$.
    \item Sniðmengi endanlega margra mengja í $\mathcal T$ er í $\mathcal T$.
  \end{enumerate}
  Tvenndin $(X,\mathcal T)$ kallast \emph{grannrúm}\index{grannrúm}. Ef ekki fer
  á milli mála við hvaða $\mathcal T$ er átt, þá tölum við um $X$ sem
  grannrúm. Köllum stökin í $X$ oft \emph{punkta}\index{punktur} og stökin í
  $\mathcal T$ \emph{opin mengi}\index{opið mengi}.
\end{skilgr}
\begin{ath}
  $X$ er sniðmengi tómu fjölskyldunnar. $\emptyset$ er sammengi tómu
  fjölskyldunnar.
\end{ath}
\begin{daemi}
  (1) Látum $(M,d)$ vera firðrúm. Þá mynda opnu mengin (í firðrúmaskilningi)
  grannmynstur á $M$.

  (2) $X$ mengi. Grannmynstrið sem samanstendur eingöngu af $\emptyset$ og $X$
  kallast \emph{grófa grannmynstrið}\index{grófa grannmynstrið}. Grannmynstrið
  sem samanstendur af öllum hlutmengjum $X$ kallast \emph{dreifða
    grannmynstrið}\index{dreifða grannmynstrið} (á $X$) (sérhvert mengi af
  gerðinni $\left\{ x \right\}$ með $x\in X$ er opið).
\end{daemi}
\begin{skilgr}
  $\mathcal T$ og $\mathcal T'$ tvö grannmynstur á mengi $X$. Segjum að
  $\mathcal T'$ sé \emph{fínna en $\mathcal T$}\index{fínna grannmynstur} ef
  $\mathcal T' \supseteq \mathcal T$. Segjum þá að $\mathcal T$ sé \emph{grófara
    en}\index{grófara grannmynstur} $\mathcal T'$.
\end{skilgr}
\begin{skilgr}
  (ekki í bók) $X$ grannrúm og $A\subseteq X$. Við segjum að $V\subseteq X$ sé
  \emph{grennd um}\index{grennd} $A$ ef til er opið mengi $U$ í $X$ þannig að
  $A\subseteq U\subseteq V$. Grenndir einstökungs $\left\{ x \right\}$ kallast
  líka grenndir $x$.
\end{skilgr}
\begin{skilgr}
  Safn $\mathcal B$ af hlutmengjum í $X$ kallast \emph{grunnur fyrir
    grannmynstur á}\index{grunnur} $X$ ef eftirfarandi gildir:
  \begin{enumerate}[(i)]
    \item Fyrir öll $x\in X$ er til a.m.k. eitt $B$ úr $\mathcal B$ þ.a. $x\in B$.
    \item Ef $B_1,B_2\in\mathcal B$ og $x\in B_1\cap B_2$, þá er til
      $B_3\in\mathcal B$ þ.a. $x\in B_3\subseteq B_1\cap B_2$.
  \end{enumerate}
\end{skilgr}
\begin{setn}
  Látum $\mathcal T_{\mathcal B}$ vera safn allra hlutmengja $U$ í $X$ sem
  fullnægja eftirfarandi skilyrði:
  \begin{quote}
    Fyrir sérhvert $x\in U$ er til $B$ úr $\mathcal B$ þ.a. $x\in B\subseteq U$.
  \end{quote}
  Þá er $\mathcal T_{\mathcal B}$ grannmynstur á $X$ og við köllum það
  \emph{grannmynstrið sem $\mathcal B$ framleiðir}\index{framleitt
    grannmynstur}\index{grannmynstur!framleitt}.
\end{setn}
\begin{proof}
  (i) Að $\emptyset \in \mathcal T_{\mathcal B}$ er augljóst.

  (ii) $X\in\mathcal T_{\mathcal B}$: Fyrir sérhvert $x \in X$ er til (skv. (i))
  $B\in\mathcal B$ þ.a. $x\in B\subseteq X$.

  (iii) Sammengi mengja úr $\mathcal T_{\mathcal B}$ er í $\mathcal T_{\mathcal
    B}$: Látum $(U_\alpha)_{\alpha\in J}$ vera fjölskyldu í $\mathcal
  T_{\mathcal B}$ og setjum $U:=\bigcup_{\alpha\in J}U_\alpha$. Ef $x\in U$ þá
  er til $\alpha \in J$ þ.a. $x\in U_{\alpha}$ og því er til $B\in\mathcal B$
  þ.a. $x\in B\subseteq U_\alpha$ og þar með er $x\in B\subseteq U$. Þetta hefur
  í för með sér að $U\in\mathcal T_{\mathcal B}$.

  (iv) Endanleg sniðmengi mengja í $\mathcal T_{\mathcal B}$ eru í $\mathcal
  T_{\mathcal B}$: Nóg að sýna að sniðmengi tveggja mengja í $\mathcal
  T_{\mathcal B}$ sé í $\mathcal T_{\mathcal B}$ og beita svo þrepun. Látum
  $U_1,U_2\in\mathcal T_{\mathcal B}$ og $x\in U_1,U_2$. Þá eru til
  $B_1,B_2\in\mathcal B$ þ.a. $x\in B_1\subseteq U_1$ og $x\in B_2\subseteq
  U_2$. Skv. skilyrði (ii) fyrir grunna er þá til $B_3$ úr $\mathcal B$
  þ.a. $x\in B_3\subseteq B_1\cap B_2\subseteq U_1\cap U_2$. Þar með er $U_1\cap
  U_2$ í $\mathcal T_{\mathcal B}$.
\end{proof}
\begin{daemi}
  (1) Ef $\mathcal T$ er grannmynstur á $X$, þá er $\mathcal T$ grunnur fyrir
  $\mathcal T$.

  (2) Safn allra einstökunga $\left\{ x \right\}$ er grunnur fyrir dreifða
  grannmynstrið á $X$.

  (3) $(M,d)$ firðrúm og $\mathcal T_{d}$ grannmynstrið sem $d$ skilgreinir
  (safn opnu mengjanna). Setjum $B_{\varepsilon}(x) := \left\{ y\in M : d(x,y) <
    \varepsilon \right\}$ fyrir öll $\varepsilon > 0$ og öll $x\in M$. Þá er
  $\mathcal B := \left\{ B_{\varepsilon}(x) : \varepsilon > 0, x\in M \right\}$
  grunnur fyirr $\mathcal T_{d}$.
\end{daemi}
\begin{setn}
  $\mathcal B$ grunnur fyrir grannmynstur $\mathcal T$ á mengi $X$. Þá er
  $\mathcal T$ jafnt safni allra sammengja af mengjum í $\mathcal B$.
\end{setn}
\begin{proof}
  Ef $(B_\alpha)_{\alpha\in J}$ er fjölskylda í $\mathcal B$, þá er
  $(B_\alpha)_{\alpha\in J}$ jafnframt fjölskylda í $\mathcal T$ og þar með er
  $\bigcup_{\alpha\in J}B_\alpha$ í $\mathcal T$ (vegna þess að $\mathcal T$ er
  grannmynstur).

  Öfugt, ef $U$ er í $\mathcal T$, þá er fyrir sérhvert $x$ úr $U$ til $B_x$ úr
  $\mathcal B$ þannig að $x\in B_x\subseteq U$. Þar með er $U=\bigcup_{x\in U}
  B_x$.
\end{proof}
\begin{setn}
  $\mathcal B$ og $\mathcal B'$ grunnar fyrir grannmynstur á mengi $X$. Þá eru
  eftirfarandi skilyrði jafngild:
  \begin{enumerate}[(i)]
    \item $\mathcal T_{\mathcal B'}$ er fínna en $\mathcal T_{\mathcal B}$.
    \item Fyrir sérhvert $x\in X$ og $B\in \mathcal B$ sem inniheldur $x$ er til
      $B'$ úr $\mathcal B'$ þannig að $x\in B'\subseteq B$.
  \end{enumerate}
\end{setn}
\begin{proof}
  \emph{(i)$\Rightarrow$(ii):} Gefið $x\in X$ og $B\in \mathcal B$ með $x\in B$.
  Nú er
  \[
  B\in \mathcal B\subseteq \mathcal T_{\mathcal B} \underset{(i)}{\subseteq}
  \mathcal T_{\mathcal B'}
  \]
  svo þar sem $\mathcal B'$ framleiðir $\mathcal T_{\mathcal B'}$ þá er til $B'$
  $\mathcal B'$ úr $\mathcal B'$ þ.a. $x\in B'\subseteq B$.  g
  \emph{(ii)$\Rightarrow$(i):} Gefið $U$ úr $\mathcal T_{\mathcal B}$. Viljum
  sýna að $U\in \mathcal T_{\mathcal B'}$. Ef $x\in U$, þá er til $B$ úr
  $\mathcal B$ með 4$x\in B\subseteq U$. Skv. (ii) er þá til $B'$ úr $\mathcal
  B'$ með $x\in B'\subseteq B$ og því $x\in B'\subseteq U$, en það þýðir að
  $U\in \mathcal T_{\mathcal B'}$.
\end{proof}
\begin{ath}
  Oft hagstæðara að vinna með grunnana.
\end{ath}
\begin{setn}
  Látum $(X,\mathcal T)$ vera grannrúm og $\mathcal C\subseteq \mathcal T$
  þannig að fyrir sérhvert $U\in \mathcal T$ og sérhvert $x\in U$ er til $C$ úr
  $\mathcal C$ þannig að $x\in C\subseteq U$. Þá er $\mathcal C$ grunnur fyirr
  grannmynstrið $\mathcal T$.
\end{setn}
\begin{proof}
  Sýnum fyrst að $\mathcal C$ sé grunnur fyrir grannmynstur á $X$.

  (i) Gefið $x$ úr $X$. Viljum sýna að til sé $C$ úr $\mathcal C$ þannig að
  $x\in C$.  En $X\in \mathcal T$ svo að til er $C\in \mathcal C$þannig að $x\in
  C\subseteq X$.

  (ii) Gefin séu $C_1$ og $C_2$ úr $\mathcal C$ og $x\in C_1\cap C_2$. Viljum
  sýna að til sé $C_3$ úr $\mathcal C$ þ.a. $x\in C_3\subseteq C_1\cap C_2$. En
  $C_1,C_2$ eru í $\mathcal T$ og því $C_1\cap C_2 \in \mathcal T$ og þar með er
  til $C_3\in \mathcal C$ þannig að $x\in C_3\subseteq C_1\cap C_2$.

  Tökum nú eftir að $\mathcal T = \mathcal T_{\mathcal T}$ svo skv. síðustu
  setningu er $\mathcal T_{\mathcal C}$ fínna en $\mathcal T$ og ljóst er að
  $\mathcal T = \mathcal T_{\mathcal T}$ er fínna en $\mathcal T_{\mathcal C}$
  vegna þess að $\mathcal C\subseteq \mathcal C$. Þar með er $\mathcal T =
  \mathcal T_{\mathcal C}$.
\end{proof}
\begin{daemi}
  $\mathcal B$ safn allra opinna bila $\left] a,b\right[$ í $\R$ og $\mathcal
  B'$ safn allra hálfopinna bila af gerðinni $\left[a,b\right[$ í $\R$. Þá eru
  $\mathcal B$ og $\mathcal B'$ grunnar fyrir grannmynstur á $\R$. $\mathcal
  T_{\mathcal B}$ er venjulega grannmynstrið á $\R$, en $\mathcal T_{\mathcal
    B'}$ er stranglega fínna en $\mathcal T_{\mathcal B}$, þ.e.a.s. $\mathcal
  T_{\mathcal B} \subsetneq \mathcal T_{\mathcal B'}$ [lesa sjálf].
\end{daemi}
\begin{skilgr}
  $\mathcal S$ Safn hlutmengja í mengi $X$ sem þekur $X$ kallast
  \emph{hlutgrunnur fyrir grannmynstur
    á}\index{hlutgrunnur}\index{grunnur!hlutgrunnur} $X$.
\end{skilgr}
\begin{setn}
  Ef $\mathcal S$ er hlutgrunnur (fyrir grannmynstur á $X$) og $\mathcal B$ er
  safn allra endanlegra sniðmengja af mengjum í $\mathcal S$ þá er $\mathcal B$
  grunnur fyrir grannmynstur á $X$. Við segjum að $\mathcal S$
  \emph{framleiði}\index{framleiða} $\mathcal T_{\mathcal B}$.
\end{setn}
\begin{proof}
  (i) Ef $x\in X$, þá er til $S$ úr $\mathcal S$ þannig að $x\in S$ og $\mathcal
  S\subseteq \mathcal B$.

  (ii) Ef $B_1,B_2\in \mathcal B$, þá eru til $S_1,\dots,S_m$ og
  $S_1',\dots,S_n'$ úr $\mathcal S$ þannig að $B_1 = \bigcap_{j=1}^m S_j$ og
  $B_2 = \bigcap_{j=1}^n S_j'$ og því $B_1\cap B_2 = S_1 \cap\cdots\cap S_m \cap
  S_1'\cap \cdots \cap S_n'$ líka úr $\mathcal B$.
\end{proof}

% 11.01.2010
\section{Röðunargrannmynstur}
$(X,<)$ línulega raðað mengi og $a,b\in X$. Köllum
\begin{itemize}
\item $\left]a,b\right[ := \left\{ x\in X: a<x<b \right\}$ \emph{opið
    bil}\index{opið bil}.
\item $\left[a,b\right[ := \left\{ x\in X: a\leq x < b \right\}$ \emph{hálfopið
    bil}\index{hálfopið bil}.
\item $\left]a,b\right] := \left\{ x\in X: a < x \leq b \right\}$ \emph{hálfopið
    bil}\index{hálfopið bil}.
\item $\left[a,b\right] := \left\{ x\in X: a\leq x \leq b \right\}$ \emph{lokað
    bil}\index{lokað bil}.
\item $\left]a,+\infty\right[ := \left\{ x\in X : a < x \right\}$ \emph{opinn
    geisla}\index{opinn geisli}.
\item $\left]-\infty,a\right[ := \left\{ x\in X : x < a \right\}$ \emph{opinn
    geisla}\index{opinn geisli}.
\item $\left[a,+\infty\right[ := \left\{ x\in X : a\leq x \right\}$
  \emph{lokaðan geisla}\index{lokaður geisli}.
\item $\left]-\infty,a\right] := \left\{ x\in X : x\leq a \right\}$
  \emph{lokaðan geisla}\index{lokaður geisli}.
\end{itemize}
\begin{skilgr}
  $(X,<)$ línulega raðað. Látum $\mathcal B$ vera safn allra mengja af
  eftirfarandi gerðum:
  \begin{enumerate}[(i)]
    \item opin bil í $X$
    \item hálfopin bil $\left[a_0, b\right[$ þar sem $a_0 := \min X$ (ef til)
    \item hálfopin bil $\left]a, b_0\right]$ þar sem $b_0 := \max X$ (ef til). 
  \end{enumerate}
  $\mathcal B$ er greinilega grunnur fyrir grannmynstur á $X$. Við köllum
  $\mathcal T_{\mathcal B}$
  \emph{röðunargrannmynstur}\index{röðunargrannmynstur} línulega raðaða
  mengisins $(X,<)$.
\end{skilgr}
\begin{daemi}
  (1) $\R\times\R$ með \emph{orðabókarröðun}\index{orðabókarröðun}.

  (2) $\N = \left\{ 0,1,2,\dots \right\}$. Röðunargrannmynstrið er dreifða
  grannmynstrið.

  (3) $X = \left\{ 1,2 \right\}\times\N$ með orðabókarröðun.  Dreifða
  grannmynstrið er fínna en röðunargrannmynstrið, vegna þess að $\left\{ 2\times
    0 \right\}$ er ekki opið.

  (4) Opnir geislar eru opin mengi í röðunargrannmynstrinu.
\end{daemi}
\begin{skilgr}
  $X$ línulega raðað. 
  \begin{enumerate}[(i)]
    \item Stak $c$ úr $X$ kallast \emph{eftirfari}\index{eftirfari} $a$ ef $a <
      c$ og $\left] a,c \right[=\emptyset$.
    \item Stak $b$ úr $X$ kallast \emph{undanfari}\index{undanfari} $b$ ef $d<b$
      og $\left] d,b \right[ = \emptyset$. 
  \end{enumerate}
  \end{skilgr}

\section{Faldgrannmynstur}
\begin{skilgr}
  Látum $X$ og $Y$ vera grannrúm. Öll mengi af gerðinni $U\times V$ þar sem $U$
  er opið í $X$ og $V$ er opið í $Y$, mynda grunn fyrir grannmynstur á $X\times
  Y$ sem við köllum \emph{faldgrannmynstrið}\index{faldgrannmynstur} á $X\times
  Y$.
\end{skilgr}
\begin{setn}
  Ef $\mathcal B$ er grunnur fyrir á $X$ og $\mathcal C$ er grunnur fyrir
  grannmynstrið á $Y$, þá er safnið $\mathcal D := \left\{ B\times C :
    B\in\mathcal B, C\in\mathcal C \right\}$ grunnur fyrir faldgrannmynstrið á
  $X\times Y$.
\end{setn}
\begin{proof}
  Látum $W$ vera opið í $X\times Y$ og $(x,y)\in W$. Þá er til opið $U$ í $X$ og
  opið $V$ í $Y$ þ.a. $(x,y)\in U\times V\subseteq W$. Þar eð $\mathcal B$ er
  grunnur fyrir $\mathcal T_X$ og $\mathcal C$ er grunnur fyrir $\mathcal T_Y$,
  þá eru til $B$ úr $\mathcal B$ og $C$ úr $\mathcal C$ þ.a. $x\in B\subseteq U$
  og $y\in C\subseteq V$ og þar með $(x,y)\in B\times C\subseteq U\times
  V\subseteq W$.
\end{proof}
\begin{daemi}
  [Ganga vel úr skugga um það!] Faldgrannmynstrið á $\R^2 = \R\times\R$ er
  venjulega grannmynstrið.
\end{daemi}
\begin{setn}
  Táknum ofanvörpin $\pi_1:X\times Y\to X;(x,y)\mapsto x$ og $\pi_2:X\times Y\to
  Y;(x,y)\mapsto y$. Safnið
  \[
  \mathcal S := \left\{ \pi_1^{-1}(U): U\text{ opið í } X \right\}\cup\left\{
    \pi_{2}^{-1}(V) : V\text{ opið í } Y \right\}
  \]
  er hlutgrunnur fyrir faldgrannmynstrið.
\end{setn}
\begin{proof}
  Leiðir beint af því að $\pi_1^{-1}(U)\cap\pi_2^{-1}(V) = U\times V$.
\end{proof}

\section{Hlutrúmsgrannmynstur}
\begin{skilgr}
  $(X,\mathcal T)$ grannrúm og $Y\subseteq X$. Safnið $\mathcal T_Y := \left\{
    Y\cap U : U\in \mathcal T \right\}$ er grannmynstur á $Y$, kallað
  \emph{hlutrúmsgrannmynstrið}\index{hlutrúmsgrannmynstur} á $Y$. Hlutmengi í
  $X$ með hlutrúmsgrannmynstrinu kallast \emph{hlutrúm}\index{hlutrúm} í $X$.
\end{skilgr}
\begin{setn}
  Ef $\mathcal B$ er grunnur fyrir grannmynstrið á $X$, þá er 
  \[
  \mathcal B_{Y} := \left\{ B\cap Y : B\in \mathcal B \right\}
  \]
  grunnur fyrir hlutrúmsgrannmynstrið á $Y$.
\end{setn}
\begin{proof}
  Augljóst.
\end{proof}
\begin{setn}
  $Y$ hlutrúm í grannrúmi $X$. Ef $U$ er opið í $Y$ og $Y$ er opið í $X$, þá er
  $U$ opið í $X$.
\end{setn}
\begin{proof}
  Augljóst.
\end{proof}
\begin{daemi}
  Lítum á hlutmengin $Y_1 = \left[ 0,1 \right]$ og $Y_2 = \left[ 0,1
  \right[\cup\left\{ 2 \right\}$ í $\R$. Þá er hlutrúmsgrannmynstrið (í $\R$) og
  röðunargrannmynstrið eins á $Y_1$, en ekki á $Y_2$ ($\left\{ 2 \right\}$ er
  ekki opið í röðunargrannmynstrinu).
\end{daemi}
% Ég er svangur
\begin{ath}
  (dæmi 6 á bls. 91 í kennslubók): Á bili eða geisla í línulega röðuðu mengi
  $(X,<)$ eru röðunargrannmynstrin og hlutrúmsgrannmynstrin eins (sjá einnig
  almennari útgáfu: Setning 16.4 í kennslubók).
\end{ath}
\begin{setn}
  $A$ hlutrúm í grannrúmi $X$ og $B$ hlutrúm í grannrúmi $Y$. Þá er
  faldgrannmynstrið á $A\times B$ það sama og hlutrúmsgrannmynstrið sem $A\times
  B$ erfir frá $X\times Y$.
\end{setn}
\begin{proof}
  $U$ opið í $X$ og $V$ opið í $Y$. Þá er 
  \begin{align*}
    \underbrace{(U\times V)\cap(A\times B)}_{\substack{\text{mynda grunn fyrir}\\\text{hlutrúmsgrannmynstrið}}}
    &= \underbrace{(U\cap A)\times(V\cap B)}_{\substack{\text{mynda grunn fyrir}\\\text{faldgrannmynstrið}}}.
  \end{align*}
\end{proof}

\section{Lokuð mengi}
\begin{skilgr}
  Hlutmengi $A$ í grannrúmi $X$ er sagt \emph{lokað}\index{lokað mengi} ef
  $X\setminus A$ er opið.
\end{skilgr}
\begin{setn}
  $X$ grannrúm. Þá gildir:
  \begin{enumerate}[(i)]
    \item $\emptyset$ og $X$ eru lokuð.
    \item Sniðmengi lokaðra mengja í $X$ er lokað.
    \item Endanleg sammengi lokaðra mengja eru lokuð.
  \end{enumerate}
\end{setn}
\begin{proof}
  (i) $\emptyset = X\setminus X$ og $X = X\setminus\emptyset$.
  
  (ii) og (iii) leiðir beint af reglum de Morgans og samsvarandi reglum fyrir
  opin mengi.
\end{proof}
\begin{daemi}
  Í dreifðu grannmynstri eru öll mengi bæði opin og lokuð.
\end{daemi}
\begin{setn}
  $Y$ hlutrúm í grannrúmi $X$. Hlutmengi $Y$ er lokað í $Y$ \emph{þ.þ.a.a.} það
  sé sniðmengi $Y$ og lokaðs mengis í $X$.
\end{setn}
\begin{proof}
  Látum $A\subseteq Y$. Ef $A = C\cap Y$ með $C$ lokað í $X$, þá er $X\setminus
  C$ opið í $X$ og því $Y\setminus A = (X\setminus C)\cap Y$ opið í $Y$, svo $A$
  er lokað í $Y$.

  \emph{Öfugt}, ef $A$ er lokað í $Y$, þá er $Y\setminus A$ opið í $Y$ og því er
  $Y\setminus A = U\cap Y$, þar sem $U$ er opið í $X$. Mengið $X\setminus U$ er
  þá lokað í $X$ og $A = (X\setminus U)\cap Y$.
\end{proof}
\begin{setn}
  $Y$ hlutrúm í grannrúmi $X$. Ef $A$ er lokað í $Y$ og $Y$ er lokað í $X$, þá
  er $A$ lokað í $X$.
\end{setn}
\begin{proof}
  Augljóst.
\end{proof}

% 12.10.2010
\section{Lokanir og innri mengi}
\begin{skilgr}
  $A$ hlutmengi í grannrúmi $X$.
  \begin{enumerate}[(i)]
  \item Sammengi allra opinna mengja sem eru innihaldin í $A$ er stærsta opna
    mengið sem $A$ inniheldur og kallast \emph{innra mengi}\index{innra mengi}
    eða \emph{innmengi}\index{innmengi} mengisins $A$, táknað $\overset \circ A$
    eða $\innmengi(A)$.
  \item Sniðmengi allra lokaðra mengja sem innihalda $A$ er minnsta lokaða
    mengið sem inniheldur $A$ og kallast \emph{lokun}\index{lokun} (eða
    \emph{lokunarhjúpur}\index{lokunarhjúpur}) $A$, táknað $\overline A$.
  \end{enumerate}
\end{skilgr}
\begin{ath}
  $\overset \circ A \subseteq A \subseteq \overline A$.
\end{ath}
\begin{daemi}
  $X = \R^2, Y = \left[0,1\right[\times\left\{ 0 \right\}$ og $A =
  \left]0,1\right[\times\left\{ 0 \right\}$. Lokun $A$ í $Y$,
  þ.e.a.s. $\overline A_Y = Y$. Lokun $A$ í $X$ er jöfn lokun $Y$ í $X$,
  þ.e.a.s. $\overline A_X = \overline Y_X = \left[ 0,1 \right]\times\left\{ 0
  \right\}$. Innmengi $A$ í $Y$ er $A$. Innmengi $A$ í $X$ er $\emptyset$.
\end{daemi}
\begin{setn}
  $Y$ hlutrúm í grannrúmi $X$ og $A\subseteq Y$. Látum $\overline A$ tákna lokun
  $A$ í $X$. Þá er $\overline A\cap Y$ lokun $A$ í $Y$.
\end{setn}
\begin{proof}
  Látum $B$ tákna lokun $A$ í $Y$. Þar eð $\overline A\cap Y$ er lokað í $Y$, þá
  er $B\subseteq \overline A\cap Y$. En þar eð $B$ er lokað í $Y$, þá er til $C$
  í $X$ þ.a. $B = C\cap Y$ og því $\overline A \subseteq C$ og það gefur að
  $\overline A\cap Y \subseteq C\cap Y = B$.
\end{proof}
\begin{setn}
  $A$ hlutmengi í grannrúmi $X$
  \begin{enumerate}[(i)]
    \item $X\in\overline A$ \emph{þ.þ.a.a.} sérhver grennd $U$ um $x$ skeri $A$.
    \item Ef $\mathcal B$ er grunnur fyrir $\mathcal T_X$, þá gildir:
      $x\in\overline A$ \emph{þ.þ.a.a.} $B\cap A\neq\emptyset$ fyrir öll
      $B\in\mathcal B$ þ.a. $x\in B$.
  \end{enumerate}
\end{setn}
\begin{proof}
  (i) $x\in X\setminus \overline A$ \emph{þ.þ.a.a.} til sé opin grennd $U$ um
  $x$ þ.a. $x\in U\subseteq X\setminus \overline A$.

  (ii) Augljós.
\end{proof}

\section{Þéttipunktar}
\begin{skilgr}
  $A$ hlutmengi í grannrúmi $X$. Punktur $x$ úr $X$ er kallaður
  \emph{þéttipunktur}\index{zettipunktur@þéttipunktur} $A$ ef $(A\setminus\left\{ x
  \right\})\cap U\neq\emptyset$ fyrir allar grenndir $U$ um $x$.
\end{skilgr}
\begin{ath}
  Þéttipunktur $A$ getur hvort sem er tilheyrt $A$ eða ekki. $A$ getur haft
  punkta sem eru ekki þéttipunktar.
\end{ath}
\begin{setn}
  $A$ hlutmengi í grannrúmi $X$ og $A'$ mengi allra þéttipunkta $A$. Þá er
  $\overline A = A\cup A'$.
\end{setn}
\begin{proof}
  Ef $x\in A'$, þá gildir um sérhverja grennd $U$ um $x$ að $(A\setminus\left\{
  x \right\})\cap U \neq \emptyset$ og því $A\cap U\neq \emptyset$. Skv.
  niðurstöðu (i) í síðustu setningu er þá $x\in\overline A$ við höfum því sýnt
  að $A'\subseteq \overline A$ og þar sem $A\subseteq \overline A$ skv.
  skilgreiningu þá gildir $A\cup A' \subseteq \overline A$.

  \emph{Öfugt}, sýnum að $\overline A\subseteq A\cup A'$. Látum $x\in\overline
  A$. Ef $x\in A$, þá þarf ekkert að gera. Gerum ráð fyrir að $x\in\overline
  A\setminus A$. Þá gildir um sérhverja grennd $U$ um $x$ að $\emptyset \neq
  A\cap U = (A\setminus \left\{ x \right\})\cap U$. Þar með er $x\in A$.
\end{proof}
\begin{fylgisetn}
  Hlutmengi í grannrúmi er lokað \emph{þ.þ.a.a.} það innihaldi alla þéttipunkta
  sína.
\end{fylgisetn}
\begin{proof}
  $A$ er lokað \emph{þ.þ.a.a.} $A = \overline A$ \emph{þ.þ.a.a.} $A'\subseteq A$.
\end{proof}

\section{$T_1$-rúm og Hausdorff-rúm}
\begin{skilgr}
  Við segjum að grannrúm $X$ sé
  \begin{enumerate}[(i)]
  \item \emph{$T_1$-rúm}\index{T-rúm@$T_1$-rúm} (eða uppfylli
    \emph{$T_1$-frumsenduna}) ef fyrir sérhverja tvo punkta $x,y$ úr
    $X$ eru til grenndir $U_x$ um $x$ og $U_y$ um $y$ þannig að
    $y\notin U_x$ og $x\notin U_y$.
  \item \emph{Hausdorff-rúm}\index{Hausdorff-rúm} (eða uppfylli
    \emph{$T_2$-frumsenduna}) ef fyrir sérhverja tvo punkta $x,y$ úr
    $X$ eru til grenndir $U_x$ um $x$ og $U_y$ um $y$ þ.a. $U_x\cap
    U_y = \emptyset$.
  \end{enumerate}
\end{skilgr}
\begin{setn}
  Endanleg mengi í $T_1$-rúmi $X$ eru lokuð.
\end{setn}
\begin{proof}
  Látum $A\subseteq X$ vera endanlegt og $x\in X\setminus A$. Þá er fyrir
  sérhvert $a$ úr $A$ til grennd $U_a$ um $x$ þ.a. $a\notin U_a$ og þar með er
  $\bigcap_{a\in A}U_a$ grennd um $x$ í $X\setminus A$. Mengið $X\setminus A$ er
  því opið og þar með er $A$ lokað.
\end{proof}
\begin{setn}
  $A$ hlutmengi í $T_1$-rúmi $X$. Þá er $x$ þéttipunktur $A$ \emph{þ.þ.a.a.}
  sérhver grennd um $x$ skeri $A$ í óendanlega mörgum punktum.
\end{setn}
\begin{proof}
  Ef sérhver grennd um $x$ sker $A$ í óendanlega mörgum punktum, þá sker hún
  $A\setminus\left\{ x \right\}$ og þar með er $x$ þéttipunktur $A$.

  \emph{Öfugt}, g.r.f. að $x$ sé punktur úr $X$ þ.a. til sé grennd $U$ um $x$
  sem sker $A$ í aðeins endanlega mörgum punktum. Þá fæst
  \[
  (A\setminus \left\{ x \right\})\cap U = \left\{ x_1,\dots,x_m \right\}.
  \]
  En skv. síðustu setningu er $X\setminus\left\{ x_1,\dots,x_m \right\}$ opið
  mengi í $X$ og þar með grennd um $x$, sem sker ekki $A\setminus\left\{ x
  \right\}$. Af því sést að $x$ er ekki þéttipunktur $A$.
\end{proof}
\begin{setn}
  \begin{enumerate}[(i)]
    \item Línulega raðað mengi er Hausdorff-rúm með tilliti til
      röðunargrannmynstursins.
    \item Faldrúm tveggja Hausdorff-rúma er Hausdorff-rúm. 
    \item Hlutrúm í Hausdorff-rúmi er Hausdorff-rúm. 
  \end{enumerate}
\end{setn}
\begin{proof}
  Dæmi 10,11 og 12 á bls. 101 í kennslubók.
\end{proof}

\section{Samfelldar varpanir}
\begin{skilgr}
  Vörpun $f:X\to Y$ milli grannrúma er sögð \emph{samfelld}\index{samfelld
  vörpun} ef frummynd $f$ opnu mengi í $Y$ er opið mengi í $X$; þ.e.a.s.
  $f^{-1}(U)$ er opið í $X$ ef $U$ er opið í $Y$.
\end{skilgr}
\begin{ath}
  (i) $f:X\to Y$ er samfelld \emph{þ.þ.a.a.} vörpunin $\mathcal P(Y)\to \mathcal
  P(X), Z\mapsto f^{-1}(Z)$ varpi $\mathcal T_Y$ inn í $\mathcal T_X$.

  (ii) Ef $\mathcal B$ er grunnur fyrir $\mathcal T_Y$, þá er $f:X\to Y$
  samfelld \emph{þ.þ.a.a.} $f^{-1}(B)$ sé opið í $X$ fyrir sérhvert $B$ úr
  $\mathcal B$: Ef $V$ er opið í $Y$, þá er $V=\bigcup_{i\in I}B_i$ ($B_i\in
  \mathcal B \; \forall i\in I$) og $f^{-1}(V) = \bigcup_{i\in I} f^{-1}(B_i)$.

  (iii) Ef $\mathcal S$ er hlutgrunnur fyrir $\mathcal T_Y$, þá er $f:X\to Y$
  samfelld \emph{þ.þ.a.a.} $f^{-1}(S)$ sé opið í $X$ fyrir sérhvert $S$ úr
  $\mathcal S$: Látum $\mathcal B$ vera grunninn sem $\mathcal S$ framleiðir. Ef
  $B\in\mathcal B$, þá
  \[
  B = S_1\cap\cdots\cap S_m \quad (S_j\in\mathcal S, \; j=1,\dots,m),
  \]
  \[
  f^{-1}(B) = f^{-1}(S_1)\cap\cdots\cap f^{-1}(S_m).
  \]
\end{ath}
\begin{daemi}
  $(X,d_X)$ og $(Y,d_Y)$ firðrúm. Vörpun $f:X\to Y$ er samfelld í
  firðrúmaskilningi \emph{þ.þ.a.a.} $f$ sé samfelld í grannrúmaskilningi.
\end{daemi}
\begin{ath}
  $X$ mengi og $\mathcal T, \mathcal T'$ tvö grannmynstur á $X$. Þá er vörpunin
  \[
  \id_X: (X,\mathcal T') \to (X,\mathcal T), x\mapsto x
  \]
  samfelld \emph{þ.þ.a.a.} $\mathcal T\subseteq \mathcal T'$, þ.e.a.s. $\mathcal
  T'$ sé fínni en $\mathcal T$.
\end{ath}
\begin{setn}
  $X$ og $Y$ grannrúm og $f:X\to Y$. Þá eru eftirfarandi skilyrði jafngild:
  \begin{enumerate}[(i)]
    \item $f$ er samfelld.
    \item Fyrir öll $A\subseteq X$ er $f(\overline A) \subseteq \overline{f(A)}$.
    \item Fyrir sérhvert lokað hlutmengi $B$ í $Y$ er $f^{-1}(B)$ lokað í $X$. 
  \end{enumerate}
\end{setn}
% 18.01.2010
\begin{proof}
  \emph{(i)$\Rightarrow$(ii):} Látum $A\subseteq X$ og $x\in\overline A$. Viljum
  sýna að $f(x)\in\overline{f(A)}$. Ef $V$ er grennd um $f(x)$ í $Y$, þá er
  $f^{-1}(V)$ grennd um $x$ í $X$ (vegna þess að $f$ er samfelld), svo að
  $f^{-1}(V)\cap A\neq \emptyset$ (vegna þess að $x\in\overline A$). Veljum
  punkt $y$ úr $f^{-1}(V)\cap A$. Þá er $f(y)\in f(f^{-1}(V)\cap A) = V\cap
  f(A)$ og þar með er $V\cap f(A)\neq \emptyset$. Við höfum því sýnt að
  $f(x)\in\overline{f(A)}$.

  \emph{(ii)$\Rightarrow$(iii):} Látum $B$ vera lokað í $Y$ og sýnum að
  $f^{-1}(B)\subseteq f^{-1}(B)$. Ljóst er að $f(f^{-1}(B))\subseteq B$ svo ef
  $x\in \overline{f^{-1}(B)}$, þá gildir að 
  \[
  f(x)
  \in
  f(\overline{f^{-1}(B)}) 
  \underset{\text{(ii)}}{\subseteq}
  \overline{f(f^{-1}(B))}
  \subseteq
  \overline B
  = B.
  \]
  Þar með er $x\in f^{-1}(B)$.

  \emph{(iii)$\Rightarrow$(i):} Látum $V$ vera opið í $Y$. Viljum sýna að
  $f^{-1}(V)$ sé opið í $X$. Nú er $Y\setminus V$ lokað í $Y$ svo að 
  \[
  X\setminus f^{-1}(V)
  = f^{-1}(Y)\setminus f^{-1}(V)
  = f^{-1}(Y\setminus V)
  \]
  er lokað í $X$ og því $f^{-1}(V)$ opið í $X$.
\end{proof}
\begin{skilgr}
  $X,Y$ grannrúm og $f:X\to Y$. Við segjum að $f$ sé \emph{samfelld í
  punkti}\index{samfelldni í punkti} $x_0$ úr $X$ ef fyrir sérhverja grennd
  $V$ um $f(x_0)$ í $Y$ er til grennd $U$ um $x_0$ í $X$ þannig að
  $f(U)\subseteq V$.
\end{skilgr}
\begin{ath}[Æfing]
  $X,Y$ grannrúm og $f:X\to Y$. Vörpunin $f$ er samfelld \emph{þ.þ.a.a.}
  $f$ sé samfelld í sérhverjum punkti úr $X$.
\end{ath}


\section{Grannmótanir}
\begin{skilgr}
  $X$ og $Y$ grannrúm og $f:X\to Y$ \emph{gagntæk} vörpun. Ef bæði $f$ og
  $f^{-1}$ eru samfelldar, þá er $f$ kölluð \emph{grannmótun}\index{grannmótun}
  (þ.e. einsmótun milli grannrúma).
\end{skilgr}
\begin{daemi}
  (1) $\R\to\R,x\mapsto x^n$ er grannmótun ef $n$ er oddatala.

  (2) $\arctan:\R\to\left]-\frac\pi 2, \frac\pi 2\right[$ er grannmótun.
\end{daemi}
\begin{skilgr}
  $X$ og $Y$ grannrúm og $f:X\to Y$ samfelld og eintæk. Þá fæst gagntæk vörpun
  $\hat f:X\to f(X)$. Hún er \emph{samfelld}, en ef hún er grannmótun þá kallast
  $f$ \emph{greyping}\index{greyping} og við segjum að $f$ \emph{greypi}
  $X$ í $Y$.
\end{skilgr}
\begin{ath}
  Hér er $f(X)$ að sjálfsögðu með \emph{hlutrúmsgrannmynstrið}.
\end{ath}
\begin{daemi}
  (1) $f:\left]0,2\pi\right[\to\R^2, f(t):=(\cos(t),\sin(t))$ er greyping.

  (2) $g:\left[0,2\pi\right[\to\R^2, g(t):=(\cos(t),\sin(t))$ er \emph{ekki}
  greyping: T.d. er $f\left( \left[ 0,\pi \right[ \right)$ ekki opið í
  $f\left( \left[ 0,2\pi \right[ \right)$.
\end{daemi}

\section{Reglur varðandi samfelldni}
\begin{setn}
  $X,Y$ og $Z$ grannrúm.
  \begin{enumerate}[(i)]
    \item Ef $y_0\in Y$, þá er vörpunin $f:X\to Y, f(x):=y_0$ fyrir öll $x\in
      X$, samfelld.
    \item Ef $A$ er hlutrúm í $X$, þá er ívarpið $j:A\hookrightarrow X, a\mapsto
      a$ samfellt.
    \item Ef $f:X\to Y$ og $g:Y\to Z$ eru samfelldar, þá er $g\circ f:X\to Z$
      samfelld. 
    \item Ef $f:X\to Y$ er samfelld og $A$ er hlutrúm í $X$, þá er einskorðunin
      $f|_A:A\to Y$ samfelld.
    \item Ef $f:X\to Y$ og $B$ hlutrúm í $Y$ þannig að $f(X)\subseteq B$, þá er
      vörpunin $g:X\to B, g(x) := f(x)$ samfelld.
    \item[(v')] Ef $f:X\to Y$ er samfelld og $Y$ er hlutrúm í $Z$, þá er
      $h:X\to Z, x\mapsto f(x)$ samfelld.
    \item  Vörpunin $f:X\to Y$ er samfelld \emph{þ.þ.a.a.} til sé opin þakning
      $(U_\alpha)_{\alpha\in I}$ á $X$ þannig að $f|_{U_\alpha}:U_\alpha\to Y$
      sé samfelld fyirr öll $\alpha$ úr $I$.
  \end{enumerate}
\end{setn}
\begin{proof}
  (i) Ef $V$ er opið í $Y$ þá fæst 
  \[
  f^{-1}(V) =
  \begin{cases}
    \emptyset, & \text{ef $y_0\notin V$},\\
    X,         & \text{ef $y_0\in V$}.
  \end{cases}
  \]

  (ii) Fyrir sérhvert opið $U$ í $X$ er $j^{-1}(U) = A\cap U$ opið í $A$ skv.
  skilgreiningu á hlutrúmsgrannmynstrinu.

  (iii) $V$ opið í $Z$. Þá fæst 
  \[
  (g\circ f)^{-1}(V)
  = f^{-1}(g^{-1}(V))
  \]
  opið í $x$ vegna þess að $f$ og $g$ eru samfelldar.

  (iv) Skv. (ii) er ívarpið $j_A:A\hookrightarrow X$ samfellt svo skv. (iii) er
  $f|_A = f\circ j_A$ samfelld.

  (v) Fyrir $U$ opið í $B$ er til opið mengi $V$ í $Y$ þannig að $U = B\cap Y$.
  En þar sem $f(X)\subseteq B$, þá gildir að 
  \[
  g^{-1}(U)
  = g^{-1}(B\cap V)
  = f^{-1}(B\cap V)
  = f^{-1}(V)
  \]
  sem er opið í $X$.

  (v') $h=j_Y \circ f$, þar sem $j_Y : Y\hookrightarrow Z$ er ívarpið. Þar með
  er $h$ samfelld skv. (iii).

  (vi) Vitum: Ef $f$ samfelld og $(U_\alpha)_{\alpha\in I}$ opin þakning, þá er
  $f|_{U_{\alpha}}:U_\alpha\to Y$ samfelld fyrir öll $\alpha\in I$. Öfugt, látum
  $(U_\alpha)_{\alpha\in I}$ vera opna þakningu á $X$ og sýnum að $f$ sé
  samfelld ef $f|_{U_\alpha}$ er samfelld $\forall\,\alpha\in I$. Sönnum á tvo
  vegu:

  \emph{1. sönnun:} Látum $x\in X$ og $V$ vera grennd um $f(x)$ í $Y$. Til er
  $\alpha$ úr $I$ þannig að $x\in U_\alpha$ og þar sem $f|_{U_\alpha}$ er
  samfelld þá er til opin grennd $W$ um $x$ í $U_\alpha$ þannig að
  $f|_{U_\alpha}(W)\subseteq V$. En $U_\alpha$ er opið í $X$ svo að $W$ er opin
  grennd um $x$ í $X$ og $f|_{U_\alpha}(W)=f(W)$.

  \emph{2. sönnun:} Látum $V$ vera opið mengi í $Y$. Þá fæst
  \[
  f^{-1}(V)
  = f^{-1}(V)\cap \underbrace{\bigcup_{\alpha\in I} U_\alpha}_{X}
  = \bigcup_{\alpha\in I} (f^{-1}(V)\cap U_\alpha)
  = \bigcup_{\alpha\in I} \left( f|_{U_\alpha} \right)^{-1}(V)
  \]
  sem er opið í $X$.
\end{proof}
\begin{setn}
  [Samlíming]\index{samlíming}
  $X$ grannrúm og $A,B$ lokuð hlutmengi í $X$ þannig að $X= A\cup B$. Látum
  $Y$ vera grannrúm og $f:A\to Y$, $g:B\to Y$ vera samfelldar. Ef $f|_{A\cap B}
  = g_{A\cap B}$, þá er vörpunin $h:X\to Y$, 
  \[
  h(x) :=
  \begin{cases}
    f(x) & \text{ef $x\in A$}\\
    g(x) & \text{ef $x\in B$}
  \end{cases}
  \]
  samfelld.
\end{setn}
\begin{proof}
  Fyrir sérhvert lokað mengi $C$ í $Y$ er 
  \begin{align*}
    h^{-1}(C)
    &= h^{-1}(C)\cap (A\cup B) \\
    &= (h^{-1}(C)\cap A)\cup (h^{-1}(C)\cap B) \\
    &= (h|_A)^{-1}(C)\cup (h|_B)^{-1}(C) \\
    &= f^{-1}(C)\cup g^{-1}(C)
  \end{align*}
  sem er lokað í $X$ vegna þess að $f^{-1}(C)$ er lokað í $A$ sem er lokað í
  $X$ og $g^{-1}(C)$ er lokað í $B$, sem er lokað í $X$.
\end{proof}
\begin{setn}
  $A,X,Y$ grannrúm, $\pi_1:X\times Y\to X$ og $\pi_2:X\times Y\to Y$ venjulegu
  ofanvörpin. Vörpun $f:A\to X\times Y$ er samfelld \emph{þ.þ.a.a.} $\pi_1\circ
  f$ og $\pi_2\circ f$ séu samfelldar.
\end{setn}
\begin{proof}
  Ef $f$ er samfelld, þá eru $\pi_1\circ f$ og $\pi_2\circ f$ samfelldar vegna
  þess að $\pi_1$ og $\pi_2$ eru samfelldar.

  Öfugt, ef $\pi_1\circ f$ og $\pi_2\circ f$ eru samfelldar, þá gildir um
  sérhvert grunnmengi $U\times V$ þar sem $U$ er opið í $X$ og $V$ er opið í
  $Y$, að 
  \[
  f^{-1}(U\times V)
  = (\pi_1\circ f)^{-1}\cap (\pi_2\circ f)^{-1}(V)
  \]
  sem er opið í $A$.
\end{proof}

% 19.01.2010
\section{Kartesísk margfeldi grannrúma}
\begin{skilgr}
  $(X_\alpha)_{\alpha\in J}$ fjölskylda af grannrúmum.

  \begin{enumerate}[(i)]
    \item Mengin $\prod_{\alpha\in J}U_\alpha$ þar sem $U_\alpha$ er opið í
      $X_\alpha$ mynda grunn fyrir grannmynstur á $\prod_{\alpha\in J}X_\alpha$.
      Umrætt grannmynstur kallast
      \emph{kassagrannmynstrið}\index{kassagrannmynstur} á $\prod_{\alpha\in
      J}X_\alpha$.
    \item Setjum fyrir sérhvert $\beta\in J$
      \[
      \mathcal S_{\beta} 
      := \left\{ \pi^{-1}_\beta (U_\beta) : U_\beta\text{ opið í }
      X_\beta\right\}
      \]
      og $\mathcal S := \bigcup_{\beta\in J} \mathcal S_\beta$. Grannmynstrið
      sem er framleitt af hlutgrunninum $\mathcal S$ kallast
      \emph{faldgrannmynstrið}\index{faldgrannmynstur} á $\prod_{\alpha\in J}
      X_{\alpha}$. Mengið $\prod_{\alpha\in J}X_\alpha$ með þessu grannmynstri
      kallast \emph{faldrúm}\index{faldrúm} grannrúmanna $X_\alpha$.
  \end{enumerate}
  \end{skilgr}
\begin{ath}
  (i) Þar sem grunnur er búinn til út frá hlutgrunni með því að taka öll
  \emph{endanleg} sniðmengi, þá fæst:
  \begin{itemize}
    \item Grunnur fyrir kassagrannmynstrið samanstendur af öllum hlutmengjum af
      gerðinni $\prod_{\alpha\in J}U_\alpha$ með $U_\alpha$ opið í $X_\alpha$.
    \item Grunnur fyrir faldgrannmynstrið samanstendur af öllum hlutmengjum af
      gerðinni $\prod_{\alpha\in J}U_\alpha$, þar sem $U_\alpha$ er opið í
      $X_\alpha$ og $U_\alpha = X_\alpha$ fyrir öll nema endanlega mörg
      $\alpha$ úr $J$ (m.ö.o. fyrir næstum öll). 
  \end{itemize}

  (ii) Kassagrannmynstrið og faldgrannmynstrið eru eins ef $J$ er
  \emph{endanlegt} mengi.
\end{ath}
\begin{setn}
  $(X_\alpha)_{\alpha\in J}$ fjölskylda grannrúma og $\mathcal B_{\alpha}$
  grunnur fyrir grannmynstur á $X_\alpha$ fyrir sérhvert $\alpha\in J$. 
  \begin{enumerate}[(i)]
    \item Mengin $\prod_{\alpha\in J}B_\alpha$ þar sem $B_\alpha\in\mathcal
      B_\alpha\;\forall\,\alpha\in J$ mynda grunn fyrir kassagrannmynstrið á
      $\prod_{\alpha\in J}X_\alpha$.
    \item Mengin $\prod_{\alpha\in J}B_\alpha$, þar sem $B_\alpha\in\mathcal
      B_\alpha \cup \left\{ X_\alpha \right\}\; \forall\,\alpha$ og $B_\alpha =
      X_\alpha$ fyrir næstum öll $\alpha$ úr $J$, mynda grunn fyrir
      faldgrannmynstrið. 
  \end{enumerate}
\end{setn}
\begin{proof}
  Æfing!
\end{proof}
\begin{setn}
  $(X_\alpha)_{\alpha\in J}$ fjölskylda af grannrúmum. Faldrúmið
  $\prod_{\alpha\in J}X_\alpha$ uppfyllir eftirfarandi allsherjareiginleika:

  Fyrir sérhvert $\beta\in J$ látum við $\pi_\beta:\prod X_\alpha\to X_\beta$
  tákna náttúrlega ofanvarpið, þ.e.a.s. ef $(x_\alpha)_{\alpha\in
  J}\in\prod_{\alpha\in J}X_\alpha$, þá $\pi_\beta( (x_\alpha)_{\alpha\in J})
  = x_\beta$. Tvenndin $\left( \prod_{\alpha\in J}X_\alpha,
  (\pi_\alpha)_{\alpha\in J} \right)$ fullnægir:

  \begin{quote}
    Látum $Z$ vera grannrúm og $f:Z\to \prod_{\alpha\in J}X_\alpha$. Þá er
    $f$ samfelld \emph{þ.þ.a.a.} vörpunin $\pi_\alpha\circ f$ sé samfelld
    fyrir sérhvert $\alpha\in J$.
  \end{quote}
\end{setn}
\begin{proof}
  Ef $f:Z\to\prod_{\alpha\in J}X_\alpha$ er samfelld, þá er
  $f_\alpha:=\pi_\alpha\circ f$ samfelld fyrir öll $\alpha$ vegna þess að
  $\pi_\alpha$ er augljóslega samfelld.

  Ef $f_\alpha$ er samfelld fyrir öll $\alpha$ úr $J$ og $U$ er opið í
  $X_\beta$ fyrir sérhvert $\beta$ úr $J$, þá er $f^{-1}(\pi^{-1}_\beta
  (U)) = f^{-1}_\beta (U)$ opið í $Z$. En mengi af gerðinni $\pi^{-1}_\beta(U)$
  mynda hlutgrunn fyrir faldgrannmynstrið, svo að $f$ er samfelld.
\end{proof}
Með allsherjareiginleika einhvers fyrirbrigðis er átt við eiginleika sem
ákvarðar það burtséð frá \emph{einkvæmt ákvarðaðri} einsmótun.

Gerum ráð fyrir að $(Y,(g_\alpha)_{\alpha\in J})$ sé þannig að $Y$ sé grannrúm
og fyrir sérhvert $\alpha$ úr $J$ sé $g_\alpha:Y\to X_\alpha$ samfelld og um
sérhvert grannrúm $Z$ og sérhverja vörpun $f:Z\to Y$ gildi að hún er samfelld
\emph{þ.þ.a.a.} allar varpanirnar $g_{\alpha}\circ f$ séu samfelldar. Þá er til
nákvæmlega ein grannmótun $h:Y\to\prod_{\alpha\in J}X_\alpha$ sem uppfyllir
$g_\alpha = \pi_\alpha\circ h$ fyrir öll $\alpha$ úr $J$.
\begin{proof}
  \[
  \xymatrix{
     & X_\beta 
     \\
     Y \ar[r]^{\exists ! g}\ar@/^/[ur]^{g_\beta}\ar@/_/[dr]_{g_\gamma}
     & \prod_{\alpha\in J} X_\alpha \ar[u]^{\pi_\beta}\ar[d]_{\pi_\gamma} 
     \\
     & X_\gamma
  }
  \]
  \emph{Ath.} Það er víst eitthvað vitlaust í þessari efnisgrein eftir síðustu
  sönnun. Jón Ingólfur lætur okkur fá dæmi úr þessu.
\end{proof}
\begin{setn}
  Látum $A_\alpha$ vera hlutrúm í $X_\alpha$ fyrir sérhvert $\alpha$ úr
  $J$.
  \begin{enumerate}[(i)]
    \item $\prod_{\alpha\in J}$ með kassagrannmynstrinu eru hlutrúm í
      $\prod_{\alpha\in J} X_\alpha$ með kassagrannmynstrinu.
    \item Faldrúmið $\prod_{\alpha\in J} A_\alpha$ er hlutrúm í faldrúminu
      $\prod_{\alpha\in J}X_\alpha$. 
  \end{enumerate}
\end{setn}
\begin{proof}
  Æfing.
\end{proof}
\begin{setn}
  $(X_\alpha)_{\alpha\in J}$ fjölskylda af Hausdorff-rúmum, þá er
  $\prod_{\alpha\in J}X_\alpha$ Hausdorff-rúm, hvort sem er í kassa- eða
  faldgrannmynstrinu.
\end{setn}
\begin{proof}
  Æfing (nægir að sanna fyrir faldgrannmynstrið, því það er fínna).
\end{proof}
\begin{daemi}
  Setjum $\R^\omega := \prod_{n\in\N^*} X_n$  með $X_n = \R$ fyrir öll $n$. Setjum
  $f: \R\to\R^{\omega}, t\mapsto (t,t,t,\dots)$ þ.e.a.s. $f = (f_n)_{n\in\N^*}$
  með $f_n(t) = t$ fyrir öll $n\in\N^*$ og öll $t$ úr $\R$. Ljóst er að $f$ er
  samfelld ef $\R^\omega$ er með faldgrannmynstrið. $f$ er hins vegar ekki
  samfelld ef $\R^\omega$ er með kassagrannmynstrið:
  \[
  V := \left] -1,1 \right[\times\prod_{n\geq 1}\left] -\frac 1n,\frac 1n \right[
  \]
  er opið í kassagrannmynstrinu, en 
  \[
  f^{-1}(V) 
  = \bigcap_{n\geq 1} \left] -\frac 1n,\frac 1n \right[ 
  = \left\{ 0 \right\}
  \]
  sem er ekki opið í $\R$.
\end{daemi}

\section{Grannfræði firðrúma}
Látum $(X,d)$ vera firðrúm. Munum að opnu mengin í $X$ (í firðrúmaskilningi)
mynda grannmynstur, þar sem opnu kúlurnar mynda grunn. Það kallast
\emph{grannmynstrið sem $d$ framleiðir}.
\begin{skilgr}
  Grannrúm $X$ er sagt \emph{firðanlegt}\index{firðanlegt grannrúm} ef til er
  firð á $X$ sem framleiðir grannmynstrið.
\end{skilgr}
% 25.10.2010
\begin{skilgr}
  $(X,d)$ firðrúm. Hlutmengi $A$ í $X$ er sagt \emph{takmarkað}\index{takmarkað
  hlutmengi í firðrúmi} ef $\diam(A) := \sup\left\{ d(a,b):a,b\in A
  \right\}<+\infty$.
\end{skilgr}
\begin{ath}
  \emph{Takmörkun} er ekki grannfræðilegt hugtak, þ.e.a.s. grannmynstur á
  tilteknu mengi getur verið framleitt á ólíkum firðum þannig að ákveðið
  hlutmengi sé takmarkað í sumum þeirra, en ekki í öðrum.
\end{ath}
\begin{setn}
  $(X,d)$ firðrúm og setjum\[
  \overline d: X\times X\to\R_+, (x,y)\mapsto\min\left\{ d(x,y) , 1 \right\}.
  \]
  Þá er $\overline d$ firð á $X$ sem framleiðir sama grannmynstur og $d$ á
  $X$.
\end{setn}
\begin{skilgr}
  Firðin $\overline d$ kallast \emph{staðlaða takmarkaða firðin}\index{staðlaða
  takmarkaða firðin} sem tilheyrir $d$.
\end{skilgr}
\begin{proof}
  [Sönnun á síðustu setningu]
  Ljóst að $\overline d(x,y) \geq 0$ og að $\overline d(x,y) = \overline d(y,x)$
  fyrir öll $x,y$ úr $X$. Sýnum nú að fyrir öll $x,y,z$ úr $X$ gildi\[
  \overline d(x,z)\leq \overline d(x,y)+ \overline d(y,z).
  \]
  Ef $d(x,y)\geq 1$ eða $d(y,z)\geq 1$, þá er $\overline d(x,z)\leq 1 \leq
  \overline d(x,y) + \overline d(y,z)$. Ef hins vegar $d(x,y) < 1$ og
  $d(y,z)<1$, þá fæst\[
  \overline d(x,z)
  \leq d(x,z)
  \leq d(x,y) + d(y,z)
  = \overline d(x,y) + \overline d(y,z).
  \]
  Ennfremur gildir: Ef $\varepsilon > 0$, þá $B_d(x,\varepsilon)\subseteq
  B_{\overline d}(x,\varepsilon)$ og ef $\delta \leq \min\left\{ \varepsilon,1
  \right\}$ þá $B_{\overline d}(x,\delta)\subseteq B_d(x,\varepsilon)$. Þar með
  er sýnt að firðirnar framleiða sama grannmynstur.
\end{proof}
\begin{skilgr}
  Látum $(V,\|\cdot\|)$ vera staðalrúm (staðlað vigurrúm)\index{staðalrúm} og
  skilgreinum firð á $V$ með $d:V\times V\to \left[ 0,+\infty \right[,
  (x,y)\mapsto \|x-y\|$. Við segjum að grannmynstrið sem þessi firð framleiðir
  sé \emph{framleitt af staðlinum}. Segjum að tveir staðlar séu
  \emph{jafngildir} ef þeir framleiða sama grannmynstrið.
\end{skilgr}
\begin{setn}
  $(V,\|\cdot\|)$ staðlað rúm. 
  \begin{enumerate}[(i)]
    \item Línuleg vörpun $L:V\to V$ er samfelld \emph{þ.þ.a.a.}
      til sé fasti $k>0$ þ.a.  $\|L(x)\|\leq k\cdot\|x\|$ fyrir öll $x$ úr $V$.
    \item Staðall $\|\cdot\|_1$ á $V$ er jafngildur $\|\cdot\|$ \emph{þ.þ.a.a.}
      til séu fastar $m>0$ og $M>0$ þ.a. $m\cdot\|x\|\leq \|x\|_1\leq
      M\cdot\|x\|$ fyrir öll $x$ úr $V$.
    \item Allir staðlar á endanlega víðu $\R$-vigurrúmi eru jafngildir. 
  \end{enumerate}
\end{setn}
\begin{proof}
  Æfing (sjá dæmablað 4).
\end{proof}
\begin{setn}
  Sérhver staðall á $\R^n$ framleiðir faldgrannmynstrið.
\end{setn}
\begin{proof}
  Nóg að sanna niðurstöðuna fyrir staðalinn\[
  \|x\|:=\max\left\{ |x_1|,\dots,|x_n| \right\}
  \]
  fyrir sérhvert $x = (x_1,\dots,x_n)$ úr $\R^n$. Látum
  $\rho:\R^n\times\R^n\to\left[ 0,+\infty \right[$ vera tilheyrandi firð. Þá
  gildir fyrir sérhvert $x=(x_1,\dots,x_n)$ úr $\R^n$ og sérhvert $\varepsilon
  >0$ að\[
  B_\rho(x,\varepsilon) 
  = \left] x_1-\varepsilon,x_1+\varepsilon \right[
  \times\cdots\times\left] x_n-\varepsilon,x_n+\varepsilon \right[
  \]
  tilheyrir grunninum sem framleiðir faldgrannmynstrið. Hins vegar gildir fyrir
  sérhvert hlutmengi\[
  B = \left] a_1,b_1 \right[\times\cdots\times\left] a_n,b_n \right[
  \]
  úr umræddum grunni og $x = (x_1,\dots,x_n)$ úr $B$ að til eru
  $\varepsilon_1,\dots,\varepsilon_n>0$ þ.a.\[
  \left] x_i-\varepsilon,x_i+\varepsilon \right[
  \subseteq
  \left] a_i,b_i\right[
  \]
  fyrir $i=1,\dots,n$. Setjum $\varepsilon:=\min\left\{
  \varepsilon_1,\dots,\varepsilon_n \right\}$ og fáum
  $B_\rho(x,\varepsilon)\subseteq B$.
\end{proof}
\begin{skilgr}
  Látum $d$ vera venjulegu firðina á $\R$ og $\overline d$ vera tilheyrandi
  staðlaða takmarkaða firð. Fyrir mengið $J$ skilgreinum við firðina $\overline
  \rho$ á $\R^J$ með\[
  \overline\rho(x,y)
  := \sup\left\{ \overline d(x_\alpha,y_\alpha) : \alpha\in J \right\}
  \]
  þar sem $x = (x_\alpha)_{\alpha\in J}$ og $y = (y_\alpha)_{\alpha\in J}$.
  Köllum þessa firð \emph{jafnmælisfirðina}\index{jafnmælisfirðin}
  á $\R^J$ og tilheyrandi grannmynstur
  \emph{jafnmælisgrannmynstrið}\index{jafnmælisgrannmynstrið} á $\R^J$.
\end{skilgr}
\begin{setn}
  \emph{Jafnmælisgrannmynstrið} á $\R^J$ er stranglega fínna en
  faldgrannmynstrið ef $J$ er óendanlegt mengi.
\end{setn}
\begin{proof}
  Látum $\prod_{\alpha\in J}U_\alpha$ vera grunnmengi fyrir faldgrannmynstrið og
  $x = (x_\alpha)_{\alpha\in J}\in\prod_{\alpha\in J}U_\alpha$. Látum
  $\alpha_1,\dots,\alpha_n$ vera þau stök ú $J$ þ.a. $U_\alpha\neq \R$. Þá eru
  til $\varepsilon_1,\dots,\varepsilon_n>0$ þ.a. $B_{\overline
  d}(X_{\alpha_i},\varepsilon_i)\subseteq U_{\alpha_i}$ fyrir $i=1,\dots,n$.
  Setjum $\varepsilon := \min \left\{ \varepsilon_1,\dots,\varepsilon_n
  \right\}$ og fáum að $B_{\overline
  \rho}(x,\varepsilon)\subseteq\prod_{\alpha\in J}U_\alpha$. Hins vegar er ljóst
  að $B_{\overline\rho}(x,1/2)$ getur ekki innihaldið mengi af gerðinni
  $\prod_{\alpha\in J}U_\alpha$ þar sem $U_\alpha = \R$ fyrir eitthvert
  $\alpha$ úr$J$.
\end{proof}
\begin{setn}
  Látum $\overline d$ vera stöðluðu takmörkuðu firðina á $\R$ og setjum
  \[
  D:\R^\omega\times\R^\omega\to\R,
  (x,y)\mapsto\sup\left\{ \frac{\overline d(x_i,y_i)}{i} : i\in\N^* \right\},
  \]
  þar sem $x = (x_i)_{i\in\N^*}$ og $y=(y_i)_{i\in\N^*}$. Þá er $D$ firð sem
  framleiðir faldgrannmynstrið á $\R^\omega$.
\end{setn}
\begin{proof}
  (1) \emph{$D$ er firð:} Ljóst að $D(x,y)\geq 0$, $D(x,y)=0$ þ.þ.a.a.
  $x= y$; og $D(x,y) = D(y,x)$ fyrir öll $x,y$ úr $\R^\omega$. Látum $x =
  (x_i)$, $y=(y_i)$ og $z=(z_i)$. Þá gildir fyrir sérhvert $i$ úr $\N^*$ að\[
  \frac{\overline d(x_i,z_i)}{i}
  \leq \frac{\overline d(x_i,y_i)}{i} + \frac{\overline d(y_i,z_i)}{i}
  \leq D(x,y)+D(y,z)
  \]
  og þar með $D(x,z) = \sup_{i\in\N^*}\frac{\overline d(x_i,z_i)}{i}\leq
  D(x,y)+D(y,z)$.

  (2) \emph{$D$ framleiðir faldgrannmynstrið:} Látum $U$ vera opið í
  firðgrannmynstrinu og $x=(x_i)\in U$. Sýnum að til sé opið mengi $V$ í
  faldgrannmynstrinu þ.a. $x\in V\subseteq U$. Veljum $\varepsilon$-kúlu
  $B_D(x,\varepsilon)$ og veljum svo $N$ úr $\N^*$ þannig að $1/N<\varepsilon$.
  Setjum \[
  V := \left] x_1-\varepsilon,x_1+\varepsilon \right[\times\cdots\times\left]
  x_N-\varepsilon,x_N+\varepsilon \right[\times\R\times\R\times\cdots
  \]
  Fyrir öll $y = (y_i)_{i\in\N^*}$ úr $\R^\omega$ og öll $i\geq N$ gildir að
  \[
  \frac{\overline d(x_i,y_i)}{i}\leq \frac 1N,
  \]
  svo að\[
  D(x,y) \leq\max\left\{ \frac{\overline d(x_1,y_1)}{1}, \frac{\overline
  d(x_2,y_2}{2},\dots,\frac{\overline d(x_N,y_N)}{N}, \frac{1}{N} \right\}.
  \]
  Þar með er $D(x,y)<\varepsilon$ fyri öll $y$ úr $V$ og því $V\subseteq
  B_D(x,\varepsilon)$.

  Öfugt, látum $U = \prod_{n\in\N^*}U_n$ vera grunnmengi fyrir faldgrannmynstrið
  á $\R^\omega$ og $x = (x_n)$ vera stak úr $U$. Viljum sýna að til sé opið
  mengi í firðgrannmynstrinu þ.a. $x\in V\subseteq U$. Látum $I$ vera endanlegt
  hlutmengi í $\N^*$ þ.a. $U_n = \R$ ef $n\notin I$. Fyrir sérhvert $n$ úr
  $I$ veljum við $\varepsilon_n$ þ.a. $0<\varepsilon_n<1$ og 
  $\left] x_n-\varepsilon_n,x_n+\varepsilon_n \right[\subseteq U_n$. Setjum svo
  $\varepsilon := \min\left\{ \varepsilon_n/n : n\in I \right\}$. Þá er fljótséð
  að $B_D(x,\varepsilon)\subseteq U$.
\end{proof}
\begin{skilgr}
  Runa $(x_n)$ í grannrúmi $X$ er sögð \emph{samleitin að
  punkti\index{samleitni runu að punkti} $x$ úr $X$} ef fyrir sérhverja grennd
  $U$ um $x$ er til jákvæð heil tala $N$ þ.a. $x_n\in U$ fyrir öll $n\geq N$.
  Köllum þá $x$ \emph{markgildi}\index{markgildi runu} rununnar. Táknum þetta
  $x_n\longrightarrow x$ eða $\lim_{n\to+\infty}x_n = x$.
\end{skilgr}
\begin{ath}
  Í Hausdorff-rúmi getur runa í mesta lagi haft eitt markgildi.
\end{ath}
\begin{skilgr}
  $X$ grannrúm og $x\in X$. 
  \begin{enumerate}[(i)]
    \item Setjum að $x$ hafi \emph{teljanlegan grenndagrunn\index{teljanlegur
      grenndagrunnur} í} $X$ eða að $X$ hafi \emph{teljanlegan grenndagrunn um}
      $x$ ef til er safn $(U_n)_{n\in\N}$ af grenndum um $x$ þannig að sérhver
      grennd $U$ um $x$ innihaldi a.m.k. eitt $U_n$.
    \item Segjum að $X$ fullnægi \emph{fyrstu teljanleika-frumsendu
      (f.t.f)}\index{f.t.f.}\index{fyrsta teljanleika-frumsendan} ef $X$ hefur
      teljanlegan grenndagrunn um sérhvern punkt sinn.
  \end{enumerate}
\end{skilgr}
\begin{ath}
  Firðrúm fullnægir f.t.f.
\end{ath}
% 26.01.2010
\begin{setn}
  $X$ grannrúm, $A\subseteq X$ og $x\in X$.
  \begin{enumerate}[(i)]
    \item Ef til er runa $(x_n)$ í $A$ þ.a. $x_n\longrightarrow x$, þá er
      $x\in\overline A$.
    \item Ef $X$ fullnægir f.t.f., þá gildir öfugt: Ef $x\in\overline A$, þá er
      til runa $(x_n)$ í $A$ þ.a. $x_n\longrightarrow x$. 
  \end{enumerate}
\end{setn}
\begin{proof}
  (i) Látum $(x_n)$ vera runu í $A$ þ.a. $x_n\longrightarrow x$. Þá inniheldur
  sérhver grennd um $x$ punkt úr $A$ svo að $x\in\overline A$.

  (ii) G.r.f. að $X$ uppfylli f.t.f. og $x\in\overline A$. Látum
  $(U_n)_{n\geq 1}$ vera teljanlegan grenndagrunn um $x$ í $X$ og setjum
  $U_n' := U_1\cap\cdots\cap U_n$ fyrir öll $n\geq 1$. Veljum svo fyrir sérhvert
  $n$ punkt $x_n$ úr $A\cap U_n'$. Þá er ljóst að $x_n\longrightarrow x$.
\end{proof}
\begin{setn}
  $X,Y$ grannrúm og g.r.f. að $X$ fullnægi f.t.f. Vörpun $f:X\to Y$ er samfelld
  \emph{þ.þ.a.a.} um sérhverja samleitna runu $(x_n)$ í $X$ gildi að
  \begin{equation}
    \lim_{n\to\infty}f(x_n) = f\left(\lim_{n\to\infty}x_n\right)
    \label{eq:ftf_limit}
  \end{equation}
\end{setn}
\begin{proof}
  G.r.f. að $f$ sé samfelld, $x_n\longrightarrow x$, og $V$ sé grennd um
  $f(x)$ í $Y$. Þá er $f^{-1}(V)$ grennd um $x$ (vegna þess að $f$ er samfelld),
  svo að til er $N$ þ.a. $x_n\in f^{-1}(V)$ $\forall\,n\geq N$ og þar með
  $f(x_n)\in V$ $\forall\,n\geq N$.

  Öfugt, g.r.f. að um allar samleitnar runur $(x_n)$ í $X$ gildi
  \eqref{eq:ftf_limit}. Látum $C$ vera lokað mengi í $Y$ og $x\in
  \overline{f^{-1}(C)}$. Viljum sýna að $x\in f^{-1}(C)$. Veljum runu $(x_n)$ úr
  $f^{-1}(C)$ þ.a. $x_n\longrightarrow x$. Þá gildir að $f(x_n)\longrightarrow
  f(x)$, og þar með er $f(x)\in C$ vegna þess að $C$ er lokað, og þar með er
  $x\in f^{-1}(C)$.
\end{proof}
\begin{setn}
  $V$ staðalrúm (yfir $\R$). Þá eru varpanirnar $\R\times V\to V,
  (\lambda,x)\mapsto \lambda x$ og $V\times V\to V, (x,y)\mapsto x+y$ samfelldar
  ($\R\times V$ og $V\times V$ með faldgrannmynstri).
\end{setn}
\begin{proof}
  Æfing.
\end{proof}
\begin{setn}
  $X$ grannrúm og $f,g:X\to\R$ samfelldar. Þá eru $f+g$, $f-g$ og $f\cdot g$ frá
  $X$ til $\R$ líka samfelldar.
\end{setn}
\begin{proof}
  $h:X\to\R\times\R$, $h(x) := (f(x),g(x))$ er samfelld. Notum svo annars vegar
  að samskeyting samfelldra varpana er samfelld vörpun og hins vegar síðustu
  setningu.
\end{proof}
\begin{skilgr}
  $X$ mengi, $(Y,d)$ firðrúm og $(f_n)$ runa af vörpunum frá $X$ í $Y$. Við
  segjum að runan $(f_n)$ sé \emph{samleitin í jöfnum mæli}\index{samleitni í
  jöfnum mæli} (j.m) að vörpun $f:X\to Y$ ef fyrir sérhvert $\varepsilon > 0$ er
  til jákvæð heil tala $N$ þannig að $d(f_n(x),f(x))<\varepsilon$ fyrir sérhvert
  $x$ úr $X$ og öll $n\geq N$.
\end{skilgr}
\begin{setn}
  $(f_n)$ runa af samfelldum vörpunum frá grannrúmi $X$ yfir í firðrúm
  $Y$. Ef $(f_n)$ er samleitin í j.m. að vörpun $f:X\to Y$, þá er $f$ samfelld.
\end{setn}
\begin{proof}
  (Þekkt) æfing.
\end{proof}
\begin{daemi}
  (1) $\R^\omega$ með kassagrannmynstrinu er \emph{ekki} firðanlegt. Setjum\[
  A := \left\{ (x_1,x_2,\dots)\in\R^\omega : x_n>0\;\forall\,n\in\N^* \right\}.
  \]
  Þá er $\mathbf 0 = \left( 0,0,\dots \right)\in \overline A$ vegna þess að
  sé $\mathbf 0\in B = \left] a_1,b_1 \right[\times\left] a_2,b_2
  \right[\times\cdots$, þá er $(\frac 12 b_1, \frac 12 b_2,\dots)\in B\cap A$.
  Hins vegar er ekki til nein runa í $A$ sem hefur $\mathbf 0$ sem markgildi:
  Látum $(\mathbf a_n)$ vera runu í $A$ og skrifum $\mathbf a_n =
  (a_{1n},a_{2n},\dots,a_{in},\dots)$ með $a_{in}>0$ $\forall\,i$ og $\forall\,n$.
  Mengið $B' := \left] -a_{11},a_{11} \right[\times\left] -a_{22},a_{22}
  \right[\times\cdots$ er þá grennd um $\mathbf 0$ sem inniheldur ekki neitt
  $\mathbf a_n$, því að $a_{nn}\notin \left]-a_{nn},a_{nn}\right[$ fyrir
  sérhvert $n$.

  (2) $\R^{J}$ með faldgrannmynstrinu er ekki firðanlegt ef $J$ er óteljanlegt.
  Látum $J$ vera óteljanlegt og látum $A$ vera hlutmengi allra
  $(x_{\alpha})_{\alpha\in J}$ úr $\R^J$ þ.a. $x_\alpha = 0$ fyrir endanlega
  mörg $\alpha$ og $x_\alpha = 1$ fyrir hin $\alpha$. Þá er $\mathbf 0\in
  \overline A$: Ef $\prod_{\alpha\in J} U_\alpha$ er grunngrennd um $\overline
  0$ með $U_\alpha = \R$ fyrir öll $\alpha$ úr $J\setminus I$, þar sem
  $I$ er endanlegt hlutmengi í $J$, þá eru allir punktar $(x_\alpha)_{\alpha\in
  J}$ úr $A$ sem uppfylla $x_\alpha = 0$ ef $\alpha\in I$ líka í
  $\prod_{\alpha\in J} U_\alpha$.

  Sýnum nú að engin runa í $A$ hafi $\mathbf 0$ sem markgildi: Látum
  $(\mathbf a_n)$ vera runu í $A$ og skrifum $\mathbf a_n =
  (a_{n\alpha})_{\alpha\in J}$. Skilgreinum mengi $J_n := \left\{ \alpha\in J :
  a_{n\alpha} = 0 \right\}$. Þá er $J_n$ endanlegt og því $\bigcup_{n} J_n$
  teljanlegt, þar með er $J\setminus\bigcup_n J_n \neq \emptyset$. Veljum
  $\beta$ úr $J\setminus \bigcup_n J_n$ og fáum $a_{n\beta}=1$ fyrir öll $n$.
  Þar með inniheldur opna grenndin $\pi_\beta^{-1}\left( \left] -1,1 \right[
  \right)$ ekkert $\mathbf a_n$.
\end{daemi}


\section{Deildagrannrúm}
\emph{Upprifjun:} $X$ mengi, $R$ jafngildisvensl á $X$. $R$ skilgreinir
\emph{deildaskiptingu}\index{deildaskipting} á $X$ (skiptingu ekki tómra
hlutmengja $X_i$ í $X$, $i\in I$, þ.a. $X_i\cap X_j=\emptyset$ ef $i\neq j$ og
$\bigcup_{i\in I} X_i = X$). Öfugt, ef $(X_i)_{i\in I}$
er deildaskipting á $X$, þá skilgreinir hún jafngildisvensl $R$: Segjum að
$xRy$ þ.þ.a.a. $x$ og $y$ séu úr sömu deild.
Táknum með $X/R$ mengið af deildunum, þ.e.a.a. $X/R = \left\{ X_i: i\in I
\right\}$, og ef $x\in X$, þá látum við $[x]$ tákna tilheyrandi deild.

\begin{skilgr}
  Látum $X$ vera grannrúm og $R$ vera jafngildisvensl á $X$. Látum $\pi: X\to
  X/R$ vera ofanvarpið á deildamengið sem $R$ skilgreinir. Skilgreinum
  grannmynstur á $X/R$ með því að $U\subseteq X/R$ sé opið \emph{þ.þ.a.a.}
  $\pi^{-1}(U)$ sé opið í $X$. Köllum þetta
  \emph{deildagrannmynstrið}\index{deildagrannmynstur} á $X/R$ og segjum að
  $X/R$ sé \emph{deildagrannrúm $X$ m.t.t. $R$}\index{deildagrannrúm}.
\end{skilgr}
\begin{ath}
  Deildagrannmynstrið á $X/R$ er \emph{fínasta} grannmynstrið á $X/R$ þ.a.
  ofanvarpið $\pi: X\to X/R$ sé samfellt!
\end{ath}
\begin{setn}
  $Y$ grannrúm og $X,R$ eins og áður. Vörpun $f:X/R\to Y$ er samfelld
  \emph{þ.þ.a.a.} $f\circ \pi: X\to Y$ sé samfelld.
  \[
  \xymatrix{
  X\ar[d]_\pi \ar[r]^{f\circ\pi}&  Y\\
  X/R \ar[ur]_f &
  }
  \]
\end{setn}
\begin{proof}
  Ef $V$ er opið í $Y$, þá er $f^{-1}(V)$ opið í $X/R$ \emph{þ.þ.a.a.}
  $(f\circ \pi)^{-1}(V) = \pi^{-1}(f^{-1}(V))$ sé opið í $X$.
\end{proof}
\begin{fylgisetn}
  Ef $Y$ er grannrúm ($X,\pi,R$ eins og áður) og $f:X\to Y$ er samfelld og föst
  á trefjum $\pi$ (þ.e. ef $xRy$, þá $f(x) = f(y)$) þá er ótvírætt ákvarðaða
  vörpunin $\tilde f : X/R \to Y$ líka samfelld ($\tilde f = f\circ\pi$).\[
  \xymatrix{
  X \ar[r]^f\ar[d]_\pi & Y\\
  X/R\ar[ur]_{\exists!\,\tilde f} &
  }
  \]
\end{fylgisetn}
\begin{proof}
  Leiðir beint af síðustu setningu.
\end{proof}
% 01.02.2010
\begin{setn}
  $X$ og $Y$ grannrúm, $R$ jafngildisvensl á $X$. Ef $f:X\to Y$ er samfelld og
  föst á trefjum $\pi:X\to X/R$, þá getum við skrifað $f$ sem samskeytingu
  þriggja samfelldra varpana:
 \[
 \xymatrix{
 X       \ar[rr]^\pi_{\text{ofanv.}}           &&
 X/R   \ar[rr]^g_{[x]\mapsto f(x)}             &&
 f(X)  \ar@{^{(}->}[rr]^\iota_{\text{ívarpið}} &&
 Y
 }
 \]
 Ennfremur gildir að vörpunin $g$ er grannmótun \emph{þ.þ.a.a.}
 \begin{enumerate}[(i)]
   \item $f(x)=f(y)$ \emph{þ.þ.a.a.} $xRy$ ($g$ gagntæk)
   \item $V\subseteq f(X)$ er opið í $f(X)$ (með hlutrúmsgrannmynstrinu)
     \emph{þ.þ.a.a.} $f^{-1}(V)$ sé opið í $X$ (\emph{þ.þ.a.a.} $g^{-1}(V)$ sé
     opið í $X/R$ vegna $f^{-1}(V) = (g\circ \pi)^{-1}(V)$). 
 \end{enumerate}
\end{setn}
\begin{proof}
  Vörpunin $g$ er ótvírætt ákvörðuð og samfelld. Skilyrði (i) þýðir að
  $g$ sé gagntæk og skilyrði (ii) að $g$ sé opin vörpun. 
\end{proof}
\begin{skilgr}
  $X,Y$ grannrúm og $f:X\to Y$. Skilgreinum jafngildisvensl $R$ á $X$ með
  $x_1 R x_2$ \emph{þ.þ.a.a.} $f(x_1)=f(x_2)$. Við segjum að $f$ sé
  \emph{deildavörpun}\index{deildavörpun} ef $f$ er samfelld og átæk og
  $g$-ið úr síðustu setningu er grannmótun, m.ö.o. ótvírætt ákvarðaða vörpunin
  $\tilde f:X/R\to Y$ sé grannmótun. 
\end{skilgr}
\begin{skilgr}
  Hlutmengi $U$ í $X$ er sagt \emph{mettað}\index{mettað mengi} (\emph{e}.
  saturated) m.t.t. jafngildisvensla $R$ á $X$ ef eftirfarandi gildir:
  \begin{quote}
    Ef $x\in U$ og $xRy$, þá $y\in U$.
  \end{quote}
\end{skilgr}
\begin{ath}
  (i) Mengi $U$ í $X$ er mettað \emph{þ.þ.a.a.} $\pi^{-1}(\pi(U))=U$ þar sem
  $\pi:X\to X/R$.

  (ii) $X,Y$ grannrúm og $f:X\to Y$ átæk. Þá er $f$ deildavörpun \emph{þ.þ.a.a.}
  $f$ sé samfelld og varpi opnum mettuðum mengjum á opin mettuð mengi (jafngilt
  því að $V$ sé opið í $Y$ \emph{þ.þ.a.a.} $f^{-1}(V)$ sé opið í $X$).
\end{ath}
\begin{daemi}
  (1) $X$ og $Y$ grannrúm, $A$ hlutmengi í $Y$ og $f:A\to X$ samfelld. Á
  sundurlæga sammenginu $X\coprod Y$ skilgreinum við vensl $R$ með því að
  skilgreina deildirnar $\left\{ y \right\}$ ef $y\in Y\setminus A$ og
  $\left\{ y,f(y) \right\}$ ef $y\in A$ og $\left\{ x \right\}$ ef $x\in
  X\setminus f(A)$. Deildagrannrúmið $(X\coprod Y)/R$ er táknað $X\cup_f Y$. Við
  segjum að $X\cup_f Y$ fáist með því að \emph{líma $Y$ við $X$ með $f$}. 

  T.d. fæst áhugavert dæmi með því að skoða $X=\R^2$, $S^1=\partial D(0,1)$,
  $f:S^1\to\R^2, s\mapsto s$ og $Y=\overline{D(0,1)}$. Þá verður $X\cup_f Y$
  grannmóta rúmi sem fæst með því að líma hálfkúlu í $\R^3$ með geisla $1$
  ofan á einingarskífuna í $\R^2$.

  (2) Ef $X = \left\{ * \right\}$ er einn punktur, þá skrifum við $\left\{ *
  \right\}\cup_A Y$ í stað $\left\{ *\right\}\cup_f Y$ og segjum að $\left\{
  * \right\}\cup_A Y$ fáist með því að \emph{draga $A$ saman í einn punkt}. T.d.
  $Y=\left[ 0,1 \right]$ og $A=\left\{ 0,1 \right\}$, þá er $\left\{ *
  \right\}\cup_A Y\cong S'$ (grannmóta) vegna þess að $Y\to S'; t\mapsto
  e^{2\pi it}$ er deildamótun.

  Annað dæmi: $D = \left\{ (x,y)\in\R^2:x^2+y^2 \leq 1 \right\}$ og $A =
  S^1 =\partial D$. Þá er $\left\{ * \right\}\cup_{S^1}D\cong S^2$ (sjáum það
  síðar).
\end{daemi}



\chapter{Samhengni og þjöppun}

\section{Samanhangandi rúm}
\begin{skilgr}
  \begin{enumerate}[(1)]
    \item \emph{Strjált rúm}\index{strjált rúm} er grannrúm með strjála
      grannmynstrinu.
    \item \emph{Opin tvískipting}\index{opin tvískipting} á grannrúmi $X$ er
      safn $\left\{ U,V \right\}$ þar sem $U$ og $V$ eru opin í $X$, 
      $X = U\cap V$, $U\cap V= \emptyset$ og $U,V\neq \emptyset$.
  \end{enumerate}
\end{skilgr}
\begin{setn}
  Eftirfarandi skilyrði eru jafngild fyrir grannrúm $X$:
  \begin{enumerate}[(i)]
    \item $X$ á sér enga opna tvískiptingu.
    \item Einu hlutmengin í $X$ sem eru bæði opin og lokuð eru $\emptyset$ og
      $X$.
    \item Sérhver samfelld vörpun frá $X$ inn í strjált grannrúm er föst
      (þ.e.a.s. tekur bara eitt gildi). 
  \end{enumerate}
\end{setn}
\begin{skilgr}
  Grannrúm sem fullnægir skilyrðum (i), (ii) og (iii) að ofan er sagt
  \emph{samanhangandi}\index{samanhangandi} (\emph{e}. connected).
\end{skilgr}
\begin{proof}
  [Sönnun á síðustu setningu]
  \emph{(i)$\Rightarrow$(ii):} Ef $U$ er opið og lokað í $X$, þá er $X\setminus
  U$ líka opið og lokað í $X$. Þá er ljóst að $\left\{ U,X\setminus U \right\}$
  er opin tvískipting á $X$ nema $U=\emptyset$ eða $U=X$.

  \emph{(ii)$\Rightarrow$(iii):} Ef $Y$ er strjált, $f:X\to Y$ samfelld og
  $x\in X$, þá er $f^{-1}(f(x))$ bæði opið og lokað í $X$. Þar með er
  $f^{-1}(f(x))=X$ vegna þess að $x\in f^{-1}(f(x))$.

  \emph{(iii)$\Rightarrow$(i):} G.r.f. að $X=U\cup V$ þar sem $U$ og $V$ eru
  opin og $U\cap V=\emptyset$. Þá er vörpunin $f:X\to \left\{ 0,1 \right\}$, \[
  f(x) := 
  \begin{cases}
    0 & x\in U\\
    1 & x\in V,
  \end{cases}
  \]
  samfelld og þá fasti. Þar með gildir að $U = \emptyset$ eða $V = \emptyset$.
\end{proof}
\begin{daemi}
  \begin{enumerate}[(1)]
    \item Grannrúm, sem inniheldur bara einn punkt, er samanhangandi.
    \item Strjált rúm er samanhangandi \emph{þ.þ.a.a.} það innihaldi bara einn
      eða engan punkt.
    \item Samanhangandi hlutrúm í $\R$ er bil ($\emptyset$ og $\R$ meðtalin).
    \item Samanhangandi hlutrúm í $\Q$ er $\emptyset$ eða einstökungur.
  \end{enumerate}
\end{daemi}
\begin{setn}
  $X$ grannrúm.
  \begin{enumerate}[(i)]
    \item Látum $Y$ vera grannrúm og $f:X\to Y$ samfellda. Ef $X$ er
      samanhangandi, þá er $f(X)$ samanhangandi.
    \item Ef $A\subseteq X$ er samanhangandi og $A\subseteq B\subseteq \overline
      A$, þá er $B$ samanhangandi.
    \item Ef $X=\bigcup_{i\in I}A_i$ með $A_i$ samanhangandi fyrir öll $i\in I$
      og $\bigcap_{i\in I} A_i \neq \emptyset$, þá er $X$ samanhangandi.
  \end{enumerate}
\end{setn}
\begin{proof}
  (i) Ef $\left\{ U,V \right\}$ opin tvískipting á $f(X)$, þá er $\left\{
  f^{-1}(U), f^{-1}(V) \right\}$ opin tvískipting á $X$.

  (ii) Ef $f:B\to Y$ er samfelld og $Y$ strjált, þá er $f$ föst á $A$ og því
  einnig á $B$ vegna þess að $f(B)\subseteq f(\overline A) \subseteq
  \overline{f(A)} = \overline{\left\{ * \right\}} = \left\{ * \right\}$.

  (iii) Ef $f$ er samfelld vörpun frá $X$ inn í strjált rúm, þá er $f$ föst á
  sérhverju $A_i$ og þá einnig á $X$.
\end{proof}
% 02.02.2010
\begin{daemi}
  $A := \left\{ (x,y)\in\R^2 : x>0, \sin(1/x)=y \right\}$ er
  samanhangandi vegna þess að $A$ er mynd samfelldu vörpunarinnar\[
  \left] 0,+\infty \right[\to\R^2, x\mapsto (x,\sin(x)),
  \]
  þar með er lokun $A$ í $\R^2$,\[
  \overline A = A\cup \left\{ (0,y)\in\R^2:-1\leq y\leq 1 \right\},
  \]
  líka samanhangandi.
\end{daemi}
\begin{setn}
  $A\neq \emptyset$ samanhangandi hlutrúm í grannrúmi $X$. Sammengi allra
  samanhangandi hlutrúma í $X$ sem innihalda $A$ er lokað og samanhangandi
  hlutrúm í $X$.
\end{setn}
\begin{proof}
  Sammengi þessara rúma er samanhangandi vegna þess að sniðmengi þeirra er ekki
  tómt (sjá síðustu setningu) og þar sem lokun samanhangandi hlutrúms er
  samanhangandi, þá er það lokað (sjá síðustu setningu).
\end{proof}
\begin{skilgr}
  Ef $x$ er punktur í grannrúmi $X$, þá táknum við með $C_x$ sammengi allra
  samanhangandi hlutmengja í $X$ sem hafa $x$ sem stak. Köllum $C_x$
  \emph{samhengisþátt $x$ í $X$}\index{samhengisþáttur}.
\end{skilgr}
\begin{ath}
  (i) Samhengisþættir $X$ eru hér stök í safni allra samanhangandi hlutmengja í
  $X$ (m.t.t. íveruröðunar\footnote{röðunin í veldismenginu, $\mathcal P(X)$}).

  (ii) Samhengisþættir eru lokuð hlutmengi, en ekki endilega opin; t.d.
  $X:=\left\{ 0 \right\}\cup\left\{ 1/n:n\in\N^*\right\}$ sem hlutrúm í $\R$:
  $C_{1/n} = \left\{ 1/n \right\}$ bæði opið og lokað, en $C_0 = 0$ er ekki
  opið.
\end{ath}
\begin{skilgr}
  Grannrúm $X$ er sagt \emph{algerlega ósamanhangandi}\index{algerlega
  ósamanhangandi} ef $C_x = \left\{ x \right\}$ fyrir öll $x$ úr $X$. 
\end{skilgr}
\begin{daemi}
  (1) Strjál grannrúm eru algerlega ósamanhangandi. Hins vegar þurfa algerlega
  ósamanhangandi grannrúm ekki að vera strjál; t.d. er $X$ úr athugasemd (ii) að
  ofan ekki strjált. 

  (2) Ef $A\subseteq \R$ hefur þann eiginleika að fyrir sérhver $x$ og $y$ úr
  $A$ er til $z$ úr $\R\setminus A$ þ.a. $x<z<y$, þá er $A$ algerlega
  ósamanhangandi. Þetta hefur m.a. í för með sér að $\Q$ er algerlega
  ósamanhangandi.
\end{daemi}
\begin{setn}
  Á grannrúmi $X$ eru venslin 
  \begin{quote}
    $x\sim y$ þ.þ.a.a. $x\in C_y$
  \end{quote}
  jafngildisvensl og deildarúmið $X/\sim$ er algerlega ósamanhangandi.
\end{setn}
\begin{proof}
  Æfing.
\end{proof}
\begin{skilgr}
  \emph{Vegur}\index{vegur} í grannrúmi $X$ er samfelld vörpun $\gamma:\left[
  0,1 \right]\to X$. Segjum að $\gamma$ tengi (saman) punktana $\gamma(0)$ og
  $\gamma(1)$. Köllum $\gamma(0)$ \emph{upphafspunkt}, $\gamma(1)$
  \emph{lokapunkt}; þeir kallast saman \emph{endapunktar} $\gamma$.
\end{skilgr}
\begin{skilgr}
  Grannrúm er sagt \emph{vegsamanhangandi}\index{vegsamanhangandi} ef sérhverja
  punkta í $X$ er unnt að tengja með vegi í $X$.
\end{skilgr}
\begin{setn}
  Vegsamanhangandi grannrúm $X$ er samanhangandi.
\end{setn}
\begin{proof}
  Tökum $a$ úr $X$. Fyrir sérhvert $x$ úr $X$ veljum við veg $\gamma_x$ frá $a$
  til $x$. Þá er\[
  X = \bigcup_{x\in X}\gamma_x(\left[ 0,1 \right])
  \]
  samanhangandi vegna þess að $\bigcup_{x\in X}\gamma_x(\left[ 0,1 \right])\ni
  a$
\end{proof}
\begin{skilgr}
  \emph{Vegsamhengisþáttur}\index{vegsamhengisþáttur} punkts $x$ í grannrúmi $X$
  er mengi allra punkta í $X$ sem unnt er að tengja við $x$ með vegi í $X$.
\end{skilgr}
\begin{ath}
  (i) Vegsamhengisþáttur $x$ er innihaldinn í $C_x$ en getur verið ólíkur og
  þarf ekki að vera lokaður.

  (ii) Vegsamhengisþættir grannrúms $X$ mynda deildaskiptingu á $X$.
\end{ath}
\begin{daemi}
  $A = \left\{ (x,y)\in\R^2 : x>0, y=\sin(1/x) \right\}$ er vegsamanhangandi, en
  $\overline A$ er ekki vegsamanhangandi (Sönnun: æfing).
\end{daemi}
\begin{skilgr}
  Segjum að grannrúm sé \emph{staðsamanhangandi}\index{staðsamanhangandi}
  [\emph{staðvegsamanhangandi}\index{staðvegsamanhangandi}]
  ef það hefur grunn af samanhangandi [vegsamanhangandi] opnum mengjum.
  Þetta þýðir að fyrir sérhvert opið mengi $U$ og sérhvern punkt $x\in U$ er til
  [veg]samanhangandi opið mengi $V$ þ.a. $x\in V\subseteq U$.
\end{skilgr}
\begin{daemi}
  (1) Fyrir öll $n\geq 0$ er $\R^n$ ($\R^0 = \left\{ 0 \right\}$)
  vegsamanhangandi og staðvegsamanhangandi.

  (2) $\R\setminus\left\{ 0 \right\}$ er staðvegsamanhangandi, en ekki
  samanhangandi.

  (3) $A = \left\{ 0 \right\}\cup\left\{ 1/n:n\in\N^* \right\}$. Hlutmengið
  $(A\times \left[ 0,1 \right])\cup(\left[ 0,1 \right]\times\left\{ 0
  \right\})$ í $\R^2$ kallast \emph{hárgreiðan}\index{hárgreiðan}. Það er
  vegsamanhangandi, en ekki staðsamanhangandi.
\end{daemi}
\begin{ath}
  (i) Í staðsamanhangandi grannrúmi eru samhengisþættirnir bæði opin og lokuð
  hlutmengi.

  (ii) Í staðvegsamanhangandi grannrúmi eru samhengisþættirnir opin mengi. En
  þar sem þeir mynda skiptingu þá eru þeir líka lokaðir.
\end{ath}
\begin{setn}
  Grannrúm sem er samanhangandi og staðvegsamanhangandi er vegsamanhangandi.
\end{setn}
\begin{proof}
  Sérhver vegsamhengisþáttur er bæði opið og lokað hlutmengi í samanhangandi
  rúmi. Þar með er hann allt rúmið.
\end{proof}
\begin{fylgisetn}
  Í staðvegsamanhangandi grannrúmi eru vegsamhengisþættirnir og
  samhengisþættirnir þeir sömu.
\end{fylgisetn}
\begin{proof}
  Ljóst.
\end{proof}
\begin{setn}
  $(X_\alpha)_{\alpha\in I}$ fjölskylda af grannrúmum.
  \begin{enumerate}[(i)]
    \item Ef $X_\alpha$ er samanhangandi fyrir öll $\alpha\in I$, þá er
      $\prod_{\alpha\in I} X_\alpha$ samanhangandi.
    \item Ef $X_\alpha$ er vegsamanhangandi fyrir öll $\alpha\in I$, þá er
      $\prod_{\alpha\in I} X_\alpha$ vegsamanhangandi.
  \end{enumerate}
\end{setn}
\begin{proof}
  (i) Sjá dæmi 10 á bls. 152. (ii) Er létt æfing.

  (ii) Létt æfing.
\end{proof}

\section{Stefnan tekin á pólsku martröðina}

\paragraph{Upprifjun} 
\begin{itemize}
  \item Ef $(X,d)$ firðrúm, þá kallast talan \[
    \diam(X) := \sup\left\{ d(x,y) : x,y\in X \right\} \]
    \emph{þvermál}\index{zvermal firdrums@þvermál firðrúms} $X$.
  \item Firðrúm er sagt fullkomið ef sérhver Cauchy-runa í því er samleitin.
\end{itemize}


\begin{setn}
  Ef $(X_n)_{n\geq 0}$ er minnkandi runa (m.t.t. $\subseteq$) af ekki-tómum,
  lokuðum hlutmengjum í fullkomnu firðrúmi $X$ þ.a.\[
  \lim_{n\to\infty}\diam(X_n) = 0,
  \]
  þá inniheldur $\bigcup_{n\geq 0} X_n$ nákvæmlega einn punkt.
l\end{setn}
\begin{proof}
  Fyrir sérhvert $n\in\N$ tökum við $a_n$ úr $X_n$. Sýnum að $(a_n)$ sé
  Cauchy-runa: Látum $\varepsilon>0$, þá er til $N$ þ.a.
  $\diam(X_n)<\varepsilon$ fyrir öll $n\geq N$; ef $n,m\geq N$ er þá $a_n,a_m\in
  X_N$ og því $d(a_n,a_m) < \varepsilon$.

  Setjum $a := \lim_{n\to\infty} a_n$ og fáum auðveldlega að $a\in\bigcup X_n$
  vegna þess að $X_n$-in eru lokuð.
\end{proof}
% 08.10.2010
\begin{setn}
  [Baire]\index{Baire-setning}
  Látum $X$ vera fullkomið firðrúm og $(A_n)_{n\geq 0}$ vera teljanlega
  fjölskyldu af lokuðum mengjum sem hvert um sig hefur engan innri punkt,
  þ.e.a.s. $\innmengi(A) = \emptyset$. Þá hefur $\bigcup_{n\geq 0} A_n$
  engan innri punkt, þ.e.a.s. $\innmengi\left( \bigcup_{n\geq 0} A_n \right) =
  \emptyset$.
\end{setn}
\begin{proof}
  $\innmengi\left(\bigcup_{n\geq 0}A_n\right)=\emptyset$ jafngildir því að
  $X\setminus\bigcup_{n\geq 0}A_n$ sé þétt í $X$, þ.e.a.s. skeri sérhvert opið
  og ekki tómt mengi í $X$. Látum $U$ vera opið og ekki tómt mengi í $X$. Þá er
  $U\nsubseteq A_0$ og því $U\setminus A_0 \neq\emptyset$. Þar eð $U\setminus
  A_0$ er opið þá er til opin kúla $B_1$ í $X$ með $\overline B_1\subseteq
  U\setminus A_0$ og $\diam(B_1)<1$. Nú er $B_1\nsubseteq A_1$ svo að
  $B_1\setminus A_1 \neq \emptyset$ og því er til opin kúla $B_2$ með $\overline
  B_2 \subseteq B_1\setminus A_1$ og $\diam(B_2)<1/2$. Með þrepun fæst minnkandi
  runa af kúlum $B_0\supseteq B_1\supseteq \cdots$ þ.a. $\overline B_{n+1} \cap
  A_n = \emptyset$ og $\diam(B_n)<1/n$ fyrir öll $n\geq 0$. Skv. síðustu
  setningu er til $x$ úr $X$ þ.a. $\{x\} = \bigcap_{n\geq 0} \overline B_n$. Þá
  er $x\in U$ og $x\notin A_n$ fyrir öll $n\geq 0$, svo að $x\in U\cap
  \left(X\setminus\bigcup_{n\geq 0} A_n\right)$.
\end{proof}
\begin{skilgr}
  Grannrúm $X$ er sagt vera \emph{Baire-rúm}\index{Baire-rúm} eða fullnægja
  \emph{Baire-eiginleikanum} ef sérhver teljanleg fjölskylda af lokuðum
  hlutmengjum innri punkta í $X$ hefur sammengi, sem hefur engan innri punkt.
\end{skilgr}
\begin{ath}
  (i) Baire-eiginleikinn er oft orðaður svona: Ef $(A_n)_{n\geq 0}$ er runa af
  hlutmengjum í $X$ þ.a. $\innmengi(\overline A_n)=\emptyset\;\forall\,n$, þá er
  $X\setminus\bigcup_{n\geq 0}A_n$ þétt í $X$. Mengi $B$ sem uppfyllir
  $\innmengi(\overline B)=\emptyset$ kallast \emph{hvergi þétt}\index{hvergi
    þétt}.

  (ii) Jafngild framsetning á Baire-setningu fæst með því að skoða fyllimengi:
  Ef $(U_n)_{n\geq 0}$ er runa af opnum þéttum hlutmengjum í firðrúmi $X$, þá er
  $\bigcap_{n\geq 0} U_n$ líka þétt.
\end{ath}
\begin{fylgisetn}
  $X$ fullkomið firðrúm og $X =\bigcup_{n\geq 0} A_n$ þar sem $A_n$ er lokað
  fyrir öll $n$, þá er til $n$ úr $\N$ þ.a. $\innmengi(A_n) \neq \emptyset$.
\end{fylgisetn}

\section{Pólsk martröð}
\label{sec:polsk_martrod}
Látum $C$ tákna Cantor-mengið með hlutrúmsgrannmynstrinu. Það er búið til með
því að taka burt opna miðþriðjunginn úr $\left[0,1\right]$, og síðan opnu
miðþriðjungana úr bilunum sem eftir verða, o.s.frv.

Sérhverja tölu úr $\left[0,1\right]$ er unnt að setja fram sem
$\sum_{n=1}^\infty \frac{a_n}{3^n}$ með $a_n\in\{0,1,2\}$. Cantor-mengið er
myndað úr þeim tölum þar sem öll $a_n$-in eru annaðhvort $0$ eða
$2$. Endapunktar allra miðþriðjunganna sem numdir eru burt mynda þétt hlutmengi
í $C$, táknum það með $C_{\text A}$, og setjum $C_{\text{Ó}} := C\setminus
C_{\text A}$ (hér stendur A fyrir aðgengilegur, Ó fyrir óaðgengilegur). Látum
$X_{\text A}$ vera keiluna yfir $C_{\text A}$ með toppunkt $(\frac 12, 1)$,
þ.e.a.s. mengi allra línustrika frá $(\frac 12,1)$ til punktanna í $C_{\text
  A}$. Eins látum við $X_{\text{Ó}}$ tákna keiluna yfir $C_{\text{Ó}}$ með
toppunkt $(\frac 12, 1)$. Látum $Y_{\text A}$ vera mengi allra punkta $(x,y)$ úr
$X_\text{A}$ þ.a. $y$ sé ræð. Látum $Y_\text{Ó}$ vera mengi allra punkta $(x,y)$
úr $X_\text{Ó}$ þ.a. $y$ sé óræð \emph{eða} 0 \emph{eða} 1. Setjum $Y :=
Y_\text{A} \cup Y_\text{Ó}$.
\begin{setn}\label{setn:polska_martrod}
  $Y$ er samanhangandi, en $Y\setminus\{(\frac 12, 1)\}$ er algerlega
  ósamanhangandi. 
\end{setn}
\begin{fylgisetn}
  $Y$ er samanhangandi, en allir vegsamhengisþættir $Y$ eru einstökungar.
\end{fylgisetn}
Við þurfum á eftirfarandi niðurstöðu að halda:
\begin{hjalparsetn}
  $Y$ hlutmengi í firðrúmi $X$. Ef $Y$ er \emph{ekki} samanhangandi, þá eru til
  opin mengi $U$ og $V$ í $X$ þ.a. $Y\subseteq U\cup V$, $U\cap Y\neq\emptyset$,
  $V\cap Y\neq\emptyset$ og $U\cap V=\emptyset$.
\end{hjalparsetn}
\begin{proof}
  Æfing.
\end{proof}
\begin{proof}
  [Sönnun á setningu \ref{setn:polska_martrod}] Beitum óbeinni sönnun; g.r.f. að
  $U$ og $V$ séu opin sundurlæg mengi í $\R^2$ þannig að $Y\subseteq U\cup V$ og
  $(\frac 12, 1)\in U$ og $V\cap Y\neq \emptyset$. Fyrir sérhvert $r > 0$ setjum
  við $W(r):= \{(x,y)\in\R^2 : y>r\}$ og fyrir sérhvert $x$ úr $C$ látum við
  $\ell(x)$ vera lokaða línustrikið sem tengir $(x,0)$ og $(\frac 12,
  1)$. Skilgreinum (aðgreiningar-)fall $f:C\to \left[0,1\right]$ með \[ f(x) :=
  \inf\{r : W(r)\cap\ell(x)\subseteq U\}.
  \]
  Til hagræðis skilgreinum við svo fall \[
  g:C\to\left[0,1\right]
  \]
  með því að \[
  (g(x),f(x))\in\ell(x).
  \]
  Tökum eftir \[
  (g(x),f(x)) =\begin{cases}
    \text{hæsti punktur á $\ell(x)$, sem er ekki í $U$}, & \text{ef
      $\ell(x)\nsubseteq U$}.\\
    (x,0) & \text{ef $\ell(x)\subseteq U$}.
  \end{cases}
  \]
  Þar með er $(g(x),f(x))$ hvorki í $U$ né $V$ ef $f(x)\neq 0$. Tökum einnig
  eftir að $f(x) > 0$ \emph{þ.þ.a.a.} $\ell(x')\cap V\neq 0$ í grennd við
  $x$. Þetta þýðir að til er opið bil í $\left[0,1\right]$ þ.a. $f(x)>0$ fyrir
  öll $x$ úr bilinu. Nú er $C$ búið til með því að endurtaka í sífellu sömu
  aðgerðina (þ.e. brottnám miðþriðjunga), svo að umrætt opið bil inniheldur
  mengi $C'$ sem er eins og $C$. Við getum því einfaldlega gert ráð fyrir að
  $f(x)>0$ fyrir öll $x$ úr $C$. Setjum \[
  Z := \{ (g(x),f(x)) : x\in C_\text{Ó}\}.
  \]
  Nú er $Z\cap (U\cup V) = \emptyset$, og því $\overline Z \cap (U\cup
  V)=\emptyset$ vegna þess að $U\cup V$ er opið; sér í lagi $\overline Z\cap Y =
  \emptyset$. Einnig er ljóst að $\overline Z$ er innihaldið í keilunni
  $X_\text{A} \cup X_\text{Ó}$ yfir $C$ vegna þess að hún er lokuð. Af þessu
  leiðir að fyrir $(a,b)$ úr $\overline Z$ gildir:
  \begin{itemize}
  \item Ef $b$ er óræð, þá er $(a,b)\in\ell(x)$ með $x\in C_\text{A}$.
  \item Ef $b$ er ræð, þá er $(a,b)\in\ell(x)$ með $x\in C_\text{Ó}$.
  \end{itemize}
  % 09.02.2010
  Fyrir sérhverja ræða tölu $q$ er mengið  $\overline Z\cap \{y=q\}$ lokað og
  því er keiluofanvarp þess á $x$-ásinn líka lokað, þ.e.a.s. mengið
  \[
  Z(q):=\{x\in C : \ell(x) \cap \{ y = q\}\cap \overline Z \neq\emptyset\}
  \]
  er lokað. Fyrir $x$ úr $C_\text{Ó}$ er $f(x)$ ræð svo að $x\in Z(f(x))$ og þar
  með er \[
  C_\text{Ó} = \bigcup_{\substack{q\in\Q\\0\leq q<1}}Z(q).
  \]
  Fáum þá \[
  C = C_\text{Ó} \coprod C_\text{A}
  =\underbrace{
   \left(\bigcup_{z\in\Q} Z(q)\right) \cup\left( \bigcup_{a\in C_\text{A}}\{a\}\right).
  }_{\substack{\text{teljanlegt sammengi}\\\text{lokaðra hlutmengja í $C$}}}
  \]
  Skv. Baire-setningu er til $q$ úr $\Q$ þ.a. $Z(q)$, og þar með $C_\text{Ó}$,
  innihaldi opið mengi í $C$ ($C$ er þjappað firðrúm og því fullkomið), en það
  er í mótsögn við að $C_\text{A}$ er þétt í $C$ og $C_\text{A}\cap C_\text{Ó}
  =\emptyset$.

  Sýnum að $Y\setminus\{(\frac 12, 1)\}$ sé algerlega ósamanhangandi:
  Keiluofanvarpið $\pi: Y\setminus\{(\frac 12,1)\}\to\R$ er samfelld
  vörpun. Látum $B$ vera samanhangandi og ekki tómt hlutmengi á
  $Y\setminus\{(\frac 12,1)\}$. Þá er $\pi(B)$ samanhangandi. En þar sem $C$ er
  algerlega ósamanhangandi, þá er til $x$ úr $C$ þ.a. $\pi(B) = \{x\}$,
  þ.e.a.s. $B\subseteq \ell(x)$. Látum $p_2 : \R^2\to\R$ vera venjulega
  ofanvarpið á $y$-hnitið, þá er $p_2(B)$ samanhangandi og tilheyrandi vörpun
  $B\to p_2(B)\subseteq\left[0,1\right[$ gagntæk.
  \begin{itemize}
  \item Ef $x\in C_\text{A}$, þá er $p_2(B)\subseteq \Q$ og þar með er $p_2(B)$
    einstökungur.
  \item Ef $x\in C_\text{Ó}$, þá er $p_2(B)\subseteq(\R\setminus\Q)\cup\{0\}$,
    og þar með er $p_2(B)$ einstökungur.
  \end{itemize}
\end{proof}


\section{Þjöppuð rúm}
\label{sec:thjoppud_rum}

\begin{skilgr}
  Grannrúm $X$ er sagt \emph{þjappað}\index{zjappad@þjappað} ef sérhver opin þakning á
  $X$ hefur endanlega hlutþakningu.
\end{skilgr}
\begin{ath}
  Víða í stærðfræðiritum er \emph{hálfþjappað}\index{halfzjappad@hálfþjappað}
  (quasi-compact) notað í stað \emph{þjappað} og þau sögð \emph{þjöppuð} sem
  bæði eru hálfþjöppuð og Hausdorff.
\end{ath}
\begin{setn}
  \begin{enumerate}[(i)]
  \item Hlutrúm $Y$ í grannrúmi $X$ er þjappað \emph{þ.þ.a.a.} sérhver þakning á
    $Y$ með opnum mengjum í $X$ hefur endanlega hlutþakningu.
  \item Lokuð hlutmengi í þjöppuðum rúmum eru þjöppuð.
  \end{enumerate}
\end{setn}
\begin{proof}
  (Næstum augljós) æfing.
\end{proof}
\begin{setn}
  Þjappað hlutrúm $Y$ í Hausdorff-rúmi $X$ er lokað.
\end{setn}
\begin{proof}
  Sýnum að $X\setminus Y$ sé opið. Látum $x_0\in X\setminus Y$. Fyrir sérhvert
  $y$ úr $Y$ eru til opnar grenndir $W_y$ um $x_0$ og $V_y$ um $y$ í $X$
  þ.a. $W_y\cap V_y = \emptyset$ vegna þess að $X$ er Hausdorff. Þá er
  $Y\subseteq \bigcup_{y\in Y} V_y$ svo til eru $y_1,\dots, y_l$ úr $Y$
  þ.a. $Y\subseteq V_{y_1}\cup\cdots\cup V_{y_l}$ vegna þess að $Y$ er
  þjappað. Af því leiðir að $Y\cap (W_{y_1}\cap\cdots\cap W_{y_l})\subseteq
  (V_{y_1}\cup\cdots\cup V_{y_l})\cap(W_{y_1}\cap\cdots\cap W_{y_l})=\emptyset$
  svo að $W_{y_1}\cap\cdots\cap W_{y_l}$ er grennd um $x_0$ í $X\setminus Y$.
\end{proof}
\begin{setn}
  $X$ þjappað grannrúm og $f: X\to Y$ samfelld. Þá er $f(X)$ þjappað.
\end{setn}
\begin{proof}
  Ef $(U_\alpha)_{\alpha\in I}$ er opin þakning á $f(X)$, þá er $\left(
    f^{-1}(U_\alpha)\right)_{\alpha\in I}$ opin þakning á $X$. Þar með er til
  endanlegt hlutmengi $J$ í $I$ þ.a. $X\subseteq \bigcup_{\alpha\in J}
  f^{-1}(U_\alpha)$ og því \[
  f(X)
  \subseteq f\left(\bigcup_{\alpha\in J} f^{-1}(U_\alpha)\right)
  \subseteq \bigcup_{\alpha\in J} U_\alpha
  \]
  þ.e.a.s. $(U_\alpha)_{\alpha\in J}$ er endanleg hlutþakning á $f(X)$.
\end{proof}
\begin{setn}
  Látum $f:X\to Y$ vera gagntæka og samfellda vörpun. Ef $X$ er þjappað og $Y$
  er Hausdorff, þá er $f$ grannmótun.
\end{setn}
\begin{proof}
  Okkur nægir að sýna að $f$ sé lokuð vörpun. Ef $A$ er lokað í $X$, þá er $A$
  þjappað og þar með $f(A)$ þjappað í $Y$. En $Y$ er Hausdorff, svo að $f(A)$ er
  lokað skv. þarsíðustu setningu.
\end{proof}
\begin{hjalparsetn}
  $X,Y$ grannrúm og $Y$ þjappað. Ef $x_0\in X$ og $W$ er opin grennd um
  $\{x_0\}\times Y$ í $X\times Y$, þá er til grennd $U$ um $x_0$ í $X$
  þ.a. $U\times Y\subseteq W$.
\end{hjalparsetn}
\begin{proof}
  Fyrir sérhvert $y$ úr $Y$ veljum við opin mengi $U_y$ í $X$ og $V_y$ í $Y$
  þ.a. \[
  (x_0, y)\in U_y\times V_y\subseteq W.
  \]
  Þá er $(V_y)_{y\in Y}$ opin þakning á $Y$. Tökum endanlega undirþakningu
  $\{V_{y_1},\dots,V_{y_k}\}$ og setjum $U = \bigcap_{i=1}^k U_{y_i}$. Þá fæst
  að \[
  \{x_0\}\times Y\subseteq U\times Y
  = U\times\left(\bigcup_{i=1}^k V_{y_i}\right)
  = \bigcup_{i=1}^k (U\times V_{y_i})
  \subseteq \bigcup_{i=1}^k (U_{y_i}\times V_{y_i})
  \subseteq W.
  \]
\end{proof}
% 15.02.2010
\begin{setn}
  Faldrúm endanlelga margra þjappaðra rúma er þjappað.
\end{setn}
\begin{proof}
  Nóg að sanna setninguna fyrir tvö þjöppuð rúm og beita svo
  þrepun. Látum $X$ og $Y$ vera þjöppuð og $\mathcal A$ vera opna
  þakningu á $X\times Y$. Ef $x\in X$, þá er $\{x\}\times Y$ þjappað
  og því til $A_1,\dots,A_l$ úr $\mathcal A$ þ.a. $\{x\}\times
  Y\subseteq A_1\cup\cdots\cup A_l$. Skv. síðustu setningu er þá til
  opin grennd $W_x$ um $x$ í $X$ sem uppfyllir $W_x\times Y\subseteq
  A_1\cup\cdots\cup A_l$. Nú er $\{ W_x : x\in X\}$ opin þakning á $X$
  svo hún hefur endanlega hlutþakningu $\{W_1,\dots,W_k\}$. Þar með er
  $X\times Y = \left( \bigcup_{j=1}^k W_j\right)\times Y =
  \bigcup_{j=1}^k(W_j\times Y)$ og sérhvert $W_j\times Y$ er þakið með
  endanlega mörgum stökum úr $\mathcal A$.
\end{proof}
\begin{ath}
  Almennt gildir að faldrúm hvaða fjölskyldu sem er af þjöppuðum
  grannrúmum er þjappað . Þetta er hin svokallað
  \emph{Tychonoff-setning}\index{Tychonoff}, sem verður sönnuð síðar
  (umtalsvert erfiðari).
\end{ath}
\begin{setn}
  Grannrúm $X$ er þjappað \emph{þ.þ.a.a.} það fullnægi eftirfarandi
  skilyrði:
  \begin{quote}
    Ef $(A_\alpha)_{\alpha\in I}$ er fjölskylda af lokuðum mengjum
    þ.a. $\bigcap_{\alpha\in I} A_\alpha \neq\emptyset$ fyrir öll
    endanleg hlutmengi $J$ í $I$, þá er $\bigcap_{\alpha\in
      I}A_\alpha\neq \emptyset$.
  \end{quote}
\end{setn}
\begin{proof}
  G.r.f. að $X$ sé þjappað og $(A_\alpha)_{\alpha\in I}$ sé fjölskylda
  af lokuðum hlutmengjum í $X$ þ.a. $\bigcap_{\alpha\in I} A_\alpha =
  \emptyset$, þá er \[
  \bigcup_{\alpha\in I}(\underbrace{X\setminus A}_\text{opið})
  = X\setminus \bigcap A_\alpha
  = X.
  \]
  Þar sem $X$ er þjappað, þá hefur opna þakningin $\{X\setminus
  A_\alpha : \alpha\in I\}$ endanlega hlutþakningu, þ.e.a.s. til er
  endanlegt hlutmengi $J$ í $I$ þ.a. $\bigcup_{\alpha\in J}(X\setminus
  A_\alpha) = X$ og þá $\bigcap_{\alpha\in J} A_\alpha = \emptyset$
  skv. de Morgan.

  Öfugt, g.r.f. að $X$ fullnægi skilyrðinu og látum
  $(U_\alpha)_{\alpha\in I}$ vera opna þakningu á $X$. Setjum
  $A_\alpha = X\setminus U_\alpha$. Þá er $\bigcap_{\alpha\in
    I}A_\alpha=\emptyset$ skv. de Morgan og þar með er til endanleg
  hlutþakning á $J$ í $I$ þ.a. $\bigcap_{\alpha\in J}A_\alpha =
  \emptyset$, en það jafngildir því að $\bigcup_{\alpha\in J} U_\alpha
  = X$.
\end{proof}
\begin{fylgisetn}
  Ef $(C_n)_{n\in\N}$ er minnkandi runa (þ.e.a.s. $C_n\supseteq
  C_{n+1}\forall n$) af lokuðum hlutmengjum í þjöppuðu rúmi
  þ.a. $C_n\neq\emptyset\forall n\in\N$, þá er $\bigcap_{n\in\N}
  C_n\neq\emptyset$.
\end{fylgisetn}

\paragraph{Upprifjun}
$(X,\leq)$ línulega raðað og $A\subseteq X$. Stak $b$ úr $X$ þ.a. $a\leq
b$ fyrir öll $a$ úr $A$ kallast \emph{yfirstak}\index{yfirstak} $A$.
Ef mengi allra yfirstaka mengisins $A$ á sér minnsta stak, þá kallast
það \emph{efra mark}\index{efra mark} $A$.
\begin{daemi}
  (1) Vitum að sérhvert hlutmengi í $\R$, sem er takmarkað að ofan, á
  sér efra mark.

  (2) Hlutmengið $A = \{r\in\Q : 0<r, r^2 < 2\}$ í $\Q$ á sér ekki efra
  mark í $\Q$.
\end{daemi}
Höfum samsvarandi hluti fyrir \emph{neðra mark}\index{neðra mark}.

\begin{setn}
  Látum $X$ vera línulega raðað mengi þar sem sérhvert hlutmengi, sem
  er takmarkað að ofan, hefur efra mark í $X$. Þá er sérhvert lokað
  bil í $X$ þjappað (í röðunargrannmynstrinu).
\end{setn}
\begin{proof}
  Látum $a,b\in X$ og $\mathcal A$ vera opna þakningu á
  $\left[a,b\right]$. Sýnum að unnt sé að þekja $\left[a,b\right]$ með
  endanlega mörgum stökunum úr $\mathcal A$.

  (1) Sýnum fyrst: Ef $x\in\left[a,b\right]$ og $x\neq b$, þá er til
  $y>x$ úr $\left[a,b\right]$ og $A,B\in\mathcal A$
  þ.a. $\left[x,y\right]\subseteq A\cup B$.
  \begin{itemize}
  \item Ef $\left]x,b\right]$ hefur minnsta stak, $y$, þá er
    $\left[x,y\right] =\{x,y\}$. Veljum þá $A$ og $B$ úr $\mathcal A$
    þ.a. $x\in A$ og $y\in B$.
  \item Ef $\left]x,b\right]$ hefur ekki minnsta stak, þá veljum við
    eitthvert $A$ úr $\mathcal A$ þ.a. $x\in A$. Þa rsem $x\neq b$ og
    $A$ er opið, þá er til $c>x$ þ.a. $\left[x,c\right[\subseteq
    A$. Veljum síðan $y$ úr $\left[x,c\right[$ og fáum að
    $\left[x,y\right]\subseteq A$; tökum svo $B = A$.
  \end{itemize}

  (2) Setjum \[
  C := \{ y\in \left]a,b\right] : \text{hægt er að þekja
    $\left[a,y\right]$ með endanlega mörgum stökum úr $\mathcal A$}\}.
  \]
  Setjum $c := \sup C$. Þá, skv. (1), er $a<c\leq b$. Sýnum að $c\in
  C$, þ.e. að hægt sé að þekja $\left[a,c\right]$ með endanlega mörgum
  stökum úr $\mathcal A$. Veljum $A$ úr $\mathcal A$ þ.a. $c\in
  A$. Þar sem $A$ er opið, þá er til $d<c$ úr $\left[a,b\right]$ sem
  uppfyllir $\left]d,c\right]\subseteq A$. Þar sem $c$ er efra mark
  $C$, þá er til $c'$ úr $C$ þ.a. $d < c'\leq c$, og því
  $\left[c',c\right]\subseteq A$. Nú eru til $A_1,\dots,A_m$ úr
  $\mathcal A$ þ.a. $\left[a,c'\right]\subseteq A_1\cup \cdots\cup
  A_m$ og þar með \[ \left[a,c\right] =
  \left[a,c'\right]\cup\left[c',c\right]\subseteq A_1\cup\cdots\cup
  A_m\cup A.\]

  Ljúkum nú sönnuninni með því að sýna að $c = b$: Ef ekki, þá er er
  til $y>c$ úr $\left[a,b\right]$  þ.a. $\left[c,y\right]$ sé hægt að
  þekja með endanlega mörugm stökum úr $\mathcal A$. Það sama gildir
  þá um $\left[a,y\right] = \left[a,c\right]\cup\left[c,y\right]$ og
  það þýðir að $y\in C$; í mótsögn við að $y>c =\sup C$.
\end{proof}
\begin{fylgisetn}
  Lokuð bil í $\R$ eru þjöppuð.
\end{fylgisetn}
\begin{proof}
  Augljóst.
\end{proof}
\begin{setn}
  [Heine-Borel]\index{Heine-Borel}
  Hlutmengi í $\R^n$ er þjappað \emph{þ.þ.a.a.} það sé lokað og takmarkað.
\end{setn}
\begin{proof}
  Látum $A\subseteq\R^n$. Ef $A$ er þjappað, þá er $A$ lokað vegna
  þess að $\R^n$ er Hausdorff. Nú er $A\subseteq \R^n = \bigcup_{r>0}
  B(0,r)$ svo til eru $r_1,\dots,r_k > 0$ þ.a. $A\subseteq
  B(0,r_1)\cup\cdots\cup B(0,r_k) = B(0,r_0)$ þar sem $r_0 :=
  \max_{1\leq i\leq k}$. Þar með er $A$ takmarkað.

  Öfugt, g.r.f. að $A$ sé lokað og takmarkað. Þá er til $r>0$ þ.a.
  \[
  A\subseteq B(0,r)
  \subseteq \left[-r,r\right]\times\cdots\times\left[-r,r\right]
  \]
  sem er þjappað, vegna þess að endanlegt faldrúm þjappaðra rúma er
  þjappað. Af þessu sést að $A$ er lokað mengi í þjöppuðu rúmi, og þar
  með þjappað.    
\end{proof}
\begin{setn}
  $f: X\to Y$ samfelld þar sem $Y$ er línulega raðað með
  röðunargrannmynstrinu. Ef $X$ er þjappað, þá eru til $c$ og $d$ úr
  $X$ þ.a. $f(c)\leq f(x)\leq f(d)$ fyrir öll $x$ úr $X$.
\end{setn}
\begin{proof}
  Ef $f(X)$ hefur ekkert stærsta stak, þá mynda mengin
  $\left]-\infty,y\right[$ með $y\in f(X)$ opna þakningu á $f(X)$. En
  $f(X)$ er þjappað, svo að til eru $y_1,\dots,y_l$ úr $f(X)$
  þ.a. $f(X)\subseteq
  \left]-\infty,y_1\right[\cup\cdots\cup\left]-\infty,y_l\right[ =
  \left]-\infty,y_0\right[$, þar sem $y_0 = \max_{1\leq i\leq l} y_i$,
  í mótsögn við að $y_0\in f(X)$.

  Tilvist minnsta staks í $f(X)$ fæst á svipaðan hátt.
\end{proof}
\begin{setn}
  $X$ þjappað og ekki-tómt Hausdorff-rúm. Ef sérhver punktur úr $X$ er
  þéttipunktur, þá er $X$ óteljanlegt.
\end{setn}
\begin{proof}
  Sýnum að ekki sé til átæk vörpun $f:\N\to X$. G.r.f að $f:\N\to X$
  sé átæk og setjum $x_n := f(n)$. Veljum opið mengi $V_1$ í $X$
  þ.a. $x_1\notin\overline V_1$. Almennt veljum við $V_n$ opið í $X$
  þ.a. $\overline V_n\subseteq \overline V_{n-1}$ og $x_n\notin
  \overline V_n$. Þar sem $X$ er þjappað, þá er $\bigcap_{n\geq
    1}V_n\neq\emptyset$, en ljóst er að $f(\N)\cap\left(\bigcap_{n\geq
    1} V_n\right) = \emptyset$ svo að $f(\N) \neq X$.
\end{proof}
% 16.02.2010
\begin{fylgisetn}
  Látum $a,b\in\R$  þ.a. $a<b$. Þá er $\left[a,b\right]$ óteljanlegt.
\end{fylgisetn}
\begin{proof}
  Leiðir beint af síðustu setningu.
\end{proof}

\paragraph{Upprifjun}

Línulega raðað mengi er sagt \emph{velraðað}\index{velraðað} ef
sérhvert ekki-tómt hlutmengi þess hefur minnsta stak. Setjum þá að
röðunin sé \emph{velröðun}\index{velröðun}. \emph{Velröðunarfrumsenda}
segir að sérhverju mengi megi velraða. Af henni leiðir: Til er
velraðað óteljanlegt mengi. Látum nú $X$ vera línulega raðað mengi og
$\alpha\in X$. Hlutmengið \[
S_\alpha := \{ x\in X: x<\alpha \}
\]
kallast \emph{snið}\index{snið} (í $X$).


\begin{setn}
  Til er óteljanlegt velraðað mengi þar sem sérhvert snið er teljanlegt.
\end{setn}
\begin{proof}
  Sýnum fyrst að til er velraðað mengi sem hefur óteljanlegt snið:
  Láum $X$ vera óteljanlegt og velraðað. Þá er $\{1,2\}\times X$
  velraðað með orðabókarröðun, og sérhvert snið af gerðinni
  $S_{(2,x)}$ er óteljanlegt.

  Látum nú $Y$ vera eitthvert velraðað mengi sem hefur óteljanlegt
  snið og setjum \[
  \Omega := \min\{\alpha\in Y: S_\alpha\text{ er óteljanlegt}\}.
  \]
  Þá er $S_\Omega$ óteljanlegt og velraðað og sérhvert snið í því er
  teljanlegt.
\end{proof}
\begin{skilgr}
  [ritháttur]
  $\overline S_\Omega = S_\Omega \cup \{\Omega\}$.
\end{skilgr}
\begin{ath}
  $S_\Omega$ er þétt í $\overline S_\Omega$.
\end{ath}
\begin{fylgisetn}
  Ef $A$ er teljanlegt hlutmengi í $S_\Omega$, þá er $A$ takmarkað að ofan.
\end{fylgisetn}
\begin{proof}
  Mengið $\bigcup_{a\in A} S_a =: B$ er teljanlegt svo að $B\subsetneq
  S_\Omega$. Ef $x\notin B$, þá er $a\leq x\;\forall\,a\in A$, því
  annars fengist $a>x$ fyrir eitthvert $a\in A$, og þar með $x\in
  S_a\subseteq B$, sem er mótsögn.
\end{proof}


\subsection{Nokkrir eiginleikar $S_\Omega$ og $\overline S_\Omega$}
Táknum minnsta stakið í $S_\Omega$ með $0$.
\begin{enumerate}[(i)]
\item Séhrvert takmarkað hlutmengi í $S_\Omega$ hefur efra mark í $S_\Omega$.
\item Sérhvert hlutmengi á sér efra mark í $\overline S_\Omega$.
\item $\overline S_\Omega$ er þjappað og $S_\Omega$ er þétt í
  $\overline S_\Omega$.
\item Sérhver runa í $S_\Omega$ hefur samleitna hlutrunu.
\end{enumerate}
\begin{proof}
  (i) Látum $A\subseteq S_\Omega$ vera takmarkað. Þá er $B := \{b\in
  S_\Omega : a\leq b \,\forall a\in A\}$ ekki tómt $\sup A = \min B$.

  (ii)  Ef $A\subseteq \overline S_\Omega$, þá er $\sup A =
  \min\{b\in\overline S_\Omega : a\leq b \;\forall\,a\in A\}$.

  (iii) Þar sem öll hlutmengi í $\overline S_\Omega$ hafa efra mark,
  þá er $\overline S_\Omega = \left[0,\Omega\right]$ þjappað.

  (iv) Æfing.
\end{proof}
\begin{aefing}
  Látum $X$ tákna $\N\times\left[0,1\right[$ með
  orðabókargrannmynstrinu. Sýnið að $X$ sé grannmóta
  $\left[0,+\infty\right[$.
\end{aefing}

\subsection{Langa (hálf)línan og langa bilið}

$L := S_\Omega \times\left[0,1\right[$ með orðabókarröðun kallast
\emph{langa (hálf)línan}\index{langa (hálf)línan} og $L^+ :=
(S_\Omega\times\left[0,1\right[)\cup(\Omega,0)$ kallast \emph{langa
  bilið}\index{langa bilið}. Unnt er að sýna: Fyrir sérhvert
$\alpha\in L$ er $S_\alpha$ einsmóta
$\left[0,1\right[\cong\left[0,+\infty\right[$, en $L$ er ekki
grannmóta $\left[0,1\right[$. Skoðum þetta rúm betur í heimadæmum.


\section{Runulegur þjappleiki (eða runuþjappleiki)}

\begin{setn}\label{setn:thjoppun}
  $X$ firðanlegt grannrúm. Þá eru eftirfarandi skilyrði jafngild:
  \begin{enumerate}[(i)]
  \item $X$ er þjappað.
  \item Sérhvert óendanlegt hlutmengi í $X$ hefur þéttipunkt.
  \item Sérhver runa í $X$ hefur samleitna hlutrunu.
  \end{enumerate}
\end{setn}
\begin{proof}
  Þekkt.
\end{proof}
\begin{skilgr}
  Grannrúm sem uppfyllir (iii) kallast
  \emph{runuþjappað}\index{runuþjappað rúm}.
\end{skilgr}
\begin{ath}
  $S_\Omega$ er ekki þjappað, vegna þess að $S_\Omega$ er ekki lokað í
  $\overline S_\Omega$ og $\overline S_\Omega$ er Hausdorff. Hins
  vegar er $S_\Omega$ runuþjappað. Sér í lagi er $S_\Omega$ ekki firðanlegt.
\end{ath}
\begin{setn}[Viðbót við setningu \ref{setn:thjoppun}]
  Fyrir grannrúm $X$ gildir að (i)$\Rightarrow$(ii) og
  (iii)$\Rightarrow$(ii).
\end{setn}
\begin{proof}
  \emph{(iii)$\Rightarrow$(ii):} Látum $A$ vera óendanlegt mengi í
  $X$. Þá er til eintæk vörpun $\N\to A$, þ.e.a.s. til er runa
  $(x_n)_{n\in\N}$ í $A$ þ.a. $x_n\neq x_m$ ef $n\neq m$. Skv. (iii)
  hefur hún samleitna hlutrunu og markgildi hennar er bersýnilega
  þéttipunktur $A$.

  \emph{(i)$\Rightarrow$(ii):} Ef $A$ hefur enga þéttipunkta í $X$, þá
  er $A$ lokað í $X$ og því þjappað í strjála grannmynstrinu, svo það
  er endanlegt.
\end{proof}
% 22.02.2010
Hinar áttirnar gilda almennt ekki í grannrúmi: Auðséð er að við höfum
(iii)$\nRightarrow$(i), (ii)$\nRightarrow$(i). Að
(i)$\nRightarrow$(iii) (og þar með (ii)$\nRightarrow$(iii)) fæst með
$X := \left\{ 0,1 \right\}\subseteq \R$. Það er greinilega þjappað,
svo að skv. setningu Tychonoffs (verður sönnuð síðar) er $X^{\left[ 0,1
\right]}$ líka þjappað. Minnumst þess að sérhver tala $t\in\left[ 0,1
\right]$ hefur tvíundarframsetningu $t = \sum_{n\geq 1}
\frac{a_n}{2^n}$ þar sem $a_n\in\left\{ 0,1 \right\}$, þar af er
nákvæmlega ein framsetning fyrir hverja tölu gefin með óendanlegri
summu. Skilgreinum $\phi_n:\left[ 0,1 \right]\to\left\{ 0,1
\right\} = X$ þ.a. $\phi_n(t) = a_n$ ef $t = \sum_{n\geq 1}
\frac{a_n}{2^n}$ er óendanleg. Sýnum að Sýnum að $(\phi_n)_{n\geq 1}$
eigi sér enga samleitna hlutrunu: Ef $(\varphi_{n_k})_{k\geq 0}$ væri
slík hlutruna, þá gilti sér í lagi að $(\phi_{n_k}(t))_{k\geq 0}$ væri
samleitin fyrir sérhvert $t$ úr $\left[ 0,1 \right]$. Skilgreinum runu
$(a_n)_{n\geq 1}$ með því að setja $a_n=0$ ef $n\notin\left\{ n_k:k\geq
0 \right\}$, $a_{n_{2k}} = 1$ og $a_{n_{2k+1}}=0$ og setjum $t_0 =
\sum_{n\geq 1} \frac{a_n}{2^n}$. Þá fæst \[
\phi_{n_k}(t_0) =
\begin{cases}
  1 &\text{ef $k$ er slétt},\\
  0 &\text{ef $k$ er oddatala}
\end{cases}
\]
sem er ekki samleitin; mótsögn!


\section{Staðþjöppuð rúm}
\begin{skilgr}
  $X$ grannrúm og $x\in X$. Ef til er þjöppuð grennd um $x$ í $X$, þá er
  sagt að $X$ sé \emph{staðþjappað í $x$}\index{staðþjappað}. Segjum að
  $X$ sé \emph{staðþjappað} ef það er staðþjappað í öllum punktum sínum.
\end{skilgr}
\begin{ath}
  Víða í stærðfræðibókum þýðir staðþjappað að rúmið sé staðþjappað 
  \emph{og Hausdorff}.
\end{ath}
\begin{setn}
  Ef Hausdorff-rúm $X$ er staðþjappað í $x$, þá mynda þjöppuðu
  grenndirnar grenndagrunn um $x$; nánar tiltekið: Ef $U$ er grennd um
  $x$ í $X$, þá er til þjöppuð grennd $B$ um $x$ þ.a. $B\subseteq U$.
\end{setn}
\begin{proof}
  Látum $C$ vera þjappaða grennd um $x$ í $X$ og $U$ vera opna grennd um
  $x$ í $X$. Þá er $C\cap(X\setminus U)$ lokað mengi í $C$ og þar með
  þjappað. Þa rsem $x\notin C\cap(X\setminus U)$, þá eru til opin
  sundurlæg mengi $V'$ og $W'$ þ.a. $f\in V'$ og $C\cap(X\setminus
  U)\subseteq W'$. Af þessu leiðir að $\overline{V'}\cap C\cap
  (X\setminus U) = \emptyset$. Setjum nú $V := V'\cap C^\circ$ og fáum
  að $\overline V\subseteq C$ og því \[
  \overline V\cap (X\setminus U) 
  = \overline V \cap (X\setminus U)\cap C
  = \emptyset
  \]
  og þar með $\overline V \subseteq U$.
\end{proof}
\begin{setn}
  Látum $X$ vera staðþjappað grannrúm og veljum punkt $\infty_X\notin
  X$. Setjum $\hat{X}:=X\coprod\left\{ \infty_x \right\}$ og skilgreinum
  grannmynstur á $\hat X$ með því að kalla hlutmengi $U$ í $\hat X$
  \emph{opið} ef annað hvort eftirfarandi skilyrða er uppfyllt
  \begin{enumerate}[(a)]
    \item $U\subseteq X$ og $U$ er opið í $X$.
    \item $\infty_X\in U$ og $U\cap X = X\setminus K$ þar sem $K$ er
      þjappað (þ.e. grenndirnar um $\infty_X$ eru fyllimengi þjappaðra
      hlutmengja í $X$).
  \end{enumerate}
  Þá gildir:
  \begin{enumerate}[(i)]
    \item $\hat X$ er þjappað og $X$ er hlutrúm í $\hat X$ (þ.e.a.s.
      hlutrúmsgrannmynstur $X$ er sama og upphaflega grannmynstur $X$).
    \item Ef $X$ er ekki þjappað, þá er $X$ þétt í $\hat X$.
    \item Ef $X$ er þjappað, þá er $\infty_X$ einangraður punktur í
      $\hat X$.
    \item Ef $X$ er Hausdorff, þá er $\hat X$ Hausdorff. 
  \end{enumerate}
\end{setn}
\begin{proof}
  Til þess að sýna að (a) og (b) ákvarði grannmynstur á $\hat X$, þá
  nægir að sýna að endanlegt sniðmengi opinna gennda um $\infty_X$ sé
  opin grennd um $\infty_X$ (sjá vikublað 1). En ef $K_1$ og $K_2$ eru
  þjöppuð í $X$, þá er $K_1\cup K_2$ þjappað í $X$ og $(X\setminus
  K_1)\cap (X\setminus K_2) = X\setminus (K_1\cup K_2)$.

  (i) Látum $\mathcal U$ vera opna þakningu á $\hat X$ og veljum $V$ úr
  $\mathcal U$ þ.a. $\infty_X\in V$. Þá er til þjappað hlutmengi $K$ í
  $X$ þ.a. $X\cap V = X\setminus K$ og safnið $\left\{ X\cap U :
  U\in\mathcal U \right\}$ er opin þakning á $X\setminus (V\cap X) = K$.
  Þar með eru til $U_1,\dots,U_l$ úr $\left\{ X\cap U : U\in\mathcal U
  \right\}$, þ.e. $K\subseteq U_1\cup\cdots\cup U_l$. En það hefur
  bersýnilega í för með sér að $\hat X \subseteq V\cup U_1\cup\cdots\cup
  U_l$.

  (ii) og (iii). $\infty_X$ er einangraður í $\hat X$ \emph{þ.þ.a.a.}
  $\left\{ \infty_X \right\}$ sé opið í $\hat X$ \emph{þ.þ.a.a.}
  $\emptyset = \left\{ \infty_X \right\}\cap X = X\setminus K$ þar sem
  $K$ er þjappað \emph{þ.þ.a.a.} $X$ sé þjappað.

  (iv) Látum $x, y\in \hat X$. Ef $x,y\in X$, þá þarf ekkert að gera. Ef
  $x = \infty_X$, þá tökum við þjappaða grennd $C$ um $y$ og setjum $U
  :=\hat X\setminus C$, þá eru $U$ og $C^\circ$ sundurlægar opnar
  grenndir um $\infty_X$ og $y$.
\end{proof}
\begin{skilgr}
  Rúmið $\hat X$ kallast
  \emph{Alexandroff-þjöppun}\index{Alexandroff-þjöppun} (eða
  \emph{einspunktsþjöppun}\index{einspunktsþjöppun}) grannrúmsins $X$.
\end{skilgr}
\begin{ath}
  Þó svo $X$ sé ekki staðþjappað, þá má búa $\hat X$ til á sama hátt og
  áður, en þá verður $\hat X$ aldrei Hausdorff (lesandi skal sanna það!).
\end{ath}


\chapter{Ýmis dæmi og skilgreiningar}

\section{Zariski-grannmynstur á $K^n$}
\index{Zariski-grannmynstur}
Zariski-grannmynstur á $K^n$ þar sem $K$ er kroppur/svið. Segjum að
$A\subseteq K^n$ sé \emph{algebrulegt}\index{algebruleg mengi}
ef til er mengi $M$ af margliðum í $K[X_1,\dots,X_n]$ þ.a. $A=N(M)
:=\left\{ x\in K^n : f(x) = 0\forall f\in M \right\}$. Fáum
\begin{enumerate}[(i)]
  \item $K^n = N(\{0\})$ og $\emptyset = N(K\left[ x_1,\dots,x_n
    \right]) = N(\{1\})$.
  \item Ef $A = N(M)$ og $B = N(L)$, þá er $A\cup B = N(M\cdot L)$, þar
    sem $M\cdot L = \left\{ p\cdot q:p\in M,q\in L \right\}$:  Ef $x\in
    A$, þá er $p(x) = 0$ $\forall p\in M$ og $(p\cdot q)(x) = 0$
    $\forall p\in M$ og $\forall q\in L$, og þar með $x\in N(M\cdot L)$.
    Þar með er sýnt að $A\subseteq N(M\cdot L)$ og á sama hátt er
    $B\subseteq N(M\cdot L)$. Öfugt, ef $x\notin A\cup B$, þá eru til
    $p$ úr $M$ og $q$ úr $L$ þ.a. $p(x)\neq 0$, $q(x)\neq 0$ og því
    $(p\cdot q)(x)\neq 0$ og þar með $x\notin N(M\cdot L)$.
  \item Ef $A_\alpha = N(M_\alpha)$, $\alpha\in I$, þá gildir \[
    \bigcap_{\alpha\in I} A_\alpha
    = N\left( \cup_{\alpha\in I} M_\alpha \right).
    \]
\end{enumerate}
(i), (ii) og (iii) hafa í för með sér að algebrulegu mengin í $K^n$ eru
lokuðu mengin í grannmynstrinu á $K^n$; svokölluðu
\emph{Zariski-grannmynstri}. 

% 23.02.2010
Almennt er þetta \emph{ekki} Hausdorff (m.ö.o. $T_2$), en það er $T_1$.
Algebruleg mengi í $K$ eru bara endanleg mengi í $K$ svo að
Zariski-grannmynstrið á $K$ er Hausdorff \emph{þ.þ.a.a.} $K$ sé
endanlegur kroppur.


\section{(Grannfræðilegar) víðáttur}
Grannrúm $X$ kallast \emph{$n$-víð víðátta}\index{víðátta} [e. manifold]
ef sérhver punktur $x$ úr $X$ hefur opna grennd $U$, sem er grannmoa
opnu mengi í $\R^n$. 
\begin{ath}
  Aðeins eitt $n$ kemur til greina því unnt er að sýna fram á með
  algebrulegri grannfræði að ef opið mengi í $\R^n$ er grannóta opnu
  mengi í $\R^k$, þá er $n=k$.
\end{ath}
Af skilgreiningunni leiðir strax:
\begin{enumerate}[(i)]
  \item Víðáttur eru staðvegsamanhangandi og þar með staðsamanhangandi.
  \item Samanhangandi víðáttur eru vegsamanhangandi. 
  \item Víðáttur eru staðþjappaðar. 
\end{enumerate}
\begin{daemi}
  (a) Algebrulega mengið $A := \left\{ (x,y) \in\R^2 : xy = 0 \right\}$
  er ekki víðátta. Hins vegar er $A\setminus\{(0,0)\}$ einvíð víðátta
  sem hefur 4 samhengisþætti.

  (b) Algebrulegt mengi\[
  A := \left\{ x\in\R^n : f_1(x)=\cdots = f_k(x)=0 \right\}
  \qquad(k\leq n)
  \]
  er $(n-k)$-víð víðátta ef Jacobi-fylkið $J(f_1,\dots,f_k)$ hefur
  metorðið $k$, skv. setningu um fólgin föll.

  (c) Í skilgreiningu á víðáttu er þess oftast krafist að $X$ sé
  Hausdorff. Þó svo að víðátta sé \emph{stað-Hausdorff}, þá kemur ekki í
  sjálfu sér að hún sé Hausdorff. T.d. $\R\coprod\R / \sim$ með $x\sim
  x$ ef $x\neq 0$. Þessi víðátta er $T_1$.

  (d) $I$ bil, vefjum $I\times I$ upp í sívalning og sívalningnum í
  kleinuhring, þá fæst víðátta sem er grannmóta $S^1\times S^1$.

  (e) Langa línan (þ.e.a.s. langa hálflínan án upphafspunkts) er
  víðátta.
\end{daemi}

\begin{skilgr}
  Tvívíðar víðáttur kallast yfirleitt \emph{fletir}\index{flötur} og
  einvíðar \emph{ferlar}\index{ferill}.
\end{skilgr}

\section{Varprúm (yfir $\R$)}
Látum $\mathbb P_n(\R)$ tákna mengi allra lína gegnum núllpunkt í
$\R^{n+1}$. Fyrir $x$ úr $\R^{n+1}\setminus\left\{ 0 \right\}$ látum við
$\overline{0x}$ tákna línuna gegnum $0$ og $x$ og skilgreinum átæka
vörpun\[
\R^{n+1}\setminus \left\{ 0 \right\}\xrightarrow{\pi} \mathbb P_n(\R),
x\mapsto \overline{0x}.
\]
Setjum deildagrannmynstrið á $\mathbb P_n(\R)$ og köllum grannrúmið
$\mathbb P_n(\R)$ \emph{$n$-víða varprúmið yfir $\R$}\index{varprúm}.

Sýnum að $\mathbb P_n(\R)$ sé $n$-víð víðátta: Táknum einingarkúluhvelið
í $\R^{n+1}$ með $S^n$. Vörpunin $\pi|_{S^n} : S^n \to \mathbb P_n(\R)$
er átæk og hefur trefjarnar $\left\{ -x,x \right\}$. Skilgreinum vensl
$\sim$ á $S^n$ með $x\sim y$ \emph{þ.þ.a.a.} $y=\pm x$. Þá fæst samfelld
og gagntæk vörpun $\hat\pi : S^n/\sim \to \mathbb P_n(\R)$. En
$S^n/\sim$ er þjappað og $\mathbb P_n(\R)$ er Hausdorff (gangið úr
skugga um það), svo að $\hat\pi$ er grannmótun. Hins vegar er ljóst að
$S^n/\sim$ er $n$-víð víðátta: Ef $x\in S^n$ og $U$ er opið hálfhvel sem
inniheldur $x$, þá er $\hat\pi|_U : U \to \hat\pi(U)$ grannmótun og
$\hat\pi(U)$ er opin grennd um $\hat\pi(x)$. 

\begin{skilgr}
  \emph{Verkun granngrúpu $G$ á grannrúm $X$}\index{verkun granngrúpu á
  grannrúm} er samfelld vörpun $G\times X\to X, (g,x)\mapsto gx$ þ.a.
  fyrir öll $x$ úr $X$ og $g,h$ úr $G$ gildi\[
  e_G x = x,
  \qquad (gh)x = g(hx),
  \]
  þar sem $e_G$ er hlutleysa á $G$. Fáum jafngildisvensl á $X$ með\[
  x\sim y
  \qquad\text{þ.þ.a.a.}\qquad
  \text{til sé } g\in G \text{ með } gx = y.
  \]
  Jafngildisflokkur staks $x$ úr $X$ er\[
  Gx := \left\{ gx : g\in G \right\}
  \]
  og kallast \emph{braut}\index{braut} staksins $x$ (m.t.t.
  verkunarinnar). Setjum $X/G := X/\sim$ með deildagrannmynstrinu og
  köllum $X/G$ \emph{brautarúm}\index{brautarúm} verkunarinnar.
\end{skilgr}
\begin{ath}
  Fyrir sérhvert $g$ úr $G$ er vörpunin\[
  \phi_g : X\to X, x\mapsto gx
  \]
  grannmótun; andhverfan er $\phi_{g^{-1}}$. 
\end{ath}
Fáum sértilfelli: Ef $G = (\R,+)$, þá kallast grannrúm $X$, sem $\R$
verkar á, \emph{hreyfikerfi}\index{hreyfikerfi} (e. dynamical system). Í
þessu tilfelli er litið á $X$ sem mengi af \emph{ástöndum} einhvers
kerfis. Fyrir hvert ástand $x$ er ákveðið hvert ástandið verður eftir
$t$ tímaeiningar; táknum það ástand $tx$. Höfum $0\cdot x = x$, $(t+s)x
= t(sx)$. 

\begin{daemi}
  (1) $\mathbb U\times\C\to\C, (u,w)\mapsto uw$. Brautirnar eru annars
  vegar hringir með miðju $o$ og hins vegar $\left\{ o \right\}$.
  Brautarúmið $\C/\mathbb U$ er grannmóta $[0,+\infty[$. 

  (2) $\Z^2\times\R^2\to\R^2,\left( (m,n),(x,y)
  \right)\mapsto(x+m,y+n)$. \emph{Sýnið:} Brautarúmið er grannmóta
  $S^1\times S^1$.
\end{daemi}







\chapter{Teljanleiki og aðskiljanleiki}
\section{Teljanleikaskilyrði}

\paragraph{Upprifjun}
Grenndagrunnur fyrir punkt $x$ í grannrúmi er mengi $\mathcal U(x)$ af
(opnum) grenndum um $x$ þ.a. sérhver grennd um $x$ innihaldi stak úr
$\mathcal U(x)$. 
\begin{daemi}
  Ef $(X,d)$ er firðrúm, þá er \[
  \mathcal U(x) := \left\{ B_d(x,\frac 1n):n\in\N^* \right\}
  \]
  grenndagrunnur fyrir $x$ úr $X$.
\end{daemi}

\paragraph{}
\begin{skilgr}
  Segjum að 
  \begin{enumerate}[(i)]
    \item fullnægi \emph{fyrsta teljanleikaskilyrðinu}\index{fyrsta
      teljanleikaskilyrðið} ef sérhver punktur úr $X$ hefur teljanlegan
      grenndagrunn,
    \item fullnægi \emph{öðru teljanleikaskilyrðinu}\index{annað
      teljanleikaskilyrðið} ef $\mathcal T_X$ hefur teljanlegan grunn,
    \item sé \emph{Lindelöf-rúm}\index{Lindelöf-rúm} ef sérhver opin
      þakning á $X$ hefur teljanlega hlutþakningu,
    \item  sé \emph{sundurgreinanlegt}\index{sundurgreinanlegt} ef það
      hefur teljanlegt þétt hlutmengi.
  \end{enumerate}
\end{skilgr}
\begin{setn}
  Annað teljanleikaskilyrðið hefur öll hin í för með sér (en almennt
  ekki öfugt). Ef $X$ er firðanlegt, þá er (i) alltaf uppfyllt og hin
  þrjú eru jafngild.
\end{setn}
\begin{proof}
  Einföld æfing, sjá bls. 191 og dæmi 5 í grein 30 í kennslubók.
\end{proof}
\begin{setn}
  \begin{enumerate}[(i)]
    \item $X$ grannrúm og $Y\subseteq X$. Ef $X$ fullnægir fyrsta [öðru]
      teljanleikaskilyrðinu, þá gerir $Y$ það líka.
    \item Ef grannrúm $X_1,\dots,X_n$ uppfylla fyrsta [annað]
      teljanleikaskilyrðið, þá gerir $\prod_{i=1}^n X_i$ það líka.
  \end{enumerate}
\end{setn}
\begin{proof}
  Næsta ljóst, sjá bls. 191 í kennslubók.
\end{proof}
\begin{daemi}
  Látum $\R_\ell$ vera $\R$ með grannmynstrinu sem grunnurinn $\left\{
  \left[a,b\right[ : a,b\in\R, a<b \right\}$ framleiðir.
  \begin{enumerate}[(i)]
    \item $\R_\ell$ fullnægir fyrsta teljanleikaskilyrðinu: Ef
      $x\in\R_\ell$, þá er $\left\{ \left[x,x+1/n\right[ : n\in\N^*
      \right\}$ teljanlegur grenndagrunnur fyrir $x$.
    \item $\R_\ell$ er sundurgreinanlegt vegna þess að $\Q$ er þétt í
      $\R_\ell$. 
    \item $\R_\ell$ fullnægir ekki öðru teljanleikaskilyrðinu: Látum
      $\mathcal B$ vera grunn fyrir $\R_\ell$. Fyrir
      sérhvert $x\in\R_\ell$ veljum við $B_x$ úr $\mathcal B$ þ.a. $x\in
      B_x\subseteq \left[x,x+1\right[$. Þá er auðséð að $B_x\neq B_y$ ef
      $x\neq y$ og þar með er $\mathcal B$ ekki teljanlegur.
    \item $\R_\ell$ er Lindöf: Látum $\mathcal A$ vera opna þakningu á
      $\R_\ell$. Þá er til fínni þakning á $\R_\ell$ af grunnstökum,
      við getum því gert ráð fyrir að\[
      \mathcal A 
      = \left\{ 
        \left[a_\alpha, b_\alpha\right[ : \alpha\in J
      \right\}.
      \]
      Setjum $C := \bigcup_{\alpha\in J}\left]a_\alpha,b_\alpha\right[$
      og lítum á $C$ sem hlutrúm í $\R$ (með venjulega grannmynstrinu).
      Þar sem $\left\{ \left]a_\alpha,b_\alpha \right[ : \alpha\in J
      \right\}$ er opin þakning á $C$, þá  hefur hún teljanlega
      hlutþakningu\[
      \mathcal A'
      = \left\{ \left]a_\alpha,b_\alpha\right[ :
      \alpha=\alpha_1,\alpha_2,\dots \right\}.
      \]
      Okkur nægir því að sýna að $\R\setminus C$ sé teljanlegt: Ef
      $x\in\R\setminus C$, þá er til $\alpha$ úr $J$ þ.a. $x=a_\alpha$.
      Veljum $q_x$ úr $\left]a_\alpha,b_\alpha\right[\cap\Q$ og fáum að
      $\left]x,q_x\right[\subseteq\left]a_\alpha,b_\alpha\right[\subseteq
      C$. Fyrir $x,y\in\R\setminus C$ þ.a. $x<y$ gildir að $q_x<q_y$ því 
      annars væri $y\in\left]x,q_x\right[\subseteq C$. Þetta gefur
      stranglega vaxandi fall $\R\setminus C\to \Q$, svo að $\R\setminus
      C$ er teljanlegt.
  \end{enumerate}
\end{daemi}
\begin{daemi}
  Faldrúm tveggja Lindelöf-rúma þarf ekki að vera Lindelöf. $\R_\ell$ er
  Lindelöf, en $\R_\ell^2 := \R_\ell\times\R_\ell$, svokölluð
  \emph{Sorgenfrey-slétta}\index{Sorgenfrey-slétta}, er ekki Lindelöf:
  \begin{itemize}
    \item $L := \left\{ x\times(-x) : x\in\R_\ell \right\}$ er lokað í
      $\R_\ell^2$.
    \item Þekjum $\R_\ell^2$ með öllum mengjum af gerðinni
      $\left[a,b\right[\times\left[-a,d\right[$ og $\R_\ell^2\setminus
      L$. Sýnum að þessi opna þakning eigi sér enga teljanlega
      hlutþakningu: $L\cap (\R_\ell^2\setminus L) = \emptyset$ og
      $L\cap(\left[a,b\right[ \times \left[-a,d\right[ ) = \left\{
      a\times(-a) \right\}$ og $L$ er óteljanlegt.
  \end{itemize}
\end{daemi}
\begin{daemi}
  Hlutrúm í Lindelöf-rúmi þarf ekki að vera Lindelöf: $I^2 = I\times
  I$ ($I = [0,1]$) með orðabókargrannmynstrinu er þjappað og þar
  með Lindelöf-rúm. Hlutrúmið $I\times\left]0,1\right[$ í $I^2$ er hins
  vegar ekki Lindelöf-rúm því að opna þakningin $\left\{ \left\{ t
  \right\}\times\left]0,1\right[ :t\in I \right\}$ á sér enga teljanlega
  hlutþakningu (þetta er skipting).
\end{daemi}


\section{Aðskilnaðarskilyrði}

\begin{skilgr}
  \begin{enumerate}[(i)]
    \item Grannrúm $X$ er sagt \emph{reglulegt}\index{reglulegt
      grannrúm} eða $T_3$\index{T3-rum@$T_3$-rúm}, ef það er $T_1$ (þ.e.
      einstökungarnir eru lokuð mengi) og fyrir sérhvert lokað mengi
      $A\subseteq X$ og sérhvert $b$ úr $X\setminus A$ eru til
      sundurlægar opnar grenndir $U$ um $a$ og $V$ um $b$. 
    \item Grannrúm $X$ er sagt \emph{normlegt}\index{normlegt grannrúm}
      eða $T_4$\index{T4-rum@$T_4$-rúm}, ef það er $T_1$ og fyrir öll
      sundurlæg lokuð mengi $A$ og $B$ í $X$ eru til sundurlægar opnar
      grenndir $U$ um $A$ og $V$ um $B$.
  \end{enumerate}
\end{skilgr}
\begin{setn}
  Látum $X$ vera $T_1$-rúm.
  \begin{enumerate}[(i)]
    \item $X$ er reglulegt \emph{þ.þ.a.a.} fyrir sérhvern punkt $x$ úr
      $X$ og sérhverja grennd $U$ um $x$ sé til grennd $V$ um $x$ þ.a.
      $\overline V\subseteq U$.
    \item $X$ er normlegt \emph{þ.þ.a.a.} fyrir sérhvert lokað mengi $A$
      í $X$ og sérhverja grennd $U$ um $A$ sé til grennd $V$ um $A$ þ.a.
      $\overline V\subseteq U$.
  \end{enumerate}
\end{setn}
\begin{proof}
  Auðséð (sjá bls. 196 í kennslubók).
\end{proof}
\begin{setn}
  \begin{enumerate}[(i)]
    \item Hlutrúm í Hausdorff-rúmi [reglulegu rúmi] er Hausdorff
      [reglulegt].
    \item Faldrúm Hausdorff-rúma [reglulegra rúma] er Hausdorff
      [reglulegt].
  \end{enumerate}
\end{setn}
\begin{proof}
  Höfum þegar séð Hausdorff-tilfellin svo við látum okkur nægja að sanna
  fullyrðingarnar fyrir regluleg rúm.

  (i) $Y$ hlutrúm í reglulegu rúmi $X$, $B$ lokað í $Y$ og $x\in
  Y\setminus B$. Þar sem $\overline B \cap Y = B$, þá er $x\notin
  \overline B$ og því eru til sundurlægar opnar grenndir $U$ um
  $\overline B$ og $V$ um $x$ í $X$. Þar með eru $U\cap Y$ og $V\cap Y$
  sundurlægar opnar grenndir um $B$ og $x$ í $Y$. 

  (ii) $(X_\alpha)_{\alpha\in I}$ fjölskylda af reglulegum grannrúmum og
  $x = (x_\alpha)_{\alpha\in I}\in \prod_{\alpha\in I} X_\alpha$. Látum
  $U$ vera opna grennd um $x$ og sýnum að $x$ eigi sér lokaða grenndí
  $\prod_{\alpha\in I} X_\alpha$ sem sé innihaldin í $U$ (sjá síðustu
  setningu). Veljum grunngrennd $\prod_{\alpha\in I} U_\alpha$ um $x$ í
  $U$. Ef $U_\alpha\neq X_\alpha$, þá veljum við grennd $V_\alpha$ um
  $x_\alpha$ í $X_\alpha$ þ.a. $\overline V_\alpha\subseteq U_\alpha$.
  Ef $U_\alpha = X_\alpha$, þá setjum við $V_\alpha=X_\alpha$. Þá er
  $\prod_{\alpha\in I} V_\alpha$ grennd um $x$ og\[
  \overline{\prod_{\alpha\in I} V_\alpha}
  = \prod_{\alpha\in I} \overline V_\alpha
  \subseteq \prod_{\alpha\in I} U_\alpha
  \subseteq U.
  \]
\end{proof}
% 02.03.2010
\begin{daemi}
  $\R_K$ er Hausdorff en er ekki reglulegt. Mengið $K=\left\{
  1/n:n\in\N^* \right\}$ er lokað í $\R_K$. Sýnum að ekki séu til opnar
  sundurlægar grenndir um $0$ og $K$. Ef $U$ er opin grennd um $0$ og
  $V$ er opin grennd um $K$, þá er til grunngrennd
  $\left]a,b\right[\setminus K$ um $0$ sem er innihaldin í $U$. Tökum
  $1/n$ úr $\left]a,b\right[$ og veljum $c,d$ úr $\R$ þ.a. $0<c<d$ og
  $\frac 1n\in\left]c,d\right[\subseteq V$. Þá er
  $\left]c,d\right[\cap\left( \left]a,b\right[\setminus K \right) \neq
  \emptyset$.
\end{daemi}
\begin{daemi}
  $\R_\ell$ er  normlegt: Látum $A$ og $B$ vera sundurlæg lokuð mengi í
  $\R_\ell$. Fyrir sérhvert $a$ úr $A$ veljum við $x_a>a$ þ.a.
  $\left[a,x_a\right[\cap B =\emptyset$ og fyrir sérhvert $b$ úr $B$
  veljum við $x_b>b$ þ.a. $\left[b,x_b\right[\cap A=\emptyset$. Þá er
  $U:= \bigcup_{a\in A}\left[a,x_a\right[$ opin grennd um $A$ og $V :=
  \bigcup_{b\in B}\left[b,x_b\right[$ opin grennd um $B$, $U\cap V =
  \emptyset$ því að $\left[a,x_a\right[\cap\left[b,x_b\right[=\emptyset$
  fyrir öll $a$ úr $A$ og $b$ úr $B$.
\end{daemi}
\begin{setn}
  \begin{enumerate}[(i)]
    \item Firðanleg grannrúm eru normleg.
    \item Þjöppuð Hausdorff-rúm eru normleg.
    \item Regluleg Lindelöf-rúm eru normleg. 
  \end{enumerate}
\end{setn}
\begin{proof}
  (i) Látum $A$ og $B$ vera sundurlæg lokuð mengi í firðŕumi $(X,d)$.
  Fyrir sérhvert $x$ úr $A$ veljum við $\varepsilon_x>0$ þ.a.
  $B_d(x,\varepsilon_x)\cap B =\emptyset$ og fyrir sérhvert $y$ úr $B$
  veljum við $\varepsilon_y>0$ þ.a. $B_d(y,\varepsilon_y)\cap
  A=\emptyset$. Þá eru $U :=\bigcup_{x\in A} B_d(x,\varepsilon_x/2)$ og
  $V:=\bigcup_{x\in B} B_d(y,\varepsilon_x/2)$ opnar sundurlægar
  grenndir um $A$ og $B$ í $X$. 

  (ii) Látum $A$ og $B$ vera sundurlæg lokuð mengi í þjöppuðu
  Hausdorff-rúmi $X$. Þá eru $A$ og $B$ þjöppuð svo fyrir sérhvert $x$
  úr $B$ eru til sundurlægar opnar grenndir $U_x$ um $A$ og $V_x$ um
  $x$. Nú er $\left\{ V_x : x\in B \right\}$ opin þakning á $B$ svo til
  eru $x_1,\dots,x_k$ úr $B$ þ.a. $B\subseteq V_{x_1}\cup\cdots\cup
  V_{x_k}$. Setjum $U := U_{x_1}\cap\cdots\cap U_{x_k}$ og $V:=
  V_{x_1}\cup\cdots\cup V_{x_k}$. Þá fæst að $B\subseteq V$, $A\subseteq
  U$ og $V\cap U=\emptyset$.

  (iii) Tökum fyrst eftir því að þó svo hlutrúm í Lindelöf-rúmi þurfi
  ekki að vera Lindelöf (sjá æfingu 5 bls. 193 í kennslubók), þá er
  \emph{lokað} hlutrúm í Lindelöf-rúmi alltaf Lindelöf-rúm (bætum bara
  fyllimengi þess við þakninguna). Látum $A$ og $B$ vera sundurlæg lokuð
  hlutmengi í reglulegu Lindelöf-rúmi $X$. Fyrir sérhvert $x$ úr $A$
  veljum við opnar sundurlægar grenndir $U_x$ um $x$ og $W_x$ um $B$.
  Þá er $U_x\subseteq X\setminus W_x$ sem er lokað svo að $\overline
  U_x\subseteq X\setminus W_x$, þar með er $\overline U_x\cap B
  =\emptyset$. Safnið $(U_x)_{x\in A}$ er opin þakning á $A$ og hefur
  því teljanlega hlutþakningu $(U_n)_{n\in\N}$ vegna þess að $A$ er
  lokað. Með því að setja $U_0\cup\cdots\cup U_n$ í stað $U_n$, þá getum
  við gert ráð fyrir að $(U_n)_{n\in\N}$ sé vaxandi. Með sama hætti fáum
  við opna vaxandi þakningu $(V_n)_{n\in\N}$ á $B$ þ.a. $\overline
  V_n\cap A = \emptyset$ fyrir öll $n\in\N$. Setjum nú\[
  U_n' := U_n\setminus\overline V_n, \qquad
  V_n' := V_n\setminus\overline U_n, \qquad
  U    := \bigcup_{n\in\N} U_n',     \qquad
  V    := \bigcup_{n\in\N} V_n'.
  \]
  Þá er $A\subseteq U$ og $B\subseteq V$. Sýnum að $U\cap V =
  \emptyset$. Ef $x\in U\cap V$, þá eru til $n,m\in\N$ með $x\in U_n'$
  og $x\in V_m'$. Segjum t.d. að $n\leq m$. Þá er $x\in
  V_m\setminus\overline U_m$ og $x\in U_n'\subseteq U_n\subseteq U_m$,
  sem er mótsögn!
\end{proof}
\begin{fylgisetn}\label{fylgisetn:regluleg_hausdorff}
  Staðþjöppuð Hausdorff-rúm eru regluleg. 
\end{fylgisetn}
\begin{proof}
  Látum $X$ vera staðþjappað Hausdorff-rúm. Þá er Alexandroff-þjöppunin
  $\hat X$ þjappað Hausdorff-rúm, svo skv. (ii) úr síðustu setningu er
  það normlegt. Af því leiðir að $\hat X$ er reglulegt og þar með er $X$
  reglulegt sem hlutrúm í $\hat X$.
\end{proof}
\begin{fylgisetn}
  Staðþjöppuð Hausdorff-rúm eru Baire-rúm.
\end{fylgisetn}
\begin{proof}
  Látum $X$ vera staðþjappað Hausdorff-rúm. Látum $(U_n)_{n\in\N}$ vera
  fjölskyldu af opnum mengjum í $X$ þ.a. um öll $n\in\N$ gildi
  $\overline U_n = X$. Sýnum að $\overline{\bigcap_{n\geq 0}U_n}=X$.
  Látum $V$ vera opið ekki-tómt hlutmengi í $X$. Skilgreinum runu
  $(V_n)_{n\geq 0}$ af ekki-tómum opnum mengjum í $X$ með eftirfarandi
  hætti
  \begin{itemize}
    \item $V_0 := V$.
    \item G.r.f. að við höfum valið opið-ekki-tómt mengi $V_n$. Þar sem
      $U_n$ er þétt í $X$, þá er $U_n\cap V_n\neq\emptyset$ og þar sem
      $X$ er reglulegt skv. fylgisetningu
      \ref{fylgisetn:regluleg_hausdorff}, þá er til opið ekki-tómt
      hlutmengi $V_{n+1}$ í $X$ þ.a. $\overline V_{n+1} \subseteq
      U_n\cap V_n$.
  \end{itemize}
  Þar eð $X$ er staðþjappað, þá getum við valið $V_1$ þ.a. $\overline
  V_1$ sé þjappað og fáum þá minnkandi runu af lokuðum ekki-tómum
  hlutmengjum \[
  \overline V_1
  \supseteq \overline V_2
  \supseteq \overline V_3
  \supseteq \cdots
  \]
  í þjappaða rúminu $\overline V_1$ og þar með\[
  \emptyset
  \neq\bigcap_{n\geq 0}\overline V_n
  \subseteq V \cap \left( \bigcap_{n\geq 0} U_n \right).
  \]
\end{proof}
\begin{setn}
  Velröðuð mengi eru normleg m.t.t. röðunargrannmynstursins.
\end{setn}
\begin{proof}
  Látum $X$ vera velraðað mengi. Tökum fyrst eftir að sérhvert bil af
  gerðinni $\left]x,y\right]$ er opið í $X$. Látum $A$ og $B$ vera lokuð
  sundurlæg hlutmengi í $X$. Fyrir sérhvert $a\in A$ látum við $I_a$
  vera opið mengi af gerðinni $\left]x_a,a\right]$ þ.a. $I_a\cap B =
  \emptyset$ ef $a\neq \min X$. Ef $a = \min X$ þá látum við $I_a$ vera
  opna mengið $\{a\}=\left]-\infty,a^+\right[$. Á samsvarandi hátt
  skilgreinum við $I_b$ fyrir öll $b$ úr $B$. Þá er fljótséð að
  $U:=\bigcup_{a\in A} I_a$ og $V:=\bigcup_{b\in B}I_b$ eru opnar
  sundurlægar grenndir um $A$ og $B$.
\end{proof}
\begin{ath}
  Hægt á að vera (JIM sagði að stæði í Munkres) að sanna tilsvarandi
  niðurstöðu fyrir hvaða línulega röðuð mengi sem er.
\end{ath}
% 08.03.2010
\begin{daemi}
  $S_\Omega\times \overline S_\Omega$ er \emph{ekki} normlegt rúm. Áður
  en við sýnum fram á þetta skulum við veita nokkrum atriðum eftirtekt
  \begin{itemize}
    \item $S_\Omega$ og $\overline S_\Omega$ eru normleg skv. síðustu
      setningu og því regluleg. Hér er því dæmi um reglulegt rúm, sem er
      ekki normlegt. Jafnframt sýnir þetta að faldrúm normlegra rúma
      þarf ekki að vera normlegt.
    \item $\overline S_\Omega\times\overline S_\Omega$ er þjappað
      Hausdorff-rúm og þar með normlegt svo þetta dæmi sýnir að hlutrúm
      í normlegu rúmi er ekki endilega normlegt.
  \end{itemize}
  Látum nú $\Delta$ tákna hornalínuna í $\overline
  S_\Omega\times\overline S_\Omega$. Þá er $\Delta$ lokað í $\overline
  S_\Omega\times \overline S_\Omega$,vegna þess að $\overline S_\Omega$
  er Hausdorff, og þar með er $A := \Delta\cap(S_\Omega\times\overline
  S_\Omega)=\Delta\setminus\left\{ \Omega\times\Omega \right\}$ lokað í
  $S_\Omega\times\overline S_\Omega$. Einnig er ljóst að $B := S_\Omega
  \times\left\{ \Omega \right\}$ er lokað i $S_\Omega\times\overline
  S_\Omega$ því að $\overline S_\Omega$ er $T_1$. $A$ og $B$
  eru því sundurlæg lokuð mengi í $S_\Omega\times\overline S_\Omega$.
  Sýnum að þau eigi sér ekki sundurlægar opnar grenndir. Beitum óbeinni
  sönnun og g.r.f. að til séu sundurlæg opin mengi $U$ og $V$ í
  $S_\Omega\times\overline S_\Omega$ þ.a. $A\subseteq U$ og $B\subseteq
  V$. Fyrir sérhvert $x$ úr $S_\Omega$ setjum við\[
  \beta(x) := 
  \min\left\{ 
    \beta\in \overline S_\Omega : x<\beta,x\times\beta\notin U
  \right\}
  \]
  sem er vel skilgreint þvi að $x\times\Omega\notin U$ og þar sem $V\cap
  U = \emptyset$ þá er $\beta(x)\in S_\Omega$. Skilgreinum runu
  $(x_n)_{n\geq 0}$ í $S_\Omega$ sem svo:
  \begin{align*}
    x_0     &:= \text{einhver punktur í } S_\Omega, \\
    x_{n+1} &:= \beta(x_n) \text{ fyrir öll } n\geq 0.
  \end{align*}
  Þá er runan stranglega vaxandi: $x_0<x_1<\cdots$. Setjum
  $b:=\sup_{n}x_n$ (alltaf til í $S_\Omega$) og fáum
  $\lim_{n\to\infty}x_n = b$ og því $\beta(x_n) = x_{n+1}\longrightarrow
  b$ og þar með $x_n\times\beta(x_n)\longrightarrow b\times b$. Þetta er
  í mótsögn við að $U$ er opin grennd um $b\times b$ og $x_n \times
  \beta(x_n)\notin U$ fyrir öll $n$. 
\end{daemi}
\begin{skilgr}
  Látum $A$ og $B$ vera lokuð hlutmengi í grannrúmi $X$. Við segjum að
  unnt sé að \emph{aðskilja $A$ og $B$ með samfelldu
  falli}\index{aðskilja með samfelldu falli} ef til er samfellt fall
  $f:X\to\left[ 0,1 \right]$ þ.a. $f|_A = 0$ og $f|_B = 1$.
\end{skilgr}
\begin{skilgr}
  Grannrúm er sagt \emph{fullkomlega reglulegt}\index{fullkomlega
  reglulegt grannrúm} ef það er $T_1$ og hægt er að aðskilja sérhvert
  lokað hlutmengi $A$ í $X$ og sérhvern einstökung $\left\{ x
  \right\}$ þar sem $x\notin A$ með samfelldu falli.
\end{skilgr}
\begin{setn}
  [Hjálparsetning Urysohns]\index{Urysohn!hjálparsetning}
  Tvö lokuð og sundurlæg hlutmengi í normlegu rúmi er unnt að aðskilja
  með samfelldu falli.
\end{setn}
\begin{proof}
  Látum $A$ og $B$ vera sundurlæg lokuð mengi í normlegu rúmi $X$.
  Segjum að endanlega runu af opnum mengjum $C=(C_0,C_1,\dots,C_m)$ sé
  \emph{(leyfileg) keðja af lengd} $m$ ef\[
  A\subseteq C_0 \subseteq C_1 \subseteq\cdots\subseteq C_m\subseteq
  X\setminus B
  \]
  og $\overline C_{k-1}\subseteq \overline C_k$; $k=1,\dots,m$. Setjum
  $C_{-1}=\emptyset$ og $C_{m+1} = X$ og köllum opnu mengin
  $C_{k+1}\setminus \overline C_{k-1}$ \emph{opnu stalla} keðjunnar $C$,
  $k=0,\dots,m$. Tökum eftir að opnu stallarnir mynda (opna) þakningu á
  $X$. Fyrir slíkt $C$ skilgreinum við $\chi_C:X\to \left[ 0,1 \right]$
  með því að setja fyrir sérhvert $k$ úr $\left\{ 0,\dots,m \right\}$\[
  \chi_C(x) = \begin{cases}
    \frac km, &\text{ef $x\in C_k\setminus \overline C_{k-1}$}\\
    1,        &\text{ef $x\in X\setminus C_m$}.
  \end{cases}
  \]
  Tökum eftir að á stallinum $C_{k+1}\setminus \overline C_{k-1}$ tekur
  $\chi_C$ bara gildin $\frac{k+1}{m}$ og $\frac{k}{m}$ og munur þeirra
  er $\frac 1m$. Búum til runu $(C^{(n)})_{n\geq 1}$ af leyfilegum
  keðjum, þar sem $C^{(n)}$ er af lengd $2^n$:\[
  C^{(1)} := (C_0,C_1)
  \text{ þar sem }
  C_1 = X\setminus B
  \text{ og $C_0$ er valið þ.a. }
  A\subseteq C_0\subseteq\overline C_0 \subseteq C_1.
  \]
  Eftir að hafa búið til leyfilega keðju $C^{(n)}$ af lengd $2^n$, þá
  búum við til keðju $C^{(n+1)}$ af lengd $2^{n+1}$ með eftirfarandi
  hætti: Fyrir sérhvert $k$ úr $\left\{ 1,\dots,2^{n} \right\}$ veljum
  við opið mengi $C'$ þ.a.\[
  \overline{C^{(n)}_{k-1}}
  \subseteq C'
  \subseteq \overline{C'}
  \subseteq C^{(n)}_k
  \]
  og bætum því við keðjuna; tölusetjum síðan upp á nýtt í vaxandi röð.
  Ljóst er að $C^{n+1}$ er af lengd $2^{n+1}$.

  Setjum $\chi_n := \chi_{C^{(n)}}$  og fáum minnkandi runu af föllum
  $X\to\left[ 0,1 \right]$. Hún hefur markgildi $f:=\lim_{n\to\infty}
  \chi_n$. Ljóst er að $f|_A = 0$ og $f|_B = 1$. Sýnum að $f$ sé
  samfellt:

  Fljótséð er að\[
  0
  \leq \chi_n -f 
  = \sum_{k=n}^{\infty}(\chi_k-\chi_{k+1})
  \leq \sum_{k=n}^{\infty} \frac{1}{2^{k+1}}
  = \frac{1}{2^n},
  \]
  og þar með $\chi_n - \frac{1}{2^n}\leq f \leq \chi_n$. Munurinn á
  stærsta og minnsta gildi $\chi_n$ á opnu stöllum keðjunar
  $C^{(n)}$ er $\frac{1}{2^n}$ svo að munurinn á gildum $f$ í tveimur
  punktum úr sama stalli $C^{(n)}$ er $\leq
  \frac{1}{2^n}+\frac{1}{2^n}= \frac{1}{2^{n+1}}$. Látum $x\in X$ og
  $\varepsilon>0$. Veljum $n$ þannig að $\frac{1}{2^{n-1}}<\varepsilon$.
  Opnu stallar keðjunnar $C^{(n)}$ mynda opna þakningu á $X$. Tökum
  stall sem inniheldur $x$. Hann er opin grennd um $x$ í $X$ og
  $|f(x) - f(y)\leq \frac{1}{2^{n-1}}<\varepsilon$ fyrir öll $y$ úr
  honum.
\end{proof}

\paragraph{Upprifjun \textnormal{(sjá líka næsta dæmablað)}}
Látum $X$ vera mengi, $f:X\to \R$ vera takmarkað fall og $A\subseteq X$.
setjum \[
\| f \|_A := \sup\left\{ |f(x)| : x\in A \right\}.
\]
Þá er $\|\cdot\|_X$ staðall á vigurrúmi allra takmarkaðra falla á $X$.
Ef $X$ er grannrúm, þá er vigurrúm allra \emph{samfelldra} takmarkaðra
falla $X\to\R$ með stalinum $\|\cdot\|_X$ fullkomið staðalrúm (þ.e.a.s.
Banach-rúm).

\paragraph{}
\begin{setn}
  [Framlengingarsetning
  Tietze-Urysohns]\index{Urysohn!framlengingarsetning
  Tietze-Urysohns}\index{Tietze!framlengingarsetning
  Tietze-Urysohns}
  Fyrir grannrúm $X$ eru eftirtalin skilyrði jafngild:
  \begin{enumerate}[(i)]
    \item $X$ er normlegt.
    \item Fyrir sérhvert lokað mengi $A$ í $X$ og sérhvert samfellt fall
      $f:A\to\left[ 0,1 \right]$ er til samfellt fall $F:X\to\left[ 0,1
      \right]$ með $F|_A = f$. 
  \end{enumerate}
\end{setn}
\begin{hjalparsetn}\label{hjalp:urysohn.1}
  Látum $X$ vera normlegt rúm, $A$ vera lokað í $X$ og $f:A\to\R$ vera
  takmarkað og samfellt. Þá er til samfellt fall $F:X\to\R$ með
  $\|f-F\|_A\leq \frac 23 \|f\|_A$ og $\|F\|_X = \frac 13\|f\|_A$.
\end{hjalparsetn}
% 09.03.2010
\begin{ath}
  Ef $A$, $B$ lokuð í grannrúmi $X$ og unnt er að aðskilja $A$ og $B$
  með samfelldu falli $f:X\to\left[ 0,1 \right]$, þ.e.a.s. $f|_A = 0$ og
  $f|_B = 1$, þá er, fyrir öll $a,b$ úr $\R$ þ.a. $a<b$, hægt að finna
  samfellt fall $g:X\to\left[ a,b \right]$ þ.a. $g|_A = a$ og $g|_B =
  b$, þ.e. fallið\[
  g(x) := a + (b-a)f(x).
  \]
\end{ath}
\begin{proof}
  [Sönnun á hjálparsetningu \ref{hjalp:urysohn.1}]
  Látum $r := \|f\|_A = \sup\left\{ |f(a)|:a\in A \right\}$. Mengin
  $B:=f^{-1}(\left[ r/3,r \right])$ og $C:=f^{-1}(\left[-r,-r/3\right])$
  eru sundurlæg og lokuð í $X$, svo skv. Urysohn er til samfellt fall
  $F:X\to\left[-r/3,r/3\right]$ þ.a. $F|_C=-r/3$ og $F|_B=r/3$. Fljótséð
  er að $F$ fullnægir umbeðnum skilyrðum.
\end{proof}
\begin{setn}
  [Framlengingarsetning Tietze-Urysohn]
  Fyrir grannrúm $X$ eru eftirtalin skilyrði jafngild: 
  \begin{enumerate}[(i)]
    \item $X$ er normlegt.
    \item Fyrir sérhvert lokað mengi $A$ í $X$ og sérhvert samfellt
      fall $f:A\to\left[ 0,1 \right]$ er til samfellt fall $F:X\to\left[
      0,1 \right]$ með $f|_A = f$.
  \end{enumerate}
\end{setn}
\begin{proof} 
  \emph{(ii)$\Rightarrow$(i):} Látum $A$ og $B$ vera lokuð og sundurlæg
  í $X$. Þá er $f:A\cup B\to\left[ 0,1 \right]$, $f|_A=0$ og $f|_B=1$
  samfellt fall. Það framlengist í samfellt fall $F:X\to\left[ 0,1
  \right]$ skv. (ii) og $F^{-1}(\left[ 0,1/2 \right[)$ og
  $F^{-1}(\left] 1/2,1 \right])$ eru opnar sundurlægar grenndir um $A$
  og $B$.

  \emph{(i)$\Rightarrow$(ii):} Með þrepun búumm við til runu
  $(F_n)_{n\geq 1}$ af samfelldum föllum $F_n : X\to\R$ þ.a. 
  \begin{equation}
    \left\| f - \sum_{k=1}^n F_k \right\|_A
    \leq
    \left( \frac 23 \right)^n \|f\|_A
    \label{eq:tietze_proof.1}
  \end{equation}
  og
  \begin{equation}
    \|F_n\|_X
    \leq \frac 13 \left( \frac 23 \right)^{n-1}\cdot\|f\|_A
    \label{eq:tietze_proof.2}.
  \end{equation}
  Sýnum á eftir hvernig smíða má rununa.
  Út frá \eqref{eq:tietze_proof.2} fæst að\[
  \sum_{n=1}^{\infty}
  \leq \sum_{n=1}^{\infty}\frac 13\left( \frac 23
    \right)^{n-1}\cdot\|f\|_A
  = \|f\|_A
  < +\infty
  \]
  og þar með er $\sum_{n\geq 1} F_n$ samleitin í jöfnum mæli á $X$ og
  hefur samfellt fall $F$ sem markgildi. Út frá
  \eqref{eq:tietze_proof.1} fæst svo að \[
  \|f-F\|_A
  = \lim_{n\to\infty} \left\| f - \sum_{k=1}^{n}F_k\right\|_A
  = 0
  \]

  \emph{Smíði rununnar:} Skv. síðustu hjálparsetningu er til samfellt
  fall $F_1 : X\to\R$ þ.a. $\|f-F_1\|_A\leq\frac 23\|f\|_A$ og
  $\|F_1\|_X=\frac 13\|f\|_A$. Þegar $F_1,\dots,F_n$ eru fengin þá getum
  við beitt hjálparsetningu á flalið $g:= f-\sum_{k=1}^{n}F_k$ til þess
  að finna samfellt fall $F_{n+1}:X\to\R$ þ.a.\[
  \left\|f-\sum_{k=1}^{n+1} F_k \right\|_A
  = \|g-F_{n+1}\|_A
  \leq \frac 23 \|g\|_A
  \leq \frac 23\left( \frac 23 \right)^n \|f\|_A
  = \left( \frac 23 \right)^{n+1} \|f\|_A.
  \]
\end{proof}
\emph{Viðbót:} Skilyrðin (i) og (ii) eru jafngild eftirfarandi skilyrði:
\begin{itemize}
  \item[(iii)] Fyrir sérhvert lokað mengi $A$ í $X$ og sérhvert
    samfellt $f:A\to\R$ er til samfellt fall $F:X\to\R$ með $F|_A = f$.
\end{itemize}

\emph{(iii)$\Rightarrow$(i):} Ef $A$ og $B$ eru sundurlæg lokuð
hlutmengi í $X$, þá er fallið $f:A\cup B\to\R$, $f|_A=0$ og $f|_B=1$
samfellt svo skv. (iii) framlengist það í samfellt fall $F:X\to\R$ og
$F^{-1}(\left]-\infty,1/2\right[)$ og $F^{-1}(\left]1/2,+\infty\right[)$
eru þá opnar sundurlægar grenndir um $A$ og $B$.

\emph{(ii)$\Rightarrow$(iii):} $A$ lokað í $X$ og $f:A\to\R$ samfellt.
Þá er $g=\frac 2\pi \arctan\circ f: A\to\left]-1,1,\right[$ samfellt og
hefur samfellda framlengingu $G:X\to\left[ -1,1 \right]$. Mengið
$B:=G^{-1}(\{-1,1\})$ er lokað og $A\cap B=\emptyset$; þar með er til
(skv. (ii)) samfellt fall $\varphi:X\to\left[ 0,1 \right]$ þ.a.
$\varphi|_A=1$ og $\varphi|_B=0$. Þá er $\varphi\cdot
G:X\to\left]-1,1\right[$ og $\varphi\cdot G|_A=G|_A=g$ (ath að
$\varphi|_A=1$) svo að fallið
$F:=\tan\circ(\frac{\pi}{2}\cdot\varphi\cdot G):X\to\R$ er samfellt og
$F|_A=f$.

\begin{setn}
  [Greypingarsetning]\index{greypingarsetning}

  Látum $X$ vera grannrúm og 
  \[  \left\{ f_\alpha:X\to Y_\alpha :\alpha\in I\right\}  \]
  vera fjölskyldu af samfelldum vörpunum þ.a.
  \begin{enumerate}[(i)]
    \item Ef $x,y\in X$ og $x\neq y$, þá er til $\alpha\in I$ þ.a.
      $f_\alpha(x)\neq f_\alpha(y)$ (sagt að fjölskyldan
      $(f_\alpha)_{\alpha\in I}$ aðgreini punkt í $X$).
    \item Ef $A$ er \emph{lokað} í $X$ og $x\in X\setminus A$, þá er til
      $\alpha\in I$ þ.a. $f_\alpha(x)\notin f_\alpha(A)$. 
  \end{enumerate}
  Þá er vörpunin $f:=(f_\alpha)_{\alpha\in I}:X\to\prod_{\alpha\in
  I}Y_\alpha$ greyping.
\end{setn}
\begin{proof}
  Vörpunin $f:X\to\prod_{\alpha\in I} Y_\alpha$ er samfelld og skv. (i)
  er hún eintæk. Viljum sýna að hún gefi af sér grannmótnun $X\to f(X)$,
  en til þess nægir að sýna að $f(U)$ sé opið í $f(X)$ fyrir sérhvert
  opið $U$ í $X$.

  Látum $U$ vera opið í $X$ og $y\in f(U)$. Veljum $x$ úr $U$ þ.a.
  $f(x)=y$. Þar eð $X\setminus U$ er lokað og $x\notin X\setminus U$, þá
  er skv. (ii) til $\alpha$ úr $I$ þ.a.
  $f_\alpha(x)\notin\overline{f_\alpha(X\setminus U)}$. Mengið
  $V:=\pi_\alpha^{-1}(Y_\alpha\setminus\overline{f_\alpha(X\setminus
  U)}) = \left\{ (y_\beta)_{\beta\in I}
  y_\alpha\notin\overline{f_\alpha(X\setminus U)}\right\}$ er opið í
  $\prod_{\alpha\in I} Y_\alpha$. Sýnum að $V\cap f(X)\subseteq f(Y)$:
  Ef $z\in X$ og $f(z)\in V$, þá er $f_\alpha(z)\notin
  \overline{f_\alpha(X\setminus U)}$ og því $z\notin X\setminus U$ sem
  þýðir að $z\in U$. Þar með er sýnt að $f(U)$ er opið í $f(X)$.
\end{proof}
% 15.03.2010
\begin{ath}
  Ef $X$ er $T_1$-rúm, þá leiðir (ii) til (i).
\end{ath}
\begin{setn}
  [Firðanleikasetning Urysohns]
  \index{Urysohn!firðanleikasetning}\index{firðanleikasetning Urysohns}
  Reglulegt grannrúm sem hefur teljanlegan grunn er firðanlegt.
\end{setn}
\begin{proof}
  Látum $X$ vera reglulegt rúm þ.a. $\mathcal T_X$ eigi sér teljanlegan
  grunn $\mathcal B$. Þar sem $X$ er reglulegt Lindelöf-rúm, þá er það
  normlegt. Fyrir sérhvert $(B,C)$ úr $\mathcal B\times\mathcal B$ sem
  uppfyllir $\overline{B}\subseteq C$, getum við fundið
  samfellt fall $f: X\to\left[ 0,1 \right]$ þ.a. $f|_{\overline B}=0$ og
  $f|_{X\setminus C} = 1$. Fjöldi slíkra spyrða (tvennda) er
  teljannlegur; látum $(f_n)_{n\in\N}$ vera upptalningu á tilheyrandi
  föllum. Sýnum að safnið $\left\{ f_n:X\to [0,1]\mid n\in\N \right\}$
  fullnægi skilyrði (ii) úr greypingarsetningunni og þar með einnig
  skilyrði (i) úr greypingarsetningunni:

  Ef $A$ er lokað í $X$ og $x\in X\setminus A$, þá er til $C$ úr
  $\mathcal B$ með $x\in C\subseteq X\setminus A$. Finnum opna grennd
  $V$ um $X$ með $\overline V\subseteq C$ og síðan $B$ úr $\mathcal B$
  þ.a. $x\in B\subseteq V$. Þá er $x\in B\subseteq\overline B\subseteq
  C$. Látum $f_n$ vera fallið sem tilheyrir $(B,C)$. Ljóst er að $f_n(x)
  = 0$ og $f_n |_A = 1$ svo að sýnt er að $(f_n)_{n\in\N}$ fullnægir
  forsendum greypingarsetningarinnar.

  Fáum því greypingu $f=(f_n)_{n\in\N}: X\to [0,1]^\omega$. En
  $[0,1]^\omega$ er firðanlegt vegna þess að það er hlutrúm í
  $\R^\omega$, sem er firðanlegt.
\end{proof}

\section{Hlutun á einum}

\begin{skilgr}
  Látum $X$ vera grannrúm og $f:X\to\C$ vera vörpun. Skilgreinum \[
  \supp(f) := \left\{ x\in X: f(x)\neq 0 \right\},
  \]
  og köllum þetta mengi \emph{stoð}\index{stoð} $f$.
\end{skilgr}
\begin{ath}
  Almennar, í staðinn fyrir $\C$ getum við sett $\R$, eða bara eitthvert
  rúm sem hefur núllstak.
\end{ath}
\begin{skilgr}
  Látum $(U_\alpha)_{\alpha\in I}$ vera opna þakningu á grannrúmi $X$.
  \emph{Hlutun á einum (á $X$) m.t.t. þakningarinnar
  $(U_\alpha)_{\alpha\in I}$ } er fjölskylda af samfelldum föllum
  $\left( \phi_\alpha : X\to \left[ 0,1 \right] \right)_{\alpha\in I}$
  sem fullnægir eftirfarandi skilyrðum:
  \begin{enumerate}[(i)]
    \item $\supp(\phi_\alpha)\subseteq U_\alpha$ fyrir öll $\alpha\in
      I$.
    \item Mengjafjölskyldan $(\supp\phi_\alpha)_{\alpha\in I}$ er
      \emph{staðendanleg}\index{staðendanleiki} (þ.e. sérhver punktur
      $x$ úr $X$ á sér grennd sem sker aðeins endanlega mörg mengi úr
      fjölskyldunni). 
    \item $\sum_{\alpha\in I}^{}\phi_\alpha(x) = 1$ fyrir sérhvert $x$
      úr $X$. 
  \end{enumerate}
\end{skilgr}
\begin{ath}
  Skilyrði (iii) hefur merkingu vegna (ii).
\end{ath}
\begin{setn}
  Látum $U_1,\dots,U_n$ vera opna þakningu á normlegu rúmi $X$. Þá er
  til hlutun á einum m.t.t. þakningarinnar.
\end{setn}
\begin{proof}
  Sýnum fyrst með þrepun að $X$ eigi sér opna þakning $V_1,\dots,V_n$
  þ.a. $\overline V_i\subseteq U_i$ fyrir $i=1,\dots,n$.
  \begin{itemize}
    \item $A := X\setminus U_2\cup\cdots\cup U_n$ er lokað í $X$ og
      innihaldið í $U_1$ svo til er opin grennd $V_1$ um $A$ þ.a.
      $\overline V_1\subseteq U_1$ vegna þess að $X$ er normlegt. Ljóst
      er að $V_1,U_2,\dots,U_n$ er opin þakning á $X$.
    \item G.r.f. að við höfum opin mengi $V_1,\dots,V_{k-1}$ ($k\geq 2$)
      þ.a. $V_1,\dots,V_{k-1},U_k,\dots,U_n$ þeki $X$ og $\overline
      V_i\subseteq U_i$ fyrir $i=1,\dots,k-1$. Þá er $A := X\setminus
      V_1\cup\cdots\cup V_{k-1}\cup U_{k+1}\cup\cdots\cup U_n$ lokað í
      $X$ og innihaldið í $U_k$. Veljum þá opna grennd $V_k$ um $A$ þ.a.
      $\overline V_k\subseteq U_k$ og fáum opna þakningu
      $V_1,\dots,V_k,U_{k+1},\dots,U_n$ með $\overline V_i\subseteq U_i$
      fyrir $i=1,\dots,k$.
  \end{itemize}
  Sýnum nú fram á tilvist hlutunarinnar. Athugum að skilyrðið (ii) er
  sjálfkrafa uppfyllt. Veljum opna þakningu $W_1,\dots,W_n$ á $X$ þ.a.
  $\overline W_i\subseteq V_i$ fyrir $i=1,\dots,n$. Skv. hjálparsetningu
  Urysohns er, fyrir sérhvert $i$ úr $\left\{ 1,\dots,n \right\}$, til
  samfellt fall $\psi_i:X\to\left[ 0,1 \right]$ þ.a. $\psi_i|_{\overline
  W_i} = 1$ og $\psi_i|_{X\setminus V_i} = 0$. Þar sem $\psi_i$ er núll
  alls staðar utan lokaða mengisins $\overline V_i$, þá er
  $\supp\psi_i\subseteq\overline V_i\subseteq U_i$. Skilgreinum samfellt
  fall $\Psi:X\to\R$ með $\Psi(x):=\sum_{i=1}^{n}\Psi_i(x)$ fyrir öll
  $x\in X$. Þar eð $X=W_1\cup\cdots\cup W_n$, þá er $\Psi(x)>0$ fyrir
  öll $x\in X$. Föllin $\phi_1,\dots,\phi_n$ sem skilgreind eru með
  $\phi_i(x) := \psi_i(x)/\Psi(x)$ $\forall x\in X$ mynda því
  bersýnilega hlutun á einum með tilliti til $U_1,\dots,U_n$. 
\end{proof}
\begin{fylgisetn}
  Á þjappaðri víðáttu er til hlutun á einum m.t.t. hvaða endanlegrar opinnar
  þakningar sem er.
\end{fylgisetn}
\begin{proof}
  Þjöppuð víðátta er normleg.
\end{proof}
\begin{ath}
  Reyndar er hægt að sýna fram á eftirfarandi niðurstöðu: Á víðáttu sem
  er teljanleg í óendanlegu (þ.e. $\infty$ hefur teljanlegan grunn í
  Alexandroff-þjöppuninni) eru alltaf til hlutanir á einum.
\end{ath}
\begin{daemi}
  $X$ þjöppuð $m$-víð víðátta. Sýnum að unnt sé að greypa $X$ inn í
  $\R^N$ fyrir einhverja náttúrlega tölu $N$. Þar sem $X$ er þjöppuð, þá
  eru til opin mengi $U_1,\dots,U_n$ í $X$ þannig að $X =
  U_1\cup\cdots\cup U_n$ og fyrir sérhvert $i$ er til grannmótun
  $g_i:U_i\to\underbrace{g(U_i)}_{\text{opið}}\subseteq \R^m$. Veljum
  hlutun á einum $\phi_1,\dots,\phi_n$ m.t.t. $U_1,\dots,U_n$ og
  skilgreinum fyrir sérhvert $i$ úr $\left\{ 1,\dots,n \right\}$
  samfellda vörpun $h_i: X\to\R^m$ með \[
  h_i(x) := \begin{cases}
    \phi_i(x)\cdot g_i(x),&\text{ef }x\in U_i,\\
    0_{\R^m},             &\text{ef }x\in X\setminus\supp\phi_i.
  \end{cases}
  \]
  Sýnum að vörpunin \[
  F: X\to (\underbrace{\R\times\cdots\times\R}_{n\text{ sinnum}}) \times
  (\underbrace{\R^m\times\cdots\times\R^m}_{n\text{ sinnum}}),
  \quad
  F(x) := (\phi_1(x),\dots,\phi_n(x), h_1(x),\dots,h_n(x))
  \]
  sé greyping: Ljóst er að $F$ er samfelld, svo okkur nægir að sýna að
  $F$ sé eintæk vegna þess að $X$ er þjappað. G.r.f. að $F(x) = F(y)$.
  Þar sem $\sum_{i=1}^{n}\phi_i (x) = 1$, þá er til $i$ þ.a.
  $\phi_i(x)=\phi_i(y)>0$ svo að $x,y\in U_i$. Þá fæst að $h_i(x) =
  \phi_i(x)g_i(x) = \phi_i(y)g_i(y) = h_i(y)$ og því $g_i(x)=g_i(y)$. En
  $g_i$ er eintæk svo að $x=y$.
\end{daemi}

% 16.03.2010

\section{Tychonoff-setningin}

\begin{skilgr}
  Segjum að safn hlutmengja í titleknu mengi hafi \emph{ekki tómt
  endanlegt snið}\index{ekki tómt endanlegt snið}\index{e.e.s.}
  (skammstafað e.e.s.) ef sniðmengi sérhvers endanlegs hlutsafns er ekki
  tómt.
\end{skilgr}

\begin{ath}
  [Upprifjun] 
  Grannrúm $X$ er þjappað \emph{þ.þ.a.a.} um sérhvert safn
  $\mathcal C$ í $\mathcal P(X)$ sem hefur e.e.s., gildir að
  $\bigcap_{C\in\mathcal C}\overline C \neq \emptyset$.
\end{ath}

\begin{hjalparsetn}
  \label{hjalp:tych_1}
  Látum $X$ vera mengi og $\mathcal A\subseteq\mathcal P(X)$ þ.a.
  $\mathcal A$ hafi e.e.s.  Þá er til $\mathcal D\subseteq\mathcal P(X)$
  þ.a. $\mathcal A\subseteq \mathcal D$, $\mathcal D$ hefur e.e.s. og
  $\mathcal D$ er óstækkanlegt m.t.t. þessara eiginleika, þ.e.a.s. ef
  $\mathcal D'$ hefur e.e.s. og $\mathcal D\subseteq \mathcal D'$, þá er
  $\mathcal D =\mathcal D'$.
\end{hjalparsetn}
\begin{proof}
  Látum $\mathbb A$ vera mengi allra safna $\mathcal B$ í $\mathcal
  P(X)$ þannig að $\mathcal A\subseteq \mathcal B$ og $\mathcal B$ hafi
  e.e.s. og röðum stökunum í $\mathbb A$ með
  \emph{íveruröðuninni}\index{iverurodun@íveruröðun} $\subsetneq$.
  Við viljum sýna að $\mathbb A$ hafi óstækkanlegt stak. Skv.
  hjálparsetningu Zorn nægir að sýna að sérhvert línulega raðað
  hlutmengi í $\mathbb A$ sé takmarkað að ofan. Látum $\mathbb B$ vera
  línulega raðað hlutmengi í $\mathbb A$ og setjum $\mathcal
  C:=\bigcup_{\mathcal B\in\mathbb B} \mathcal B$ og sýnum að $\mathcal
  C\in\mathbb B$ (þá er $\mathcal C$ stærsta stakið í $\mathbb B$ og þar
  með $\mathbb B$ takmarkað að ofan).
  \begin{itemize}
    \item Augljóslega gildir $\mathcal A\subseteq \mathcal C$. 
    \item \emph{$\mathcal C$ hefur e.e.s.:} Látum
      $C_1,\dots,C_n\in\mathcal C$. Þá gildir um sérhvert $i$ að til er
      $\mathcal B_i$ úr $\mathbb B$ þannig að $C_i\in\mathcal B_i$. Nú
      er $\left\{ \mathcal B_1,\dots,\mathcal B_n \right\}$ endanlegt
      hlutmengi í línulega raðaða menginu $\mathbb B$ og hefur því
      stærsta stak, m.ö.o. er til $k$ úr $\left\{ 1,\dots,n \right\}$
      þ.a. $\mathbb B_i\subseteq \mathbb B_k$ fyrir $i=1,\dots,n$; þar
      með eru $C_1,\dots,C_n$ stök úr $\mathcal B_k$ og $\mathcal B_k$
      hefur e.e.s., svo að $C_1\cap\cdots\cap C_n\neq\emptyset$.
  \end{itemize}
\end{proof}
\begin{hjalparsetn}
  \label{hjalp:tych_2}
  Látum $X$ vera mengi og $\mathcal D$ vera safn hlutmengja í $X$ sem
  hefur e.e.s. og er óstækkanlegt m.t.t. þess eiginleika. Þá gildir
  \begin{enumerate}[(a)]
    \item Endanlegt sniðmengi af stökum úr $\mathcal D$ er stak í
      $\mathcal D$. 
    \item Ef $A\subseteq X$ og $A$ sker sérhvert stak úr $\mathcal D$,
      þá er $A$ í $\mathcal D$.  
  \end{enumerate}
\end{hjalparsetn}
\begin{proof}
  (a) Látum $\mathcal B$ vera sniðmengi endanlega margra staka úr
  $\mathcal D$ og setjum $\mathcal E := \mathcal D\cup\left\{ B
  \right\}$. Okkur nægir að sýna að $\mathcal E$ hafi e.e.s., því þá er
  $\mathcal D=\mathcal E$ og þar með $B\in\mathcal D$. Skrifum
  $B=\bigcap_{C\in\mathcal C'} C$ með $\mathcal C'$ sem endanlegt
  hlutmengi í $\mathcal D$. Látum nú $\mathcal C$ vera endanlegt
  hlutmengi í $\mathcal E$. Ef $B\notin \mathcal C$, þá er $\mathcal
  C\subseteq \mathcal D$ og því $\bigcap_{C\in\mathcal C}C\neq\emptyset$
  vegna þess að $D$ hefur e.e.s. Ef hins vegar $B\in\mathcal C$, þá
  setjum við $\mathcal C'':=\left( C\setminus\left\{ B \right\}
  \right)\cup \mathcal C'$, þá er $\mathcal C''$ endanlegt hlutmengi í
  $\mathcal D$ og þar með 
  \begin{align*}
    \emptyset
    &\neq \bigcap_{C\in\mathcal C''} C
    = \left( \bigcap_{C\in\mathcal C\setminus\left\{ B \right\}} C \right) 
    \cap \left(\bigcap_{C\in\mathcal C'} C\right)
    = \left( \bigcap_{C\in\mathcal C\setminus\left\{ B \right\}} C
    \right)\cap B 
    = \bigcap_{C\in\mathcal C} C.
  \end{align*}

  (b) Setjum $\mathcal E:=\mathcal D\cup\left\{ A \right\}$. Ef
  $D_1,\dots,D_n\in\mathcal D$, þá er $D_1\cap\cdots\cap D_n\in\mathcal
  D$ skv. (a) og því $D_1\cap\cdots\cap D_n\cap A\neq\emptyset$; þar
  með hefur $\mathcal E$ e.e.s., svo $\mathcal E = \mathcal D$ og því
  $A\in\mathcal D$.
\end{proof}
\begin{setn}
  [Tychonoff]\index{Tychonoff}\label{setn:tychonoff}
  Faldrúm þjappaðra grannrúma er þjappað.
\end{setn}
\begin{proof}
  Látum $X = \prod_{\alpha\in J} X_\alpha$ með $X_\alpha$ þjappað fyrir
  öll $\alpha\in J$. Látum $\mathcal A$ ver safn hlutmengja í $X$ þ.a
  $\mathcal A$ hafi e.e.s. Sýnum að $\bigcap_{A\in\mathcal A}\overline
  A\neq\emptyset$: Skv. hjálparsetningu \ref{hjalp:tych_1} er til
  óstækkanlegt $\mathcal D$ þannig að $\mathcal A\subseteq\mathcal D$.
  Látum $\alpha\in J$ og látum $\pi_\alpha:X\to X_\alpha$ vera
  $\alpha$-ta ofanvarpið. Safnið $\left\{ \pi_\alpha(D)\mid D\in\mathcal
  D\right\}$ í $\mathcal P(X_\alpha)$ hefur e.e.s. svo
  $\bigcap_{D\in\mathcal D}\overline{\pi_\alpha(D)}\neq\emptyset$ vegna
  þess að $X_\alpha$ er þjappað. Fyrir sérhvert $\alpha\in J$ veljum við
  $x_\alpha$ úr $\prod_{\alpha\in J}\overline{\pi_\alpha(D)}$ og setjum
  $\mathbf x = (x_\alpha)_{\alpha\in J} \in X$. Okkur nægir að sýna að
  $\mathbf x\in\overline D$ fyrir öll $D\in\mathcal D$. Látum nú
  $D\in\mathcal D$. Þá gildir um sérhvert $\alpha\in J$ og sérhverja
  grennd $U_\alpha$ um $x_\alpha$ að
  $U_\alpha\cap\pi_\alpha(D)\neq\emptyset$ vegna þess að
  $x_\alpha\in\overline{\pi_\alpha(D)}$. Þar með er\[
  \pi_\alpha\left( \pi_\alpha^{-1}(U_\alpha)\cap D \right)
  = U_\alpha\cap\pi_\alpha(D)
  \neq \emptyset
  \]
  og því $\pi_\alpha^{-1}(U_\alpha)\neq\emptyset$. Skv. hjálparsetningu
  \ref{hjalp:tych_2} (b) er því $\pi_\alpha^{-1}(U_\alpha)\in\mathcal D$
  fyrir sérhverja grennd $U_\alpha$ um $x_\alpha$ og skv.
  hjálparsetningu \ref{hjalp:tych_2} (a) sker sérhver grunngrennd um $x$
  öll $D$ úr $\mathcal D$ (því að slík grennd er endanlegt sniðmengi af
  gerðinni $\pi^{-1}(U_\alpha)$). Af því leiðir að $\mathbf
  x\in\overline D$ fyrir öll $D\in\mathcal D$. 
\end{proof}

\section{Grannmynstur á fallarúmum}

\begin{skilgr}
  Látum $X$ vera mengi og $Y$ grannrúm. Fyrir sérhvert $x\in X$ og
  sérhvert opið mengi $U$ í $Y$ setjum við\[
  S(x,U) := \left\{ f\in Y^X\mid f(x)\in U \right\}.
  \]
  Mengin $S(x,U)$ mynda hlutgrunn fyrir grannmynstur á $Y^X$ sem kallast
  \emph{grannmynstur vejulegrar samleitni}\index{grannmynstur!
  venjulegrar samleitni}\index{venjuleg samleitni}.
\end{skilgr}
\begin{ath}
  Umrætt grannmynstur er bara faldgrannmynstur á $Y^X$.
\end{ath}
\begin{setn}
  Látum $X$ vera mengi og $Y$ grannrúm. Fallaruna $(f_n:X\to
  Y)_{n\in\N}$ stefnir á fall $f:X\to Y$ í ofangreindu grannmynstri
  \emph{þ.þ.a.a.} $f_n(x)\to f(x)$ fyrir öll $x$ úr $X$.
\end{setn}
\begin{proof}
  Létt æfing.
\end{proof}
\begin{daemi}
  [Æfing]
  $X$ grannrúm og $(Y,d)$ firðrúm. Fyrir sérhvert $f$ úr $Y^X$,
  sérhvert þjappað $C$ í $X$og sérhvert $\varepsilon>0$ setjum við\[
  B_C(f,\varepsilon)
  := \left\{ g\in Y^X \mid \sup_{x\in C} d(f(x),g(x))<\varepsilon
  \right\}.
  \]
  Sýna á að mengin $B_C(f,\varepsilon)$ myndi grunn fyrir grannmynstur á
  $Y^X$.
\end{daemi}
\begin{skilgr}
  Þetta grannmynstur kallast \emph{grannmynstur þjappaðrar
  samleitni}\index{grannmynstur!þjappaðrar samleitni}\index{zjoppud
  samleitni@þjöppuð samleitni}.
\end{skilgr}
% 22.03.2010
\begin{setn}
  $X$ grannrúm og $(Y,d)$ firðrúm. Runa $(f_n)$ í $Y^X$ stefnir á $f$ í
  grannmynstri þjappaðrar samleitni \emph{þ.þ.a.a.} $(f_n|_C)$ stefni í
  jöfnum mæli á $f|_C$ fyrir sérhvert $C$ þjappað í $X$.
\end{setn}
\begin{proof}
  Augljóst.
\end{proof}
\begin{skilgr}
  Grannrúm $X$ er sagt \emph{þjapplega framleitt}\index{zjapplega
  framleitt@þjapplega framleitt} ef það fullnægir eftirfarandi
  skilyrði
  \begin{quote}
    Hlutmengi $A$ í $X$ er opið [lokað] \emph{þ.þ.a.a.} $A\cap C$ sé
    opið [lokað] í $C$ fyrir sérhvert þjappað $C$ í $X$.
  \end{quote}
\end{skilgr}
\begin{hjalparsetn}
  \label{hjalp:thjapplega_framleitt.1}
  Grannrúm $X$ er þjapplega framleitt ef það fullnægir öðru hvoru
  eftirfarandi skilyrða:
  \begin{enumerate}[(i)]
    \item $X$ er staðþjappað.
    \item Sérhver punktur í $X$ á sér teljanlegan grenndagrunn. 
  \end{enumerate}
\end{hjalparsetn}
\begin{proof}
  (i) Látum $A\subseteq X$ þ.a. $A\cap C$ sé opið fyrir sérhvert þjappað
  $C$ í $X$. Ef $x\in A$, þá er til opin grennd $U$ um $x$ í $X$ þ.a.
  $\overline U$ sé þjappað. Þá er $A\cap\overline U$ opið í $\overline
  U$ og þar með er $A\cap U$ opið í $U$ og því einnig í $X$. Punkturinn
  $x$ er því innri punktur í $A$.

  (ii) Látum $A\subseteq X$ þ.a. $A\cap C$ sé lokað í $C$ fyrir sérhvert
  þjappað $C$ í $X$. Látum $x\in\overline A$ og sýnum að $x\in A$. Til
  er runa $(x_n)_{n\in\N}$ í $A$ þannig að $x_n\longrightarrow x$ (í
  $X$). Þar með er $C:=\left\{ x \right\}\cup\left\{ x_n:n\in\N
  \right\}$ þjappað í $X$ svo að $A\cap C$ er lokað í $C$ og inniheldur
  þar með $x$. 
\end{proof}
\begin{hjalparsetn}
  \label{hjap:thjapplega_framleitt.2}
  Látum $X$ og $Y$ vera grannrúm og g.r.f. að $X$ sé þjapplega
  framleitt. Þá er vörpun $f:X\to Y$ samfelld \emph{ef og aðeins ef}
  $f|_C:C\to Y$ er samfelld fyrir sérhvert þjappað $C$ í $X$.
\end{hjalparsetn}
\begin{proof}
  Ef $f:X\to Y$ er samfelld, þá er $f|_C:C\to Y$ samfelld fyrir sérhvert
  hlutmengi $C$ í $X$. Öfugt, g.r.f. að $f:X\to Y$sé vörpun þ.a.
  $f|_C:C\to Y$ sé samfelld fyrir sérhvert þjappað $C$ í $X$: Látum $V$
  vera opið í $Y$. Fyrir sérhvert þjappað $C\subseteq X$ fæst þá að
  $f^{-1}(V)\cap C = (f|_C)^{-1}(V)$ er opið í $C$ og þar með er
  $f^{-1}(V)$ það líka, því $X$ er þjapplega framleitt.
\end{proof}
\begin{ath}
  [Ritháttur]
  Fyrir grannrúm $X$ og $Y$ táknum við með $\mathcal C(X,Y)$ mengi allra
  samfelldra varpana frá $X$ inn í $Y$. 
\end{ath}
\begin{setn}
  Látum $X$ vera þjapplega framleitt grannrúm og $(Y,d)$ vera firðrúm.
  Þá er $\mathcal C(X,Y)$ lokað í $Y^X$ í grannmynstri þjapplegrar
  samleitni.
\end{setn}
\begin{proof}
  Látum $f$ vera þéttipunkt $\mathcal C(X,Y)$ í $Y^X$ og sýnum að
  $f\in\mathcal C(X,Y)$. Til þess nægir að sýna að $f|_C:C\to Y$ sé
  samfelld fyrir sérhvert þjappað $C\subseteq X$. Látum $C$ vera þjappað
  í $X$. Veljum, fyrir sérhvert $n$ úr $\N^*$, $f_n$ úr
  $B_C(f,1/n)\cap\mathcal C(X,Y)$. Þá er auðséð að
  $f_n|_C\longrightarrow f|_C$ í jöfnum mæli og þar með er $f|_C$
  samfelld.
\end{proof}
\begin{fylgisetn}
  Látum $X$ vera þjapplega framleitt grannrúm og $(Y,d)$ vera firðrúm.
  Ef runa af samfelldum vörpunum $(f_n:X\to Y)_{n\in\N}$ stefnir á $f$ í
  grannmynstri þjappaðrar samleitni á $Y^X$, þá er $f$ samfelld.
\end{fylgisetn}
\begin{proof}
  Augljóst.
\end{proof}
\begin{daemi}
  [Æfing]
  Látum $X$ vera grannrúm og $(Y,d)$ vera firðrúm. Látum $\mathcal T_1$
  vera grannmynstur venjulegrar samleitni á $Y^X$, $\mathcal T_2$ vera
  grannmynstur þjappaðrar samleitni á $Y^X$ og $\mathcal T_3$ vera
  jafnmælisgrannmynstrið á $Y^X$. Þá gildir\[
  \mathcal T_1\subseteq \mathcal T_2\subseteq \mathcal T_3.
  \]
  Ef $X$ er strjált, þá er $T_1 = T_2$; ef $X$ er þjappað þá er
  $T_2=T_3$. 
\end{daemi}
Látum $X$ og $Y$ vera grannrúm. Fyrir sérhvert þjappað hlutmengi $C$ í
$X$ og sérhvert opið hlutmengi $U$ í $Y$ setjum við\[
S(C,U) := \left\{ f\in\mathcal C(X,Y) : f(C)\subseteq U \right\}.
\]
Ljóst er að mengin $S(C,U)$ þekja $\mathcal C(X,Y)$ og mynda því
hlutgrunn fyrir grannmynstur á $\mathcal C(X,Y)$.
\begin{skilgr}
  Umrætt grannmynstur kallast
  \emph{þjappað-opna}\index{zjappad-opna@þjappað-opna} (eða
  \emph{þjopna})\index{zjopna@þjopna} grannmynstrið á $\mathcal C(X,Y)$. 
\end{skilgr}
\begin{setn}
  Látum $X$ vera grannrúm og $(Y,d)$ vera firðrúm. Þá er þjopna
  grannmynstrið á $\mathcal C(X,Y)$ jafnt grannmynstri þjappaðrar
  samleitni.
\end{setn}
\begin{proof}
  Fyrir sérhvert $A$ í $Y$ og sérhvert $\varepsilon>0$ setjum við\[
  U(X,\varepsilon) := \left\{ y\in Y:d(y,A)<\varepsilon \right\}.
  \]
  Tökum eftir: Ef $A$ er þjappað og $V$ er opin grennd um $A$, þá er til
  $\varepsilon>0$ þ.a. $U(A,\varepsilon)\subseteq V$.
  \begin{itemize}
    \item Látum $S(C,U)$ vera eitt af hlutgrunnsstökum þjopna
      grannmynstursins og $f\in S(C,U)$. Þá er $f(C)$ þjappað hlutmengi
      í $U$ svo að til er $\varepsilon>0$ þ.a.
      $U(f(C),\varepsilon)\subseteq U$. Fyrir sérhvert $g$ úr
      $B_C(f,\varepsilon)$ gildir ljóslega að $g(x)\in
      U(f(C),\varepsilon)$ fyrir öll $x$ úr $C$ svo að $g(C)\subseteq U$
      og þar með $B_C(f,\varepsilon)\subseteq S(C,U)$. Af þessu sést að
      grannmynstur þjappaðrar samleitni er fínna en þjopna
      grannmynstrið.
    \item Öfugt, látum $f\in\mathcal C(X,Y)$, $C$ vera þjappað í $X$ og
      $\varepsilon>0$. Sérhver punktur $x$ úr $X$ á sér opna grennd
      $V_x$ þ.a. $f(\overline V_x)\subseteq B(f(x),\varepsilon/3)$. Þar
      sem $C$ er þjappað, þa'eru til $x_1,\dots,x_n$ úr $C$ þ.a.
      $C\subseteq V_{x_1}\cup\cdots\cup V_{x_n}$. Setjum $C_j
      :=\overline V_{x_j}\cap C$ fyrir $j=1,\dots,n$. Þá eru $C_j$-in
      þjöppuð og fljótséð er að $f\in S(C,B(f(x),\varepsilon/3))\cap
      \cdots S(C_n,B(f(x_n),\varepsilon/3))\subseteq
      B_C(f,\varepsilon)$. Þar með er sýnt að þjopna grannmynstrið er
      fínna en grannmynstur þjappaðrar samleitni.
  \end{itemize}
\end{proof}
\begin{ath}
  $X$ grannrúm og $Y$ firðanlegt rúm. Þá segir setningin okkur að
  grannmynstur þjappaðrar samleitni á $C(X,Y)$ er óháð valinu á firð sem
  skilgreinir grannmynstur á $Y$. Sér í lagi gildir þetta þegar $X$ er
  þjappað; þ.e.a.s. jafnmælisgrannmynstrið er óháð valinu á firð sem
  skilgreinir grannmynstrið á $Y$.
\end{ath}
% 23.03.2010 && 24.03.2010
\begin{setn}
  Látum $X$ vera staðþjappað Hausdorff-rúm og $Y$ vera grannrúm. Þá er
  vörpunin\[
  e : X\times\mathcal C(X,Y)\to Y, e(x,f) := f(x)
  \]
  samfelld þegar $\mathcal C(X,Y)$ er með þjopna grannmynstrið.
\end{setn}
\begin{skilgr}
  Vörpunin $e$ kallst
  \emph{gildistökuvörpunin}\index{gildistökuvörpunin}
\end{skilgr}
\begin{proof}
  [Sönnun á síðustu setningu]
  Látum $(x,f)\in X\times\mathcal C(X,Y)$ og $V$ vera opna grennd um
  $f(x)=e(x,f)$ í $Y$. Þar sem $f$ er samfelld og $X$ er staðþjappað og
  Hausdorff, þá er til opin grennd $U$ um $x$ í $X$ þ.a. $\overline U$
  sé þjappað og $f(\overline U)\subseteq V$: Þá er $U\times S(\overline
  U,V)$ opin grennd um $(x,f)$ í $X\times\mathcal C(X,Y)$ og fyrir
  sérhvert $(x',f')$ úr $U\times S(\overline U,V)$ gildir augljóslega að
  $e(x',f')=f'(x')\in V$. Þar með er sýnt að $e$ er samfelld.
\end{proof}
Sérhver vörpun $f: X\times Z\to Y$ gefur af sér vörpun $F:Z\to Y^X$ sem
er skilgreind með $(F(z))(x):=f(x,z)$, m.ö.o. er $F(z) = f(\cdot,z)$.
Öfugt, sérhver vörpun $F:Z\to Y^X$ gefur af sér vörpun $f:X\times Z\to
Y$, sem skilgreind er með $f(x,z):=(F(z))(x)$. Þessar varpanir eru
sagðar tilheyra hvor annarri. 
\begin{ath}
  Ef $f:X\times Z\to Y$ er samfelld, þá tekur $F$ gildi sín í $\mathcal
  C(X,Y)$ og þá lítum við ávallt á $F$ sem vörpun $Z\to \mathcal
  C(X,Y)$.
\end{ath}
\begin{setn}
  Látum $X,Y$ og $Z$ vera grannrúm og setjum þjopna grannmynstrið á
  $\mathcal C(X,Y)$.
  \begin{enumerate}[(i)]
    \item Ef $f:X\times Z\to Y$ er samfelld, þá er tilheyrandi vörpun
      $F:Z\to \mathcal C(X,Y)$ líka samfelld.
    \item G.r.f. að $X$ sé staðþjappað og Hausdorff. Ef $F:Z\to\mathcal
      C(X,Y)$ er samfelld, þá er tilheyrandi vörpun $f:X\times Z\to Y$
      líka samfelld.
  \end{enumerate}
\end{setn}
\begin{proof}
  (i) G.r.f. að $f:X\times Z\to Y$ sé samfelld og látum $z_0$ vera
  punkt úr $Z$. Til að sýna að $F$ sé samfelld í $z_0$ nægir að sýna að
  fyrir sérhvert opið hlutmengi af gerðinni $S(C,U)$ í $\mathcal C(X,Y)$
  sem inniheldur $F(z_0)$ sé til opin grennd $W$ um $z_0$ í $Z$ þ.a.
  $F(W)\subseteq S(C,U)$. Nú er $F(z_0)\in S(C,U)$, svo
  $(F(z_0))(C)\subseteq U$, en það jafngildir því að $f(x,z_0)\in U$
  fyrir öll $x\in C$, svo $f(C\times\left\{ z_0 \right\})\subseteq U$.
  Þar eð $f$ er samfelld, þá er $f^{-1}(U)$ opin grennd um $z_0$ af
  gerðinni tiltekið $C\times\left\{ z_0 \right\}$, í $C\times Z$. Skv.
  hólkasetningunni er þá til opin grennd $W$ um $z_0$ í $Z$ þ.a.
  $C\times W\subseteq f^{-1}(U)$ því $C$ er þjappað. Það þýðir að
  $(F(z))(x) = f(x,z)\in U$ fyrir öll $x\in C$ og öll $z\in W$, svo
  $F(W)\subseteq S(C,U)$.

  (ii) Athugum að $f = \id_X\times F\circ e$
  svo $f$ er samskeyting tveggja samfelldra varpana og því samfelld.
\end{proof}

\part{Algebruleg grannfræði}
\chapter{Ýmis verkefni í grannfræði}
Meðal verkefna sem við höfum áhuga á eru
\begin{enumerate}
  \item Er til samfelld vörpun $f:B^n\to B^n$ sem hefur engan kyrrapunkt
    ($n=0$ er fáfengilegt, $n=1$ er milligildissetningin)?
  \item Er til samfelld vörpun $f:B^n \to S^{n-1}$ þ.a. $f(x) = x$ fyrir
    öll $x$ úr $S^{n-1}$? Slík vörpun kallast
    \emph{inndráttur}\index{inndráttur} (tilfellið $n=1$ er einfalt,
    þar er þetta ekki hægt því ekki til samfelld vörpun frá
    samanhangandi bilinu $\left[ -1,1 \right]$ yfir í $\left\{ -1,1
    \right\}$).
  \item Er til samfellt vigursvið á $S^n$ sem verður hvergi núll?
    M.ö.o., er til samfelld vörpun $f:S^n \to \R^{n+1}$ þ.a. $f(x)\cdot
    x = 0$ (innfeldi) fyrir öll $x\in S^n$? (Ef $n=1$ þá greinilega
    hægt). 

    Ef $S$ er hlutvíðátta í $\R^n$ og fyrir öll $x\in S$ látum við $T_x
    S$ vera snertisléttu $S$ í $x$, þá er
    \emph{vigursvið}\index{vigursvið} samfelld vörpun $F:S\to\R^n$ þ.a.
    $F(x)\in T_x S$.
  \item Er fyrir öll $\varepsilon>0$ til samfelld vörpun
    $\gamma:S^n\to S^n$ sem hefur engan kyrrapunkt og uppfyllir
    $\|\gamma(x)-x\|<\varepsilon$ fyrir öll $x\in S^n$?

    Ef $\varepsilon>2$, þá er $\gamma = -\id$ slík vörpun. Ef
    $\varepsilon$ er nógu lítið, $n=0$, þá er þetta ekki hægt, en ef
    $n=1$ þá er þetta auðveldlega hægt með því að snúa kúluhvelinu.
  \item Getur $\R^n$ verið grannmóta $\R^m$ ef $n\neq m$? ($n=1$ og
    $m>1$ er einfalt tilfelli, sést með því að taka einn punkt úr hvoru
    menginu).
  \item Er $S^2$ grannmóta hringfeldinu $S^1\times S^1$?
  \item Er hringfeldið $S^1\times S^1$ (kleinuhringur) grannmóta
    kleinuhring með tveimur götum? 
\end{enumerate}
\begin{ath}
  [Ritháttur]
  Fyrir $x:=(x_1,\dots,x_n)\in\R^n$ setjum við
  $\|x\|:=\sqrt{x_1^2+\cdots+x_n^2}$. Látum hér 
  \begin{align*}
    B^n &:= \left\{ x\in\R^n : \|x\|\leq 1 \right\}\\
    S^n &:= \left\{ x\in\R^n : \|x\|   = 1 \right\}.
  \end{align*}
\end{ath}

\begin{skilgr}
  Látum $X$ vera grannrúm og $A\subseteq X$. Samfelld vörpun $r: X\to A$
  þ.a. $r(a) = a$ fyrir öll $a\in A$ kallast
  \emph{inndráttur}\index{inndráttur}; ef slík vörpun er til þá er sagt
  að $A$ sé \emph{inndrægi (af $X$)}\index{inndrægi}.
\end{skilgr}

\section{Samtoganir}
\begin{skilgr}
  Samfelldar varpanir $f,g:X\to Y$ eru sagðar
  \emph{samtoga}\index{samtoga}, táknað
  $f\simeq g$, ef til er samfelld vörpun $F:X\times I\to Y$ ($I = \left[
  0,1 \right]$) þannig að $F(x,0) = f(x)$ og $F(x,1) = g(x)$ fyrir öll
  $x\in X$. Vörpunin $F$ kallast \emph{samtogun frá $f$ til
  $g$}\index{samtogun}.

  Almennar, látum $A\subseteq X$. Við segjum að $F$ sé \emph{samtogun
  frá $f$ til $g$ með tilliti til $A$} (og þá að \emph{$f$ sé samtoga
  $g$ m.t.t. $A$}) ef $F:X\times I\to Y$ er samtogun frá $f$ til $g$ og
  $F(a,t)=f(a)$ fyrir öll $t\in I$ ef $a\in A$. Táknum þetta með
  $f\simeq g \prel A$.
\end{skilgr}
% 29.03.2010
\begin{setn}
  $f\simeq g \prel A$ eru jafngildisvensl á $Y^X$.
\end{setn}
\begin{proof}
  \begin{description}
    \item[Sjálfhverfa] $f\simeq f \prel A$ vegna þess að
      $F:X\times[0,1]\to Y, F(x,t) = f(x)$ er samtogun frá $f$ til $f$
      með tilliti til $A$.
    \item[Samhverfa] Ef $H:X\times[0,1]\to Y$ er samtogun frá $f$ til
      $g$ m.t.t. $A$, þá er $G:X\times[0,1]\to Y, G(x,t):= H(x,1-t)$
      samtogun frá $g$ til $f$ m.t.t. $A$. 
    \item[Gegnvirkni] Ef $F:f\simeq g \prel A$ og $G:g\simeq
      h\prel A$, þá fæst $H:f\simeq h\prel A$ með því
      að setja\[
      H(x,t) := 
      \begin{cases}
        F(x,2t),   & 0 \leq t \leq 1/2,\\
        G(x,2t-1), & 1/2  \leq t\leq 1.
      \end{cases}
      \]
  \end{description}
\end{proof}
\begin{ath}
  $g(x) = F(x,1) = G(x,0)$.
\end{ath}
\begin{daemi}
  Til er inndráttur $f:B^n \to S^{n-1}$ \emph{þ.þ.a.a.}
  $\id:S^{n-1}\to S^{n-1}$ sé samtoga fastri vörpun.
\end{daemi}
\begin{proof}
  Ef $f:B^{n}\to S^{n-1}$ er inndráttur, þá setjum við\[
  H: S^{n-1}\times[0,1]\to S^{n-1}; (x,t)\mapsto f(tx).
  \]
  Þá er $H(x,1) = f(x) = x$ fyrir öll $x\in S^{n-1}$ og $H(x,0) = f(0)$,
  þ.e.a.s. $H:f(0)\simeq \id_{S^{n-1}}$.

  Öfugt, ef $H:S^{n-1}\times[0,1]\to S^{n-1}$ er samfelld með $H(x,0)=
  c$ og $H(x,1) = x$ fyrir öll $x\in S^{n-1}$ þá skilgreinum við
  inndrátt $f:B^n\to S^{n-1}$ með\[
  f(x) := 
  \begin{cases}
    H(x/\|x\|, \|x\|), & x \neq 0,\\
    c,                 & x = 0.
  \end{cases}
  \]
  Fljótséð er að $f$ er samfelld vörpun (æfing).
\end{proof}
\begin{daemi}
  Köllum $a_n : S^n\to S^n, x\mapsto -x$
  \emph{andfætisvörpun}\index{andfætisvörpun}. Ef til er samfellt
  vigursvið á $S^n$ sem er hvergi núll, þá er $a_n\simeq \id_{S^n}$.
\end{daemi}
\begin{proof}
  G.r.f. að $f:S^n \to \R^{n+1}$ sé samfelld og uppfylli $f(x)\neq 0$ og
  $f(x)\cdot x = 0$ fyrir öll $x\in S^{n-1}$. Setjum $H(x,t)=a(t)\cdot x
  + b(t,x)\cdot f(x)$. Til þess að $H$ sé vörpun frá $S^n\times[0,1]$ yfir
  í $S^n$, þá þarf að gilda $\|H(x,t)\|=1$, þ.e.a.s.
  \begin{align*}
    1
    &= \| a(t)x + b(t,x)f(x) \|^2 \\
    &= a(t)^2 \underbrace{\|x\|}_{1} + b(t,x)^2 \|f(x)\|^2  \\
    &= a(t)^2 + b(t,x)^2\underbrace{\|f(x)\|^2}_{0}.
  \end{align*}
  Tökum $a(t)=1-2t$ og þá $b(t,x) := \frac{2\sqrt{t-t^2}}{\|f(x)\|}$.
\end{proof}
\begin{setn}\label{setn:samtog_ohad}
  Látum $f_1,f_2:X\to Y$ og $g_1,g_2: Y\to Z$ vera samfelldar og
  $A\subseteq X$. Ef $f_1\simeq f_2\prel A$ og $g_1\simeq
  g_2\prel{f_1(A)}$, þá er $g_1\circ f_1\simeq g_2\circ f_2\prel A$.
\end{setn}
\begin{proof}
  $F:f_1\simeq f_2\prel A$, $G:g_1\simeq g_2\prel{f(A)}$. Skilgreinum
  $H:X\times[0,1]\to Z$ með $H(x,t) := G(F(x,t),t)$. Þá er $H:g_1\circ
  f_1\simeq g_2\circ f_2 \prel A$.
\end{proof}

\section{Ríkjafræði}

\begin{skilgr}
  \emph{Ríki $\mathcal C$}\index{ríki} samanstendur af 
  \begin{enumerate}[(i)]
    \item Safni, $\Ob \mathcal C$, af \emph{hlutum}\index{hlutir (í ríki)}.
    \item Fyrir sérhverja raðspyrðu $(X,Y)$ af hlutum úr $\Ob \mathcal
      C$ er gefið mengi $\Hom(X,Y)$ (eða $\Hom_{\mathcal C}(X,Y)$).
      Stökin í $\Hom(X,Y)$ nefnast \emph{mótanir frá $X$ til $Y$}.
    \item Fyrir sérhverja raðaða þrennd $(X,Y,Z)$ af hlutum úr $\Ob
      \mathcal C$ er gefin vörpun $\Hom(X,Y)\times\Hom(Y,Z)\to\Hom(X,Z),
      (f,g)\mapsto g\circ f$ sem við köllum
      \emph{samskeytingu}\index{samskeyting (ríkjafræði)}.
  \end{enumerate}
  Skrifum oft $f:X\to Y$ eða $X\xrightarrow{f} Y$ í stað
  $f\in\Hom(X,Y)$. Jafnframt er þess krafist að eftirfarandi gildi:
  \begin{description}
    \item[Tengiregla] Ef $f:X\to Y$, $g:Y\to Z$ og $h:Z\to W$, þá er
      $h\circ (g\circ f) = (h\circ g)\circ f$.
    \item[Tilvist hlutleysu] Fyrir sérhvert $Y$ úr $\Ob\mathcal C$ er
      til mótun $1_Y$ úr $\Hom(Y,Y)$ þ.a. fyrir öll $f$ úr $\Hom(X,Y)$
      og öll $g$ úr $\Hom(Y,Z)$ gildi að $1_Y\circ f = f$ og $g\circ 1_Y
      = g$.
  \end{description}
\end{skilgr}
\begin{ath}
  Stundum er skrifað $\id_Y$ í stað $1_Y$. Tökum eftir að $1_Y$ er
  ótvírætt ákvörðuð.
\end{ath}
\begin{skilgr}
  Við segjum að $f$ úr $\Hom(X,Y)$ sé \emph{einsmótun}\index{einsmótun
  (ríkjafræði)} ef til er $g$ úr $\Hom(Y,X)$ þ.a. $g\circ f = 1_X$ og
  $f\circ g = 1_Y$.
\end{skilgr}
\begin{daemi}
  \begin{enumerate}[(1)]
    \item Mengjaríkið $\mathbf{Men}$: $\Ob\mathbf{Men}$ er safn
      sallra mengja, \[\Hom_{\mathbf{Men}}(X,Y) = Y^X.\]
    \item Grúpuríkið $\mathbf{Grp}$: $\Ob\mathbf{Grp}$ er safn allra
      grúpa, $\Hom_{\mathbf{Grp}}$ er mengi allra grúpumótana $X\to Y$.
    \item Mótlaríkið $\mathbf{Mot}_R$, þar sem $R$ er baugur:
      $\Ob\mathbf{Mot}_R$ er safn allra (vinstri) $R$-mótla,
      $\Hom_{\mathbf{Mot}_R}(X,Y)$ er mengi allra $R$-línulegra varpana
      ($R$-mótla mótlana) $X\to Y$. 
    \item Grannrúmaríkið $\mathbf{Top}$: $\Ob\mathbf{Top}$ er safn allra
      grannrúma, $\Hom_{\mathbf{Top}}(X,Y)=\mathcal C(X,Y)$. 
    \item $X$ og $Y$ grannrúm og $f:X\to Y$ samfelld. Látum $[f]$ vera
      \emph{samtogunarflokk}\index{samtogunarflokkur} $f$, þ.e.a.s.
      jafngildisflokk $f$ m.t.t. venslanna $\simeq$. Þá getum við
      skilgreint $[g]\circ[f] := [g\circ f]$ (óháð valinu á fulltrúum
      skv. setningu \ref{setn:samtog_ohad}); fáum
      \emph{samtogunarríkið}\index{samtogunarríkið} $\mathbf{Hot}$:
      $\Ob\mathbf{Hot}$ er safn allra grannrúma,
      $\Hom_{\mathbf{Hot}}(X,Y)$ er mengi allra samtogunarflokka
      samfelldra varpana $X\to Y$. 
  \end{enumerate}
\end{daemi}
\begin{skilgr}
  Látum $\mathcal C$ og $\mathcal C'$ vera tvö ríki. 
  \begin{enumerate}
    \item \emph{Varpi}\index{varpi} (functor) $F:\mathcal C\to\mathcal
      C'$ er vörpun sem úthlutar hverjum hlut $X$ úr $\Ob\mathcal C$
      ákveðnum hlut $F(X)$ í $\Ob\mathcal C'$ og sérhverri mótun $f$ úr
      $\Hom_{\mathcal C}(X,Y)$ ákveðinni mótun $F(f)$ úr
      $\Hom_{\mathcal C}(F(X),F(Y))$ þannig að eftirfarandi skilyrði séu
      uppfyllt:
      \begin{enumerate}[(i)]
        \item Ef $f:X\to Y$ og $g:Y\to Z$ eru mótanir í $\mathcal C$, þá
          er $F(g\circ f) = F(g)\circ F(f)$.
        \item Fyrir öll $X$ úr $\Ob\mathcal C$ er $F(1_X)=1_{F(X)}$. 
      \end{enumerate}
    \item \emph{Hjávarpi}\index{hjávarpi} (cofunctor) $F:\mathcal C\to
      \mathcal C'$ er vörpun sem úthlutar hverjum hlut $X$ úr
      $\Ob\mathcal C$ ákveðnum hlut $F(X)$ úr $\Ob\mathcal C'$ og hverri
      mótun $f$ úr $\Hom_{\mathcal C}(X,Y)$ ákveðinni mótun $F(f)$ úr
      $\Hom_{\mathcal C'}(F(Y),F(X))$ þannig að eftirfarandi gildi:
      \begin{enumerate}[(i)]
        \item Ef $f:X\to Y$ og $g:Y\to Z$ eru mótanir í $\mathcal C$, þá
          er $F(g\circ f) = F(f)\circ F(g)$. 
        \item Fyrir öll $X$ úr $\Ob\mathcal C$ er $F(1_X) = 1_{F(X)}$. 
      \end{enumerate}
  \end{enumerate}
\end{skilgr}
\begin{figure}[h]
  \begin{align*}
    \xymatrix{
    X\ar[r]^f & 
    Y \ar[r]^g &
    Z \ar@{~>}[r] &
    F(X) \ar[r]^{F(f)} &
    F(Y) \ar[r]^{F(g)} &
    F(Z)
    }\tag{varpi}
    \\
    \xymatrix{
    X\ar[r]^f & 
    Y \ar[r]^g &
    Z \ar@{~>}[r] &
    F(X) &
    F(Y) \ar[l]^{F(f)} &
    F(Z) \ar[l]^{F(g)}
    }\tag{hjávarpi}
  \end{align*}
  \caption{Varpi og hjávarpi}
  \label{fig:varpi_hjavarpi}
\end{figure}
\begin{daemi}
  \begin{enumerate}[(1)]
    \item  $R$-víxlbaugur. Fyrir $R$-mótul $M$ setjum við $M^* :=
      \Hom_R(M,R)$ (svokallaður \emph{nykurmótull}\index{nykurmótull}
      $M$). Setjum $D:\mathbf{Mot}_R\to\mathbf{Mot}_R, D(M) = M^*$ og
      fyrir $f:M\to N, D(f): N^*\to M^*, D(f) = f^*$. Þá er $D$
      hjávarpi.  
      \[ \xymatrix{ 
      M \ar[r]^f\ar[dr]_{l\circ f = f^*(l) = D(f)(l)}
      & N \ar[d]^{l\in N^*} \\ 
      & R
      }
      \] 
      þar sem $n \cdot x =
      (\underbrace{1_R+\cdots+1_R}_{n\text{liðir}})\cdot x$.
    \item Fyrir grannrúm $X$ setjum við $F(X) = X$ og fyrir $f:X\to Y$
      setjum við $F(f) = [f]$ sem samtogunarflokk $f$. Þá er
      $F:\mathbf{Top}\to\mathbf{Hot}$ varpi.
    \item \emph{Gleymskuvarpar}\index{gleymskuvarpar}:
      \begin{enumerate}[(i)]
        \item $\mathbf{Mot}_R\to \mathbf{Mot}_\Z$.
        \item $\mathbf{Mot}_\Z\to \mathbf{Grp}$.
        \item $\mathbf{Grp}\to \mathbf{Men}$.
      \end{enumerate}
  \end{enumerate}
\end{daemi}
% 30.03.2010 
\section{Vegsamtoganir}
Skrifum $I:=[0,1]$. \emph{Vegur}\index{vegur} í grannrúmi $X$ er
samfelld vörpun $I\to X$.
\begin{skilgr}
  $X$ grannrúm og $\alpha,\beta:I\to X$ vegir. Segjum að $\alpha$ og
  $\beta$ séu \emph{vegsamtoga}\index{vegsamtoga} ef
  $\alpha\simeq\beta\prel{\{0,1\}}$ og skrifum þá $\alpha\simeq_p\beta$
  ($p$ stendur fyrir \emph{path}).
  
  Látum $\alpha,\beta:I\to X$ vera vegi þannig að $\alpha(1)=\beta(0)$
  og skilgreinum $\alpha*\beta:I\to X$ með 
  \[
  (\alpha*\beta)(t) :=
  \begin{cases}
    \alpha(2t)  & 0\leq t \leq 1/2\\
    \beta(2t-1) & 1/2\leq t\leq 1
  \end{cases}
  \]
  og köllum þennan nýja veg frá $\alpha(0)$ til $\beta(1)$
  \emph{samsetningu}\index{samsetning vega} veganna $\alpha$ og $\beta$.
\end{skilgr}
\begin{setn}
  Ef $\alpha_1\simeq_p\alpha_2$ og $\beta_1\simeq_p\beta_2$ og
  $\alpha_1(1)=\beta_1(0)$ (og þar með einnig $\alpha_2(1)=\beta_2(0)$),
  þá er $\alpha_1*\beta_1\simeq_p\alpha_2*\beta_2$.
\end{setn}
\begin{proof}
  Fyrir $H_1:\alpha_1\simeq_p\alpha_2$ og $H_2:\beta_1\simeq_p\beta_2$
  setjum við\[
  H(t,s) := 
  \begin{cases}
    H_1(2t,s)   & 0\leq t\leq 1/2\\
    H_2(2t-1,s) & 1/2\leq t\leq 1.
  \end{cases}
  \]
  Þá er $H:\alpha_1*\beta_1\simeq_p\alpha_2*\beta_2$.
\end{proof}
\begin{ath}
  [ritháttur]
  Látum $[\alpha]$ tákna jafngildisflokk $\alpha$ með tilliti til
  venslanna $\simeq_p$.
\end{ath}
\begin{skilgr}
  [byggir á síðustu setningu]
  Ef $\alpha(1)=\beta(0)$ þá setjum við
  $[\alpha]*[\beta]:=[\alpha*\beta]$ (vorum að sanna að þetta er óháð
  vali á fulltrúum flokkanna).
\end{skilgr}
\begin{setn}
  Fyrir sérhvert grannrúm $X$ getum við skilgreint ríki $\Pi(X)$ þannig
  að hlutirnir í $\Pi(X)$ séu punktarnir í $X$ og mótanir frá $x$ til
  $y$ séu jafngildisflokkarnir $[\alpha]$ þar sem $\alpha$ er vegur frá
  $x$ til $y$. Allar mótanir þessa ríkis eru einsmótanir.
\end{setn}
\begin{ath}
  [ritháttur]
  Skrifum $\Pi(x,y)$ í stað $\Hom_{\Pi(X)}(x,y)$ fyrir mengi allra
  jafngildisflokka vega milli $x$ og $y$.
\end{ath}
\begin{ath}
  Um $\Pi(X)$ gildir að $\Ob\Pi(X)=X$ er mengi og allar mótanirnar eru
  einsmótanir. Slík ríki kallast \emph{grýpi}\index{grýpi} (e.
  \emph{groupoid}) og sér í lagi köllum við $\Pi(X)$
  \emph{undirstöðugrýpi}\index{undirstöðugrýpi} $X$. 
\end{ath}
\begin{proof}
  [Sönnun á síðustu setningu]
  Sýnum fyrst að $\Pi(X)$ sé ríki.
  \begin{description}
    \item[Tengiregla] Látum $\alpha,\beta$ og $\gamma$ vera vegi í $X$
      þannig að $\alpha(1)=\beta(0)$ og $\beta(1)=\gamma(0)$. Við viljum
      sýna að $(\alpha*\beta)\gamma\simeq_p\alpha*(\beta*\gamma)$. Höfum
      nú að \[
      (\alpha*\beta)*\gamma =
      \begin{cases}
        \alpha(4t),   & 0\leq   t\leq 1/4,\\
        \beta(4t-1),  & 1/4\leq t\leq 1/2,\\
        \gamma(2t-1), & 1/2\leq t\leq 1
      \end{cases}
      \]
      og \[
      \alpha*(\beta*\gamma) = 
      \begin{cases}
        \alpha(2t),   & 0\leq t\leq 1/2,\\
        \beta(4t-2),  & 1/2\leq t\leq 3/4,\\
        \gamma(4t-3), & 3/4\leq t\leq 1.
      \end{cases}
      \]
      (Æfing) Finnum
      $H:(\alpha*\beta)*\gamma\simeq_p\alpha*(\beta*\gamma)$. Nú gildir
      að ef $\alpha:I\to X$ er vegur í $X$, $h:I\to I$ er samfelld þ.a.
      $h(0)=0$ og $h(1)=1$, $\beta=\alpha\circ h$, þá er
      $\alpha\simeq_p\beta$ því vörpunin 
      \[
      H:I\times I\to X, 
      H(t,s):= \alpha(sh(t)+(1-s)t)
      \]
      er vegsamtogun frá $\alpha$ til $\beta$.
    \item[Tilvist hlutleysa] Fyrir sérhvert $x\in X$ setjum við
      $e_x:I\to X,t\mapsto x$ (fastavegur). Látum $\alpha:I\to X$ vera
      veg þannig að $\alpha(0)=x$ og $\alpha(1)=:y$. Þá er
      $e_x*\alpha\simeq_p\alpha$ og $\alpha\simeq_p e_y\simeq_p \alpha$
      skv. fullyrðingunni hér að ofan. Þar með er $[e_x]$ hlutleysan í
      $\Pi(x,x)$ og sýnt er að $\Pi(X)$ er ríki. 
  \end{description}
  Sýnum nú að allar mótanirnar séu einsmótanir: Ef $\alpha$ er vegur frá
  $x$ til $y$ í $X$, þá er $\overline\alpha:I\to X, t\mapsto\alpha(1-t)$
  vegur frá $y$ til $x$ og $H:I\times I\to X$,
  \[
  H(t,s) := 
  \begin{cases}
    \alpha(2st),      & 0\leq t \leq 1/2, \\
    \alpha(2s(1-t)),  & 1/2\leq t\leq 1
  \end{cases}
  \]
  er vegsamtogun frá $e_x$ til $\alpha*\overline\alpha$. En það þýðir að
  $[\alpha]*[\overline\alpha]=[e_x]$ og þar með er $[\alpha]$ einsmótun.
\end{proof}
\begin{ath}
  Fyrir sérhvert $x\in X$ er $\Pi(x,x)$ grúpa m.t.t. aðgerðarinnar $*$.
\end{ath}
\begin{skilgr}
  Látum $X$ vera grannrúm og $x_0\in X$. Setjum
  $\pi_1(X,x_0):=\Pi(x_0,x_0)$ og köllum
  \emph{undirstöðugrúpu}\index{undirstöðugrúpa} (eða
  \emph{Poincaré-grúpu}\index{Poincaré-grúpa} $X$ í punkti $x_0$ (eða
  með \emph{grunnpunkt $x_0$}).
\end{skilgr}
\begin{ath}
  $\pi_1(X,x_0)$ er grúpa jafngildisflokka vega sem byrja og enda í
  $x_0$.
\end{ath}
\begin{setn}
  Látum $\alpha$ vera veg frá $x_0$ til $x_1$ í grannrúmi $X$. Vörpunin
  \[
  \hat\alpha:\pi_1(X,x_0)\to\pi_1(X,x_1),\;
  \hat\alpha([\beta]) := [\overline\alpha*\beta*\alpha] =
  [\overline\alpha]*[\beta]*[\alpha]
  \]
  er grúpueinsmótun. Ef $\alpha_1$ er annar vegur frá $x_0$ til $x_1$,
  þá eru einsmótanirnar $\hat\alpha$ og $\hat\alpha_1$ samoka.
\end{setn}
\begin{proof}
  Höfum nú að 
  \begin{align*}
    \hat\alpha([\beta]*[\beta'])
    &= [\overline\alpha]*[\beta]*[\beta']*[\alpha]\\
    &= [\overline\alpha]*[\alpha]*[\overline\alpha]*[\beta']*[\alpha]\\
    &= \hat\alpha([\beta])*\hat\alpha([\beta']),
  \end{align*}
  svo $\hat\alpha$ er grúpumótun. En ljóst er að $\hat{\overline\alpha}$
  er andhverfa $\hat\alpha$ svo að $\hat\alpha$ er einsmótun. Fáum svo
  að 
  \begin{align*}
    \hat\alpha_1([\beta)]
    &= [\overline\alpha_1]*[\beta]*[\alpha_1]\\
    &= [\overline\alpha_1]*[\alpha]*[\overline\alpha]*[\beta]*[\alpha]*[\overline\alpha]*[\alpha_1] \\
    &= [\overline\alpha_1*\alpha]*[\overline\alpha]*[\beta]*[\alpha]*[\overline\alpha*\alpha_1] \\
    &= \underbrace{[\overline\alpha_1*\alpha]}_{\in\pi_1(X,x_1)}*\hat\alpha([\beta])*[\overline\alpha*\alpha]^{-1}.
  \end{align*}
\end{proof}
\begin{fylgisetn}
  Ef $x_0$ og $x_1$ eru í sama vegsamhengisþætti, þá eru $\pi_1(X,x_0)$
  og $\pi_1(X,x_1)$ einsmóta.
\end{fylgisetn}
\begin{proof}
  Augljóst.
\end{proof}
\begin{skilgr}
  Grannrúm $X$ er sagt \emph{einfaldlega samanhangandi} ef það er
  vegsamanhangandi og $\pi_1(X,x_0)$ er einstökungur (bara hlutleysa)
  fyrir eitthvert (og þar með öll) $x_0$ úr $X$. Skrifum þetta of
  $\pi_1(X,x_0)=0$.
\end{skilgr}
\begin{setn}
  Í einfaldlega samanhangandi grannrúmi $X$ eru sérhverjir vegir í $X$
  sem hafa sama upphafs- og endapunkt vegsamtoga.
\end{setn}
\begin{proof}
  Látum $\alpha,\beta:I\to X$ með $\alpha(0)=\beta(0) =: x_0$ og
  $\alpha(1)=\beta(1)=:x_1$. Þá er
  $[\overline\alpha*\beta]=[e_{x_1}]$ og þar með
  \[
  [\beta]
  =[e_{x_0}*\beta]
  =[(\alpha*\overline\alpha)*\beta]
  =[\alpha*(\overline\alpha*\beta)]
  =[\alpha*e_{x_1}]
  =[\alpha].
  \]
\end{proof}

% 12.04.2010
\section{Undirstöðugrúpan og samfelldar varpanir}

Látum $f:X\to Y$ vera samfellda vörpun. Ef $\alpha$ er vegur frá $x$ til
$y$, þá er $f\circ\alpha$ vegur frá $f(x)$ til $f(y)$ í $Y$. Jafnframt
gildir:
\begin{enumerate}[(i)]
  \item Ef $[\alpha] = [\beta]$, þá eer $[f\circ\alpha]=[f\circ\beta]$.
  \item Ef $\alpha(1) = \beta(0)$, þá er $f\circ(\alpha*\beta) =
    (f\circ\alpha)*(f\circ\beta)$. 
\end{enumerate}
Af þessu sést að $f$ gefur af sér vörpun\[
 f_* : \Pi(X)\to\Pi(Y) \quad\text{þ.a.}\quad
 f_*([\alpha]*[\beta]) = f_*([\alpha])*f([\beta])
 \quad\text{ef}\quad
 \alpha(1) = \beta(0).
\]
\begin{setn}
  Látum $f:X\to Y$ vera samfellda vörpun, $x_0\in X$. 
  \begin{enumerate}[(i)]
    \item $f$ skilgreinir grúpumótun $f_*:\pi_1(X,x_0)\to\pi_1(Y,f(x_0))$
    \item Ef $f=\id_X$, þá $f_* = \id_{\pi_1(X,x_0)}$
    \item Ef $g:Y\to Z$ er samfelld, þá er $(g\circ f)_* = g_*\circ
      f_*$. 
  \end{enumerate}
\end{setn}
\begin{proof}
  Augljóst.
\end{proof}
\begin{fylgisetn}
  Ef $f:X\to Y$ er grannmótun, þá er
  $f_*:\pi_1(X,x_0)\to\pi_1(Y,f(x_0))$ grúpumótun fyrir hvaða $x_0$ úr
  $X$ sem er. Ennfremur gildir að $(f_*)^{-1} = (f^{-1})_*$.
\end{fylgisetn}
\begin{proof}
  Augljóst.
\end{proof}
\begin{ath}
  Ofangreint er unnt að setja fram á máli ríkjafræðinnar:
  \begin{enumerate}[(1)]
    \item $\Pi:\mathbf{Top}\to\mathbf{Grypi}$ sem tekur $X$ í $\Pi(X)$
      og $f$ í $\Pi(f)$ er varpi.
    \item Látum $\mathbf{Top}^*$ vera ríki allra tvennda $(X,x)$ þar sem
      $X$ er grannrúm og $x\in X$ og mótunin $f:(X,x)\to(Y,y)$ er
      samfelld vörpun $X\to Y$ þ.a. $f(x) = y$ (stundum kallað
      \emph{ríki punktaðra rúma}\index{ríki punktaðra
      rúma}\index{punktað rúm}). Fáum þá varpa
      $\pi_1:\mathbf{Top}^*\to\mathbf{Grp}$,
      $(X,x)\rightsquigarrow\pi_1(X,x)$ og \[
      ( (X,x)\xrightarrow{f}(Y,y))
      \rightsquigarrow
      (\pi_1(X,x)\xrightarrow{f_*}\pi_1(Y,y))
      \]
  \end{enumerate}
\end{ath}
\begin{setn}
  Ef $f,g:(X,x_0)\to(Y,y_0)$ eru samfelldar (mótanir í $\mathbf{Top}^*$)
  og $f\simeq g \prel{\{x_0\}}$, þá er
  $f_*=g_*:\pi_1(X,x_0)\to\pi_1(Y,y_0)$.
\end{setn}
\begin{proof}
  $f_*([\alpha])=[f\circ\alpha] = [g\circ\alpha] = g_*([\alpha])$.
\end{proof}
\begin{ath}
  Skilgreinum ríkið $\mathbf{Hot}^*$ þar sem hlutirnir eru punktuð rúm
  $(X,x)$ og mótunin $(X,x)\to(Y,y)$ er samtogunarflokkurinn af
  samfelldum vörpunum $(X,x)\to(Y,y)$ m.t.t. venslanna
  $\simeq\prel{\{x\}}$. Skv. síðustu setningu fæstþá varpi
  $\pi_1:\mathbf{Hot}^*\to\mathbf{Grp}$. Fáum víxlið örvarit
  \[
  \xymatrix{
  \mathbf{Top}^*\ar[r]^{\pi_1}\ar[d]_{\text{sjálfgefna vörpunin}}
  & \mathbf{Grp} 
  \\
  \mathbf{Hot}^*\ar[ur]_{\pi_1}
  }
  \]
\end{ath}
\begin{skilgr}
  Grannrúm $X$ er \emph{samdraganlegt}\index{samdraganlegt} ef $\id_X$
  er samtoga fastri vörpun $X\to X$.
\end{skilgr}
\begin{setn}
  Ef $X$ er samdraganlegt, þá er $\pi_1(X,x_0) = 0$ fyrir öll $x_0$ úr
  $X$.
\end{setn}
\begin{proof}
  Ljóst.
\end{proof}
\begin{daemi}
  $\pi_1(\R^n,0) = 0$ og almennt er $\pi_1(X,x)=0$ ef $X$ er $*$-laga
  hlutmengi í $\R^n$.
\end{daemi}
\begin{ath}[upprifjun]
  $X$ grannrúm og $Y$ hlutrúm í $X$, $i:Y\hookrightarrow X$ ívarpið.
  Segjum að $Y$ sé \emph{inndragi}\index{inndragi} (af) $X$ ef til er
  samfelld vörpun $r: X\to Y$ þ.a. $r\circ i = \id_Y$. Þá kallast $r$
  \emph{inndráttur}\index{inndráttur}.
\end{ath}
\begin{setn}
  Ef $Y$ er inndragi $X$ og $x_0\in Y$ og $i:Y\hookrightarrow X$
  ívarpið, þá er $i_*:\pi(Y,x_0)\to\pi_1(X,x_0)$ eintæk.
\end{setn}
\begin{proof}
  Látum $r:X\to Y$ vera inndrátt. Þá er $r\circ i = \id_Y$ og því 
  \[
  \xymatrix{
  \pi_1(Y,x_0)  \ar[r]^{i_*}\ar@/^1.5pc/[rr]^{\id}
  & \pi_1(X,x_0)\ar[r]^{r_*}
  & \pi_1(Y,x_0)
  }
  \]
  svo að $i_*$ er átæk (og $r_*$ átæk).
\end{proof}
\begin{skilgr}
  $Y$ hlutrúm í grannrúmi $X$ og $i:Y\hookrightarrow X$ ívarpið. Segjum
  að vörpun $r:X\to Y$ sé \emph{inndráttur með
  samtogun}\index{inndráttur með samtogun} ef $i\circ r\simeq\id_X\prel
  Y$. Segjum þá að $Y$ sé \emph{inndragi með samtogun af
  $X$}\index{inndragi með samtogun}.
\end{skilgr}
\begin{ath}
  \begin{enumerate}[(1)]
    \item Inndráttur með samtogun er inndráttur.
    \item $r:X\to Y$ er inndráttur með samtogun ef til er samtogun
      $H:X\times I\to X$ þ.a. 
      \begin{enumerate}[(i)]
        \item $H(x,1) = x\;\forall\,x\in X$.
        \item $H(x,0) = r(x) \;\forall\,x\in X$.
        \item $H(y,t) = y \;\forall\,y\in Y,\forall\,t\in I$. 
      \end{enumerate}
  \end{enumerate}
\end{ath}
\begin{daemi}
  \begin{enumerate}[(1)]
    \item $S^n$ er inndragi með samtogun af $\R^{n+1}\setminus\{0\}$, \[
      H:(\R^{n+1}\setminus\{0\})\times I\to\R^{n+1}\setminus\{0\}, 
      \qquad
      H(x,t) = tx + (1-t)\frac{x}{\|x\|}. \]
    \item $\{0\}\hookrightarrow\R^n$ er inndragi með samtogun: 
      \[
      H(x,t) = tx.
      \]
  \end{enumerate}
\end{daemi}
\begin{setn}
  Látum $X$ og $Y$ vera grannrúm og $P_X:X\times Y\to X$, $P_Y:X\times
  Y\to Y$ vera ofanvörpin. Ef $(x_0,y_0)\in X\times Y$, þá er \[
  \left((P_X)_*,(P_Y)_*\right):
  \pi_1(X\times Y, (x_0,y_0))
  \to
  \pi_1(X,x_0)\times \pi_1(Y,y_0)
  \]
  einsmótun.
\end{setn}
\begin{proof}
  Ef $[\alpha]\in\pi_1(X,x_0)$ og $[\beta]\in\pi_1(Y,y_0)$ þá er
  $\alpha\times\beta$ vegur í $X\times Y$ sem byrjar og endar í
  $(x_0,y_0)$ og $\left( (P_X)_*,(P_Y)_*
  \right)([\alpha\times\beta])=([\alpha],[\beta])$. Þar með er sýnt að
  $\left( (P_X)_*,(P_Y)_* \right)$ er átæk. 
  
  Sýnum að $( (P_X)_*,(P_Y)_*)$ sé eintæk: Ef $\alpha$ er vegur í
  $X\times Y$ með $(x_0,y_0)$ sem upphafspunkt og endapunkt og 
  \[
  H_X: P_X\circ\alpha\simeq_P e_{x_0},
  \qquad(
  \text{jafngilt því að }
  (P_X)_*([\alpha])=0
  )
  \]
  og
  \[
  H_Y: P_Y\circ\alpha\simeq_P e_{y_0},
  \qquad(
  \text{jafngilt því að }
  (P_Y)_*([\alpha])=0
  )
  \]
  þá er 
  \[
  H_X\times H_Y : 
  \alpha = (P_X\circ\alpha)\times(P_Y\circ\alpha)
  \simeq_P e_{(x_0,y_0)}.
  \]
\end{proof}
\begin{fylgisetn}
  Ef $Y$ er einfaldlega samanhangandi, þá leiðir ofanvarpið $P_X:X\times
  Y\to X$ af sér grúpueinsmótun $(P_X)_*:\pi_1(X\times Y, (x_0,
  y_0))\to\pi_1(X,x_0)$ fyrir öll $(x_0,y_0)\in X\times Y$.
\end{fylgisetn}
\begin{proof}
  Augljóst.
\end{proof}
\begin{fylgisetn}
  Ef $X$ og $Y$ eru einfaldlega samanhangandi, þá er $X\times Y$ það
  líka.
\end{fylgisetn}
\begin{proof}
  Ljóst.
\end{proof}
\begin{setn}
  Ef $X$ er grannrúm og $i: Y\hookrightarrow X$ er \emph{inndragi með
  samtogun}\index{inndragi með samtogun}, þá er
  $i_*:\pi_1(Y,x_0)\to\pi_1(X,x_0)$ einsmótun fyrir sérhvert $x_0$ úr
  $Y$.
\end{setn}
\begin{proof}
  Vitum að $i_*$ er eintæk. Nú er \[
  (\id_X)_* = (i\circ r)_* = i_*\circ r_*
  \]
  svo að $i_*$ er líka átæk.
\end{proof}
\begin{daemi}
  \begin{enumerate}[(1)]
    \item $X = S^1\times S^1$ og $y_0\in S^1$. Setjum $Y =
      S^1\times\{y_0\}$: Þá er $Y$ inndrægi af $X$ vegna þess að
      $S^1\times S^1\to S^1\times\{y_0\}, (x,y)\mapsto(x,y_0)$ er
      samfelld. Hér er hins vegar ekki um inndrátt með samtogun að ræða
      því að $\pi_1(S^1,y_0)\neq 0$ (sjáum það síðar).
    \item $\pi_1(S^n,x)$ er einsmóta $\pi_1(\R^{n+1}\setminus \{0\},x)$
      fyrir öll $x\in S^n$. 
    \item Ef $0\leq n < m$, þá er $\R^m\setminus\R^n$ grannmóta
      $\R^n\times (\R^{m-n}\setminus \{0\})$ svo að
      $\pi_1(\R^m\setminus\R^n,x)$ er einsmóta $\pi_1(S^{m-n-1},x)$
      fyrir öll $x\in \{0\}\times S^{m-n-1}$.
    \item $\pi_1(\C^2\setminus\C,x)$, $x\in\C^2\setminus\C$, er einsmóta
      $\pi_1(S^1,*)$ ($*$ þýðir að hér getum við tekið hvaða punkt sem
      er).
  \end{enumerate}
\end{daemi}

\section{Þekjurúm}
\begin{skilgr}
  Látum $B$ vera grannrúm. \emph{Þekjurúm}\index{zekjurúm@þekjurúm} yfir
  $B$ er grannrúm $E$ ásamt átækri samfelldri vörpun (eins konar
  ofanvarpi) $\pi:E\to B$, svonefndri
  \emph{þekjuvörpun}\index{zekuvörpun@þekjuvörpun}, sem uppfyllir
  eftirfarandi skilyrði:
  \begin{quote}
    Fyrir sérhvert $b\in B$ er til opin grennd $U$ um $b$ í $B$ þ.a.
    $\pi^{-1}(U)=\bigcup_{\alpha\in J}V_\alpha$ með $V_\alpha\cap
    V_\beta = \emptyset$ ef $\alpha\neq\beta$ og
    $\pi|_{V_\alpha}:V_\alpha\to U$ er grannmótun fyrir sérhvert
    $\alpha\in J$. 
  \end{quote}
\end{skilgr}
\begin{ath}
  (i) Þetta skilyrði má einnig orða svona: Fyrir sérhvert $b\in B$ er til
  opin grennd $U$ um $b$ í $B$, strjált rúm $F$ og grannmótun
  $\Phi:\pi^{-1}(U)\to U\times F$ þannig að örvaritið 
  \[\xymatrix{
  \pi^{-1}(U)\ar[rr]^{\Phi}\ar[dr]_{\pi|_{\pi^{-1}(U)}}
  &&
  U\times F\ar[dl]^{P_U}
  \\
  & U &
  }\]
  sé víxlið.

  (ii) Segjum að $\pi: E\to B$ sé \emph{lausblaða}\index{lausblaða
  vörpun} ef við höfum víxlið örvarit \[\xymatrix{
  E\ar[rr]^\sim\ar[dr]_\pi
  && B\times F\ar[dl]^{P_B}
  \\
  & B &
  }\]
  með $F$ strjált og $\sim$ þýðir að þarna er grannmótun.
\end{ath}
\begin{ath}
  Samfelld vörpun er þekjuvörpun \emph{þ.þ.a.a.} sérhvert $b$ úr $B$
  eigi sér opna grennd $U$ þ.a. vörpunin $\pi^{-1}(U)\to U, y\mapsto
  \pi(y)$ sé lausblaða þekjuvörpun.

% 13.04.2010
  (Æfing) Ef $\pi:E\to B$ er þekjuvörpun og $A\subseteq B$, þá er
  $\pi^{-1}(A)\to A, y\mapsto \pi(y)$ þekjuvörpun.
\end{ath}
\begin{daemi}
  \begin{enumerate}[(1)]
    \item $\exp:\C\to\C^*$ er þekjuvörpun.
    \item $\R\to S^1, t\mapsto e^{it}$ er þekjuvörpun (skv. æfingunni og
      (1)). 
    \item Fyrir sérhverja heiltölu $n>0$ er $\C^*\to\C^*,z\mapsto z^n$
      þekjuvörpun.
    \item $S^n\to\mathbb P_n(\mathbb R), x\mapsto [x]$ (þ.e. línan
      gegnum 0 og $x$) er þekuvörpun. Hér er $\mathbb P_n(\R)$ mengi
      lína gegnum núllpunkt í $\R^{n+1}$. Jafngildisvenslin $\sim$ á
      $\R^{n+1}\setminus\{0\}$ eru skilgreind með $x\sim y$
      \emph{þ.þ.a.a.} $x$ og $y$ liggi á sömu línu gegnum $0$. Setjum
      deildagrannmynstrið sem $\sim$ skilgreinir á $\mathbb
      P_n(\R)$. Ath að $S^n$ er þjappað og $\mathbb P_n(\R)$ er
      Hausdorff, svo deildavörpunin er grannmótun. 
      
      (Æfing) Látum $\pi:E\to B$ vera þekjuvörpun. Ef $B$ er
      samanhangandi, þá hefur $\pi^{-1}(b)$ sömu fjöldatölu fyrir öll
      $b\in B$. Þessi tala kallast \emph{blaðafjöldi}\index{blaðafjöldi}
      $\pi$.
    \item Ef $\pi_1:E_1\to B_1$ og $\pi_2:E_2\to B_2$ eru þekjuvarpanir,
      þá er $\pi_1\times\pi_2: E_1\times E_2\to B_1\times B_2$ líka
      þekjuvörpun. 
  \end{enumerate}
\end{daemi}
\begin{skilgr}
  Látum $\pi:E\to B$ vera þekjuvörpun og $f:X\to B$ vera samfellda
  vörpun. \emph{Lyfting á $f$ (m.t.t. $\pi$)}\index{lyfting}
  \[\xymatrix{
  & E\ar[d]^\pi
  \\
  X\ar[r]^f\ar[ur]^{\tilde f}
  & B
  }\]
  er samfelld vörpun $\tilde f:X\to E$ þ.a. $\pi\circ\tilde f=f$.
\end{skilgr}
\begin{ath}
  Þegar þekjuvörpunin sem um ræðir er $\exp:\C\to\C^*$, þá kallast
  $\tilde f$ \emph{(samfelldur) logri af $f$}\index{logri} (þ.e.a.s.
  $e^{\tilde f} = f$).
\end{ath}
\begin{setn}
  Látum $\pi:E\to B$ vera þekjuvörpun, $f:X\to B$ vera samfellda vörpun
  og $\tilde f_1,\tilde f_2$ vera lyftingar á $f$ m.t.t. $\pi$. Ef $X$
  er samanhangandi og til er $x_0$ úr $X$ þ.a. $\tilde f_1(x_0)=\tilde
  f_2(x_0)$, þá er $\tilde f_1 = \tilde f_2$. 
\end{setn}
\begin{proof}
  Setjum $X_1 := \{ x\in X : \tilde f_1(x) = \tilde f_2(x) \}$ og sýnum
  að $X_1$ sé bæði opið og lokað í $X$. Af því leiðir að $X_1 = X$ (og
  þar með $\tilde f_1 = \tilde f_2$) vegna þess að $X$ er samanhangandi
  og $x_0\in X_1$ (og þar með $x_1\neq\emptyset$).  Látum $x\in X$ og
  veljum opna grennd $U$ um $f(x)$ í $B$ þ.a. til sé grannmótun
  $\Phi:\pi^{-1}(U)\to U\times F$ þar sem $F$ er strjált grannmrúm og
  örvaritið \[\xymatrix{
  \pi^{-1}(U)\ar[rr]^\Phi\ar[dr]_\pi
  && U\times F\ar[dl]^{P_U}
  \\
  &U&
  }\]
  sé víxlið. Þá eru til $\alpha,\beta$ úr $F$ þ.a. $\Phi(\tilde f_1(x))
  = (f(x),\alpha)$ og $\Phi(\tilde f_2(x)) = (f(x),\beta)$. Þá er til
  opin grennd $V$ um $x$ í $X$ þ.a. $\Phi\circ\tilde f_1)(V)\subseteq
  U\times\{\alpha\}$ og $(\Phi\circ\tilde f_2)(V)\subseteq
  U\times\{\beta\}$.  Ef $x\notin X_1$, þá er $\alpha\neq\beta$ og því
  $\tilde f_1(V)\cap\tilde f_2(V) = \emptyset$ og þar með $V\subseteq
  X\setminus X_1$. Ef $x\in X_1$, þá er $\alpha = \beta$ og því
  $\tilde{f}_1(y)=\tilde f_2(y)$ fyrir öll $y\in V$; þar með er
  $V\subseteq X_1$. Við höfum því sýnt að bæði $X_1$ og $X\setminus X_1$
  eru opin, en það sýnir að $X_1$ er bæði opið og lokað.
\end{proof}
\begin{setn}
  [Samtogunarlyftingarsetningin]\index{samtogunarlyftingarsetningin}
  Látum $\pi: E\to B$ vera þekjuvörpun, $H: Y\times I\to B$ ($I=[0,1]$)
  vera samfellda vörpun (samtogun) með $H(y,0) = f(y)$ fyrir öll $y$ úr
  $Y$. Ef $\tilde f:Y\to E$ er lyfting á $f:Y\to B$ m.t.t. $\pi$, þá er
  til nákvæmlega ein lyfting $\tilde H:Y\times I\to E$ á $H$ m.t.t.
  $\pi$ þ.a. $\tilde H(y,0) = \tilde f(y)$ fyrir öll $y\in Y$.
  \[\xymatrix{
  & E\ar[d]^\pi \\
  Y\ar[r]^f\ar[ur]^{\tilde f}
  & B
  }
  \qquad
  \xymatrix{
  &E\ar[d]^\pi\\
  Y\times I\ar[r]^H\ar[ur]^{\tilde H}
  & B
  }\]
\end{setn}
\begin{proof}
  Fyrir sérhvert $y$ úr $Y$ þekjum við $H(\{y\}\times)$ með opnum
  mengjum $(U_j)_{j\in J}$ þ.a. $\pi^{-1}(U_j)\to U_j, x\mapsto\pi(x)$
  sé lausblaða fyrir sérhvert $j\in J$.  Mengin
  $(H^{-1}(U_j))_{j\in J}$ mynda opna þakningu á $\{y\}\times I$ svo
  ekki er til skipting $0=t_0<t_1<t_2<\cdots<t_m=1$ þ.a. fyrir sérhvert
  $k\in\{1,\dots,m\}$ sé $\{y\}\times[t_{k-1},t_k]$ innihaldið í
  $H^{-1}(U_{j(k)})$ fyrir eitthvert $j(k)$ úr $J$. Þar eð
  $[t_{k-1},t_k]$ er þjappað, þá er til opin grennd $V_y$ um $y$ í $Y$
  þ.a. $V_y\times[t_{k-1},t_k]\subseteq H^{-1}(U_j(k))$, fyrir
  $k=1,\dots,m$. Setjum $U:=U_{j(1)}$ og veljum grannmótun
  $\Phi:\pi^{-1}(U)\to U\times F$ þar sem $F$ er strjált þ.a. örvaritið 
  \[\xymatrix{
  \pi^{-1}(U)\ar[rr]^{\Phi}\ar[dr]
  && U\times F\ar[dl]^{P_U} \\
  & U &
  }\]
  verði víxlið. Nú er $H(V_y\times[t_0,t_1])\subseteq U$ ($t_0=0$) svo
  að $f(V_y)\subseteq U$ og þar með $\Phi(\tilde f(x)) = (f(x),j_0)$
  fyrir eitthvert $j_0\in J$ og öll $x\in V_y$ (með því að minnka e.t.v.
  grenndina $V_y$). Skilgreinum lyftingu $\tilde H_1$ á
  $H|_{V_y\times[t_0,t_1]}$ með $\tilde H_1(x,t) :=
  \Phi^{-1}(H(x,t),j_0)$. Þá er $\tilde H_1(x,0) = \tilde H_1(x,t_0) =
  \tilde f(x)$ fyrir öll $x\in V_y$. Lítum á $x\mapsto H(x,t_1)$ í stað
  $f$ og $x\mapsto \tilde H_1(x,t_1)$ í stað $\tilde f$ og fáum
  lyftingu á $H|_{V_y\times[t_1,t_2]}$ m.t.t. $\pi$. Með þessum hætti
  fæst eftirfarandi niðurstaða: Fyrir sérhvert $y\in Y$ er til opin
  grennd $V_y$ og lyfting $\tilde H_y: V_y\times I\to E$ á
  $H|_{V_y\times I}$ m.t.t. $\pi$ þ.a. $\tilde H_y(x,0)=\tilde f(x)$
  fyrir öll $x\in V_y$. Látum $y_1$ og $y_2$ vera ólíka punkta úr $Y$ og
  $z\in V_{y_1}\cap V_{y_2}$. Þá er $t\mapsto H(z,t)$ þ.a. $\tilde
  H_{y_1}(z,0) = \tilde H_{y_2}(z,0) = f(z)$ svo skv. síðustu setningu
  er $\tilde H_{y_1}(z,t) = \tilde H_{y_2}(z,t)$ fyrir sérhvert $t\in
  I$. Þetta sýnir að einskorðanir $\tilde H_{y_1}$ og $\tilde
  H_{y_2}$ við $(V_{y_1}\cap V_{y_2})\times I = (V_{y_1}\times
  I)\cap(V_{y_2}\times I)$ eru eins, og þar sem $(V_y\times
  I)_{y\in Y}$ er opin þakning á $Y\times I$, þá límast varpanirnar
  $\tilde H_y$ saman í lyftingu $\tilde H:Y\times I\to E$ þ.a.
  $\tilde H(y,0) = \tilde f(y)$ fyrir öll $y$ úr $Y$.

% 19.04.2010 
  Höfum sýnt fram á tilvist $\tilde H$. Sýnum að aðeins sé til ein slík.
  Ef $\tilde H_1:Y\times I\to E$ er önnur slík lyfting, þá eru $t\mapsto
  \tilde H(y,t)$ og $t\mapsto \tilde H_1(y,t)$ lyftingar á ferlinum
  $t\mapsto H(y,t)$ fyrir sérhvert $y\in Y$ og þar með $\tilde H(y,t) =
  \tilde H_1(y,t)$ fyrir öll $t\in I$ skv. síðustu setningu.
\end{proof}
\begin{fylgisetn}\label{fylgisetn:lyft1}
  Ef $p:E\to B$ er þekjuvörpun, $f:I^n\to B$ er samfelld vörpun og $e\in
  E$ þ.a. $p(e) = f(\mathbf 0)$, þá er til nákvæmlega ein lyfting
  $\tilde f$ á $f$ m.t.t. $p$ sem uppfyllir að $\tilde f(\mathbf 0) =
  e$. Sér í lagi er alltaf hægt að lyfta vegi ($n = 1$) og vegsamtogun
  ($n=2$).
\end{fylgisetn}
\begin{proof}
  Augljóst fyrir $n=0$. Beitum svo síðustu setningu og þrepum.
\end{proof}
\begin{fylgisetn}\label{fylgisetn:lyft2}
  Látum $p:E\to B$ vera þekjurúm, $\alpha$ og $\beta$ vera vegsamtoga
  vegi í $B$. Ef $\tilde\alpha$ og $\tilde\beta$ eru lyftingar á
  $\alpha$ og $\beta$ m.t.t. $p$ með $\tilde\alpha(0)=\tilde\beta(0)$,
  þá eru $\tilde\alpha$ og $\tilde\beta$ vegsamtoga. Sér í lagi gildir
  $\tilde\alpha(1)=\tilde\beta(1)$.
\end{fylgisetn}
\begin{proof}
  Látum $H:\alpha\simeq_p\beta$ og $\tilde H:I\times I\to E$ vera
  lyftingar á $H$ m.t.t. $p$ þ.a. $\tilde
  H(0,0)=\tilde\alpha(0)=\tilde\beta(0)$. Nú er
  $H(0,s)=\alpha(0)=\beta(0)$ fyrir öll $s\in I$ svo að $\tilde
  H(0,s)\in p^{-1}(\alpha(0))$ fyrir öll $s\in I$. En
  $p^{-1}(\alpha(0))$ er strjált rúm og því $\tilde H(0,s)$ fast, þ.e.
  $\tilde H(0,s)=\tilde\alpha(0)$ fyrir öll $s\in I$. Á sama hátt fæst
  að $\tilde H(1,s)$ er fast. Varpanirnar $t\mapsto\tilde H(t,s)$ og 
  $t\mapsto\tilde\alpha(t)$ eru lyftingar á $\alpha$ og
  $\tilde(0,0)=\tilde\alpha(0)$ svo að $\tilde H(t,0)=\tilde\alpha(t)$
  fyrir öll $t\in I$. Á sama hátt fæst að $\tilde H(t,1)=\tilde\beta(t)$
  fyrir öll $t\in I$. Þar með fæst $\tilde
  H:\tilde\alpha\simeq_p\tilde\beta$.
\end{proof}
\begin{ath}
  Lyfting á lykkju (þ.e.a.s. lokuðum vegi) þarf \emph{alls ekki} að vera
  lykkja.
\end{ath}
\begin{setn}
  $\pi_1(S^1,*)\cong\Z$
\end{setn}
\begin{proof}
  Lítum á $S^1$ sem einingarhringinn í $\C$ og reiknum út
  $\pi_1(S^1,1)$. Höfum þekjuvörpun $p:\R\to S^1,t\mapsto\exp(2\pi
  i\cdot t)$. Ljóst að $p^{-1}(1)=\Z$. Fyrir $[\alpha]$ úr
  $\pi_1(S^1,1)$ veljum við lyftingu $\tilde\alpha$ á $\alpha$ með
  $\tilde\alpha(0)=0$. Vitum að $\tilde\alpha(1)$ er óháð valinu á
  $\alpha$ úr $[\alpha]$ svo að við fáum vörpun
  $\varphi:\pi_1(S^1,1)\to\Z,[\alpha]\mapsto\tilde\alpha(1)$. Sýnum að
  $\varphi$ sé einsmótun:

  \emph{$\varphi$ er grúpumótun:} Ef $\alpha$ og $\beta$ eru lykkjur í
  $S^1$ sem byrja i $1$ og $\tilde\alpha$ og $\tilde\beta$ eru lyftingar
  á $\alpha$ og $\beta$ með $\tilde\alpha(0)=\tilde\beta(0)=0$, þá er
  $\tilde\beta_1:I\to\R,t\mapsto\tilde\alpha(1)+\tilde\beta(t)$ lyfting
  á $\beta$ sem uppfyllir $\tilde\beta_1(0)=\tilde\alpha(1)$. Þar með er
  $\tilde\alpha*\tilde\beta_1$ vel skilgreindur vegur og ljóst er að
  $\tilde\alpha*\tilde\beta_1$ er lyfting á $\alpha*\beta$ með
  $(\tilde\alpha*\tilde\beta_1)(0)=\tilde\alpha(0)=0$. Fáum því \[
  \varphi([\alpha]*[\beta])
  = (\tilde\alpha*\tilde\beta_1)(1)
  = \tilde\beta_1(1)
  = \tilde\alpha(1) + \tilde\beta(1)
  = \varphi([\alpha])+\varphi([\beta]).
  \]

  \emph{$\varphi$ er eintæk:} Ef $\alpha$ er lykkja með endapunkt $1$ í
  $S^1$, $\tilde\alpha$ lyfting á $\alpha$ með $\tilde\alpha(0)=0$ og
  $\varphi([\alpha])=0$, þ.e. $\tilde\alpha(1)=0$, þá er $\tilde\alpha$
  lykkja í $\R$ með endapunkt $0$. Þar með er til samtogun
  $H:0\simeq_p\tilde\alpha$ (0 er hér fastalykkja) sem gefur af sér
  vegsamtogun $p\circ H:1\simeq_p\alpha$ (1 er hér fastalykkja) og því
  er $[\alpha]$ hlutleysan í $\pi_1(S^1,1)$.

  \emph{$\varphi$ er átæk:} Viljum sýna að fyrir sérhvert $n\in\Z$ sé
  til $[\alpha]$ úr $\pi_1(S^1,1)$ þ.a. $\varphi([\alpha])=n$. Nú er
  $\tilde\alpha:I\to\R,t\mapsto nt$ vegur frá $0$ til $n$ í $\R$ og
  $p\circ\tilde\alpha$ er lykkja í $S^1$ með endapunkt $1$. Þar með er
  $\varphi([p\circ\tilde\alpha])=n$.
\end{proof}
\begin{fylgisetn}\label{fylgisetn:brouwer.hjalp}
  $S^1$ er ekki inndragi af $B^2$.
\end{fylgisetn}
\begin{proof}
  Ef svo væri, þá fengist eintæk grúpumótun
  $\pi_1(S^1,1)\to\pi_1(B^2,1)$, sem er í mótsögn við að
  $\pi_1(S^1,1)\cong\Z$ en $\pi_1(B^2,1)=0$.
\end{proof}
\begin{fylgisetn}
  $\pi_1(S^1\times S^1,(1,1))\cong\Z\times\Z$.
\end{fylgisetn}
\begin{proof}
  Augljóst.
\end{proof}
\begin{fylgisetn}
  $\pi_1(\C^*,1)\cong\Z$.
\end{fylgisetn}
\begin{proof}
  $S^1$ er inndragi með samtogun af $\C^*$ og því
  $\pi_1(S^1,1)\cong\pi_1(\C^*,1)$.
\end{proof}
\begin{fylgisetn}
  [Brouwer]
  Sérhver samfelld vörpun $B^2\to B^2$ hefur kyrrapunkt.
\end{fylgisetn}
\begin{proof}
  Leiðir beint af fylgisetningu \ref{fylgisetn:brouwer.hjalp} og dæmi 33
  af vikublaði 13.
\end{proof}
\begin{ath}
  [Ritháttur og talsmáti]
  $Y$ grannrúm. Fastalykkja sem hefur $y$ úr $Y$ sem
  \emph{grunnpunkt}\index{grunnpunktur} táknast hér eftir með $1_y$.
  Lykkja $\alpha$ í $Y$ með grunnpunkt $1_y$ er sögð
  \emph{núllsamtoga}\index{nullsamtoga@núllsamtoga} ef $[\alpha]=[1_y]$.
\end{ath}
\begin{ath}
  [Táknmál]
  Fyrir veg $\alpha$ er $\hat\alpha$ tilsvarandi einsmótun,
  $\overline\alpha$ vegurinn í hina áttina og $\tilde\alpha$ er lyfting.
\end{ath}
\begin{setn}
  Látum $p:E\to B$ vera þekjurúm og $b\in B$.
  \begin{enumerate}[(i)]
    \item Ef $e\in p^{-1}(b)$, þá er $p_*:\pi_1(E,e)\to\pi_1(B,b)$
      eintæk.
    \item Ef $E$ er vegsamanhangandi, þá mynda hlutgrúpurnar
      $p_*(\pi_1(E,e))$ fyrir $e$ úr $p^{-1}(b)$ samokunarflokk af
      hlutgrúpum í $\pi_1(B,b)$. 
  \end{enumerate}
\end{setn}
\begin{proof}
  (i) Ef $[\gamma]\in\pi_1(E,e)$ og
  $p_*([\gamma])=[p\circ\gamma]=[1_b]$, þá er til
  $H:p\circ\gamma\simeq_p 1_b$. Látum $\tilde H:I\times I\to E$ vera
  lyftingu á $H$, þ.e. $\tilde H(0,0)=e$. Þá er $\tilde
  H:\gamma\simeq_p 1_e$ og því $[\gamma]=[1_e]\in\pi_1(E,e)$.


  (ii) Látum $e_1,e_2\in p^{-1}(b)$ og $\omega$ vera veg frá $e_1$ til
  $e_2$ í $E$. Þá fæst einsmótun \[
  \hat\omega([\alpha]) 
  = [\omega*\alpha*\overline\omega].
  \]
  Nú er $p\circ\omega$ lykkja í $(B,b)$ svo að \[
  p_*([\omega*\alpha*\overline\omega])
  = p_*([\omega])*p_*([\alpha])*p_*([\omega])^{-1}
  \]
  og þar með 
  \[
  p_*([\omega])*p_*(\pi_1(E,e_2))*p_*([\omega])^{-1}
  = p_*(\pi_1(E,e_1)).
  \]
  Nú þarf bara að taka eftir að sérhvert $[\alpha]$ úr $\pi_1(B,b)$ er
  af gerðinni $p_*([\tilde\alpha])$, þar sem $\tilde\alpha$ er lyfting á
  $\alpha$. Þar með er sýnt að grúpurunan $(p_*(\pi_1(E,e)))_{e\in
  p^{-1}(b)}$ er mynda heilan samokunarflokk í $\pi_1(B,b)$.
\end{proof}
\begin{ath}
  Ef $p:E\to B$ er þekjurúm, þá verkar $\pi_1(B,b)$ á $p^{-1}(b)$ með
  eftirfarandi hætti: Fyrir $e$ úr $p^{-1}(b)$ og $[\alpha]$ úr
  $\pi_1(B,b)$ látum við $\tilde\alpha_e$ vera lyftinguna á $\alpha$ sem
  uppfyllir $\tilde\alpha_e(0)=e$. Setjum 
  \[  e[\alpha]:=\tilde\alpha_e. \]
  Fáum greinilega að $e[1_b] = e$ og \[
  e([\alpha]*[\beta])
  = e[\alpha*\beta]
  = (e[\alpha])[\beta].
  \]
  Öll $[\alpha]$ úr $\pi_1(B,b)$ sem halda ákveðnu $e$ úr $p^{-1}(b)$
  föstu, þ.e.a.s $e[\alpha]=e$ mnda hlutgrúpu í $\pi_1(B,b)$ sem við
  köllum \emph{stöðugleikagrúpu
  $e$}\index{stzgleikagrupa@stöðugleikagrúpa}. Auðséð er að hún er engin
  önnur en $p_*(\pi_1(E,e))$. Ef $E$ er vegsamanhangandi, þá er verkunin
  $\pi_1(B,b)$ á $p^{-1}(b)$ \emph{gegnvirk}, þ.e.
  $e\cdot\pi_1(B,b)=p^{-1}(b)$. M.ö.o. þá hefur verkunin aðeins eina
  \emph{braut}\index{braut}.
\end{ath}
\begin{fylgisetn}\label{fylgisetn:thekju.1}
  $p:E\to B$ þekjuvörpun, $E$ vegsamanhangandi, $e\in E$ og $b=p(e)$. Þá
  er $\#p^{-1}(b)=[\pi_1(B,b):p_*(\pi_1(E,e))]$.
\end{fylgisetn}
\begin{proof}
  Ljóst.
\end{proof}
\begin{fylgisetn}\label{fylgisetn:thekju.2}
  $p:E\to B$ þekjuvörpun, $B$ og $E$ vegamanhangandi, $e\in E$ og
  $b=p(e)$. Eftirfarandi skilyrði eru jafngild:
  \begin{enumerate}[(i)]
    \item $p$ er grannmótun.
    \item $p_*:\pi_1(E,E)\to\pi_1(B,b)$ er einsmótun.
    \item $p_*$ er átæk. 
  \end{enumerate}
\end{fylgisetn}
\begin{proof}
  \emph{(i)$\Rightarrow$(ii)$\Rightarrow$(iii):} Ljóst.

  \emph{(iii)$\Rightarrow$(i):} Skv. fylgisetningu
  \ref{fylgisetn:thekju.1} er $\#p^{-1}(b)=1$. Þar sem $B$ er
  vegsamanhangandi þá er $\#p^{-1}(x)=1$ fyrir öll $x\in B$. En það
  hefur í för með sér að $p$ er gagntæk, samfelld og opin, þ.e.
  grannmótun.
\end{proof}
\begin{fylgisetn}\label{fylgisetn:thekju.3}
  Ef $p:E\to B$ er þekjuvörpun, $E$ vegsamanhangandi og $B$ einfaldlega
  samanhangandi, þá er $p$ grannmótun. 
\end{fylgisetn}
\begin{proof}
  Augljóst út frá fylgisetningu \ref{fylgisetn:thekju.2}.
\end{proof}
\begin{fylgisetn}\label{fylgisetn:thekju.4}
  Ef $B$ er einfaldlega samanhangandi og vegsamanhangandi þá er sérhvert
  þekjurúm yfir $B$ lausblaða.
\end{fylgisetn}
\begin{proof}
  Ef $p:E\to B$ er þekjuvörpun, þá gildir um sérhvern (veg)samhengisþátt
  $C$ í $E$ að $p|_C:C\to B$ er þekjuvörpun (sjá dæmi sett fyrir á vbl.
  14). En þá er $p|_C:C\to B$ grannmótun skv. fylgisetningu
  \ref{fylgisetn:thekju.3}.
\end{proof}




\paragraph{}
\printindex

\end{document}
